
\documentclass[11pt,fleqn]{article} 
\usepackage[margin=0.8in, head=0.8in]{geometry} 
\usepackage{amsmath, amssymb, amsthm}
\usepackage{fancyhdr} 
\usepackage{palatino, url, multicol}
\usepackage{graphicx, pgfplots} 
\usepackage[all]{xy}
\usepackage{polynom} 
%\usepackage{pdfsync} %% I don't know why this messes up tabular column widths
\usepackage{enumerate}
\usepackage{framed}
\usepackage{setspace}
\usepackage{array,tikz}

\pgfplotsset{compat=1.6}

\pgfplotsset{soldot/.style={color=black,only marks,mark=*}} \pgfplotsset{holdot/.style={color=black,fill=white,only marks,mark=*}}


\pagestyle{fancy} 
\lfoot{}
\rfoot{2-3 Limit Laws}

\begin{document}
\renewcommand{\headrulewidth}{0pt}
\newcommand{\blank}[1]{\rule{#1}{0.75pt}}
\newcommand{\bc}{\begin{center}}
\newcommand{\ec}{\end{center}}
\renewcommand{\d}{\displaystyle}

\vspace*{-0.7in}

%%%%%%%%%intro page
\begin{center}
  \large
  \sc{Section 2-3: Limit Laws}\\
\end{center}
goals:\\
\begin{itemize}
\item Know how to evaluate limits algebraically (that is, using the limit laws from this section)
\item Recognize when a limit needs some algebraic manipulation and when it doesn't.
\item Understand the idea behind the Squeeze Theorem.
\end{itemize}

Recall that in the Section 2.2 notes we established \quad \fbox{ $\d \lim_{x \to 0} \frac{\sin(x)}{x}=1.$}\\

Rule: \hfill Example\\

\begin{enumerate}
\item \hfill $\d \lim_{x \to 5} 14 = \hspace{1in}$\\
\vfill
\item \hfill $\d \lim_{x \to 5} x = \hspace{1in}$\\
\vfill

\item \hfill $\d \lim_{x \to 0}\frac{\sin(x)}{x} +(2x+\sqrt{2})  = \hspace{1in}$\\
\vfill

\item \hfill $\d \lim_{x \to 0 } \lim_{x \to 0}\frac{\sin(x)}{x} -(2x+\sqrt{2}) = \hspace{1in}$\\
\vfill

\item \hfill $\d \lim_{x \to 0 }   \lim_{x \to 0}\frac{35\sin(x)}{x}= \hspace{1in}$\\
\vfill
\item \hfill $\d \lim_{x \to 4} (5x+20)(x-2) = \hspace{1in}$\\
\vfill
\item \hfill $\d \lim_{x \to 4}   \frac{5x+20}{x-2}= \hspace{1in}$\\
\vfill
\item \hfill $\d \lim_{x \to -2}  (8+5x)^5= \hspace{1in}$\\
\vfill
\item \hfill $\d \lim_{x \to -1} \sqrt{15-x} = \hspace{1in}$\\
\vfill


\end{enumerate}


\newpage
\begin{enumerate}
\item lesson:\\

$\displaystyle{\lim_{x \to \sqrt{2}} 5x -\sqrt{8x^2-1} }$
\vfill
\item lesson:\\

$\displaystyle{\lim_{t \to 2} \frac{x^2-4}{x-2} }$
\vfill
\item  lesson:\\

$\displaystyle{\lim_{x \to 5} \frac{3-\sqrt{x+4}}{5-x}}$
\vfill
\newpage
\item lesson:\\

$\displaystyle{\lim_{x \to 2} \frac{\frac{1}{4} - \frac{1}{2+x}}{x-2} }$
\vfill
\item lesson:\\

$\displaystyle{\lim_{x \to 2^-} \frac{x^2+4}{x-2} }$
\vfill
\item The last two problems reference the function $f(x) = \begin{cases} \frac{1}{2x} &\text{if } 0 < x \leq 2
\\ 0 &\text{if }  2<x \end{cases}$

\begin{enumerate}
\item Explain why $\displaystyle{\lim_{x \to 2} f(x)}$ does not exist.
\vfill
\item Evaluate $\displaystyle{\lim_{x \to 2^+} e^{f(x)}}.$
\end{enumerate}
\newpage
\item Squeeze Theorem
\end{enumerate}
\end{document}

