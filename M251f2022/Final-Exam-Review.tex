
\documentclass[11pt,fleqn]{article} 
\usepackage[margin=0.8in, head=0.8in]{geometry} 
\usepackage{amsmath, amssymb, amsthm}
\usepackage{fancyhdr} 
\usepackage{palatino, url, multicol}
\usepackage{graphicx, pgfplots} 
\usepackage[all]{xy}
\usepackage{polynom} 
%\usepackage{pdfsync} %% I don't know why this messes up tabular column widths
\usepackage{enumerate}
\usepackage{framed}
\usepackage{setspace}
\usepackage{array,tikz}

\pgfplotsset{compat=1.6}

\pgfplotsset{soldot/.style={color=black,only marks,mark=*}} \pgfplotsset{holdot/.style={color=black,fill=white,only marks,mark=*}}
\pgfplotsset{my style/.append style={axis x line=middle, axis y line=
middle, xlabel={$x$}, ylabel={$y$} }}

%axis equal 
\pagestyle{fancy} 
\lfoot{}
\rfoot{Review for Final}

\begin{document}
\renewcommand{\headrulewidth}{0pt}
\newcommand{\blank}[1]{\rule{#1}{0.75pt}}
\newcommand{\bc}{\begin{center}}
\newcommand{\ec}{\end{center}}
\renewcommand{\d}{\displaystyle}

\vspace*{-0.7in}

%%%%%%%%%intro page
\begin{center}
  \large
  \sc{Review for Final Exam}\\
\end{center}
\noindent\textbf{Logistics}\\

The final exam is Wednesday April 27 10:15-12:15 in Chapman 104. \\

All you need to take the exam is a writing utensil. Scratch paper or extra paper will be provided for you, if needed. Books, notes and calculators are not allowed. There will be no problems the require (or need) a calculator. \\

\noindent\textbf{Topics}\\

\noindent \fbox{Section 2.1}\\
Secant lines and tangent lines. Average velocity and instantaneous velocity. Average rate of change and instantaneous rate of change.\\

Example: Sketch the graph $y=x^3+1.$ Find the secant line between the points on the graph where $x=1$ and $x=3.$ Sketch  the secant line on the graph. Find an equation of the tangent line to the graph at $x=1$ and sketch it on the graph. (NOTE: We get to answer the second part of this question using our knowledge of the derivative!) \\

\noindent \fbox{Section 2.2}\\
One-sided and two-sided limits from a graph. Vertical asymptotes and limits.\\

Example: Sketch a graph with \emph{all} of the following properties: \\
\begin{itemize}
\item $f(x)$ is defined for all real numbers. (ie its domain is $(-\infty,\infty).$
\item $\lim_{x \to 1^-} f(x) = 0,$ $\lim_{x \to 1^+} f(x) =4,$ $f(1) =4$
\item $\lim_{x \to 3} f(x) =4,$ $f(4)=-1.$
\item $\lim_{x \to -1^-} f(x) =\infty$
\end{itemize}


\noindent \fbox{Section 2.3}\\
Evaluating limits algebraically.\\

Example: Evaluate $\lim_{x \to 9} \frac{3-\sqrt{x}}{9-x}$ and $\lim_{x \to 1/2^+} \frac{4x^2-18x}{2x-1}$\\

\noindent \fbox{Section 2.4}\\
Continuity. From a graph, determine where a graph is or is not continuous. From an algebraic description of a function, determine where a function is or is not continuous. Be able to explain why a function is not continuous at a point. The Intermediate Value Theorem.\\

Example: Look at your graph from Section 2.2. Where does it fail to be continuous and why? Where are the functions $f(x)= \frac{3-\sqrt{x}}{9-x}$ and $g(x)=  \frac{4x^2-18x}{2x-1}$ continuous?


\noindent \fbox{Section 3.1}\\
The relationship between secant lines and the derivative.\\

Example: Explain what the expression $f'(a) = \lim_{x \to a} \frac{f(x)-f(a)}{x-a}$ means in terms of secant lines, tangent lines and the derivative. Draw a picture to illustrate you idea.\\

\noindent \fbox{Section 3.2}\\
The derivative as a function. The formal definition of the derivative. The relationship between the graph of $f(x)$ and the graph of $f'(x).$\\

Example: Sketch the derivative of your graph from the Section 2.2 example. Use the definition of the derivative to find $f'(x)$ for $f(x)=1/x^2.$\\


\noindent \fbox{Section 3.3}\\
Derivative rules: power, constant, sum/difference, product, quotient.\\

Example: Find the derivative of $y=2x^{0.05}-\frac{x}{10}$ and $f(x)=\frac{x^3}{1-x}$\\

\noindent \fbox{Section 3.4}\\
The derivative as a rate of change. Interpretations of the derivative. Velocity and acceleration.\\

Example: Assume the distance traveled by a snow machine on a straight trail is given by $s(t)$ where $t$ is in hours starting at 12 noon and $s$ is in miles. Interpret $s'(4)=10.$ Interpret $s(4)-s(0).$ Interpret $(s(4)-s(1))/(4-1).$ Using the fact that that $s''(4)=-1.2,$ estimate $s'(4.5).$\\

\noindent \fbox{Sections 3.5-3.9}\\
Techniques and rule for taking derivative

\noindent \fbox{Section 4.1}\\
Related Rate Problems. All of these problems are word problems asking for a rate of change of some quantity with respect to time. 

Example:  An airplane is flying overhead at a constant elevation of 4000 feet as it passes directly over a man standing on the ground. If the plane is flying at a speed of 600 feet per second, how fast is the plane moving away from the man 5 seconds after it passes over his head? Assume the plane is flying in a straight line.\\

\noindent \fbox{Section 4.2}\\
Linear Approximations and Differentials. These problems ask you to find the linear approximation or differential of a function for particular values and then use these things (the linear approximation or differential) to estimate other things.

Example:  Find the linear approximation of $f(x)=5 \sin (x)$ when $a=0$ and use it to estimate $5 \sin (-0.1)$

Example:  Find the differential of $f(x)=4 \sqrt{x}$ when $x=9$ and use it to estimate  how much $f$ will change if $x$ changes from $9$ to $9.01$\\


\noindent \fbox{Section 4.3}\\
Maxima and minima. Absolute and local. Critical points. These problems are of two types: Finding ABSOLUTE extrema on closed-bounded intervals and finding local extrema in general.

Example: Find the absolute maximum and the absolute minimum of $f(x)=x^2-3x^{2/3}$ on $[0,8].$

Example: Identify any local extrema of $y=x^2- \frac{1}{x^2}.$\\

\noindent \fbox{Section 4.5}\\
Derivatives and the Shape of a Graph. These problems ask you to use $f'$ and $f''$ to determine when the original function, $f$, is increasing or decreasing, concave up or concave down, has extrema, has inflection points, and to draw sophisticated graphs.

Example: Draw some not-too-complicated graph. Now assume it is $f'$. What can you say about the graph of $f$?

Example: If $f' >0$ for $x>0$, $f' <0$ for $x<0$, $f'' > 0$ for $-2 \leq x \leq 2$ and $f'' < 0$ for $x<-2$ and for $x>2,$ sketch $f.$

\noindent \fbox{Section 4.6}\\
Limits at Infinity and Asymptotes. The problems either ask yo to evaluate a limit as $x \to \pm \infty$ or ask to find and justify the existence of a  horizontal asymptote.

Example: Determine if the graph of $f(x)=\frac{3x^3-e^x}{2x^3}$ has a horizontal asymptote. Justify your answer.\\

\noindent \fbox{Section 4.7}\\
Optimization Problems. These are word problems where you are asked to maximize or minimize some quantity. Crucial steps here include\\
 (a) identify the quantity to be maximized/minimized, \\
 (b) write the quantity from part (a) as a function of one variable,\\
 (c) identify the domain of the function from part (b), \\
 (d) take derivative and find critical points for function from part (b), \\
 (e) check/justify that your cp actually corresponds to a max/min,\\
 (f) answer the question.
 
 Example: Go work problems from old midterms.\\ 

\noindent \fbox{Section 4.8}\\
L'Hopital's Rule. Be able to use L'Hopital's Rule to evaluate limits of a variety of indeterminate forms.\\

Example: Evaluate $\lim_{x \to 0^+} x \ln(x^4)$\\

\noindent \fbox{Section 4.10}\\
Antiderivatives and Initial Value Problems \\

Example: Evaluate $\int (\frac{3}{sqrt{x}}-\csc^2(x)) \: dx$\\

Example: If an object as acceleration $a(t) = x + \sin(x),$ find its velocity equation assuming $v(0)=10.$\\


\noindent \fbox{Section 5.1}\\
Approximating areas. Use rectangles with left- or right-hand endpoints to estimate the area under a curve.\\ 

Example: Use $L_8$ (ie eight rectangles with left-hand endpoints) to estimate the area under $y=\sqrt{x}$ on the interval $[0,4].$ No need to get a decimal approximation. (!!)\\


\noindent \fbox{Section 5.2}\\
The Definite Integral as Signed Area under a Curve.\\

Example: Sketch the graph of $y=10 - 5x.$ Use this graph to evaluate $\int_{1}^6 (10-5x) \:dx$.\\

\noindent \fbox{Section 5.3}\\
The Fundamental Theorem of Calculus, parts I and II.\\

Example: Find the derivative of the function $F(x)=\int_1^{\cos(x)} (1-t^2) \; dt$\\

Example: Evaluate $\int_1^5 \frac{x}{1+x^4} \; dx$\\

\noindent \fbox{Section 5.4}\\
The Net Change Theorem\\

Example: If $v(t)$ is the velocity of a car along a straight road in miles per hour, interpret the meaning of $\int_1^5 v(t) \; dt = -20.$ Assume 1 and 5 are measured in hours. \\

\noindent \fbox{Section 5.5}\\
The method of substitution. (See the second example from Section 5.3 above.)\\

\noindent \fbox{Sections 5.6-5.7}\\
More integration formulas including those of exponential functions, logarithms, and inverse trigonometric functions.

\end{document}

