\documentclass[11pt,fleqn]{article} 
\usepackage[margin=0.8in, head=0.8in]{geometry} 
\usepackage{amsmath, amssymb, amsthm}
\usepackage{fancyhdr} 
\usepackage{palatino, url, multicol}
\usepackage{graphicx} 
\usepackage[all]{xy}
\usepackage{polynom} 
\usepackage{pdfsync}
\usepackage{enumerate}
\usepackage{framed}
\usepackage{setspace, adjustbox}
\usepackage{array%,tikz, pgfplots
}

\usepackage{tikz, pgfplots}
\usetikzlibrary{calc}
%\pgfplotsset{my style/.append style={axis x line=middle, axis y line=
%middle, xlabel={$x$}, ylabel={$y$}, axis equal }}
%
\pagestyle{fancy} 
\lfoot{UAF Calculus I}
\rfoot{4-5}


\newcommand{\be}{\begin{enumerate}}
\newcommand{\ee}{\end{enumerate}}

\newcommand{\bi}{\begin{itemize}}
\newcommand{\ei}{\end{itemize}}

\begin{document}
\setlength{\parindent}{0cm}
\renewcommand{\headrulewidth}{0pt}
\newcommand{\blank}[1]{\rule{#1}{0.75pt}}
\renewcommand{\d}{\displaystyle}
\vspace*{-0.7in}
\begin{center}
 {\large{ \sc{Section 4.5: Derivatives and the Shape of the Graph (day 1)}}}
\end{center}
 \begin{enumerate}
 \item When $f$ increases, decreases and its derivative.
 \vspace{2in}
 \item The First Derivative Test
 \vspace{1in}
 \item For the function $f(x)=\frac{2}{3}x^3+x^2-12x+7$: 
 \begin{enumerate}
 	\item Determine the intervals where $f(x)$ is increasing or decreasing.
	\item Use the First Derivative Test to identify the location of all local extrema.
	\item Use technology to confirm your work.
\end{enumerate}
\vfill
\newpage
\item Identify all local extrema for $f(x)=x^2e^{-x}.$
\vspace{1.5in}
\item Concavity and points of inflection
\vspace{2in}
\item Test for Concavity
\vspace{1in}
\item Determine the intervals for which the function $f(x)=\frac{2}{3}x^3+x^2-12x+7$ is concave up and concave down. Identify the $x$-coordinate of any inflection points.
\vfill
\item Do the same for $f(x)=x^2e^{-x}.$
\vfill
 
 \end{enumerate}
\end{document}