\documentclass[11pt,fleqn]{article} 
\usepackage[margin=0.8in, head=0.8in]{geometry} 
\usepackage{amsmath, amssymb, amsthm}
\usepackage{fancyhdr} 
\usepackage{palatino, url, multicol}
\usepackage{graphicx} 
\usepackage[all]{xy}
\usepackage{polynom} 
\usepackage{pdfsync}
\usepackage{enumerate}
\usepackage{framed}
\usepackage{setspace, adjustbox}
\usepackage{array%,tikz, pgfplots
}

\usepackage{tikz, pgfplots}
\usetikzlibrary{calc}
\pgfplotsset{my style/.append style={axis x line=middle, axis y line=
middle, xlabel={$x$}, ylabel={$y$}, axis equal }}
%
\pagestyle{fancy} 
\lfoot{UAF Calculus I}
\rfoot{4-6}


\newcommand{\be}{\begin{enumerate}}
\newcommand{\ee}{\end{enumerate}}

\newcommand{\bi}{\begin{itemize}}
\newcommand{\ei}{\end{itemize}}

\begin{document}
\setlength{\parindent}{0cm}
\renewcommand{\headrulewidth}{0pt}
\newcommand{\blank}[1]{\rule{#1}{0.75pt}}
\renewcommand{\d}{\displaystyle}
\vspace*{-0.7in}
\begin{center}
 {\large{ \sc{Section 4.6: Limits at Infinity and Asymptotes }}}
\end{center}
 \begin{enumerate}
  \item Limits at Infinity: In plain English, what should the symbols below mean?\\
 
 $\displaystyle{\lim_{x \to \infty} f(x) = L }$ \\
 
 $\displaystyle{\lim_{x \to -\infty} f(x) = L }$\\
 
 \item Three Principles ($a$ is a constant) and a Strategy
 \begin{itemize}
 	\item If $a$ is a constant, then $\displaystyle{\lim_{x \to \pm\infty} ax=}$
 	\item $\displaystyle{\lim_{x \to \pm\infty} \frac{1}{x}=}$
	\item If $\displaystyle{\lim_{x \to \pm\infty} f(x)=a}$ and $\displaystyle{\lim_{x \to \pm\infty} {g(x)}= \pm \infty}$, then  $\displaystyle{\lim_{x \to \pm\infty} \frac{f(x)}{g(x)}=}$
	\item Strategy: Divide numerator and denominator by the highest power of $x$ in the denominator.
 \end{itemize}
 \item Use the Principles to evaluate the limits below. Then, use your calculator to confirm your answer is correct.
 	\begin{enumerate}
	\item $\displaystyle{\lim_{x \to \infty} \frac{2x^2-x}{3x-5x^2}}$
	\vfill	
	\item $\displaystyle{\lim_{x \to \infty} \frac{2x^3-x}{3x-5x^2}}$
	\vfill
	\item $\displaystyle{\lim_{x \to \infty} \frac{3x+ \sin(x)}{x}}$
	\vfill
	\item $\displaystyle{\lim_{x \to -\infty} \frac{2x+1}{\sqrt{x^2+1}}}$
	\vfill
	
		\end{enumerate}
 \newpage
 \item Construct a function $f(x)$ with a vertical asymptote at $x=2$ and a horizontal asymptote at $x=5.$ Then \textbf{use limits} to demonstrate you are correct.

 \vspace{2in}

 \item Given $f(x)=\frac{x^2}{x^2+1},$ $f'(x)=\frac{2x}{(x^2+1)^2},$ $f''(x)= \frac{-2(3x^2-1)}{(x^2+1)^3}.$  Identify important features of $f(x)$ like: asymptotes, local extrema, inflection points, and make a rough sketch.
 \vfill
 \end{enumerate}
\end{document}