\documentclass[11pt]{article}
% Time-stamp: <homework-02.tex, saved on Fri, Sep 14, 2007 at 12:46pm>
\usepackage[margin=1in, head=1in]{geometry}
\usepackage{amsmath, amssymb, amsthm}
\usepackage{fancyhdr}
\usepackage{graphicx,color}

%\usepackage{pdfsync}
\addtolength{\textwidth}{.5in}
\addtolength{\leftmargin}{-1in}
\addtolength{\textheight}{.5in}
\addtolength{\topmargin}{-0.5in}

%\pagestyle{fancy}
%\lhead{MATH 200X }
%\chead{Fall 2007}
%\rhead{FINAL EXAM}
%\lfoot{}
%\cfoot{\thepage}
%\rfoot{}

\setcounter{secnumdepth}{0}
%\renewcommand{\theenumi}{\alph{enumi}}
%\renewcommand{\emptyset}{\varnothing}
\newcommand{\R}{\mathbb{R}}
\newcommand{\N}{\mathbb{N}}
\newcommand{\Z}{\mathbb{Z}}
\newcommand{\clm}{\par\textit{Claim:}\par}
\newcommand{\diam}{\mathrm{diam}}
\newcommand{\sect}{\textsection}

\parindent=0in
\parskip=0.5\baselineskip

\begin{document}
\begin{center}MATH 156: Precalculus  \\ Fall 2015 \\ MWF 11:45-12:45 Gruening 208; TR 11:30-12:30 Gruening 408 
\end{center}

\hrulefill

\begin{tabular}{p{2.5cm}p{5cm}p{0.25cm}p{2.5cm}p{5cm}}
\textbf{Instructor:} &Jill Faudree&& \textbf{Teaching}&Hector Ba\~{n}os\\
&&&\textbf{Assistant:}&\\
\textbf{Contact}& Chapman 301D&&& Chapman 303C\\
\textbf{Details:}&jrfaudree@alaska.edu &&&hdbanoscervantes@alaska.edu \\
& 474-7385 &&\\
\textbf{Office Hours:} (\textbf{\emph{tentative}})&  M 2:15-3:15, TF 10:15-11:15, W 2:15-3:00 and by appointment. Also, you are welcome to drop by. &&& MWF 9:15-10:15\\
\end{tabular}

\textbf{Textbook:} \emph{Precalculus: Mathematics for Calculus, 7th edition}, authors: Stewart, Redlin, \& Watson, publisher: Cengage Learning\\
\textbf{Course Web Page:} Blackboard (for now) and http://www.webassign.net\\
\textbf{Prerequisites:} ALEKS Math Placement score of 65 or higher\\
\textbf{WebAssign Class Key:} uaf 6081 2059

\hrulefill

\textsc{Course Overview and Goals:}\\
 From the Course Catalog:
\begin{quote}
Various classes of functions and their graphs are explored numerically, algebraically and graphically. Function classes include polynomial, rational, exponential, logarithmic and trigonometric. Skills and concepts needed for calculus are emphasized. This class is intended for students intending to take calculus I. Note: Only seven credits total may be earned from MATH F151X, MATH F152X and MATH F156X.  \end{quote}

\textbf{This course is exclusively for students who will take Calculus I next semester. If you are not intending to take MATH 251 Calculus I in Spring 2016 either at UAF or elsewhere, you need to talk with me after class.}

The purpose of this course is to develop mastery of the skills and background material required to be successful in Calculus.  Every topic covered in class, every skill appearing in homework or quizzes or tests, is one that you will need in a first semester Calculus course.

The mathematical topics covered include a brief review of algebra, equations and inequalities, functions with an emphasis on polynomial, rational, exponential, and logarithmic functions. The discussion of trigonometry will include both the unit circle and the right triangle approach. Some trigonometric identities will be covered. Solving systems of equations is included.

There are many great things about this course, but one of them is that we have a goal that is larger than the material and outcome of this one course, this one semester.  Our collective goal is successful completion of Calculus I in Spring 2016.

\textbf{The implication of the previous paragraphs is that every member should be aiming for a grade no less than a B (not a B-) on every assignment (homework, quiz, test). Any student whose major requires Calclulus II or higher,  should aim for a grade no less than an A on every assignment.}

\textsc{Course Mechanics}:

\textbf{Class meetings} will always begin with a problem on the board. This is to help you remember what we talked about in the last meeting. You should arrive ready to start working as class begins. Next we will address any questions or problems lingering from the previous class (homework, quizzes, etc.). The remainder of the class will alternate between a short introduction to a topic and working practice problems on that topic. 

Be prepared to be active in class!

\textbf{Quizzes} on the previous week's material will be given every Thursday and returned Friday at the beginning of class. I reserve the right to give unannounced quizzes on other class days if appropriate. Any student earning a score below 85\% on a quiz will be required to re-take it. If, for some reason, we need to change quiz dates, I reserve the right to do so. Any changes will be announced in class well in advance.

\textbf{Homework} sets for each section we will cover this semester are already posted on Webassign. They are generally due at 11:30PM one class day after the material is covered in class.  For example, if we cover section 2.3 on Tuesday, the homework on section 2.3 is due Wednesday by 11:30 PM.

The purpose of the homework is \emph{practice}. Webassign conveniently provides immediate feedback so you will know if your answers are wrong. A homework assignment is considered \emph{attempted} if all parts of all problems have been attempted prior to the due date. Your Webassign average will be calculated as the percentage of attempted assignments.\\

\textbf{Attendance} is required. While it is not explicitly a part of your final grade calculation, it is implicitly so. Students who miss class are statistically very unlikely to end the semester with a passing grade, much less an A or B.

There will be three {\textbf{tests} and a {\textbf{comprehensive final exam.} The tentative dates for the tests are Monday 5 October, Monday 2 November, and Monday 7 December. \textcolor{red}{\textbf{Our final exam is scheduled for Friday, December 18 from 10:15a.-12:15pm.}} \textbf{Make-up Midterms} will be given only for excused absences and only if approved in advance. Any student earning a score below 85\% will be required to retake the it.\\

\textbf{Grades} will be calculated according to the following rubric:
\begin{tabular}{|l|c|}
  \hline
  % after \\: \hline or \cline{col1-col2} \cline{col3-col4} ...
  Webassign average & 5\%\\
  quiz average & 5\% \\
  test 1 & 20\% \\
  test 2 & 20\%\\
  test 3 & 20\%\\
  final exam & 30\% \\
  \hline
\end{tabular}

Grade Bands: A, A- (90 - 100\%), B+,B, B- (80 - 89\%), C+, C, C- (70 - 79\%), D+, D, D-
(60 - 69\%), F (0 - 59\%).  I reserve the right to lower the thresholds. The grade of $A+$ is reserved for outstanding performance in the course overall.

\newpage

\textsc{(tentative) Schedule of Topics:}

\begin{tabular}{|l|l|l|l|l|}
  \hline
  % after \\: \hline or \cline{col1-col2} \cline{col3-col4} ...
  week  & topics &  & week & topics \\
  beginning&&&beginning&\\
  \hline
  8/31 & preliminaries,  \sect 1.1, 1.2 &  & 10/26 & \sect 4.4-4.6, Review\\
  9/7 & \sect 1.3-1.5 &  & 11/2 & Test 2, \sect 5.1-5.3 \\
  9/14 & \sect 1.7-1.10 &  & 11/9 & \sect 5.4-5.5, 6.1\\
  9/21 & \sect 1.11, 2.1-2.3 &  &  11/16 & \sect 6.2-6.4, 7.1 \\
  9/28 & \sect 2.4-2.6, Review &  & 11/23& \sect 7.1-7.2 \\
  10/5 &  Test 1, \sect 2.7-2.8, 3.1 &  & 11/30& Trig Review, \sect 10.1\\
  10/12 & \sect 3.2-3.4,3.6  &  &12/7 &  \sect Test 3, Review\\
  10/19 & \sect 3.7, 4.1-4.3 &  &   12/14 & Review, Final Exam Friday  Dec 18  \\
  \hline
\end{tabular}


\textsc{Miscellaneous Other Issues:}

\textbf{Communication:} We (your instructor and your TA) will communicate with you using three different channels: (1) class, (2) Blackboard (for general announcements) and (3) email (for private correspondence). We will not email you casually. If you receive an email from one of us, you need to read it and respond, if so requested.  Class time and email is also the best way for you to communicate with us. (See homework at the end of this syllabus.)

\textbf{Course accommodations:} If you need course adaptations or accommodations because of a
disability, please inform your instructor during the first week of the semester, after consulting
with the Office of Disability Services, 203 Whitaker (474-7403).

\textbf{University and Department Policies:} Your work in this course is governed by the UAF Honor
Code. The Department of Mathematics and Statistics has specific policies on incomplete grades,
late withdrawals, and early final exams, some of which are listed below. A complete listing
can be found at
http://www.dms.uaf.edu/dms/Policies.html.

\textbf{Late Withdrawal:} This semester the last day for withdrawing with a W  appearing on your
transcript is Friday October 30. After this date no student may withdraw from a course unless the student has a passing grade. If, in my opinion, a student is not participating adequately in the
class, I may elect to drop or withdraw this student. Inadequate participation includes but is not limited to: repeatedly missing class, not completing homework assignments, not taking a quiz or test, not re-taking a quiz or test when required,  or having a failing average (below 70\%) at the withdrawal date.

\textbf{Academic Honesty:} Academic dishonesty, including cheating and plagiarism, will not be tolerated. It is a violation of the Student Code of Conduct and will be punished according to
UAF procedures.

\textbf{Courtesies:} As a courtesy to your instructor and fellow students, please arrive to class on
time, turn off your electronic devices (phones, laptops, iPods, etc.) and pay attention in class.\\

\vfill

\begin{center} \sc{Your First Assignment}\end{center} 
By 10:00am Friday September 4, you must:
\begin{itemize}
\item send an email to Jill using your preferred email address
\item read over this entire syllabus
\item sign into Webassign and set up your account
\item open/download Homework 1.1, read through all of the problems, and attempt at least 5.
\end{itemize}





\end{document}
