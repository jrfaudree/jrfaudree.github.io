\documentclass[11pt]{article}
% Time-stamp: <homework-02.tex, saved on Fri, Sep 14, 2007 at 12:46pm>
\usepackage[margin=1in, head=1in]{geometry}
\usepackage{amsmath, amssymb, amsthm}
\usepackage{fancyhdr}
\usepackage{graphicx}
\usepackage{pgfplots}

%\usepackage{pdfsync}
\addtolength{\textwidth}{.5in}
\addtolength{\leftmargin}{-1in}
\addtolength{\textheight}{.5in}
\addtolength{\topmargin}{-0.5in}

%\pagestyle{fancy}
%\lhead{MATH 200X }
%\chead{Fall 2007}
%\rhead{FINAL EXAM}
%\lfoot{}
%\cfoot{\thepage}
%\rfoot{}

\setcounter{secnumdepth}{0}
%\renewcommand{\theenumi}{\alph{enumi}}
%\renewcommand{\emptyset}{\varnothing}
\newcommand{\R}{\mathbb{R}}
\newcommand{\N}{\mathbb{N}}
\newcommand{\Z}{\mathbb{Z}}
\newcommand{\clm}{\par\textit{Claim:}\par}
\newcommand{\diam}{\mathrm{diam}}
\newcommand{\sect}{\textsection}

\parindent=0in
\parskip=0.5\baselineskip

\begin{document} 

\begin{center}MATH 156: Precalculus  \\ Fall 2015 \\ Worksheet \sect 4.1 and 4.2: Exponential Functions and the Natural Exponential Function\end{center}

\hrulefill

\begin{enumerate}
\item For each of the following, sketch the graph of the function, find the domain and range of the function, find any $x$- or $y$-intercepts or state that none exist, identify any asymptotes, describe the end-behavior.
\begin{enumerate}
\item $f(x)=2^x $
\vfill
\item $f(x)=2^{-x} $
\vfill
\item $f(x)= -2^x$
\vfill
\item $f(x)=\frac{2^x}{3} $
\vfill
\item $f(x)= 4-2^x$
\vfill
\item $f(x)=7+2^{x-1} $
\vfill
\item $f(x)=\left(\frac{1}{2}\right)^x $
\vfill
\end{enumerate}
\newpage
\item On the same set of axes, graph $f(x)=2^x,\:f(x)=3^x,\:f(x)=4^x,\:f(x)=(1/2)^x,\:f(x)=(1/3)^x,\:f(x)=(1/4)^x.$
\vfill
\item Let $f(x)$ be an exponential function of the form $f(x)=a^x$ that contains the point $(3,\frac{1}{5})$, find $f(x).$
\vfill

\item Given that formula $A(t)=P(1+\frac{r}{n})^{nt}$ is the standard formula for compound interest, 
what do each of the following represent?
\begin{enumerate}
\item $A(t)$
\item $P$
\item $r$
\item $n$
\item $t$
\end{enumerate}
\item How much oney ahs accumulated in an account with principle of \$1000 with an interest rate of 2\% per year, compounded quarterly after 10 years?
\vfill
\newpage
\item Without using a calculator, find an exact expression and a rough approximation for each of the values below assuming $f(x)=e^x.$
\begin{enumerate}
\item $f(0)$
\item $f(1)$
\item $f(-1)$
\end{enumerate}
\item On the same set of axes, graph $f(x)=2^x,\:f(x)=3^x,\: f(x)=e^x.$
\vfill
\item (True or False) Explain your answer.\\ (a) $f(x)=e^{x^2}$ and $g(x)=(e^x)^2$ are equal. \\(b) $h(x)=e^{2x+3}$ and $k(x)=e^3\cdot (e^x)^2$ are the same function. 
\vfill
\item Are exponential functions one-to-one?

\end{enumerate}

\end{document}

