\documentclass[11pt]{article}
% Time-stamp: <homework-02.tex, saved on Fri, Sep 14, 2007 at 12:46pm>
\usepackage[margin=1in, head=1in]{geometry}
\usepackage{amsmath, amssymb, amsthm}
\usepackage{fancyhdr}
\usepackage{graphicx}
\usepackage{pgfplots}

%\usepackage{pdfsync}
\addtolength{\textwidth}{.5in}
\addtolength{\leftmargin}{-1in}
\addtolength{\textheight}{.5in}
\addtolength{\topmargin}{-0.5in}

%\pagestyle{fancy}
%\lhead{MATH 200X }
%\chead{Fall 2007}
%\rhead{FINAL EXAM}
%\lfoot{}
%\cfoot{\thepage}
%\rfoot{}

\setcounter{secnumdepth}{0}
%\renewcommand{\theenumi}{\alph{enumi}}
%\renewcommand{\emptyset}{\varnothing}
\newcommand{\R}{\mathbb{R}}
\newcommand{\N}{\mathbb{N}}
\newcommand{\Z}{\mathbb{Z}}
\newcommand{\clm}{\par\textit{Claim:}\par}
\newcommand{\diam}{\mathrm{diam}}
\newcommand{\sect}{\textsection}

\parindent=0in
\parskip=0.5\baselineskip

\begin{document} 

\begin{center}MATH 156: Precalculus  \\ Fall 2015 \\ Worksheet \sect 4.4: Laws of Logarithms\end{center}

\hrulefill

You must know all of the rules here by heart and you must know how to use them quickly.\\

\hrulefill

\begin{enumerate}
\item
\setlength{\fboxsep}{15pt}
\fbox{$y=\log_ax \Longleftrightarrow \hspace{1in}$} \hfill Example: Find $x$ for $\ln x = -2.$
\vfill

\item 
\setlength{\fboxsep}{15pt}
\fbox{$\log_a (AB) = \hspace{1in}$} \hfill Example: Evaluate $\log_6 9+\log_6 24$
\vfill


\item 
\setlength{\fboxsep}{15pt}
\fbox{$\log_a\left(\frac{A}{B}\right) = \hspace{1in}$} \hfill Example: Evaluate $\log_9 75 - \log_9 25$
\vfill


\item 
\setlength{\fboxsep}{15pt}
\fbox{$\log_a(A^c)= \hspace{1in}$} \hfill Example: Evaluate $\ln e^{-2/3}$
\vfill



\item 
\setlength{\fboxsep}{15pt}
\fbox{$\log_aA + \log_aB = \hspace{1in}$} \hfill Example: Simplify $\log_4 (16x +  4y)$
\vfill


\item 
\setlength{\fboxsep}{15pt}
\fbox{$\frac{\log_aA}{\log_aB} = \hspace{1in}$} \hfill Example: Simplify $\frac{\log_3(x+1)}{\log_3(x-1)}$
\vfill


\item 
\setlength{\fboxsep}{15pt}
\fbox{$(\log_aA)(\log_aB) = \hspace{1in}$} \hfill Example: Simplify $\left(\log_3(x+1)\right)\left(\log_3(x-1)\right)$
\vfill

\newpage
\item Change of Base Formula\\
\setlength{\fboxsep}{15pt}
\fbox{$\log_a x = \hspace{1in}$} \hfill Example: Use your calculator to find $\log_7 23.$
\vfill
\end{enumerate}

\begin{enumerate}
\item Evaluate
\begin{enumerate}
\item $\log_28^{20} $
\vfill
\item $\log_3 \frac{1}{\sqrt{81}}$
\vfill 
\item $ \log_5(\log_2 32^{25})$
\vfill
\end{enumerate}
\item Expand
\begin{enumerate}
\item $ \ln (x\sqrt[3]{y})$
\vfill
\item $\log_a\left(\frac{x^2}{2yz^3} \right)$
\vfill
\item $\ln\sqrt{x\sqrt{y+1}}$
\vfill
\end{enumerate}
\newpage
\item Combine and simplify, if possible.
\begin{enumerate}
\item $\frac{1}{3}\log_2 1000 - 3 \log_2 5 $
\vfill
\item $\log (x^2-1) - \log (x-1) $
\vfill
\item $\frac{2}{5} \log_6 (x+1)^5 + \frac{1}{2} \left[ \log_6 (x^2+4x+4) - \log_6(x+2)\right]$
\vfill
\end{enumerate}
\item Change the logarithm $\log_{13} 49 $ to base $e.$
\vfill
\item Using the previous problem, explain the relationship between the graph of $y=log_{13} x$ and $y=\ln x.$
\vfill




\end{enumerate}
\end{document}

