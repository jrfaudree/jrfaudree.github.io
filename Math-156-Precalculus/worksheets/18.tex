\documentclass[11pt]{article}
% Time-stamp: <homework-02.tex, saved on Fri, Sep 14, 2007 at 12:46pm>
\usepackage[margin=1in, head=1in]{geometry}
\usepackage{amsmath, amssymb, amsthm}
\usepackage{fancyhdr}
\usepackage{graphicx}
\usepackage{pgfplots}

%\usepackage{pdfsync}
\addtolength{\textwidth}{.5in}
\addtolength{\leftmargin}{-1in}
\addtolength{\textheight}{.5in}
\addtolength{\topmargin}{-0.5in}

%\pagestyle{fancy}
%\lhead{MATH 200X }
%\chead{Fall 2007}
%\rhead{FINAL EXAM}
%\lfoot{}
%\cfoot{\thepage}
%\rfoot{}

\setcounter{secnumdepth}{0}
%\renewcommand{\theenumi}{\alph{enumi}}
%\renewcommand{\emptyset}{\varnothing}
\newcommand{\R}{\mathbb{R}}
\newcommand{\N}{\mathbb{N}}
\newcommand{\Z}{\mathbb{Z}}
\newcommand{\clm}{\par\textit{Claim:}\par}
\newcommand{\diam}{\mathrm{diam}}
\newcommand{\sect}{\textsection}

\parindent=0in
\parskip=0.5\baselineskip

\begin{document} 

\begin{center}MATH 156: Precalculus  \\ Fall 2015 \\ Worksheet \sect 4.3: Logarithmic Functions\end{center}

\hrulefill

This is the most important section we will cover this year. Make sure to mastered the techniques here.\\

\hrulefill

The following is your life-raft for understanding and manipulating logarithmic functions. If you know this, you can reason your way out of most anything related to logarithms.\\

\begin{center}
\setlength{\fboxsep}{15pt}
\fbox{$y=\log_ax \Longleftrightarrow \hspace{1in}$}
\end{center}

Are there any restrictions on $a$?\\

What is the number $a$ called?\\

There is a word describing the relationship between logarithms and exponents that is illustrated by the box. What is is?\\

\hrulefill 

\begin{enumerate}
\item Use the box above to fill in the blanks:
\begin{enumerate}
\item The domain of \fbox{$y=\log_ax$} is\\
\item The range of \fbox{$y=\log_ax$} is\\
\item $\log_a 1= \hspace{1in}$\\
\item $\log_a a= \hspace{1in}$\\
\item $\log_a a^x= \hspace{1in}$\\
\item $a^{\log_ax} = \hspace{1in}$\\
\end{enumerate}
\item How do your answers above change is $a$ is replaced by the number $e$?
\newpage
\item Without the aid of a calculator, graph $y=\log_3 x$ by plotting at least 5 different points. Use this graph to describe the end behavior of the function and determine if the graph has any asymptotes.
\vspace{2in}
\item Use your answer to \#3 to graph each of the functions below. Include asymptotes, domain and range.
\begin{enumerate}
\item  $y=1+\log_3 (-x)$
\vspace{2in}
\item  $y=-\log_3 (x-2)$
\vspace{2in}
\end{enumerate}
\item Express in exponential form: 
\begin{enumerate}
\item $\log_{7}x=31$
\item $\log_{7}3=4y$
\end{enumerate}
\item Express in logarithmic form:
\begin{enumerate}
\item $10^{-4x}=1000$
\item $e^{2t}=3s$
\end{enumerate}
\newpage
\item Evaluate the expressions without the use of a calculator:\\

\begin{tabular}{lll}
(a) $ \log_2 32$ \hspace{1.5in} & (b) $\log_8 8^{17} $ \hspace{1.5in} & (c) $ \log_7 1 $ \\
&&\\&&\\
&&\\
(a) $ \log_{27} \frac{1}{9}$ \hfill & (b) $\ln \sqrt{e} $ \hfill & (c) $ \log 0.0001 $ \\
&&\\&&\\
\end{tabular}
\vspace{.2in}

\item Solve for $x$ in the equations below. Get an exact answer without the use of a calculator.

\begin{tabular}{lll}
(a) $ \ln x =3$ \hspace{1.5in} & (b) $\ln e^2=x $ \hspace{1.5in} & (c) $ \log_4 2=x $ \\
&&\\&&\\
&&\\
(a) $ \log_{4} x=2$ \hfill & (b) $\log_x 1000=3 $ \hfill & (c) $ \log_x 12=\frac{2}{5} $ \\
&&\\&&\\
\end{tabular}
\vspace{.2in}

\item Find the domain of the functions below
\begin{enumerate}
\item $f(x)=\log_{5}(x+4)$
\vfill
\item $f(x)=\log_{9}(x-x^2)$
\vfill
\item $f(x)=\ln x + \ln (1-x)$
\vfill
\item $f(x)=\ln( x^2)$
\end{enumerate}

\end{enumerate}
\end{document}

