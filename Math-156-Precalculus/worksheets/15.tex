\documentclass[11pt]{article}
% Time-stamp: <homework-02.tex, saved on Fri, Sep 14, 2007 at 12:46pm>
\usepackage[margin=1in, head=1in]{geometry}
\usepackage{amsmath, amssymb, amsthm}
\usepackage{fancyhdr}
\usepackage{graphicx}
\usepackage{pgfplots}

%\usepackage{pdfsync}
\addtolength{\textwidth}{.5in}
\addtolength{\leftmargin}{-1in}
\addtolength{\textheight}{.5in}
\addtolength{\topmargin}{-0.5in}

%\pagestyle{fancy}
%\lhead{MATH 200X }
%\chead{Fall 2007}
%\rhead{FINAL EXAM}
%\lfoot{}
%\cfoot{\thepage}
%\rfoot{}

\setcounter{secnumdepth}{0}
%\renewcommand{\theenumi}{\alph{enumi}}
%\renewcommand{\emptyset}{\varnothing}
\newcommand{\R}{\mathbb{R}}
\newcommand{\N}{\mathbb{N}}
\newcommand{\Z}{\mathbb{Z}}
\newcommand{\clm}{\par\textit{Claim:}\par}
\newcommand{\diam}{\mathrm{diam}}
\newcommand{\sect}{\textsection}

\parindent=0in
\parskip=0.5\baselineskip

\begin{document} 

\begin{center}MATH 156: Precalculus  \\ Fall 2015 \\ Worksheet \sect 3.2: Polynomial Functions and Their Graphs\end{center}

\hrulefill

This section is a detailed look at polynomial functions. By the end of this section, you want to be able to:
\begin{enumerate}
\item identify a polynomial function, its degree, its coefficients, and its leading coefficient.
\item be familiar with the common properties of polynomials.
\item describe the end behavior of a polynomial.
\item draw an approximate sketch of a polynomial function using its factored form.
\end{enumerate}

\hrulefill

A polynomial function has the form: \\

 
 List some properties of polynomial functions:
 
 \vspace{1in}

\hrulefill

For each function below: 
(a) Find the $x$- and $y$-intercepts of $f.$\\
(b) Describe the end behavior of $f.$\\
(c) Sketch the graph of $f.$\\
(d) Determine how many local maxima and minima $f$ has.\\

Example 1: Let $f(x)=(x+4)(x+1)(x-5).$ \\

\newpage
Example 2: $f(x)=(x+1)^2(x-5)$\\
\vfill
Example 3: $h(x)=16x^3-x^7$\\
\vfill
Example 4: $g(x)=x^4-2x^3+8x-16.$
\vfill

\newpage

\end{document}

