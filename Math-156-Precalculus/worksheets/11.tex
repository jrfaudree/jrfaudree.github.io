\documentclass[11pt]{article}
% Time-stamp: <homework-02.tex, saved on Fri, Sep 14, 2007 at 12:46pm>
\usepackage[margin=1in, head=1in]{geometry}
\usepackage{amsmath, amssymb, amsthm}
\usepackage{fancyhdr}
\usepackage{graphicx}
\usepackage{pgfplots}

%\usepackage{pdfsync}
\addtolength{\textwidth}{.5in}
\addtolength{\leftmargin}{-1in}
\addtolength{\textheight}{.5in}
\addtolength{\topmargin}{-0.5in}

%\pagestyle{fancy}
%\lhead{MATH 200X }
%\chead{Fall 2007}
%\rhead{FINAL EXAM}
%\lfoot{}
%\cfoot{\thepage}
%\rfoot{}

\setcounter{secnumdepth}{0}
%\renewcommand{\theenumi}{\alph{enumi}}
%\renewcommand{\emptyset}{\varnothing}
\newcommand{\R}{\mathbb{R}}
\newcommand{\N}{\mathbb{N}}
\newcommand{\Z}{\mathbb{Z}}
\newcommand{\clm}{\par\textit{Claim:}\par}
\newcommand{\diam}{\mathrm{diam}}
\newcommand{\sect}{\textsection}

\parindent=0in
\parskip=0.5\baselineskip

\begin{document} 

\begin{center}MATH 156: Precalculus  \\ Fall 2015 \\ Worksheet \sect 2.5: Linear Functions and Models\end{center}

\hrulefill

The purpose of this section is to add depth nuance to your understanding of lines. Thus, it is important it clarify what is OLD (you already know this) and what is NEW (the ideas introduced/emphasized here).

By the end of this section, you want to be able to recognize a linear function given ANY of the following forms and to easily go between them.
\begin{itemize}
\item algebraically ($x$'s and $y$'s)
\item graphically
\item numerically
\item verbally (sort of essay-style)
\end{itemize}

\hrulefill

{\sc{algebraically:}}  \\


{\sc{graphically:}}\\

\vspace{2in}


{\sc{numerically:}} Given the table of values below, which one(s) {\it{seem}} to belong to a linear function and which do not and WHY?? If it seems linear, write its algebraic representation.\\

\vfill
\begin{tabular}{c|c|c|c|c|c|c}
s&-1 &1&3&5&7&9\\
\hline
t&4&4.5&5&5.5&6&6.5\\
\end{tabular}
\vfill

\begin{tabular}{c|c|c|c|c|c|c}
x&2&7&12&17&22&26\\
\hline
y&1&2&3&5&8&12\\
\end{tabular}
\vfill

\begin{tabular}{c|c|c|c|c|c|c}
t&0&1&3&6&10&15\\
\hline
d&100&98&94&88&80&70\\
\end{tabular}
\vfill

\newpage

{\sc{verbally:}} Which of the following scenarios corresponds to a linear equation and why? If it seems linear, write its algebraic representation.\\


Example 1:  A population of fruit flies, denoted by $p$, doubles every 10 days. If $t$ represents time, measured in days, is $p$ a linear function of $t$?\\
\vfill

Example 2: Eleanor harvests a dozen carrots a day from her  garden through the month of August. If  $c$ represents the number of carrots Eleanor has harvested by day $d$ of August, is $c$ a linear function of $d$? How would this problem be different if we know that prior to August first, Eleanor had 
harvested a total of 40 carrots?\\

\vfill

{\sc{SUMMARY:}}

If $f(x)$ is a linear function, what is its AVERAGE RATE OF CHANGE?\\

\vfill




\end{document}

