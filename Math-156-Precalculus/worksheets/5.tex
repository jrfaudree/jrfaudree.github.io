\documentclass[11pt]{article}
% Time-stamp: <homework-02.tex, saved on Fri, Sep 14, 2007 at 12:46pm>
\usepackage[margin=1in, head=1in]{geometry}
\usepackage{amsmath, amssymb, amsthm}
\usepackage{fancyhdr}
\usepackage{graphicx}

%\usepackage{pdfsync}
\addtolength{\textwidth}{.5in}
\addtolength{\leftmargin}{-1in}
\addtolength{\textheight}{.5in}
\addtolength{\topmargin}{-0.5in}

%\pagestyle{fancy}
%\lhead{MATH 200X }
%\chead{Fall 2007}
%\rhead{FINAL EXAM}
%\lfoot{}
%\cfoot{\thepage}
%\rfoot{}

\setcounter{secnumdepth}{0}
%\renewcommand{\theenumi}{\alph{enumi}}
%\renewcommand{\emptyset}{\varnothing}
\newcommand{\R}{\mathbb{R}}
\newcommand{\N}{\mathbb{N}}
\newcommand{\Z}{\mathbb{Z}}
\newcommand{\clm}{\par\textit{Claim:}\par}
\newcommand{\diam}{\mathrm{diam}}
\newcommand{\sect}{\textsection}

\parindent=0in
\parskip=0.5\baselineskip

\begin{document} 

\begin{center}MATH 156: Precalculus  \\ Fall 2015 \\ Worksheet \sect 1.10: Lines\end{center}

\hrulefill

By the end of this section you must know how to:
\begin{itemize}
\item find the slope of a line given
\begin{itemize}
\item its equation, OR
\item its graph, OR
\item two points on the line 
\end{itemize}

\item find the equation of a line given

\begin{itemize}
\item two points on the line OR
\item its slope and a point on the line OR
\item its slope and an intercept ($x$ or $y$) OR
\item a point and a line parallel to it OR
\item a point and a line perpendicular to it OR
\item its graph
\end{itemize}

\item graph a line given its equation

\item find the $x$- and $y$-intercepts of a line (if they exist)

\item write and recognize the equation of a vertical line and a horizontal line

\end{itemize}

\hrulefill

{\sc{definition:}} The slope $m$ of a line through $A(x_1,y_1)$ and $B(x_2,y_2)$ is

\vspace{1in}

{\sc{example 1:}} Given the  points $A(2, -3),$ $B(10, 5),$ $C(-5,-3)$ and $D(2,7)$, find the slope of a line through:
\begin{enumerate}
\item $A$ and $B$
\vfill
\item $A$ and $C$
\vfill
\item $A$ and $D$
\vfill
\end{enumerate}

%newpage
\newpage

{\sc{use old to make new:}} Use the DEFINITION OF SLOPE ABOVE to show that\\

if you KNOW (1) the slope $m$ and (2) a point on the line $(x_1,x_2)$ you can write the equation of the line. (This is called {\it{ point-slope form}}.)

\vspace{1in}

{\sc{example 2:}} Write the equation of the line with a slope of $5$ and containing the point $(-20,4).$

\vspace{1in}


{\sc{Notation:}}  A line in the form $y=mx+b$ is called {\it{slope-intercept}} form.\\

{\sc{example:}}  Rewrite the  equation from example 2 so that it is in slope-intercept form.

\vspace{1in}

{\sc{Questions:}} What is the $y$-intercept of the line from example 2? The $x$-intercept?
\vspace{1in}

{\sc{Graph the line from example 2.}}

\newpage

{\sc{Example 3:}} Write the slope-intercept form of the equation of the line containing the points $(-4,3)$ and $(1,13).$
\vfill
{\sc{use old to make new:}} What slope to horizontal lines have and why?

\vspace{1in}

{\sc{use old to make new:}} What slope to vertical lines have and why?

{\sc{Example 4:}} Write an equation of the horizontal line through the point $(3,5).$
\vspace{1in}

{\sc{Example 5:}} Write an equation of the vertical line through point $(3,5).$ (HINT: Look again the the answer to Example 4 and observe that vertical lines should have the same sort of ``form" as horizontal lines....)
\vspace{1in}

{\sc{Questions:}} If two lines are parallel, what is the relationship between their slopes? What if they are perpendicular?

\vspace{1in}

{\sc{Example:}} Let line $L_1$ have equation $2x-3y=12.$
\begin{enumerate}
\item Write an equation of a line parallel to line $L_1$ containing the point $(1,7).$
\vfill
\item Write an equation of a line perpendicular to the line $L_1$ containing the point $(1,7).$
\vfill
\end{enumerate}


\newpage

{\sc{You Construct Examples:}}

\begin{enumerate}

\item Give the equation of a line that has no $x$-intercepts and sketch it.

\vfill
\item Give the equation of a line that has no $y$-intercepts and sketch it.

\vfill

\item Give the equation of a line with negative slope and sketch it.

\vfill

\end{enumerate}

{\sc{Last Example}} The relationship between Fahrenheit (F) and Celsius (C) temperature is given by $F=\frac{9}{5}C+32.$
\begin{enumerate}
\item Explain how you know $F$ and $C$ have a linear relationship.
\item Give a detailed interpretation of what the $9/5$ means in this equation.
\item Give a detailed interpretation of what the $32$ means in this equation.
\item Water boils at $212^oF$. Use the equation to determine the temperature in degrees Celsius at which water boils.
\item Find the temperature at which the scales agree.
\end{enumerate}
\end{document}
