\documentclass[11pt]{article}
% Time-stamp: <homework-02.tex, saved on Fri, Sep 14, 2007 at 12:46pm>
\usepackage[margin=1in, head=1in]{geometry}
\usepackage{amsmath, amssymb, amsthm}
\usepackage{fancyhdr}
\usepackage{graphicx}
\usepackage{pgfplots}

%\usepackage{pdfsync}
\addtolength{\textwidth}{.5in}
\addtolength{\leftmargin}{-1in}
\addtolength{\textheight}{.5in}
\addtolength{\topmargin}{-0.5in}

%\pagestyle{fancy}
%\lhead{MATH 200X }
%\chead{Fall 2007}
%\rhead{FINAL EXAM}
%\lfoot{}
%\cfoot{\thepage}
%\rfoot{}

\setcounter{secnumdepth}{0}
%\renewcommand{\theenumi}{\alph{enumi}}
%\renewcommand{\emptyset}{\varnothing}
\newcommand{\R}{\mathbb{R}}
\newcommand{\N}{\mathbb{N}}
\newcommand{\Z}{\mathbb{Z}}
\newcommand{\clm}{\par\textit{Claim:}\par}
\newcommand{\diam}{\mathrm{diam}}
\newcommand{\sect}{\textsection}

\parindent=0in
\parskip=0.5\baselineskip

\begin{document} 

\begin{center}MATH 156: Precalculus  \\ Fall 2015 \\ Worksheet \sect 4.5: Exponential and Logarithmic Equations\end{center}

\hrulefill

This is the second most important section of the book. The first pages are problems from Section 4.5 using the techniques we talked about on Tuesday. The last portion consists of sample applied problems from Section 4.6.\

\hrulefill

Solve for $x.$
\begin{enumerate}
\item $e^{0.4x}=8$
\vfill
\item $\log (x-4)=6$
\vfill
\item $e^{x^2}=e^5 $
\vfill
\item $7^{-x/40}=3$
\vfill
\item $ \log_2x + \log_2(x-3)=2$
\vfill
\newpage
\item $125^x+5^{3x-1}=200$
\vfill
\item $\frac{34}{1+e^{-x}}=2$
\vfill

\item $3^{4x}-3^{2x}-6=0$
\vfill
\item $x^210^x-x10^x=2(10^x)$
\vfill
\item $ \log_3(x+15)-\log_3(x-1)=2$
\vfill
\newpage
\item $\ln (x-\frac{1}{2})+\ln 2=2 \ln x $
\vfill
\item $\log_6(\log_5 x ) = 36 $
\vfill
\item $ e^{2\ln x}=\frac{1}{e^3}$
\vfill
\item A sum of \$1000 was invested for 4 years and the interest was compounded semiannually. If this sum amounted to \$1435.77, what was the interest rate? Recall the formula: $A(t)=P(1+\frac{r}{t})^{nt}$.
\vfill
\newpage
\begin{center}
Here are problems from Section 4.6
\end{center}
\item (exponential growth: doubling time) $n(t)=n_02^{t/a}$ where $n_0$ is the initial population, $a$ is the doubling time. Then $n(t)$ is the population at time $t.$ Note that $a$ and $t$ use the same unit of measure.\\

Example: A certain bird was introduced to an island 20 years ago. The population has been observed to double every 10 years. Now the population is 4000. (a) What was the initial size of the bird population? (b) Estimate the bird population 5 years from now.

\vfill

\item (exponential growth: relative growth rate) $n(t)=n_0(e^{rt})$ where $n_0$ is the initial population, $r$ is the relative growth rate (expressed as a proportion of the present population). Then $n(t)$ is the population at time $t.$\\

Example: It is observed that a certain bacteria culture has a relative growth rate of 12\% per hour but in the presence of an antibiotic the relative growth rate is reduce to 5\% per hour. If the initial number of bacteria in the culture is 22, find the projected population after 24 hours if \\
(a) no antibiotic is present and the relative growth rate is 12\%\\
(b) an antibiotic is present in the culture and the relative growth rate is only 5\%.
\vfill

\end{enumerate}
\end{document}

