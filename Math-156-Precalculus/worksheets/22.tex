\documentclass[11pt]{article}
% Time-stamp: <homework-02.tex, saved on Fri, Sep 14, 2007 at 12:46pm>
\usepackage[margin=1in, head=1in]{geometry}
\usepackage{amsmath, amssymb, amsthm}
\usepackage{fancyhdr}
\usepackage{graphicx}
\usepackage{pgfplots}

%\usepackage{pdfsync}
\addtolength{\textwidth}{.5in}
\addtolength{\leftmargin}{-1in}
\addtolength{\textheight}{.5in}
\addtolength{\topmargin}{-0.5in}

%\pagestyle{fancy}
%\lhead{MATH 200X }
%\chead{Fall 2007}
%\rhead{FINAL EXAM}
%\lfoot{}
%\cfoot{\thepage}
%\rfoot{}

\setcounter{secnumdepth}{0}
%\renewcommand{\theenumi}{\alph{enumi}}
%\renewcommand{\emptyset}{\varnothing}
\newcommand{\R}{\mathbb{R}}
\newcommand{\N}{\mathbb{N}}
\newcommand{\Z}{\mathbb{Z}}
\newcommand{\clm}{\par\textit{Claim:}\par}
\newcommand{\diam}{\mathrm{diam}}
\newcommand{\sect}{\textsection}

\parindent=0in
\parskip=0.5\baselineskip

\begin{document} 

\begin{center}MATH 156: Precalculus  \\ Fall 2015 \\ Worksheet \sect 5.2: Trigonometric Functions of Real Numbers\end{center}

\hrulefill

Plot the terminal points for $t=0,\: \pi/6, \: \pi/4, \: \pi/3,$ and $\pi/2.$\\

\begin{tikzpicture}[baseline=(current bounding box.center), xscale=2, yscale=2]
\draw[<->] (0,-1.5) -- (0,1.5) node[above] {$y$};
\draw[<->](-1.5,0) -- (1.5, 0) node[right] {$x$};
\draw[thick] (0,0) circle (1cm);
\end{tikzpicture}

\hrulefill

Definition of the Trigonometric Functions\\

$t$ is any real number and $P(x,y)$ is the terminal point on the unit circle associated with $t:$

\begin{tabular}{lllll}
$\sin(t) =$ & \quad\quad\quad\quad& $\cos(t) =$ & \quad\quad\quad\quad &$\tan(t) =$ \\
&&&&\\
&&&&\\
$\csc(t) =$ & \quad &$\sec(t) =$ & \quad& $\cot(t) =$ \\
&&&&\\
&&&&\\
\end{tabular}

\hrulefill

Example: For $t=2\pi/3,$ find the values of each of the six trigonometric functions.
\newpage
\begin{enumerate}\item For each value of $t$ below, find each of the six trigonometric functions:
\begin{enumerate}
\item $t=5 \pi/4$\\
\vfill
\item $t=-7\pi/6$\\
\vfill
\item $4\pi/3$\\
\vfill
\end{enumerate}
\item What is the sign of $\tan t \sec t$ if the terminal point of $t$ is in quadrant IV?
\vfill
\item If $\sin t > 0$ and $\sec t < 0$, in what quadrant is the terminal point of $t$?
\vfill
\item Find the values of the trigonometric function of $t$ from the fact that $\cos t = -7/25$ and the terminal point of $t$ is in quadrant III.
\vfill
\item Determine the domain of all six trigonometric functions.
\vfill
\newpage
\item Write $\tan t$ in terms of $\sin t$ and $\cos t.$
\vfill
\item Explain why $\csc t = \frac{1}{\sin t}.$ (This is called a reciprocal identity.)
\vfill
\item Find reciprocal identities for $\sec t$ and $\cot t.$
\vfill
\item Explain why $\sin^2 t + \cos^2 t = 1$ for any choice of $t.$ (This is called a Pythagorean Identity.)
\vfill
\item Use the equation in the previous problem and the reciprocal identities to derive two more Pythagorean Identities.
\vfill
\item Use the definition to show  that $f(t)=\sin t$ is an odd function.
\vfill
\newpage
\item Determine whether the functions below are even or odd:
\begin{enumerate}
\item $f(t)=\cos t$
\vfill
\item $f(t)=\tan t$
\vfill
\item $f(t)=t^2\cos(t)$
\vfill
\end{enumerate}
\item How do you know the equation $2-2\sin x=6$ has no solutions.
\vfill
\item Find all $t$ so that $\sin t =0.$
\vfill
\item Find all $t$ so that $\cos (t/2)=0.$
\vfill
\end{enumerate}
\end{document}

