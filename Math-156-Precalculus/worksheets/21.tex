\documentclass[11pt]{article}
% Time-stamp: <homework-02.tex, saved on Fri, Sep 14, 2007 at 12:46pm>
\usepackage[margin=1in, head=1in]{geometry}
\usepackage{amsmath, amssymb, amsthm}
\usepackage{fancyhdr}
\usepackage{graphicx}
\usepackage{pgfplots}

%\usepackage{pdfsync}
\addtolength{\textwidth}{.5in}
\addtolength{\leftmargin}{-1in}
\addtolength{\textheight}{.5in}
\addtolength{\topmargin}{-0.5in}

%\pagestyle{fancy}
%\lhead{MATH 200X }
%\chead{Fall 2007}
%\rhead{FINAL EXAM}
%\lfoot{}
%\cfoot{\thepage}
%\rfoot{}

\setcounter{secnumdepth}{0}
%\renewcommand{\theenumi}{\alph{enumi}}
%\renewcommand{\emptyset}{\varnothing}
\newcommand{\R}{\mathbb{R}}
\newcommand{\N}{\mathbb{N}}
\newcommand{\Z}{\mathbb{Z}}
\newcommand{\clm}{\par\textit{Claim:}\par}
\newcommand{\diam}{\mathrm{diam}}
\newcommand{\sect}{\textsection}

\parindent=0in
\parskip=0.5\baselineskip

\begin{document} 

\begin{center}MATH 156: Precalculus  \\ Fall 2015 \\ Worksheet \sect 5.1: The Unit Circle\end{center}

\hrulefill

The unit circle in the $xy$-plane has equation: \\
\vspace{.5in}

The circumference of the unit circle is: \\
\vspace{0.5in}

\hrulefill

\begin{enumerate}
\item Determine which of the following points lie on the unit circle: $P(\frac{-\sqrt{11}}{6},\frac{5}{6})$, $Q(0,-1)$, $R(\frac{2\sqrt{5}}{5},\frac{2}{5})$, $S(\frac{-3}{4},\frac{-2\sqrt{7}}{4}).$
\vfill

\item Find the missing coordinate of $P$ using the fact that $P$ lies on the unit circle in the given quadrant.\\
\begin{enumerate}
\item $P(1/2, y)$, quadrant IV\\
\vspace{.5in}
\item $P(x,-2/7)$, quadrant III
\vspace{.5in}
\end{enumerate}

\hrulefill

In our text, $t$, will always represent a distance along the unit circle starting at $(1,0)$ in the clockwise direction if $t>0$ and a counterclockwise direction if $t<0.$

\begin{tikzpicture}[baseline=(current bounding box.center), xscale=1, yscale=1]
\draw[<->] (0,-2) -- (0,2) node[above] {$y$};
\draw[<->](-2,0) -- (2, 0) node[right] {$x$};
\draw[thick] (0,0) circle (1cm);
\end{tikzpicture}
\quad
\begin{tikzpicture}[baseline=(current bounding box.center), xscale=1, yscale=1]
\draw[<->] (0,-2) -- (0,2) node[above] {$y$};
\draw[<->](-2,0) -- (2, 0) node[right] {$x$};
\draw[thick] (0,0) circle (1cm);
\end{tikzpicture}
\quad
\begin{tikzpicture}[baseline=(current bounding box.center), xscale=1, yscale=1]
\draw[<->] (0,-2) -- (0,2) node[above] {$y$};
\draw[<->](-2,0) -- (2, 0) node[right] {$x$};
\draw[thick] (0,0) circle (1cm);
\end{tikzpicture}
\newpage

\begin{tikzpicture}[baseline=(current bounding box.center), xscale=1, yscale=1]
\draw[<->] (0,-2) -- (0,2) node[above] {$y$};
\draw[<->](-2,0) -- (2, 0) node[right] {$x$};
\draw[thick] (0,0) circle (1cm);
\end{tikzpicture}
\quad
\begin{tikzpicture}[baseline=(current bounding box.center), xscale=1, yscale=1]
\draw[<->] (0,-2) -- (0,2) node[above] {$y$};
\draw[<->](-2,0) -- (2, 0) node[right] {$x$};
\draw[thick] (0,0) circle (1cm);
\end{tikzpicture}
\quad
\begin{tikzpicture}[baseline=(current bounding box.center), xscale=1, yscale=1]
\draw[<->] (0,-2) -- (0,2) node[above] {$y$};
\draw[<->](-2,0) -- (2, 0) node[right] {$x$};
\draw[thick] (0,0) circle (1cm);
\end{tikzpicture}

\item Use the unit circles above to draw and label the terminal point determined by the given value of $t$:
\begin{tabular}{llllll}
 $t=\pi$, $(x,y)=$& \quad& \quad&\quad& \quad&$t=\pi/2$, $(x,y)=$\\
 $t=-\pi/2$, $(x,y)=$&  \hspace{.5in}&\quad& \quad& \quad&$t=0$, $(x,y)=$\\
 $t=21\pi,$ $(x,y)=$ & \hspace{.5in}&\quad&\quad& \quad&$t=7\pi/2$, $(x,y)=$\\
\end{tabular}

\hrulefill

Three very special terminal points\\

\begin{tabular}{lll}
$t=\pi/6$ &$t=\pi/4$&$t=\pi/3$\\
&&\\
&&\\

\begin{tikzpicture}[baseline=(current bounding box.center), xscale=1, yscale=1]
\draw[<->] (0,-2) -- (0,2) node[above] {$y$};
\draw[<->](-2,0) -- (2, 0) node[right] {$x$};
\draw[thick] (0,0) circle (1cm);
\end{tikzpicture}
&
\begin{tikzpicture}[baseline=(current bounding box.center), xscale=1, yscale=1]
\draw[<->] (0,-2) -- (0,2) node[above] {$y$};
\draw[<->](-2,0) -- (2, 0) node[right] {$x$};
\draw[thick] (0,0) circle (1cm);
\end{tikzpicture}
&
\begin{tikzpicture}[baseline=(current bounding box.center), xscale=1, yscale=1]
\draw[<->] (0,-2) -- (0,2) node[above] {$y$};
\draw[<->](-2,0) -- (2, 0) node[right] {$x$};
\draw[thick] (0,0) circle (1cm);
\end{tikzpicture}\\

\end{tabular}

Three more instructive examples\\

\begin{tabular}{lll}
$t=3\pi/4$ &$t=-2\pi/3$&$t=23\pi/6$\\
&&\\
&&\\

\begin{tikzpicture}[baseline=(current bounding box.center), xscale=1, yscale=1]
\draw[<->] (0,-2) -- (0,2) node[above] {$y$};
\draw[<->](-2,0) -- (2, 0) node[right] {$x$};
\draw[thick] (0,0) circle (1cm);
\end{tikzpicture}
&
\begin{tikzpicture}[baseline=(current bounding box.center), xscale=1, yscale=1]
\draw[<->] (0,-2) -- (0,2) node[above] {$y$};
\draw[<->](-2,0) -- (2, 0) node[right] {$x$};
\draw[thick] (0,0) circle (1cm);
\end{tikzpicture}
&
\begin{tikzpicture}[baseline=(current bounding box.center), xscale=1, yscale=1]
\draw[<->] (0,-2) -- (0,2) node[above] {$y$};
\draw[<->](-2,0) -- (2, 0) node[right] {$x$};
\draw[thick] (0,0) circle (1cm);
\end{tikzpicture}\\

\end{tabular}

\hrulefill

One last definition:

$\overline{t}$ is called the reference number associated with $t$ and is the shortest distance along the unit circle between the terminal point and the $x$-axis.


\newpage

\item Find the reference number and terminal point for each value of $t.$
\begin{enumerate}
\item $t=5\pi/3$
\vfill
\item $t=7\pi/6$
\vfill

\item $t=-7\pi/4$
\vfill

\item $t=-17\pi3$
\vfill

\item $t=\pi$
\vfill

\item $t=31\pi/6$
\vfill
\end{enumerate}

\item Explain what a \emph{radian} is.
\vfill

\end{enumerate}
\end{document}

