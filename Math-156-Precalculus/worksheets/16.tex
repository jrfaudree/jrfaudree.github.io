\documentclass[11pt]{article}
% Time-stamp: <homework-02.tex, saved on Fri, Sep 14, 2007 at 12:46pm>
\usepackage[margin=1in, head=1in]{geometry}
\usepackage{amsmath, amssymb, amsthm}
\usepackage{fancyhdr}
\usepackage{graphicx}
\usepackage{pgfplots}

%\usepackage{pdfsync}
\addtolength{\textwidth}{.5in}
\addtolength{\leftmargin}{-1in}
\addtolength{\textheight}{.5in}
\addtolength{\topmargin}{-0.5in}

%\pagestyle{fancy}
%\lhead{MATH 200X }
%\chead{Fall 2007}
%\rhead{FINAL EXAM}
%\lfoot{}
%\cfoot{\thepage}
%\rfoot{}

\setcounter{secnumdepth}{0}
%\renewcommand{\theenumi}{\alph{enumi}}
%\renewcommand{\emptyset}{\varnothing}
\newcommand{\R}{\mathbb{R}}
\newcommand{\N}{\mathbb{N}}
\newcommand{\Z}{\mathbb{Z}}
\newcommand{\clm}{\par\textit{Claim:}\par}
\newcommand{\diam}{\mathrm{diam}}
\newcommand{\sect}{\textsection}

\parindent=0in
\parskip=0.5\baselineskip

\begin{document} 

\begin{center}MATH 156: Precalculus  \\ Fall 2015 \\ Worksheet \sect 3.1: Quadratic Functions and Models\end{center}

\hrulefill

This section is a detailed look at quadratic functions. In particular, given a quadratic function, you want to be able to:
\begin{enumerate}
\item put it in standard form
\item quickly sketch the graph using the standard form
\item find the maximum or minimum
\item find the $x$-intercepts, if any exist
\end{enumerate}

\hrulefill

A quadratic function is a polynomial of degree \underline{\hspace{.5in}}. The most common way to see a quadratic function\\

 is \underline{\hspace{2in}}.\\

A very useful form is the {\bf{standard form}} of a quadratic which is \underline{\hspace{2in}}.\\

Why is this form useful?

\hrulefill

Example 1: For the quadratic function $f(x)=3-6x-4x^2$, (a) express $f$ in standard form, (b) sketch a graph of $f,$ (c) find the maximum or minimum of $f$ (and state which it is), (d) find $x$-intercepts, if any exists, and (e) state the domain and range.
\newpage
Example 2: Same directions for $f(x)=\frac{1}{2}x^2+2x-6.$

\end{document}

