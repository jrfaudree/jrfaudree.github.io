\documentclass[12pt]{article}
\usepackage[margin=.8in]{geometry}
\usepackage{amsmath,amssymb,amsthm, latexsym, mathrsfs, pdfsync, 
%setspace,
%graphics, 
fancybox, fancyhdr,
%pictex, 
graphicx, enumerate,
subfig, multicol}


\usepackage{tikz}
\usepackage{pgf}
\usepackage{pgfplots}
\usetikzlibrary{calc}

\newcommand{\blankbox}[2]{\fbox{\rule{#1}{0in}\rule{0in}{#2}}}
%		Problem and Part
\newcounter{problemnumber}
\newcounter{partnumber}
\newcommand{\Problem}{\stepcounter{problemnumber}\setcounter{partnumber}{0}
       \item[\fbox{\parbox{.18in}{\hfil\theproblemnumber\hfil}}]}
\newcommand{\Part}{\stepcounter{partnumber}\item[(\alph{partnumber})]}
\newcommand{\Points}[1]{(#1 points) \quad}

\def\RR{{\mathbb R}}
\def\NN{{\mathbb N}}
\def\ZZ{{\mathbb Z}}
\def\QQ{{\mathbb Q}}
\def\CC{{\mathbb C}}
\def\bc{\begin{center}}
\def\ec{\end{center}}

\newcommand{\be}{\begin{enumerate}}
\newcommand{\ee}{\end{enumerate}}
\newcommand{\bpm}{\begin{pmatrix}}
\newcommand{\epm}{\end{pmatrix}}
\newcommand{\bv}[1]{\mathbf{#1}}
\newcommand{\spn}[1]{\text{Span}\left\{#1\right\}}

\newcommand\answerbox[3]{#3 \fbox{\rule{#1}{0cm}\rule{0cm}{#2}}}

\setlength{\headheight}{22pt}
\setlength{\headsep}{2pt}

\lhead{\sc Math F156X}
\chead{\Large \sc Test 1 -- Retake -- Version A} 
\rhead{\sc Fall 2015}
\cfoot{}
\pagestyle{fancy}
%

\begin{document}
\thispagestyle{fancy}

\begin{tabular}{l@{\hspace{.075\linewidth}}  l}
Your Name (print clearly) &\\
%Your Instructor\\
\blankbox{.6\linewidth}{.45in} & Thursday, October 8\\
%\blankbox{.3\linewidth}{.45in}\\
\end{tabular}
\bigskip

\bigskip
\bigskip

{
\renewcommand{\baselinestretch}{1.8}
\setlength{\tabcolsep}{.2in}
\normalsize
\begin{center}
\begin{tabular}{|c|c|c|}
\hline
Page&Total Points&\parbox{.8in}{\hfil Score\hfil}\\
\hline
1&20&\\
\hline
2&20&\\
\hline
3&15&\\
\hline
4&20&\\
\hline
5&10&\\
\hline
6&15&\\
\hline
\hline
Total&100&\\
\hline
\end{tabular}

\end{center}
}

\bigskip

\begin{center}
\begin{Large}
Instructions and information:
\end{Large}
\end{center}

\begin{itemize}
\item Please turn off cell phones or any other thing that will go BEEP.

\item
Calculators are {\bf not} allowed on this test.
\item
Read the directions for each problem. You must always show your work to receive partial credit.  

\item Be wary of doing computations in your head. Instead, write out your
computations on the exam paper.

\item
If you need more room, use the backs of the pages and indicate to the
grader where to look.

\item
Raise your hand (or come up to the front) if you have a question.

\end{itemize}

\newpage
%%%%%%%%%%%
%BEGIN TEST
%%%%%%%%%%%
\begin{enumerate}
%%%%%%%%%%%%
%PAGE 2
%%%%%%%%%%%%
%1.5(\#57-64)new
\item (5 points) Find all real solutions of \scalebox{1.2}{$ 3x^2-4x-6=0$} by completing the square.\\

\begin{flushright}{ Answer:\underline{\hspace{2in}}}\end{flushright}
\vfill
%1.4(\#59-72) new
\item (5 points) Simplify the compound fraction \scalebox{1.2}{$ \frac{1+\frac{1}{x+1}}{2-\frac{1}{x+1}}$}.
\begin{flushright}{ Answer:\underline{\hspace{2in}}}\end{flushright}
\vfill
%1.4(\#15-24)new
\item (5 points) Simplify the rational expression \scalebox{1.2}{$ \frac{x^3-x}{3x^2+5x+2}$}.
\begin{flushright}{ Answer:\underline{\hspace{2in}}}\end{flushright}
\vfill
%1.3(\#91-96) new
\item (5 points) Factor the expression \scalebox{1.2}{$4(x+3)^{1/2}-2(x+3)^{-1/2}$} completely by factoring out the lowest power of each common factor.
\begin{flushright}{ Answer:\underline{\hspace{2in}}}\end{flushright}
\vfill
\newpage

%%%%%%%%%%
%PAGE 1
%%%%%%%%%%

%1.2(\#61-70)new
\item (5 points) Simplify the expression below and  and eliminate any negative exponents.\\

\scalebox{1.2}{ $\left( \frac{4x^{2/3}}{y^{-1/3}} \right)^{2}\frac{1}{xy}  $ }
\hfill { Answer:\underline{\hspace{2in}}}\vfill
%1.3(\#47-62) new
\item (5 points) Multiply and simplify \scalebox{1.2}{$x^{1/5}(x^{4/5} - {x^{9/5}})$}.
\begin{flushright}{ Answer:\underline{\hspace{2in}}}\end{flushright}
\vfill
%1.3(\#97-126)new
\item (5 points) Factor \scalebox{1.2}{$8-125y^3$}.
\begin{flushright}{ Answer:\underline{\hspace{2in}}}\end{flushright}
\vspace{.75in}
%1.5(\#81-86)new
\item (5 points) Use the discriminant to determine the number of real solutions to  $3x^2=x-\frac{5}{3}$.
\begin{flushright}{ Answer:\underline{\hspace{2in}}}\end{flushright}
\vspace{.75in}
\newpage
%%%%%%%%%%%%
%PAGE 3
%%%%%%%%%%%%
%1.7(\#21-52 )new
\item (5 points) Erika bikes 10 mi/hr faster than she runs. Every morning she bikes 11 miles and runs 3.2 miles for a total of 1 hour of exercise. Let $r$ represent how fast Erika runs (measured in mi/hr). Write an equation containing $r$ that can be used to solve for $r.$ {\emph{You do not need to solve for $r$.}} You should use the {\it{units}} in this problem to help you write the equation.\\

\begin{flushright}{ Answer:\underline{\hspace{2in}}}\end{flushright}
\vfill
%1.5(\#87-116)new
\item (5 points) Find all real solutions of \scalebox{1.2}{$ x^6-7x^3-8=0$}. 
\begin{flushright}{ Answer:\underline{\hspace{2in}}}\end{flushright}
\vfill

%1.8(\#101-104) and 2.1(\#55-72)new
\item (5 points) Find the domain of the function \scalebox{1.2}{$f(x)=\frac{(x+1)^3}{\sqrt{1-3x}}$}.\\
\begin{flushright}{ Answer:\underline{\hspace{2in}}}\end{flushright}
\vfill
\newpage
%%%%%%%%%%%%
%PAGE 4
%%%%%%%%%%%%
%1.9(\#89-96)new
\item (4 points) Write an equation for a circle with center $(3,-20)$ and radius 4.
\begin{flushright}{ Answer:\underline{\hspace{2in}}}\end{flushright}
\vspace{.3in}
\item (5 points each) Solve the inequalities below. Express your solutions using interval notation.
\begin{enumerate}
%1.8(\#75-90)new
\item  \scalebox{1.2}{$15-\vert 3x+1 \vert > 2$}.
\begin{flushright}{ Answer:\underline{\hspace{2in}}}\end{flushright}
\vfill
%1.8(\#59-74)new
\item  \scalebox{1.2}{$\frac{8}{x-1}\geq\frac{8}{x}$}.
\begin{flushright}{ Answer:\underline{\hspace{2in}}}\end{flushright}
\vfill
\end{enumerate}
%1.10(\#23-50)new
\item (6 points) Write an equation of a line through $(2,-5)$ perpendicular to the line $4x-y=14.$ \begin{flushright}{ Line:\underline{\hspace{2in}}}\end{flushright}
\vfill
\newpage	
%%%%%%%%%%%%
%PAGE 5
%%%%%%%%%%%%
%2.1(\#35-38)
\item (2 points each) Use $g(x)=3x-2$ to find the expressions below. You do NOT need to simplify your answers.
\begin{enumerate}
\item $g(\frac{x}{2})$\\ 
\begin{flushright}{ Answer:\underline{\hspace{2in}}}\end{flushright}
\item $g(x+5)$\\
\begin{flushright}{ Answer:\underline{\hspace{2in}}}\end{flushright}
\end{enumerate}

%2.2(\#5-28)
\item (3 points each) Sketch the graphs of the functions below. You do not have to make a table but you must plot at least two points on the graph.
\begin{enumerate}
\item $y=\frac{1}{x}$
\begin{center}
\begin{tikzpicture}[scale=.7]
\draw[<->] (-4,0) -- (4,0)node[right] {$x$};
\draw[<->] (0,-4) -- (0,4)node[above] {$y$};
\end{tikzpicture}
\end{center}
\vfill
\item $y=\vert x \vert$
\begin{center}
\begin{tikzpicture}[scale=.7]
\draw[<->] (-4,0) -- (4,0)node[right] {$x$};
\draw[<->] (0,-4) -- (0,4)node[above] {$y$};
\end{tikzpicture}
\end{center}
\end{enumerate}
\newpage
%%%%%%%%%%%%
%PAGE 6
%%%%%%%%%%%%
%2.4(\#11-24)new
\item (5 points) Find the average rate of change of the function $f(x)=x^2+1$ from $x=a$ to $x=a+h.$
\begin{flushright}{ Answer:\underline{\hspace{2in}}}\end{flushright}
\vfill
%2.3(\#31-34,43-46) 1.11(\#17-24,33-39)new
\item (10 points) Below is graphed the function $f(x)=(x-3)^2(2x+1)$ and the line $y=2x.$ Use the graphs to answer questions (a) through (e).

\begin{tikzpicture}[baseline=(current bounding box.center), xscale=1.75, yscale=.25]
\draw[<->] (0,-11) -- (0,18) node[above] {$y$};
\draw[<->](-1.3,0) -- (4.2, 0) node[right] {$x$};
\foreach \i in {-1, 1, 2,3, 4}{\draw (\i, .25) -- (\i, -.25);
\draw (\i,0) node[below] {\i};
}
\foreach \i in {-10,-5,5,10,15}{
\draw (-.1, \i) -- (.1,\i);
\draw (0,\i) node[  left] {\i};
}
\draw[smooth,samples=200,domain=-1:4.2, ultra thick,,<->]  plot({\x},{(\x-3)*(\x-3)*(2*\x+1)});
\draw[smooth,samples=200,domain=-1.3:4.2, ultra thick,<->]  plot({\x},{2*\x});
\end{tikzpicture}
\begin{enumerate}
\item Estimate $f(1).$ \hfill { Answer:\underline{\hspace{2in}}}
\vspace{.2in}
\item Estimate the $y$-intercepts of $f(x)$.\hfill{ Answer:\underline{\hspace{2in}}}
\vspace{.2in}
\item Estimate the $x$-intercepts of $f(x)$.\hfill{ Answer:\underline{\hspace{2in}}}
\item Find solutions to the inequality \\
$(x-3)^2(2x+1)>2x $
\hfill Answer:\underline{\hspace{2in}}

\item Find the open interval(s) on which $f(x)$ is increasing.\\

\hfill Answer:\underline{\hspace{2in}}
\end{enumerate}
\end{enumerate}

\end{document}
%%%%%%
%EXTRA PROBS
%%%%%%
%1.2(\#55-60)
\item (4 points) Evaluate \scalebox{1.2}{$(-32)^{-3/5}.$}\\
\begin{flushright}{ Answer:\underline{\hspace{2in}}}\end{flushright}
\vspace{.75in}