\documentclass[12pt]{article}
\usepackage[margin=.8in]{geometry}
\usepackage{amsmath,amssymb,amsthm, latexsym, mathrsfs, pdfsync, 
%setspace,
%graphics, 
fancybox, fancyhdr,
%pictex, 
graphicx, enumerate,
subfig, multicol}


\usepackage{tikz}
\usepackage{pgf}
\usepackage{pgfplots}
\usetikzlibrary{calc}

\newcommand{\blankbox}[2]{\fbox{\rule{#1}{0in}\rule{0in}{#2}}}
%		Problem and Part
\newcounter{problemnumber}
\newcounter{partnumber}
\newcommand{\Problem}{\stepcounter{problemnumber}\setcounter{partnumber}{0}
       \item[\fbox{\parbox{.18in}{\hfil\theproblemnumber\hfil}}]}
\newcommand{\Part}{\stepcounter{partnumber}\item[(\alph{partnumber})]}
\newcommand{\Points}[1]{(#1 points) \quad}

\def\RR{{\mathbb R}}
\def\NN{{\mathbb N}}
\def\ZZ{{\mathbb Z}}
\def\QQ{{\mathbb Q}}
\def\CC{{\mathbb C}}
\def\bc{\begin{center}}
\def\ec{\end{center}}

\newcommand{\be}{\begin{enumerate}}
\newcommand{\ee}{\end{enumerate}}
\newcommand{\bpm}{\begin{pmatrix}}
\newcommand{\epm}{\end{pmatrix}}
\newcommand{\bv}[1]{\mathbf{#1}}
\newcommand{\spn}[1]{\text{Span}\left\{#1\right\}}

\newcommand\answerbox[3]{#3 \fbox{\rule{#1}{0cm}\rule{0cm}{#2}}}

\setlength{\headheight}{22pt}
\setlength{\headsep}{2pt}

\lhead{\sc Math F156X}
\chead{\Large \sc Test 2 -- Version A} 
\rhead{\sc Fall 2015}
\cfoot{}
\pagestyle{fancy}
%

\begin{document}
\thispagestyle{fancy}

\begin{tabular}{l@{\hspace{.075\linewidth}}  l}
Your Name (print clearly) &\\
%Your Instructor\\
\blankbox{.6\linewidth}{.45in} & Monday, November 2\\
%\blankbox{.3\linewidth}{.45in}\\
\end{tabular}
\bigskip

\bigskip
\bigskip

{
\renewcommand{\baselinestretch}{1.8}
\setlength{\tabcolsep}{.2in}
\normalsize
\begin{center}
\begin{tabular}{|c|c|c|}
\hline
Page&Total Points&\parbox{.8in}{\hfil Score\hfil}\\
\hline
1&15&\\
\hline
2&18&\\
\hline
3&20&\\
\hline
4&15&\\
\hline
5&20&\\
\hline
6&12&\\
\hline
extra credit &5&\\
\hline
\hline
Total&100&\\
\hline
\end{tabular}

\end{center}
}

\bigskip

\begin{center}
\begin{Large}
Instructions and information:
\end{Large}
\end{center}

\begin{itemize}
\item Please turn off cell phones or any other thing that will go BEEP.
\item Calculators are {\bf not} allowed on this test.
\item Read the directions for each problem. You must always show your work to receive partial credit.  
\item Be wary of doing computations in your head. Instead, write out your
computations on the exam paper.
\item If you need more room, use the backs of the pages and indicate to the
grader where to look.
\item Raise your hand (or come up to the front) if you have a question.
\end{itemize}

\hrulefill

%%%%%%% Formulas %%%%%%%%%%
\begin{center} Formulas \\ 
\begin{tabular}{lll}
$n(t)=n_02^{t/a}$&\hspace{-.2in}&$n(t)=n_0e^{rt}$\\
$m(t)=m_02^{-t/h}$&\hspace{-.2in}&$m(t)=m_0e^{rt}$ where $r=(\ln 2)/h,$\\

$A(t)=P(1+\frac{r}{n})^{nt}$&\hspace{-.2in}&$A(t)=Pe^{rt}$\\
$\log_b x = (\log_a x )/(\log_a b)$&&\\
\end{tabular}
\end{center}

\newpage
%%%%%%%%%%%
%BEGIN TEST
%%%%%%%%%%%

\begin{enumerate}
%%%%%%%%%%
%PAGE 1
%%%%%%%%%%

%%%%Applied Linear
%2.5 \#39-50
\item The owner of a toy factory estimates that it costs \$1300 to produce 100 toys in one day and \$1900 to produce 300 toys in one day.
\begin{enumerate}
\item (3 points) Assuming that the relationship between cost and the number of toys produced is linear, find a linear function $C$ that models the cost of producing $x$ toys in one day.\\
\vspace{.7in}
%\begin{flushright}{Answer: \underline{\hspace{2in}}}\end{flushright}
\item (2 points) At what rate does the factory's cost increase for every additional toy produced?
\vspace{.3in}
%\begin{flushright}{Answer: \underline{\hspace{2in}}}\end{flushright}
\end{enumerate}
%%%%combining functions
%2.7 \#7-16,47-58
\item (5 points each) Let $f(x)=\sqrt{16-x^2}$ and $g(x)=\sqrt{x+2}.$
\begin{enumerate}
\item Find $f/g$ and state its domain.
\vfill
\item Find $f\circ g$ and state its domain.
\vfill
\end{enumerate}
\newpage
%%%%%%%%
%Page 2
%%%%%%%%%
%%%Graph transformations; rational and roots
%2.6 \#29-52
\item (6 points each) Sketch the graphs below. Label any asymptotes and intercepts.
\begin{enumerate}
\item 
\begin{minipage}[l]{7cm}
$f(x)=2-\sqrt[3]{x}$
\end{minipage}%
\begin{minipage}[c]{10cm}
\begin{tikzpicture}[scale=0.7]
\draw[->, thick] (-4,0) -- (4,0)node [pos=1, below] {$x$};
\draw[->, thick] (0,-4) -- (0,4) node [pos=1, left] {$y$};
\end{tikzpicture}
\end{minipage}
\vfill
\item 
\begin{minipage}[l]{7cm}
$f(x)=\frac{-3}{(x-5)^2}$
\end{minipage}%
\begin{minipage}[c]{10cm}
\begin{tikzpicture}[scale=0.7]
\draw[->, thick] (-4,0) -- (4,0)node [pos=1, below] {$x$};
\draw[->, thick] (0,-4) -- (0,4) node [pos=1, left] {$y$};
\end{tikzpicture}
\end{minipage}
\vfill
\item 
\begin{minipage}[l]{7cm}
$f(x)=\vert x^2-1 \vert$
\end{minipage}%
\begin{minipage}[c]{10cm}
\begin{tikzpicture}[scale=0.7]
\draw[->, thick] (-4,0) -- (4,0)node [pos=1, below] {$x$};
\draw[->, thick] (0,-4) -- (0,4) node [pos=1, left] {$y$};
\end{tikzpicture}
\end{minipage}
\end{enumerate}
\vfill
\newpage
%%%%%%%
%Page 3
%%%%%%%
%inverse functions
%2.8\#49-70
\item (5 points) Find the inverse of $h(x)=\frac{(2-x^3)^5}{7}.$
\vfill
%quadratics
%3.1\#9-24
\item Let $g(x)=5x^2-15x +2.$
\begin{enumerate}
\item (6 points) Express $g$ in standard form.
\vfill
\item (2 points) Find the vertex of the graph of $g.$
\vspace{.5in}
\item (2 points) Determine the range of $g.$
\vspace{.5in}
\end{enumerate}
%graphing polys
%3.2 \#15-30
\item (5 points)\\
\begin{minipage}[t]{8cm}
Sketch the graph of $P(x)=-(x+4)^2(x-1)^3$
on the axes. Make sure you label all intercepts and exhibits proper end behavior.
\end{minipage}%
%\quad
%\quad
%\begin{minipage}[r]{12cm}
\vspace{-1in}
\begin{flushright}
\begin{tikzpicture}[scale=0.9]
\draw[->, thick] (-4,0) -- (4,0)node [pos=1, below] {$x$};
\draw[->, thick] (0,-4) -- (0,4) node [pos=1, left] {$y$};
\end{tikzpicture}
\end{flushright}
%\end{minipage}
\newpage
%%%%%%%%
%Page 4
%%%%%%%%
%long division
%3.2\#15-24
\item (5 points) Find the quotient, $Q(x),$ and remainder  $R(x),$ of \scalebox{1.2}{$\frac{6x^2-17x+7}{2x-3} $.}
\vfill
\item (5 points) Let \scalebox{1.2}{$r(x)=\frac{x^2-25}{3x^2+17x+10}=\frac{(x-5)(x+5)}{(3x+2)(x+5)}$}
\begin{enumerate}
\item Find any horizontal asymptotes or state that none exist.
\vspace{.6in}
\item Find any vertical asymptotes or state that none exist.
\vspace{.6in}
\end{enumerate}
\item (5 points) Solve the rational inequality \scalebox{1.2}{$\frac{6x-7}{5x-2}\geq 1.$} Give your answer in interval notation.
\vfill
\newpage
%%%%%%%%
%Page 5
%%%%%%%%
%4.4\#23-48
\item (5 points) Expand \fbox{\scalebox{1.2}{$\log_2x\sqrt{\frac{y}{z}}$}} using the Laws of Logarithms.
\vfill
\item (5 points each) Solve.
\begin{enumerate}
\item $10^{1-x}=6$
\vfill
\item $\log_6 x +\log_6 (x+1)=1$
\vfill
\item $e^{2x}+e^{x}-20=0$
\vfill
\end{enumerate}
\newpage
%%%%%%
%Page 6
%%%%%%
%%%Graph transformations; exponentials and logs
\item (6 points each) Sketch the graphs below. Label any asymptotes and intercepts.
\begin{enumerate}
\item 
%4.1 27-40
\begin{minipage}[l]{7cm}
$h(x)=4-2^{x-1}$
\end{minipage}%
\begin{minipage}[c]{10cm}
\begin{tikzpicture}[scale=0.7]
\draw[->, thick] (-4,0) -- (4,0)node [pos=1, below] {$x$};
\draw[->, thick] (0,-4) -- (0,4) node [pos=1, left] {$y$};
\end{tikzpicture}
\end{minipage}
\vfill
\item 
%4.3 61-72
\begin{minipage}[t]{7cm}
$f(x)=1+\ln(-x)$\\
\end{minipage}%
\begin{minipage}[c]{10cm}
\begin{tikzpicture}[scale=0.7]
\draw[->, thick] (-4,0) -- (4,0)node [pos=1, below] {$x$};
\draw[->, thick] (0,-4) -- (0,4) node [pos=1, left] {$y$};
\end{tikzpicture}
\end{minipage}
\end{enumerate}
\vfill
\end{enumerate}
%\newpage
%%%%%%%%%
%Extra Credit
%%%%%%%%%
{\sc{Extra Credit}} (5 points) A certain population of fish has a relative growth rate of $2.5\%$ per year. How long will it take for the population to double? (Yes. You do have enough information to complete t his problem.)
\vspace{.6in}

\end{document}
%%%%%%
%EXTRA PROBS
%%%%%%
%1.2(\#55-60)
\item (4 points) Evaluate \scalebox{1.2}{$(-32)^{-3/5}.$}\\
\begin{flushright}{ Answer:\underline{\hspace{2in}}}\end{flushright}
\vspace{.75in