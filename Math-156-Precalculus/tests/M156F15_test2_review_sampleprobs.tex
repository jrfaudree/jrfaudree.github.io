\documentclass[11pt]{article}
% Time-stamp: <homework-02.tex, saved on Fri, Sep 14, 2007 at 12:46pm>
\usepackage[margin=1in, head=1in]{geometry}
\usepackage{amsmath, amssymb, amsthm}
\usepackage{fancyhdr}
\usepackage{graphicx}
\usepackage{pgfplots}

%\usepackage{pdfsync}
\addtolength{\textwidth}{.5in}
\addtolength{\leftmargin}{-1in}
\addtolength{\textheight}{.5in}
\addtolength{\topmargin}{-0.5in}

%\pagestyle{fancy}
%\lhead{MATH 200X }
%\chead{Fall 2007}
%\rhead{FINAL EXAM}
%\lfoot{}
%\cfoot{\thepage}
%\rfoot{}

\setcounter{secnumdepth}{0}
%\renewcommand{\theenumi}{\alph{enumi}}
%\renewcommand{\emptyset}{\varnothing}
\newcommand{\R}{\mathbb{R}}
\newcommand{\N}{\mathbb{N}}
\newcommand{\Z}{\mathbb{Z}}
\newcommand{\clm}{\par\textit{Claim:}\par}
\newcommand{\diam}{\mathrm{diam}}
\newcommand{\sect}{\textsection}

\parindent=0in
\parskip=0.5\baselineskip

\begin{document} 

\begin{center}MATH 156: Precalculus  \\ Fall 2015 \\ Test 2: Review and Sample Problems \end{center}

\hrulefill

Test 2 will be Monday 2 November during our regular class time. The test will cover Chapter 2 Sections 5-8, Chapter 3 Sections 1-3, 6-7, Chapter 4 Sections 1-6. Calculators, notes, books or other aids will not be allowed.

\hrulefill

Chapter 2 Section 5\\
How to recognize linear functions. How to write the equation of a linear function.\\
Most important skill: Given a verbal description of a function, you must know how to write its equation and use it.

Chapter 2 Section 6\\
Transformations of functions. Vertical and horizontal shifts. Reflections about $x$- and $y$-axis. Vertical and horizontal stretching and shrinking. You should be able to recognize symmetry in graphs.\\
Most important skill: You must be able to recognize and graph a simple transformation of a known function. See the list at the end of the review for the list of functions you must know.\\

Chapter 2 Section 7\\
Combining functions. Given $f(x)$ and $g(x)$, know how to find $(f+g)(x),$ $(f-g)(x),$ $(fg)(x),$ $(f/g)(x),$ and $(f\circ g)(x),$ find their domains, and use them. \\


Chapter 2 Section 8\\
One-to-one functions and inverses. Know how to determine whether or not a function is one-to-one. Know how to find the inverse of a one-to-one function. Know the inverse function property and how to use it.\\

Chapter 3 Section 1\\
Quadratic functions and models. Be able to place a quadratic function into {\bf{standard form}} and pick off the {\bf{vertex}}. Using the standard form, be able to sketch the transformed quadratic, find any intercepts, any maximum or minimum, and identify its domain and range.\\

Chapter 3 Section 2\\
Polynomial functions and their graphs. Given a polynomial function, be able to (i) identify its end-behavior, (ii) use the multiplicity of roots and intercepts to sketch an approximate graph. Know what the leading terms implies about the number of roots and number of local extrema.\\


Chapter 3 Section 3\\
Know how to do long division and rewrite an expression $p(x)/d(x)$  using the quotient and remainder.\\

Chapter 3 Section 6\\
Given a rational function, be able to find any asymptotes, including slant asymptotes, find intercepts, and determine end behavior.\\
\newpage

Chapter 3 Section 7\\
Be able to solve rational and polynomial inequalities.\\

Chapter 4 Section 1\\
Exponential functions. Know how to sketch the graph of an exponential function and how to use an exponential function. Know how to use the formula for compound interest.

Chapter 4 Section 2\\
The natural exponential function. The same as Chapter 4 Section 1 but with base $e.$\\

Chapter 4 Section 3\\
Know the definition of a logarithmic function and rules about how to calculate with them, including simplifying and expanding. Know how to graph a logarithmic function. Know the standard notation associated with logarithms (such as those base 10 and base $e.$)\\

Chapter 4 Section 4\\
Laws of logarithms. Know them. Be aware of common mistakes and don't make them! Know the change of base formula.

Chapter 4 Section 5\\
Exponential and Logarithmic equations. Have facility solving equations with logarithms. Recall that our techniques include: rewriting a logarithmic equation in exponential form, rewriting an exponential equation in logarithmic form, taking the logarithm of both sides, rewriting terms in an exponential equation with the same base, identify a hidden quadratic equation.\\

Chapter 4 Section 6\\
Be able to use formulas for doubling, exponential growth, and radioactive decay.\\

Functions you must know how to graph: \\
$y=mx+b,$ \\
$y=x^2,\:y=x^3,\:y=x^4,\:y=x^5, \cdots$\\
$y=\sqrt{x},\:y=\sqrt[3]{x},\:y=\sqrt[4]{x},\:y=\sqrt[5]{x},\cdots$\\
$y=1/x,\:y=1/x^2,\:y=1/x^3,\:y=1/x^4,\cdots$\\
$y=a^x$\\
$y=\log_a x$\\
$y=|x|$

\newpage

\begin{center}Sample Problems\end{center}

Here are the formulas that will be on the front of your test:\\
$n(t)=n_02^{t/a}$\\
$n(t)=n_0e^{rt}$\\
$m(t)=m_02^{-t/h}$\\
$m(t)=m_0e^{rt}$ where $r=(\ln 2)/h,$\\
$\log_b x = (\log_a x )/(\log_a b)$\\
$A(t)=P(1+\frac{r}{n})^{nt}$\\
$A(t)=Pe^{rt}$\\

These are all odd problems from your book. At the end of the list you will find the numbers identified so you can check your answers. I just selected TWO sample problems from each section. You must also look over difficult problems from (1) quizzes, (2) homework, (3) in-class problems from lectures and worksheets.\\

\begin{enumerate}
%2.6.51
\item Sketch $f(x)=(1/2)\sqrt{x+4}-3.$
%4.6.5
\item The fox population in a certain region has a relative growth rate of $8\%$ per year. It is estimated that the population in 2013 was 18000. \\
(a) Find a function $n(t)=n_0e^{rt}$ that models the population $t$ years after 2013.\\
(b) Use the function from part (a) to estimate the fox population in the year 2021.\\
%3.7.35
\item Solve $\frac{x+2}{x+3} < \frac{x-1}{x-2}$

%3.2.29
\item Sketch the graph of the polynomial function $P(x)=x^3(x+2)(x-3)^2 $. Make sure your graph shows all intercepts and exhabits the proper end behavior.

%3.6.61
\item Find intercepts, asymptotes, and domain for $f(x)=\frac{}{}.$

%4.4.61
\item Change $\log_3 15$ to base $e.$

%3.1.23
\item For $f(x)=-4x^2-12x+1,$ (a) express $f$ in standard form, (b) find the vertex and $x$- and $y$-intercepts of $f$, (c) Find the domain and range of $f$, and (d) determine whether $f$ has a maximum or minimum and identify it.

%2.7.57
\item Find $f \circ g$, $f \circ f$, $g\circ f,$ and $g \circ g$ for $f(x)=\frac{x}{x+1}$ and $g(x)=\frac{1}{x}$ and determine their domains.

%4.3.33
\item Evaluate (a) $\log_8 0.25 $ (b) $\ln e^4 $ (c) $\ln (1/e) $

%2.5.49
\item The monthly cost of driving a car depends on the number of miles driven. Lynn found that in May her driving cost was \$380 for 480 miles and in June her cost was \$460 for 800 miles. Assume that there is a linear relationship between the monthly cost $C$ of driving a car and the distance $x$ driven.\\
(a) Find a linear function $C$ that models the cost of driving $x$ miles per month.\\
(b) Draw a graph of $C.$ What is the slope of this line?
(c) At what rate does Lynn's cost increase for eery additional mile she drives?\\

%4.5.43
\item Solve the exponential equation $2^x-10(2^{-x})+3=0.$

%4.1.37
\item Sketch $h(x)=2^{x-4}+1$  including domain, range, asymptotes, and intercepts.

%3.3.23
\item Find the quotient and remainder of $\frac{x^6+x^4+x^2+1}{x^2+1}.$

%2.6.85
\item Determine if the function $f(x)=x^2+x$ is even or odd or neither.

%4.4.47
\item Expand $\log \sqrt{\frac{x^2+4}{(x^2+1)(x^3-7)^2}}.$

%4.2.35
\item If \$600 is invested at an interest rate of $2.5\%$ per year, find the amount of the investment at the end of 10 years for the following compounding methods. (c) quarterly (d) continuously.

%2.8.21
\item Determine if the function $f(x)=x^6-3$ on $0 \leq x \leq 5$ is one-to-one.

%2.7.67
\item Express the function as a composition of functions $H(x)=|1-x^3|.$

%4.5.65
\item Solve the equation $\log_9 (x-5) +\log_9 (x+3)=1.$

%%%%%%%%%%%%%%%
%3.3.55
\item Show that $c=1/2$ is a zero of $P(x)=2x^3+7x^2+6x-5$ and show that $x-c$ is a factor of $P(x).$
%4.6.17
\item The half-life of radium-226 is 1600 years. Suppose we have a 22-mg sample. (a) Find a function $m(t)=m_02^{-t/h}$ that models the mass remaining after $t$ years. (b) Find a function $m(t)=m_0e^{rt}$ where $r=(ln 2)/h,$ that models the mass remaining after $t$ years. (c) How much of the sample will remain after 4000 years?\\

%4.5.37
\item Solve the exponential equation $\frac{50}{1+e^{-x}}=4.$

%3.2.75
\item Graph the family of polynomials on the same set of axes: $P(x)=x^4+c$ for $c=-1,0,1,2.$

%3.6.75
\item Find the slant asymptotes and the vertical asymptotes of $r(x)=\frac{x^3+x^2}{x^2-4}.$

%4.3.69 and 71
\item Sketch $y= \log_3(x-1)-2$ and $y=| \ln x|$.

%2.6.71
\item Look at the graph on page 208 with problem \#71 and sketch all the graphs from parts (a)-(f). (Sorry, it was so much easier than reproducing the problem...)

%3.1.47 
\item Find a function $f$ whose graph is a parabola with vertex $(2,-3)$ and that passes through the point $(3,1).$

%3.7.43
\item Find the domain of $h(x) = \sqrt[4]{x^4-1}.$

%2.5.27
\item Write the equation of the linear function $f$ with rate of change 3 and initial value of $-1.$

%4.2.15
\item Sketch $y=e^{x+1}-3$ include domain, range, intercepts, asymptotes.

%2.8.47
\item Use the Inverse Function Property to show that $f(x)=\frac{x+2}{x-2}$ and $g(x)=\frac{2x+2}{x-1}$ are inverses of each other.

%4.3.41
\item Solve $ \log_2 (1/2)=x$ and $log x = -3 $.

%4.4.57
\item Use the Laws of Logarithms to combine $ (1/3)\log(x+2)^3+(1/2)[\log x^4 - \log (x^2-x-6)^2].$

\end{enumerate}

\newpage

Numbers of Sample problems. \\

\begin{enumerate}
%2.6.51
\item 2.6.51 Sketch $f(x)=(1/2)\sqrt{x+4}-3.$
%4.6.5
\item 4.6.5 The fox population in a certain region has a relative growth rate of $8\%$ per year. It is estimated that the population in 2013 was 18000. \\
(a) Find a function $n(t)=n_0e^{rt}$ that models the population $t$ years after 2013.\\
(b) Use the function from part (a) to estimate the fox population in the year 2021.\\
%3.7.35
\item 3.7.35 Solve $\frac{x+2}{x+3} < \frac{x-1}{x-2}$

%3.2.29
\item 3.2.29 Sketch the graph of the polynomial function $P(x)=x^3(x+2)(x-3)^2 $. Make sure your graph shows all intercepts and exhabits the proper end behavior.

%3.6.61
\item 3.6.61 Find intercepts, asymptotes, and domain for $f(x)=\frac{}{}.$

%4.4.61
\item 4.4.61 Change $\log_3 15$ to base $e.$

%3.1.23
\item 3.1.23 For $f(x)=-4x^2-12x+1,$ (a) express $f$ in standard form, (b) find the vertex and $x$- and $y$-intercepts of $f$, (c) Find the domain and range of $f$, and (d) determine whether $f$ has a maximum or minimum and identify it.

%2.7.57
\item 2.7.57 Find $f \circ g$, $f \circ f$, $g\circ f,$ and $g \circ g$ for $f(x)=\frac{x}{x+1}$ and $g(x)=\frac{1}{x}$ and determine their domains.

%4.3.33
\item 4.3.33 Evaluate (a) $\log_8 0.25 $ (b) $\ln e^4 $ (c) $\ln (1/e) $

%2.5.49
\item 2.5.49 The monthly cost of driving a car depends on the number of miles driven. Lynn found that in May her driving cost was \$380 for 480 miles and in June her cost was \$460 for 800 miles. Assume that there is a linear relationship between the monthly cost $C$ of driving a car and the distance $x$ driven.\\
(a) Find a linear function $C$ that models the cost of driving $x$ miles per month.\\
(b) Draw a graph of $C.$ What is the slope of this line?
(c) At what rate does Lynn's cost increase for eery additional mile she drives?\\

%4.5.43
\item 4.5.43 Solve the exponential equation $2^x-10(2^{-x})+3=0.$

%4.1.37
\item 4.1.37 Sketch $h(x)=2^{x-4}+1$  including domain, range, asymptotes, and intercepts.

%3.3.23
\item 3.3.23 Find the quotient and remainder of $\frac{x^6+x^4+x^2+1}{x^2+1}.$

%2.6.85
\item 2.6.85 Determine if the function $f(x)=x^2+x$ is even or odd or neither.

%4.4.47
\item 4.4.47 Expand $\log \sqrt{\frac{x^2+4}{(x^2+1)(x^3-7)^2}}.$

%4.2.35
\item 4.2.35 If \$600 is invested at an interest rate of $2.5\%$ per year, find the amount of the investment at the end of 10 years for the following compounding methods. (c) quarterly (d) continuously.

%2.8.21
\item 2.8.21 Determine if the function $f(x)=x^6-3$ on $0 \leq x \leq 5$ is one-to-one.

%2.7.67
\item 2.7.67 Express the function as a composition of functions $H(x)=|1-x^3|.$

%4.5.65
\item 4.5.65 Solve the equation $\log_9 (x-5) +\log_9 (x+3)=1.$

%%%%%%%%%%%%%%%
%3.3.55
\item 3.3.55 Show that $c=1/2$ is a zero of $P(x)=2x^3+7x^2+6x-5$ and show that $x-c$ is a factor of $P(x).$
%4.6.17
\item 4.6.17 The half-life of radium-226 is 1600 years. Suppose we have a 22-mg sample. (a) Find a function $m(t)=m_02^{-t/h}$ that models the mass remaining after $t$ years. (b) Find a function $m(t)=m_0e^{rt}$ where $r=(ln 2)/h,$ that models the mass remaining after $t$ years. (c) How much of the sample will remain after 4000 years?\\

%4.5.37
\item 4.5.37 Solve the exponential equation $\frac{50}{1+e^{-x}}=4.$

%3.2.75
\item 3.2.75 Graph the family of polynomials on the same set of axes: $P(x)=x^4+c$ for $c=-1,0,1,2.$

%3.6.75
\item 3.6.75 Find the slant asymptotes and the vertical asymptotes of $r(x)=\frac{x^3+x^2}{x^2-4}.$

%4.3.69 and 71
\item 4.3.69 and 71 Sketch $y= \log_3(x-1)-2$ and $y=| \ln x|$.

%2.6.71
\item 2.6.71 Look at the graph on page 208 with problem \#71 and sketch all the graphs from parts (a)-(f). (Sorry, it was so much easier than reproducing the problem...)

%3.1.47 
\item  3.1.47 Find a function $f$ whose graph is a parabola with vertex $(2,-3)$ and that passes through the point $(3,1).$

%3.7.43
\item 3.7.43 Find the domain of $h(x) = \sqrt[4]{x^4-1}.$

%2.5.27
\item 2.5.27 Write the equation of the linear function $f$ with rate of change 3 and initial value of $-1.$

%4.2.15
\item 4.2.15 Sketch $y=e^{x+1}-3$ include domain, range, intercepts, asymptotes.

%2.8.47
\item 2.8.47 Use the Inverse Function Property to show that $f(x)=\frac{x+2}{x-2}$ and $g(x)=\frac{2x+2}{x-1}$ are inverses of each other.

%4.3.41
\item 4.3.41 Solve $ \log_2 (1/2)=x$ and $log x = -3 $.

%4.4.57
\item 4.4.57
\end{enumerate}


\end{document}

