\documentclass[11pt]{article}
\usepackage[margin=1in, head=1in]{geometry}
\usepackage{amsmath, amssymb, amsthm}
\usepackage{fancyhdr}
\usepackage{graphicx}
\usepackage{pgfplots}
\usepackage{verbatim}

%\usepackage{pdfsync}
\addtolength{\textwidth}{.5in}
\addtolength{\leftmargin}{-1in}
\addtolength{\textheight}{.5in}
\addtolength{\topmargin}{-0.5in}

%\pagestyle{fancy}
%\lhead{MATH 200X }
%\chead{Fall 2007}
%\rhead{FINAL EXAM}
%\lfoot{}
%\cfoot{\thepage}
%\rfoot{}

\setcounter{secnumdepth}{0}
\newcommand{\R}{\mathbb{R}}
\newcommand{\N}{\mathbb{N}}
\newcommand{\Z}{\mathbb{Z}}
\newcommand{\clm}{\par\textit{Claim:}\par}
\newcommand{\diam}{\mathrm{diam}}
\newcommand{\sect}{\textsection}

\parindent=0in
\parskip=0.5\baselineskip

\begin{document}
\newgeometry{top=3in}
\begin{center}
\vspace{2in}

\huge{Math 156 PRECALCULUS \\
Fall 2015}

\vfill

\huge{\bf{Quiz 5 -- Make up}}\\

\vspace{0.5in}

\large{Wednesday, 28 October 2015}\\

\vfill


{\huge{Name:{\underline{\hspace{2in}}}}}
\vfill
This quiz has 6 problems worth a total of 30 points. It is TWO SIDED. 
\vfill
\end{center}
\newpage
\restoregeometry
\begin{enumerate}
%Homework Problem 2.5.42
%Range of Problems 2.5.39-50
\item The amount of copper ore produced from a copper mine in Arizona is modeled by the function $f(x)=200+32x$ where $x$ is the number of years since 2005 and $f(x)$ is measured in thousands of tons.
\begin{enumerate}
\item (3 points) Find $f(1)$ and explain what this means in terms of the problem (That is, your explaination should have the words  {\it{years}} and {\it{tons of copper}}). \vspace{1in}
\item (3 points) Explain what the 32 represents in terms of the problem (That is, your explaination should have the words  {\it{years}} and {\it{tons of copper}})
\vspace{1in}
\end{enumerate}
%Homework Problems 2.6 \#40,43,47
%Range of Problems 2.6 \#29-52
\item (3 points each) For each function below, sketch the graph. (For people with limited art skills, like me, you are welcome to augment your picture with an explanation of the standard function you are transforming and how you are transforming it. For example, one might write, ``This is the parabola $y=x^2$ translated 2 units to the left and stretched horizontally by a factor of 3.")
\begin{enumerate}
\item $y=\sqrt[3]{x}-2$
\vfill
\item $y=-|x-2|$
\vfill
\end{enumerate}
%%Homework Problems 2.7 \#63,66
%%Range of Problems 2.7 \#63-68
\item (4 points) Express the function $H(x)=(2x-4)^3$ in the form $f \circ g$ in a nontrivial way. (That is, you are not allowed to choose $f(x)=x$ or $g(x)=x.$
\vspace{.75in}
\newpage
%Homework Problems 2.7 \#47,51,56,
%Range of Problems 2.7 \#47-58
\item (4 points) For $f(x)=\frac{1}{x}$ and $g(x)=\frac{x}{x-5},$ find $(g \circ f)(x)$ and its domain.
\vfill
%%Homework Problems 2.8 \#38,40,48
%%Range of Problems 2.8 \#37-48
\item (4 points) Use the Inverse Function Property to show that $f(x)=x^3+1$ and $g(x)=(x-1)^{1/3}$ are inverses of each other
\vfill
%%Homework Problems 2.8 \#50,55,61,66,68,70
%%Range of Problems 2.8 \#49-70
\item (3 points each) Find the inverse functions below: 
\begin{enumerate}
\item $f(x)=\frac{3x}{x-2}$
\vfill
\item $g(x)=2+\sqrt{x+3}.$
\vfill
\end{enumerate}
\end{enumerate}
\end{document}
