\documentclass[11pt]{article}
\usepackage[margin=1in, head=1in]{geometry}
\usepackage{amsmath, amssymb, amsthm}
\usepackage{fancyhdr}
\usepackage{graphicx}
\usepackage{pgfplots}
\usepackage{verbatim}

%\usepackage{pdfsync}
\addtolength{\textwidth}{.5in}
\addtolength{\leftmargin}{-1in}
\addtolength{\textheight}{.5in}
\addtolength{\topmargin}{-0.5in}

%\pagestyle{fancy}
%\lhead{MATH 200X }
%\chead{Fall 2007}
%\rhead{FINAL EXAM}
%\lfoot{}
%\cfoot{\thepage}
%\rfoot{}

\setcounter{secnumdepth}{0}
\newcommand{\R}{\mathbb{R}}
\newcommand{\N}{\mathbb{N}}
\newcommand{\Z}{\mathbb{Z}}
\newcommand{\clm}{\par\textit{Claim:}\par}
\newcommand{\diam}{\mathrm{diam}}
\newcommand{\sect}{\textsection}

\parindent=0in
\parskip=0.5\baselineskip

\begin{document}
\newgeometry{top=3in}
\begin{center}
\vspace{2in}

\huge{Math 156 PRECALCULUS \\
Fall 2015}

\vfill

\huge{\bf{Quiz 5 -- Version Retake}}\\

\vspace{0.5in}

\large{Tuesday, October 20, 2015}\\

\vfill


{\huge{Name:{\underline{\hspace{2in}}}}}
\vfill
This quiz has 6 problems worth a total of 30 points. It is TWO SIDED. 
\vfill
\end{center}
\newpage
\restoregeometry
\begin{enumerate}
%new
%Homework Problem 2.5.42
%Range of Problems 2.5.39-50
\item A pump empties a 2000 gallon pool at a rate of 5 cubic feet per second. Initially pool is full. \begin{enumerate}
\item (3 points) Find a linear function $V$ that models the volume of water in the pool at any time $t.$
\vspace{1in}
\item (3 points) How long does it take to completely empty?
\vspace{1in}
\end{enumerate}
%new
%Homework Problems 2.6 \#40,43,47
%Range of Problems 2.6 \#29-52
\item (3 points each) For each function below, sketch the graph. (For people with limited art skills, like me, you are welcome to augment your picture with an explanation of the standard function you are transforming and how you are transforming it. For example, one might write, ``This is the parabola $y=x^2$ translated 2 units to the left and stretched horizontally by a factor of 3.")
\begin{enumerate}
\item $y=4-(x+1)^2$
\vfill
\item $y=\sqrt[3]{x-2}$
\vfill
\end{enumerate}
%new
%%Homework Problems 2.7 \#63,66
%%Range of Problems 2.7 \#63-68
\item (4 points) Express the function $H(x)=\frac{4}{1+x^3}$ in the form $f \circ g$ in a nontrivial way. (That is, you are not allowed to choose $f(x)=x$ or $g(x)=x.$
\vspace{.75in}
\newpage
%new
%Homework Problems 2.7 \#47,51,56,
%Range of Problems 2.7 \#47-58
\item (4 points) For $f(x)=\sqrt{x-4}$ and $g(x)=x^2+x+1,$ find $(g \circ f)(x)$ and its domain.
\vfill
%new
%%Homework Problems 2.8 \#38,40,48
%%Range of Problems 2.8 \#37-48
\item (4 points) (i) Explain, in your own words, what it means for a function to be one-to-one. (ii) Determine if the function $f(x)=\frac{1}{x^2}$ is one-to-one and show your answer is correct. 
\vfill
%new
%%Homework Problems 2.8 \#50,55,61,66,68,70
%%Range of Problems 2.8 \#49-70
\item (3 points each) Find the inverse of the functions below: 
\begin{enumerate}
\item $f(x)=\sqrt{6+7x}$
\vfill
\item $g(x)=x^2+x$ for $x \geq -1/2$
\vfill
\end{enumerate}
\end{enumerate}
\end{document}
