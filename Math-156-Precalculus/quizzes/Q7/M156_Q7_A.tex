\documentclass[11pt]{article}
\usepackage[margin=1in, head=1in]{geometry}
\usepackage{amsmath, amssymb, amsthm}
\usepackage{fancyhdr}
\usepackage{graphicx}
\usepackage{pgfplots}
\usepackage{verbatim}

%\usepackage{pdfsync}
\addtolength{\textwidth}{.5in}
\addtolength{\leftmargin}{-1in}
\addtolength{\textheight}{.5in}
\addtolength{\topmargin}{-0.5in}

%\pagestyle{fancy}
%\lhead{MATH 200X }
%\chead{Fall 2007}
%\rhead{FINAL EXAM}
%\lfoot{}
%\cfoot{\thepage}
%\rfoot{}

\setcounter{secnumdepth}{0}
\newcommand{\R}{\mathbb{R}}
\newcommand{\N}{\mathbb{N}}
\newcommand{\Z}{\mathbb{Z}}
\newcommand{\clm}{\par\textit{Claim:}\par}
\newcommand{\diam}{\mathrm{diam}}
\newcommand{\sect}{\textsection}

\parindent=0in
\parskip=0.5\baselineskip

\begin{document}


\newgeometry{top=1in}

{\huge{Name:{\underline{\hspace{4in}}}}}
\vfill

\begin{center}
\vspace{1in}

\huge{Math 156 PRECALCULUS \\
Fall 2015}

\vfill

\huge{\bf{Quiz 7 -- Version A}}\\

\vspace{0.5in}

\large{Thursday, October 29, 2015}\\

\vfill

This quiz has 8 problems worth a total of 30 points. It is TWO SIDED. 
\vfill
\end{center}
%Useful Formulas:\\

%\scalebox{1.2}{$A(t)=P\left(1+\frac{r}{n}\right)^{nt}$} \hfill \scalebox{1.2}{$A(t)=Pe^{rt}$}

\vfill


\newpage
\restoregeometry
\begin{enumerate}
%4.3 #9-16
\item (2 points) Express the equation \fbox{\scalebox{1.2}{$\log 3=2t $}} in exponential form. (You don't need to solve it.)\\
\begin{flushright}{ Answer:\underline{\hspace{2in}}}\end{flushright}
\vfill
%4.3 \#17-24
\item (2 points) Express the equation \fbox{\scalebox{1.2}{$e^{0.7t}=r$}} in logarithmic form.\\
\begin{flushright}{ Answer:\underline{\hspace{2in}}}\end{flushright}
\vfill
%4.3 \#25-34
\item (2 points each) Evaluate the expressions below.
\begin{enumerate}
\item \scalebox{1.2}{$\log_9 \sqrt{3}$}\\
\begin{flushright}{ Answer:\underline{\hspace{2in}}}\end{flushright}
\vfill
\item \scalebox{1.2}{$e^{\ln 10}$}\\
\begin{flushright}{ Answer:\underline{\hspace{2in}}}\end{flushright}
\vfill
\item \scalebox{1.2}{$\log_4 8 $}\\
\begin{flushright}{ Answer:\underline{\hspace{2in}}}\end{flushright}
\vfill
\end{enumerate}
%4.3 \#73-78
\item (2 points) find the domain of the function \scalebox{1.2}{$h(x)=\ln x + \ln (2-x)$}. Give your answer in interval notation.\\
\begin{flushright}{ Answer:\underline{\hspace{2in}}}\end{flushright}
\vfill
\newpage
\item (4 points each) Sketch the graphs of the functions below and {\bf{LABEL}} (a) any asymptotes and (b) any $x$- or $y$-intercepts. State the domain and range. 
\begin{enumerate}
%4.3 \#61-72
\item $f(x)=\log_3 (x+2)$\begin{flushright}{domain: \underline{\hspace{2in}}}\end{flushright}
\begin{flushright}{range: \underline{\hspace{2in}}}\end{flushright}
\vspace{-.6in}
\begin{tikzpicture}[scale=0.9]
\draw[->, thick] (-3,0) -- (4,0)node [pos=1, below] {$x$};
\draw[->, thick] (0,-3) -- (0,4) node [pos=1, left] {$y$};
\end{tikzpicture}

%4.1 \#27-40
\item $f(x)=2^{x-4}+1$ \begin{flushright}{domain: \underline{\hspace{2in}}}\end{flushright}
\begin{flushright}{range: \underline{\hspace{2in}}}\end{flushright}
\vspace{-.6in}

\begin{tikzpicture}[scale=0.9]
\draw[->, thick] (-3,0) -- (4,0)node [pos=1, below] {$x$};
\draw[->, thick] (0,-3) -- (0,4) node [pos=1, left] {$y$};
\end{tikzpicture}

%4.2 \#7-16
\item $f(x)=-2e^{-x}$\begin{flushright}{domain: \underline{\hspace{2in}}}\end{flushright}
\begin{flushright}{range: \underline{\hspace{2in}}}\end{flushright}
\vspace{-.6in}
\begin{tikzpicture}[scale=0.9]
\draw[->, thick] (-3,0) -- (4,0)node [pos=1, below] {$x$};
\draw[->, thick] (0,-3) -- (0,4) node [pos=1, left] {$y$};
\end{tikzpicture}
\end{enumerate}
\newpage

%4.4 \# 9-22
\item (2  points) Use the Laws of Logarithms to evaluate the expression\\ 

\fbox{\scalebox{1.2}{$\frac{-1}{3}\log_5 125 $}} \begin{flushright}{ Answer:\underline{\hspace{2in}}}\end{flushright}
\vfill
%4.4 \#23-48
\item (2 points) Use the Laws of Logarithms to expand the expression\\ 

\fbox{\scalebox{1.2}{$\ln\left(\frac{\sqrt{3x^5}}{zy^2}\right)$}} \begin{flushright}{ Answer:\underline{\hspace{2in}}}\end{flushright}
\vfill
%4.4 \#49-58
\item (2 points) Use the Laws of Logarithms to combine the expression:\\

 \fbox{\scalebox{1.2}{$ \log_a(a+b)+\log_a(a-b)-2\log_ac$}} \begin{flushright}{ Answer:\underline{\hspace{2in}}}\end{flushright}
\vfill



\end{enumerate}
\end{document}
