%%% Preamble starts here.
\documentclass{amsart}
%for the heading
\usepackage{fancyhdr, enumerate}
%for the picture. 
\usepackage{tikz, calc}
%adjust the page width
\usepackage[margin=1in]{geometry}

%% The next line says how the "vertex" style of nodes should look: drawn as small circles.
\tikzstyle{vertex}=[circle, draw, inner sep=0pt, minimum size=6pt,fill=white]
%%
%% Next, we make a \vertex command as a shorthand in place of \node[vertex} to get that style.
\newcommand{\vertex}{\node[vertex]}

\linespread{1.1}


%special commands for number sets
\def\RR{{\mathbb R}}
\def\NN{{\mathbb N}}
\def\ZZ{{\mathbb Z}}
\def\QQ{{\mathbb Q}}
\def\CC{{\mathbb C}}

% header
\lhead{MATH 490/497}
\chead{\sc Midterm Review} 
\rhead{Spring 2020}
\cfoot{}
\pagestyle{fancy}

%%%% Main document starts here.

\begin{document}
\thispagestyle{fancy}
The midterm will be Monday 2 March and will cover all material discussed in class thus far. A comprehensive list is below. The midterm will be one hour. Books, notes or other aids are not allowed.\\

The midterm will consist largely of proofs (or sketches of proofs). You should be able to prove any result presented in class and any result from the homework. There will also be problems not exactly like something we have seen before. \\

\noindent{\sc{Terminology by Section}}\\


\noindent\textbf{Section 1.1:} (I am omitting things like \emph{graph} and \emph{adjacent}) complement, clique, independent set, chromatic number, isomorphism, girth

\noindent\textbf{Section 1.2:} walk, trail, length, components, cut-edge, cut-vertex, Eulerian graph, circuit, Eulerian circuit/trail, even/odd graph, maximal versus maximum

\noindent\textbf{Section 1.3:} degree sequence, graphic sequence

Technically not terminology, here is where we learned the algorithm for determining if a sequence is graphic.

\noindent\textbf{Section 3.1:} matching, saturated, unsaturated, perfect matching, maximal matching, $M$-alternating path, $M$-augmenting path, symmetric difference, vertex cover


\quad \\
{\sc{Presentation Topics:}}\\

\noindent\textbf{Formal Proof:} Lemma 1.2.5, Proposition 1.2.11, Theorem 1.2.14, Lemma 1.2.25, Proposition 1.2.28, Proposition 1.2.29, Lemma 1.2.31, Proposition 1.3.3, Corollaries 1.3.4, 1.3.5, 1.3.6, Proposition 1.3.13, Proposition 1.3.15, Theorem 1.3.19, Lemma 3.1.9, Theorem 3.1.10, Corollary 3.1.13

\quad \\
\noindent\textbf{Sketch of Proof:} Theorem 1.2.26 (recall we discussed two different proofs of this), Theorem 1.3.23, Theorem 3.1.11, Theorem 3.1.16
\end{document}



