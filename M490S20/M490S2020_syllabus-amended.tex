\documentclass[11pt]{article}
\usepackage[margin=1in, head=1in]{geometry}
\usepackage{amsmath, amssymb, amsthm}
\usepackage{fancyhdr}
\usepackage{graphicx,xcolor}

\addtolength{\textwidth}{.5in}
\addtolength{\leftmargin}{-1in}
\addtolength{\textheight}{.5in}
\addtolength{\topmargin}{-0.5in}

\setcounter{secnumdepth}{0}
\newcommand{\R}{\mathbb{R}}
\newcommand{\N}{\mathbb{N}}
\newcommand{\Z}{\mathbb{Z}}
\newcommand{\clm}{\par\textit{Claim:}\par}
\newcommand{\diam}{\mathrm{diam}}
\newcommand{\sect}{\textsection}

\parindent=0in
\parskip=0.5\baselineskip

\begin{document}
\begin{center}\textcolor{red}{AMENDED} Syllabus for \\ MATH 490: Senior Seminar -- Graph Theory  \\ Spring 2020 \\ MW 9:15am-10:15am \\ Chapman 106
\end{center}

\textcolor{red}{This syllabus is in effect starting 23 March 2020. All changes begin and end with red text.}

\hrulefill

\textbf{Instructor:} Jill Faudree\\
\textbf{Contact Details:} Chapman 306B, jrfaudree@alaska.edu, 474-7385\\
\textbf{Office Hours:} (\textbf{\emph{tentative}})  MWF 10:30-11:30, R 9:15-10:15 and by appointment. Also, you are welcome to drop by. Note that these hours may change depending on student demands and scheduling concerns.\\
\textbf{Textbook:} \emph{Introduction to Graph Theory}, 2nd edition, by Doug West (ISBN-13: 978-0130144003)\\
\textbf{Online Access to Course Materials:} 
\begin{itemize}
	\item \textbf{Blackboard} for written homework, class announcements, grades and student assignments.
	\item Dr. Faudree's webpage (\textbf{http://jrfaudree.github.io/M490S20home.html}) for the day-to-day schedule, review sheets and midterm/final exam solutions. 
	\end{itemize}
\textbf{Prerequisites:} A grade of C or better in (COJO 131 or COJO 141) and (MATH 401 or MATH 405) and (senior standing).\\

\hrulefill

\textsc{Course Overview and Goals:}\\
 
�Senior Seminar is the capstone experience for UAF's undergraduate mathematics program, and it will be conducted somewhat differently from your previous courses. Unlike most other courses, the responsibility of learning the material of the course falls largely on the students. That is, it is your job to learn the material and teach it to
each other. One of the primary goals of this course is that you learn how to effectively communicate mathematics
to your peers.
The teacher is here to help you with this process -- to provide a framework, guidance, and, alas,
grades. But collectively you will be leading the class. You will get to practice figuring out
unfamiliar mathematics on your own, as well as explaining it to others. You will be asked to
give formal oral presentations, field questions from the audience, contribute to homework problem
sessions, and actively participate in class discussions. �This is going to be like no other math
class you have taken, and I hope it will be rewarding and fun.
As for the material, we will be studying Graph Theory in a more formal, in-depth and rigorous way than seen in Discrete Math or Combinatorics.\\

\textsc{Course Mechanics}:\\

\textcolor{red}{Hour-long sessions will be moved online, for now.} We will try to proceed with our usual classroom structure except that the process will occur online. This means student presenters will have to prepare a beamer presentation instead of board presentation. Non-presenting students are still expected to show up and participate. \textcolor{red}{If this doesn't work, we will move asynchronously.}

We have two hour-long meetings for discussing material. Mondays will be devoted to student-led lectures followed by a question and answer period. Wednesdays will be split between student-led lectures and questions/discussions of homework. Written homework will be due on Friday. 

Student-led lectures will begin small, with a rigidly-defined scope, and gradually build into larger chunks of material that will include more material than can be discussed in the allotted time. This will require you to exercise judgement about what to present in class. You will need to think about what the important
points are, and how best to explain them. You are welcome (i.e., highly encouraged) to talk
to me while you are making these preparations. I will ensure that one of my office hours
each week is devoted to this course alone.
\textbf{Students who are not presenting will be expected to have read the material in advance and
come prepared to ask questions and otherwise discuss the material.} Class participation is
part of your grade.\\

I reserve the right to adjust the mechanics described here depending on the needs of the
class.\\

{\sc{Homework}}\\

Homework will be due on Friday. Late homework is not accepted except in rare catastrophic instances. All homework will be turned in online via Blackboard using \LaTeX. Students are expected to have attempted all homework problems in order to be able to participate in Wednesday class discussions.\\

{\sc{Tests}}\\

\textcolor{red}{I am now operating under the assumption that proctored assessments will not occur.} Unless something changes, you will not take the ETS Major Field Test. Moreover, we will not have a written in-class final exam. For now, I am replacing the in-class final exam with a project focused on some open problem in Graph Theory and a take-home exam. \textcolor{red}{If proctoring later becomes an option, I reserve the right to require  the ETS Major Field Test and give a proctored in-class Final.}

�There will be a midterm, tentatively scheduled for Monday, March 2. I reserve the right
to make it a take-home midterm, but I expect it will be an in-class exam.
In addition, you will take the ETS Major Fields Test in Mathematics. �This is a standardized
test used by our department to monitor student outcomes. Your participation in taking the
test will contribute to your class participation grade, but the test score you receive on it will
not impact your course grade. We will hold it at a time and day to be announced later in the
semester.\\

The final exam is scheduled for Wednesday April 29 from 8:00am-10:00am. \\


\textbf{Grades} will be calculated according to the following rubric:\\

\textcolor{red}{If proctoring is not an option, the final exam will be replaced by a final project and a take-home exam.}

\begin{tabular}{|l|c|}
  \hline
  % after \\: \hline or \cline{col1-col2} \cline{col3-col4} ...
  homework & 20\%\\
  class presentations & 20\% \\
  class participation & 15\% \\
  midterm & 20\%\\
  \textcolor{red}{project} & 10\%\\
  \textcolor{red}{take-home final}& 15\%\\
  %final exam & 25\% \\
  \hline
\end{tabular}

Grade Bands: A, A- (90 - 100\%), B+,B, B- (80 - 89\%), C+, C, C- (70 - 79\%), D+, D, D-
(60 - 69\%), F (0 - 59\%).  I reserve the right to lower the thresholds. The grade of $A+$ is reserved for outstanding performance in the course overall.\\

\textsc{(tentative) Schedule of Topics:}

Given the student-led nature of the class, a formal schedule of topics is difficult to predict but I expect we will cover one section every three class periods.

\textsc{Miscellaneous Other Issues:}

\textbf{Communication:} I will communicate with you using three different channels: (1) class, (2) Blackboard (for general announcements) and (3) email (for private correspondence). I will not email you casually. If you receive an email from me, you need to read it and respond, if necessary.  Class time and email is also the best way for you to communicate with me. (See homework at the end of this syllabus.)

\textbf{Disability Services}
The Office of Disability Services implements the
Americans with Disabilities Act (ADA), and ensures that UAF students
have equal access to the campus and course materials. The instructors will work with
the Office of Disability Services (208 Whitaker, 474-5655) to provide
reasonable accommodations to students with disabilities.

\textbf{Student Protections and Services}
Every qualified student is welcome in our classes.  As needed, we are happy to work with you, Disability Services, Military and Veteran Services, Rural Student Services, etc. to find reasonable accommodations. Students at this university are protected against sexual harassment and discrimination (Title IX), and minors have additional protections. \textit{As required,} if we notice or are informed of \textit{certain types} of misconduct, then we are required to report it to the appropriate authorities.  For more information on your rights as a student and the resources available to you, please go to the following site: {https://www.uaf.edu/handbook/}{{www.uaf.edu/handbook}}.

\textbf{Incomplete Grade} 
Incomplete (I) will only be given in DMS courses in cases where the student has completed the majority (normally all but the last three weeks) of a course with a grade of C or better, but for personal reasons beyond his/her control has been unable to complete the course during the regular term. Negligence or indifference are not acceptable reasons for the granting of an incomplete grade. 

\textbf{Late Withdrawals} 
A withdrawal after the deadline (currently 9 weeks into the semester) from a DMS course will normally be granted only in cases where the student is performing satisfactorily (i.e., C or better) in a course, but has exceptional reasons, beyond his/her control, for being unable to complete the course. These exceptional reasons should be detailed in writing to the instructor, department head and dean.

\textbf{No Early Final Examinations}
Final examinations for DMS courses shall not be held earlier than the date and time published in the official term schedule. Normally, a student will not be allowed to take a final exam early. Exceptions can be made by individual instructors, but should only be allowed in exceptional circumstances and in a manner which doesn't endanger the security of the exam.

\textbf{Academic Dishonesty}
Academic dishonesty, including cheating and plagiarism, will not be tolerated.  It is a violation of the Student Code of Conduct and will be punished according to UAF procedures.


\end{document}