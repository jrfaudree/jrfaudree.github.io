
\documentclass[11pt,fleqn]{article} 
\usepackage[margin=0.8in, head=0.8in]{geometry} 
\usepackage{amsmath, amssymb, amsthm}
\usepackage{fancyhdr} 
\usepackage{palatino, url, multicol}
\usepackage{graphicx, pgfplots} 
\usepackage[all]{xy}
\usepackage{polynom} 
%\usepackage{pdfsync} %% I don't know why this messes up tabular column widths
\usepackage{enumerate}
\usepackage{framed}
\usepackage{setspace}
\usepackage{array,tikz}

\pgfplotsset{compat=1.6}

\pgfplotsset{soldot/.style={color=black,only marks,mark=*}} \pgfplotsset{holdot/.style={color=black,fill=white,only marks,mark=*}}


\pagestyle{fancy} 
\lfoot{}
\rfoot{2-4 Continuity}

\begin{document}
\renewcommand{\headrulewidth}{0pt}
\newcommand{\blank}[1]{\rule{#1}{0.75pt}}
\newcommand{\bc}{\begin{center}}
\newcommand{\ec}{\end{center}}
\renewcommand{\d}{\displaystyle}

\vspace*{-0.7in}

%%%%%%%%%intro page
\begin{center}
  \large
  \sc{Section 2-4: Continuity}\\
\end{center}
Read Section 2.4. Work the embedded problems. \\
\hrulefill

\begin{enumerate}
\item Pictures of graph discontinuities
\vfill
\item Definition of continuity at a point
\vfill
\item Sketch the graph of a function $f(x)$ with the following properties:\\
\begin{enumerate}
\item the domain of $f(x)$ is the interval $[0,10].$
\item $f(x)$ is continuous except at $x=0$ where it has in infinite discontinuity and $x=5$ where it has a jump discontinuity.
\end{enumerate}
\vfill
\item Give an example of a function that is continuous everywhere on its domain.
\vspace{.5in}

\newpage
\item Determine the point(s), if any, at which each function is discontinuous. Justify your answer. Classify any discontinuity as jump, removable, infinite, or other.
\begin{enumerate}
	\item $g(x) = x^{-1}+1$
	\vfill
	\item $h(x)=\frac{x+2}{x^2-4}$
	\vfill
		\end{enumerate}
\item Find the value(s) of k that makes the function continuous over the given interval.\\
$f(x)=\begin{cases} e^{kx} & \text{if } 0 \leq x < 4 \\
				2x+1 & \text{if } 4 \leq x \leq 10 \\
				\end{cases}$
\vspace{2in}
\newpage
\item The Intermediate Value Theorem
\vfill
BONUS:
\item Use the Intermediate Value Theorem to show that the equation $x^4+x-3=0$ must have a solution in
the interval from $x=1$ to $x=2.$
\vfill

\end{enumerate}
\end{document}

