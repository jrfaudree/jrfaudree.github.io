
\documentclass[11pt,fleqn]{article} 
\usepackage[margin=0.8in, head=0.8in]{geometry} 
\usepackage{amsmath, amssymb, amsthm}
\usepackage{fancyhdr} 
\usepackage{palatino, url, multicol}
\usepackage{graphicx, pgfplots} 
\usepackage[all]{xy}
\usepackage{polynom} 
%\usepackage{pdfsync} %% I don't know why this messes up tabular column widths
\usepackage{enumerate}
\usepackage{framed}
\usepackage{setspace}
\usepackage{array,tikz}

\pgfplotsset{compat=1.6}

\pgfplotsset{soldot/.style={color=black,only marks,mark=*}} \pgfplotsset{holdot/.style={color=black,fill=white,only marks,mark=*}}
\pgfplotsset{my style/.append style={axis x line=middle, axis y line=
middle, xlabel={$x$}, ylabel={$y$}, axis equal }}


\pagestyle{fancy} 
\lfoot{}
\rfoot{3-3 Derivative Rules}

\begin{document}
\renewcommand{\headrulewidth}{0pt}
\newcommand{\blank}[1]{\rule{#1}{0.75pt}}
\newcommand{\bc}{\begin{center}}
\newcommand{\ec}{\end{center}}
\renewcommand{\d}{\displaystyle}

\vspace*{-0.7in}

%%%%%%%%%intro page
\begin{center}
  \large
  \sc{Section 3-3: Derivative Rules}\\
\end{center}
Goals: To establish and justify several derivative rules and use them and to learn some new notation. \\ Just FYI but on Wednesday we will begin with a complete and comprehensive summary of \emph{all} the rules from this section.\\

\begin{enumerate}
\item Use the definition to find the derivative of $f(x)=x^2.$\\
\vspace{2in}
\item Recall that at the end of class on Friday we established: \\
\vspace{1in}

\item Graph $f(x)=\cos (x)$ and use the same strategy to guess its derivative.\\
\vspace{1.5in}


\item If $f(x)=10,$ what should $f'(x)$ be and why? 
\vfill

\item Make a conjecture about the derivative of constant functions and write it down. \vfill
\newpage
\item If $f(x)=x,$ what should $f'(x)$ be and why? 

\vfill
\item What about $f(x)=5x$? Explain.
\vfill
\item What about $f(x)=5x+10$? Explain.
\vfill
\item In the 3.2 notes on the definition of the derivative, we found that if $f(x) =\sqrt{x+5}$, then its derivative was: \\

Use this to determine the derivative of $g(x)=\sqrt{x}.$
\vfill
\item The Power Rule\\
\vspace{1.5in}
\item The Sum (and Difference) Rule\\
\vspace{1.5in}

\newpage
\item The Constant Multiple Rule\\
\vspace{1.5in}

\item Apply the rules to find the derivatives of the functions below. Simplify your answers and write with positive exponents.
	\begin{enumerate}
	\item $\displaystyle{f(x)=e^3}$\\
	\vfill

	\item $\displaystyle{f(x)=x^{-4}}$\\
		\vfill

	\item $\displaystyle{H(x)=4x^{3/2}+ 15}$\\
	\vfill

	\item $\displaystyle{j(x)=\frac{\sqrt{2}}{2}+x-8x^{2.3}}$\\
	\vfill

	\end{enumerate}
\item Notation\\
\vfill
\item Higher Order Derivatives
\vfill
\newpage
\item Find examples of $f(x)$ and $g(x)$ that demonstrate that the rules below are WRONG.\\
\begin{quote} INCORRECT: If $H(x)=f(x)g(x)$, then $H'(x)=f'(x)g'(x).$ \end{quote}
\vfill

\begin{quote} INCORRECT: If $H(x)=\frac{f(x)}{g(x)}$, then $H'(x)=\frac{f'(x)}{g'(x)}.$ \end{quote}
\vfill

\item Product and Quotient Rules\\
	
	\begin{enumerate}
	\item $\displaystyle{\frac{d}{dx}\left[f(x)\: g(x)\right]=}$\\
	\vfill

	\item $\displaystyle{\frac{d}{dx}\left[\frac{f(x)}{g(x)}\right]=}$\\
	\vfill
	\end{enumerate}

\end{enumerate}
\end{document}	
\item Find the derivative of each of the following:
	\begin{enumerate}
	\item $H(x)=(3x^2+1)(\frac{1}{x}+x)$
	\vfill
	\item $G(x)=\frac{x^2}{x^2+1}$
	\vfill
	\end{enumerate}
\newpage
\item Notation:
\vspace{1in}
\item Higher order derivatives\\

Example: $y=x^3-2\sqrt{x}+ \pi$
\vspace{2in}
\item The vertical height of an object is given by $s(t)=-16t^2+20t+100.$ Find $s'(t)$ and $s''(t)$. Include units.
\vspace{2in}
\end{enumerate}
\end{document}

