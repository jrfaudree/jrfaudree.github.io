\documentclass[11pt,fleqn]{article} 
\usepackage[margin=0.8in, head=0.8in]{geometry} 
\usepackage{amsmath, amssymb, amsthm}
\usepackage{fancyhdr} 
\usepackage{palatino, url, multicol}
\usepackage{graphicx, pgfplots} 
\usepackage[all]{xy}
\usepackage{polynom,tabularx} 
%\usepackage{pdfsync} %% I don't know why this messes up tabular column widths
\usepackage{enumerate}
\usepackage{framed}
\usepackage{setspace}
\usepackage{array}
\usepackage{pgf,tikz}
\usepackage{mathrsfs}

\usepackage[parfill]{parskip}
\usetikzlibrary{arrows}

\usetikzlibrary{calc}

\pgfplotsset{compat=1.6}

\pgfplotsset{soldot/.style={color=blue,only marks,mark=*}} \pgfplotsset{holdot/.style={color=blue,fill=white,only marks,mark=*}}

\renewcommand{\headrulewidth}{0pt}
\newcommand{\blank}[1]{\rule{#1}{0.75pt}}
\newcommand{\bc}{\begin{center}}
\newcommand{\ec}{\end{center}}
\newcommand{\be}{\begin{enumerate}}
\newcommand{\ee}{\end{enumerate}}

\def\ds{\displaystyle}

\renewcommand{\d}{\displaystyle}

\newcommand{\ans}[1][2]{ \ \rule{#1 in}{.5 pt} \ }


\pagestyle{fancy} 
%\lfoot{Uses a calculator}
\rfoot{5-1}

\begin{document}

\vspace*{-0.7in}

\begin{center}
  \Large\sc{Section 5.1: Approximating Areas}
  \end{center}
Using rectangles to estimate areas of curvy curves.\\

\begin{enumerate}
\item For all parts of this problem, the goal is to estimate the area below $f(x) =\frac{1}{2}x^2+1$ and above the $x$-axis on the interval $[0,2].$\\
	\begin{enumerate}
	\item ($R_4$) Use $n=4$ rectangles and right-hand endpoints.
	\vfill
	\item ($L_4$) Use $n=4$ rectangles and left-hand endpoints.
	\vfill
	\item ($M_4$) Use $n=4$ rectangles and midpoints endpoints.
	\vfill
	\newpage
	\item Use $R_{10}$
	\vfill
	\end{enumerate}
\item Oil leaked out of a tank at a rate of $r(t)$ liters per hour. The rate decreased as time passed and values of the rate atn 2-hour time intervals are shown in the table. Estimate how much oil leaked out. What method are you using? Is is an over estimate? Underestimate? Can you tell?
\begin{center}
\begin{tabular}{ | l | c|c|c|c|c|c|}
\hline
time, $t,$ (in hours) & 0&2&4&6&8&10\\
\hline
rate, $r(t),$ (in liters/hour)&8.7&7.6&6.8&6.2&5.7&5.3\\
\hline
\end{tabular}
\end{center}
\vfill	
\end{enumerate}
\end{document}
