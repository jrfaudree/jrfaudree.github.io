
\documentclass[11pt,fleqn]{article} 
\usepackage[margin=0.8in, head=0.8in]{geometry} 
\usepackage{amsmath, amssymb, amsthm}
\usepackage{fancyhdr} 
\usepackage{palatino, url, multicol}
\usepackage{graphicx, pgfplots} 
\usepackage[all]{xy}
\usepackage{polynom} 
%\usepackage{pdfsync} %% I don't know why this messes up tabular column widths
\usepackage{enumerate}
\usepackage{framed}
\usepackage{setspace}
\usepackage{array,tikz}

\pgfplotsset{compat=1.6}

\pgfplotsset{soldot/.style={color=black,only marks,mark=*}} \pgfplotsset{holdot/.style={color=black,fill=white,only marks,mark=*}}
\pgfplotsset{my style/.append style={axis x line=middle, axis y line=
middle, xlabel={$x$}, ylabel={$y$} }}

%axis equal 
\pagestyle{fancy} 
\lfoot{}
\rfoot{3-6 Chain Rule}

\begin{document}
\renewcommand{\headrulewidth}{0pt}
\newcommand{\blank}[1]{\rule{#1}{0.75pt}}
\newcommand{\bc}{\begin{center}}
\newcommand{\ec}{\end{center}}
\renewcommand{\d}{\displaystyle}

\vspace*{-0.7in}

%%%%%%%%%intro page
\begin{center}
  \large
  \sc{Section 3-6: The Chain Rule}\\
\end{center}
\begin{enumerate}
\item Recall Two Versions of the Chain Rule
\vspace{1in}
\item Understanding what the ``formulas" in the book are trying to communicate:
\vspace{2in}
\item Find the derivatives for each function below:
	\begin{enumerate}
	\item $f(\theta)=4\tan(\theta/\pi).$
	\vfill
	\item $g(t)=\sqrt[5]{\sin(7t)}$
	\vfill
	\item $h(x)=\sin(x^2-\frac{1}{x^2+x})$
	\vfill
	\end{enumerate}
\newpage
\item (Some additional independent practice) Find the derivatives.
	\begin{enumerate}
	\item $f(x)=(\sec(3x)+\csc(2x))^5$
	\vfill
	\item $g(x)=\frac{\cot(x^2+1)}{x^3+1}$
	\vfill
	\item $h(x)=(2x-1)^3(2x+1)^5$
	\vfill
	\end{enumerate}
\item Find all $x$-values where the tangent to $f(x)=(x^2-4)^3$ is horizontal.
\vfill


\item Use the table below to evaluate the derivatives of the given functions at the indicated value.

\begin{multicols}{2}
\begin{tabular}{c||c|c|c|c}
$x$ &$f(x)$ & $f'(x)$ & $g(x)$ & $g'(x)$ \\
\hline \hline
-1 & 2&-1&0&1 \\
\hline
0 & 1&2&3&4 \\
\hline
1& -1&-2&-3&-4 \\
\hline
2&0&4&-1&2\\
\end{tabular}
	\begin{enumerate}
	\item $\displaystyle{h(x)=f(g(x))}$ at $a=2.$
	\vfill
	\item $\displaystyle{k(x)=f(x)g(x^2)}$ at $a=1$
	\vfill
	\end{enumerate}
\end{multicols}
\vspace{.3in}
\end{enumerate}
\end{document}

