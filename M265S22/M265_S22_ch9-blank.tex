\documentclass[11pt]{report}

\usepackage{geometry,amsmath,amssymb,amsthm}

\geometry{margin=1.in}

\theoremstyle{plain}
\newtheorem{thm}{Theorem}
\newtheorem{lem}[thm]{Lemma}
\newtheorem{prop}[thm]{Proposition}
\newtheorem{cor}[thm]{Corollary}


\begin{document}
\hfill Math 265

\hfill \today

\begin{center}
\Large{\textbf{Wednesday 16 March}} \\
\end{center}
\begin{enumerate}
\item Let $A,$ $B$, and $C$ be sets. If $A \times C = B \times C$,  then $A=B.$\\

Proof: Let $A,$ $B$, and $C$ be sets and assume $A \times C = B \times C.$\\

Let $a \in A$ and let $c \in C.$ Thus, by definition $(a,c) \in A \times C.$ Since by assumption $A \times C = B \times C$ and  $(a,c) \in A \times C,$  it follows that $(a,c) \in B \times C.$ Since $(a,c) \in B \times C,$ it follows that $a \in B.$ Thus, we have shown that if $a \in A,$ then $a \in B.$ Thus, $A \subseteq B.$\\

If we switch $A$ and $B$ in the previous paragraph, we prove that $B \subseteq A.$\\

Since $A \subseteq B$ and $B \subseteq A,$ it follows that $A=B.$\\

\item If $x,y \in \mathbb{R}$ and $x^3 < y^3,$ then $x < y.$\\

Proof: Let $x,y \in \mathbb{R}$ and $x^3 < y^3.$ Proceed by contradiction and assume that $x \geq y.$ Using algebra, we obtain:
\begin{equation} \label{factor} 0 < y^3-x^3=(y-x)(x^2+xy+y^2).\end{equation}
Since $x \geq y,$ we know that $y-x \leq 0.$ If $y-x=0,$ then expression \ref{factor} gives:
\begin{equation} 0 < y^3-x^3=(y-x)(x^2+xy+y^2)=0\cdot(x^2+xy+y^2)=0,\end{equation}
a contradiction. On the other hand, if $x-y<0,$ then expression \ref{factor} implies that $x^2+xy+y^2<0$ in order for the product to be positive. Thus, $xy < 0.$ So $y<0$ and $x>0.$ But this implies $y^3<0$ and $x^3 >0.$ But this contradicts the assumption that $y^3 < x^3.$

Since a contradiction is obtained in both cases, we can conclude that $x < y.$\\

\item For every $n \in\mathbb{N},$ $\displaystyle \sum_{i=1}^{n} i = \frac{n(n+1)}{2}.$

Proof: (by induction)\\
Base Case: If $n=1,$ then $\displaystyle \sum_{i=1}^{1} i =1= \frac{1(1+1)}{2}.$ Thus, the proposition holds for $n=1.$\\

Inductive Case: Let $k \in \mathbb{N}.$ (So $k \geq 1.$) Suppose that $\displaystyle \sum_{i=1}^{k} i = \frac{k(k+1)}{2}.$ We must show that $\displaystyle \sum_{i=1}^{k+1} i = \frac{(k+1)(k+2)}{2}.$\\
Observe 

\begin{align*}
\sum_{i=1}^{k+1} i  &= 1+2+3+ \cdots + (k-1)+k+(k+1) &\text{expanding summation notation} \\
 &=(1+2+3+ \cdots + (k-1)+k)+(k+1) &\text{associativity of addition} \\
 &=\left(\sum_{i=1}^{k} i \right)+(k+1) &\text{contracting summation notation} \\
  &=\left(\frac{k(k+1)}{2}\right)+(k+1) &\text{contracting summation notation} \\
  &=\left(\frac{k(k+1)}{2}\right)+\frac{2(k+1)}{2} &\text{algebra} \\
   &=\left(\frac{(k+2)(k+1)}{2}\right),&
  \end{align*}
  which is what we wanted to show.\\
  
  If follows by induction that for every $n \in\mathbb{N},$ $\displaystyle \sum_{i=1}^{n} i = \frac{n(n+1)}{2}.$
\end{enumerate}
\end{document}