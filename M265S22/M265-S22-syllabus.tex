\documentclass[12pt]{article}

% Layout.
\usepackage[top=1in, bottom=0.75in, left=1in, right=1in, headheight=1in, headsep=6pt]{geometry}

% Fonts.
\usepackage{mathptmx}
\usepackage[scaled=0.86]{helvet}
\renewcommand{\emph}[1]{\textsf{\textbf{#1}}}

% Misc packages.
\usepackage{amsmath,amssymb,latexsym}
\usepackage{graphicx,hyperref}
\usepackage{array}
\usepackage{xcolor}
\usepackage{multicol}
\usepackage{tabularx,colortbl}
\usepackage{enumitem}

\hypersetup{
    colorlinks=true,
    linkcolor=blue,
    filecolor=magenta,      
    urlcolor=blue,
    pdftitle={Overleaf Example},
    pdfpagemode=FullScreen,
    }

% Paragraph spacing
\parindent 0pt
\parskip 6pt plus 1pt
\def\tableindent{\hskip 0.5 in}
\def\ts{\hskip 1.5 em}

%header
\usepackage{fancyhdr}
\pagestyle{fancy} 
\lhead{\large\sf\textbf{MATH 265: Introduction to Mathematical Proofs}}
\rhead{\large\sf\textbf{Spring 2022 Syllabus}}

\newcommand{\localhead}[1]{\par\smallskip\textbf{#1}\nobreak\\}%
\def\heading#1{\localhead{\large\emph{#1}}}
\def\subheading#1{\localhead{\emph{#1}}}

\newenvironment{clist}%
{\bgroup\parskip 0pt\begin{list}{$\bullet$}{\partopsep 4pt\topsep 0pt\itemsep -2pt}}%
{\end{list}\egroup}%


\begin{document}

\begin{tabular}{p{0.2\linewidth}  p{0.8\linewidth}}
\textbf{Instructor:}& Jill Faudree\\
\textbf{Contact Details:}&Chapman 306B, jrfaudree@alaska.edu, 474-7385\\
\textbf{Office Hours*:}& MWF before class (10:30-11:30am) and R 11:30am-1:00pm and by appointment. Also, you are welcome to drop by.\\
\textbf{Textbook:} &\emph{Book of Proof} by Richard Hammack, 3rd Edition\\
 &(ISBN: 978-0-9894721-2-8 (paperback))\\
 & \textbf{OR}\\
 & the free online version \\
 &\url{https://www.people.vcu.edu/~rhammack/BookOfProof/}\\
\textbf{Lecture Hours:}& MWF 11:45am-12:45pm Grue 405\\
\textbf{Course Web Page:}& \url{https://jrfaudree.github.io/ProofsSpring22.html}\\
\textbf{Course Grades:}& on Canvas\\
\textbf{Midterm Dates*:}& Wednesday February 23 and Wednesday March 30\\
\textbf{Final Exam:}& Friday, April 29, 10:15am-12:15pm\\
\textbf{Prerequisites:} &MATH 252 Calculus II with a grade of C- or better \textbf{OR} concurrent enrollment in MATH 252 Calculus II\\
\end{tabular}

*Note that official Office Hours and Midterm Dates may change over the semester. Any changes will be announced in class and on the course webpage.\\

\heading{Course Overview and Learning Outcomes}

This course is an introduction to formal, rigorous mathematical proof. It is intended to give you an understanding of the logical framework and common techniques used to build proofs and to teach you how to construct and write your own. 

A mathematical proof may have many forms but essentially all are governed by some established principles. There are explicit assumptions which are used to demonstrate that a stated conclusion is guaranteed. The demonstration of this guarantee uses accepted and explicitly stated rules of logical reasoning. 

For many consumers of mathematics, the primary role of mathematics is as a tool for solving some other problem (from Physics, Economics, etc.). From this point of view, it is sufficiency if the mathematical process consistently appears to give a solution supported by observation or data. However, mathematics from the mathematicians' view generally requires proof. Indeed for most mathematicians proof is the defining property of the subject. No conjecture, however much empirical evidence supports it, becomes a theorem without proof.

In summary, this is the course that gives the insider's view of the subject.\\


\heading{Course Mechanics}

\textbf{Class meetings} will be run as an interactive lecture. Sometimes I will lecture at the board, sometimes students will present their work. Everyday you will do work in class, sometimes individually and sometimes in groups. You should always bring paper and pencil to class.

\textbf{Attendance} is mandatory. 

Being present in class and being active in class are so important to student success that \textbf{class participation} is a portion of your grade. Acceptable class participation includes asking questions, attempting to answer questions, contributing to group work, and allowing others to contribute. You will get an easy A on this portion of your grade as a reward for being present, active, and engaged in the class meetings and for being respectful to the other members of the class.

\textbf{Homework} will be assigned weekly and turned in via an upload link in Canvas. The entire homework assignment will be checked to make sure you have attempted everything. Selected problems will be graded completely.  Your written homework average will be calculated as (points earned)/(points possible). 

A selection of homework problems will be written using the mathematical typesetting tool called {\LaTeX}. 

There are many benefits to learning to use \LaTeX. It's easier to revise and edit your work. It encourages writing actual sentences since most people can type words faster than they can hand-write them. It will help you focus on the quality of your writing. 

On the course webpage, you will find a link to resources for getting started with \LaTeX. 

\textbf{Collaboration} with classmates to solve homework problems is encouraged. Talking with others about math is a great way to solidify knowledge and clarify points of confusion. \textbf{But each student should write up their solutions independently.} The line between collaboration and plagarism and/or cheating is crossed if one student gives another student a complete solution (on paper or electronically). If you receive significant help solving a problem, it is customary to make a note in your homework to give the person who helped you credit. 

There will be two \textbf{midterms} during the semester and a comprehensive \textbf{final exam}. The two midterms are tentatively scheduled for Wednesday February 23 and Wednesday March 30.  \textbf{The Final Exam will be Friday, April 29, 10:15am-12:15pm.} It is DMS policy that final exams cannot be given early or late.

\textbf{Make-up Midterms} will be given only for excused absences. 

\textbf{Grades} will be calculated according to the rubric on the left. Letter grades will be assigned according to the scale on the right.
This scale is a guarantee; the instructors reserve the right to lower the thresholds.

\begin{multicols}{2}
\begin{tabular}{|l|c|}
  \hline
  homework & 25\%\\
  participation& 5\%\\
  Test 1 & 22.5\% \\
  Test  2 & 22.5\% \\
  Final Exam & 25\% \\
  \hline
\end{tabular}

 

\def\sts{\hskip 0.5em}
\strut\hbox to\hsize{\vbox{\halign{#\hfil\sts&#\hfil\ts&#\hfil\sts&#
\hfil\ts&#\hfil\sts&#\hfil   \cr
A+ & 97--100\% & C+ & 77--79\% & F  & $<$ 60\%\cr

A & 93--96\% &  C & 73--76\%&&\cr
A- & 90--92\% & C- & 70--72\%&&\cr
B+ & 87--89\% & D+ & 67--69\%&&\cr
B &  83--86\% & D & 63--66\%&&\cr
B- & 80-82\% & D- & 60--62\%&&\cr
}}\hfil}
\end{multicols}


 

\heading{Tutoring and Resources}
\vskip -30pt\strut
\begin{clist}
	\item The Math and Stat Lab, Chapman Building Room 305, offers tutors. 
	See 

	\url{https://uaf.edu/dms/mathlab/} for schedules and availability.
	\item Free
one-on-one (or small group) tutoring is available in 
Chapman Building Room 201. You must schedule an
appointment; see \url{https://uaf.edu/dms/mathlab/}.
	\item Student Support Services offers free tutoring in many subjects to students who qualify for their program.
	\item ASUAF offers private tutoring for a small fee (based on student income).
\end{clist}

\heading{Rules and Policies}
\vskip -20pt

\subheading{Incomplete Grade} 
Incomplete (I) will only be given in
  DMS courses in cases where
  the student has completed the majority (normally all but the last
  three weeks) of a course with a grade of C or better, but for
  personal reasons beyond his/her control has been unable to complete
  the course during the regular term. Negligence or indifference are
  not acceptable reasons for the granting of an incomplete
  grade. 

\subheading{Late Withdrawals} 
A withdrawal after the deadline
  (currently 9 weeks into the semester) from a DMS course will
  normally be granted only in cases where the student is performing
  satisfactorily (i.e., C or better) in a course, but has exceptional
  reasons, beyond his/her control, for being unable to complete the
  course. These exceptional reasons should be detailed in writing to
  the instructor, department head and dean.

\subheading{No Early Final Examinations}
Final examinations for DMS
  courses shall not be held earlier than the date and time published
  in the official term schedule. Normally, a student will not be
  allowed to take a final exam early. Exceptions can be made by
  individual instructors, but should only be allowed in exceptional
  circumstances and in a manner which doesn't endanger the security of
  the exam.

\subheading{Academic Dishonesty}
Academic dishonesty, including cheating and plagiarism, will not
be tolerated.  It is a violation of the Student Code of Conduct
and will be punished according to UAF procedures.

 %\begin{center} \textsc{Syllabus Addendum} \end{center}
 
 \noindent{\bf COVID-19 statement:} Students should keep up-to-date on the university's policies, practices, and mandates related to COVID-19 by regularly checking this website: \url{https://sites.google.com/alaska.edu/coronavirus/uaf?authuser=0}

Further, students are expected to adhere to the university's policies, practices, and mandates and are subject to disciplinary actions if they do not comply.

\noindent{\bf Student protections statement:} UAF embraces and grows a culture of respect, diversity, inclusion, and caring. Students at this university are protected against sexual harassment and discrimination (Title IX). Faculty members are designated as responsible employees which means they are required to report sexual misconduct. Graduate teaching assistants do not share the same reporting obligations. For more information on your rights as a student and the resources available to you to resolve problems, please go to the following site: \url{https://catalog.uaf.edu/academics-regulations/students-rights-responsibilities/}.

\noindent{\bf Disability services statement:} I will work with the Office of Disability Services to provide reasonable accommodation to students with disabilities.

\noindent{\bf Student Academic Support:}
\begin{itemize}
\setlength\itemsep{0em}
        \item Speaking Center (907-474-5470,
        email: {uaf-speakingcenter@alaska.edu}, Gruening 507)
\item Writing Center (907-474-5314, email: {uaf-writing-center@alaska.edu}, Gruening 8th floor)
\item UAF Math Services, email: {uafmathstatlab@gmail.com}, Chapman Building (for math fee paying students only)
\item Developmental Math Lab, Gruening 406
\item The Debbie Moses Learning Center at CTC (907-455-2860, 604 Barnette St, Room 120,\\ email: {https://www.ctc.uaf.edu/student-services/student-success-center/})
\item For more information and resources, please see the Academic Advising Resource List (\url{https://www.uaf.edu/advising/lr/SKM_364e19011717281.pdf})
\end{itemize}

\noindent{\bf Student Resources:}
\begin{itemize}
\setlength\itemsep{0em}
\item Disability Services (907-474-5655, email: {uaf-disability-services@alaska.edu}, Whitaker 208)
\item Student Health \& Counseling [6 free counseling sessions] (907-474-7043, \url{https://www.uaf.edu/chc/appointments.php}, Whitaker 203)
\item Center for Student Rights and Responsibilities (907-474-7317, email: {uaf-studentrights@alaska.edu}, Eielson 110)
\item Associated Students of the University of Alaska Fairbanks (ASUAF) or ASUAF Student Government (907-474-7355, email: {asuaf.office@alaska.edu}, Wood Center 119)
\end{itemize}

\noindent{\bf Nondiscrimination statement:}
The University of Alaska is an affirmative action/equal opportunity employer and educational institution. The University of Alaska does not discriminate on the basis of race, religion, color, national origin, citizenship, age, sex, physical or mental disability, status as a protected veteran, marital status, changes in marital status, pregnancy, childbirth or related medical conditions, parenthood, sexual orientation, gender identity, political affiliation or belief, genetic information, or other legally protected status. The University's commitment to nondiscrimination, including against sex discrimination, applies to students, employees, and applicants for admission and employment. Contact information, applicable laws, and complaint procedures are included on UA's statement of nondiscrimination available at www.alaska.edu/nondiscrimination. For more information, contact:

\begin{tabular}{l}
UAF Department of Equity and Compliance\\
1760 Tanana Loop, 355 Duckering Building, Fairbanks, AK  99775\\
907-474-7300\\
email: {uaf-deo@alaska.edu}
\end{tabular}

\hfill

 \scriptsize syllabus version: \today \normalsize\end{document}