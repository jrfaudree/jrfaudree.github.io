\documentclass[11pt]{article}
%{amsart}
%\pagestyle{empty} 
\setlength{\topmargin}{-0.8in} % usually -0.25in
\addtolength{\textheight}{1.4in} % usually 1.25in
\addtolength{\oddsidemargin}{-0.8in}
\addtolength{\evensidemargin}{-0.8in}
\addtolength{\textwidth}{1.6in} %\setlength{\parindent}{0pt}

\newcommand{\normalspacing}{\renewcommand{\baselinestretch}{1.1}\tiny\normalsize}
\normalspacing

% macros
\usepackage{amsmath,amssymb,mathrsfs,amsthm,xspace,alltt,verbatim,fancyhdr,mathtools}
\usepackage[final]{graphicx}
\usepackage[pdftex,colorlinks=true]{hyperref}
\usepackage{fancyvrb}
\usepackage{tikz}

\newtheorem*{lem*}{Lemma}


\newcommand{\be}{\begin{enumerate}}
 \newcommand{\ee}{\end{enumerate}}

\newcommand{\bF}{\mathbf{F}}
\newcommand{\bN}{\mathbf{N}}
\newcommand{\bT}{\mathbf{T}}

\newcommand{\CC}{{\mathbb{C}}}
\newcommand{\RR}{{\mathbb{R}}}
\newcommand{\eps}{\epsilon}
\newcommand{\ZZ}{{\mathbb{Z}}}
\newcommand{\QQ}{{\mathbb{Q}}}
\newcommand{\ZZn}{{\mathbb{Z}}_n}
\newcommand{\NN}{{\mathbb{N}}}
\newcommand{\ip}[2]{\mathrm{\left<#1,#2\right>}}

\renewcommand{\Re}{\operatorname{Re}}
\renewcommand{\Im}{\operatorname{Im}}

\newcommand{\Log}{\operatorname{Log}}

\newcommand{\grad}{\nabla}

\newcommand{\Matlab}{\textsc{Matlab}\xspace}
\newcommand{\Octave}{\textsc{Octave}\xspace}
\newcommand{\pylab}{\textsc{pylab}\xspace}

\newcommand{\prob}[1]{\bigskip\noindent\textbf{#1.} }
\newcommand{\pts}[1]{(\emph{#1 pts})}

\newcommand{\probpts}[2]{\prob{#1} \pts{#2} \quad}
\newcommand{\ppartpts}[2]{\textbf{(#1)} \pts{#2}}
\newcommand{\epartpts}[2]{\medskip\noindent \textbf{(#1)} \pts{#2}}

\lhead{\sc{Math 265 Proofs}}
\chead{\large \sc Final Exam} 
\rhead{\sc Spring 2022}
\cfoot{}
\pagestyle{fancy}
\begin{document}
\thispagestyle{fancy}


\medskip
\large
\vspace{.1in}
\begin{tabular}{l@{\hspace{.4in}}l}
Your Name & Your Signature\\
\framebox(200,30){Solutions} & \framebox(200,30){} \\
\end{tabular}

%\bigskip

\vfill
{
\renewcommand{\baselinestretch}{1.8}
\setlength{\tabcolsep}{.2in}
\normalsize
\begin{center}
\begin{tabular}{|c|c|c|}
\hline
Problem&Total Points&\parbox{.8in}{\hfil Score\hfil}\\
\hline
1&20&\\
\hline
2&12&\\
\hline
3&10&\\
\hline
4&10&\\
\hline
5&15&\\
\hline
6&15&\\
\hline
7&10&\\
\hline
8&8&\\
\hline
extra credit&5&\\
\hline
\hline
%\hline
Total&100&\\
\hline
%Current Course Grade&\multicolumn{2}{c|  }{}\\
%\hline

\end{tabular}

\end{center}
}
\vfill
\begin{itemize}
\item 
You have 2 hours.

\item If you have a cell phone with you, it should be turned off and put away. (Not in your pocket)

\item You may not use a calculator, book, notes or aids of any kind.

\item In order to earn partial credit, you must show your work.

\end{itemize}
\newpage
\begin{enumerate}
%miscellany
\item (20 points)
	\begin{enumerate}
	\item State the negation of each statement below.
		\begin{enumerate}
		\item If $n$ is divisible by 14, then $n$ is divisible by $2$ and $n$ is divisible by $7.$\\
		
		Answer: $n$ is divisible by 14 and ($n$ is not divisible by $2$ or $n$ is not divisible by $7$.)
		\vfill
		\item For every real number $r$ there exists a rational number $q$ such that $r < q < r+1.$\\
		
		Answer: There exists a real number $r$ such that for every rational number $q$, $q \leq r$ or $r+1 \geq q.$\\
		\vfill
		\end{enumerate}
	\vfill
	\item Determine the truth value of the statements below.
		\begin{enumerate}
		\item $2 \in \mathcal{P}(\{0,1,2,3\})$\\
		
		FALSE. Elements of $\mathcal{P}(\{0,1,2,3\})$ are sets.\\
		
		\vfill
		\item $\{ \emptyset, \{0,1\}\} \subseteq  \mathcal{P}(\{0,1,2,3\})$\\
		
		TRUE. Both $\emptyset$ and $ \{0,1\}$ are elements of  $\mathcal{P}(\{0,1,2,3\})$
		\vfill
		\end{enumerate}
	\item List three different partitions of the set $S=\{1,2,3\}.$ Label your partitions $P_1, P_2,$ and $P_3.$ Use correct notation.\\
	
	Some examples: $P_1=\{\{1,2,3\}\}, P_2=\{\{1,2\},\{3\}\}, P_3=\{\{1\},\{2,3\}\}$\\
	\vfill 
	\item Let $R$ be an equivalence relation on $S=\{a,b,c,d\}$ such that $aRb$ and $dRa.$ Circle all of the following statements that \emph{must} also be true.
		\begin{enumerate}
		\item $cRc$ (TRUE)
		\item $bRd$ (TRUE)
		\item $d \in [a]$ (TRUE)
		\end{enumerate}
	\end{enumerate}
\newpage
%plain induction
\item (12 points) Prove that $1^3+2^3+3^3 + \cdots + n^3=\frac{n^2(n+1)^2}{4}$ for all $n \in \NN.$\\

\textbf{Proof:} (by induction on $n$.)\\

Base Step: Let $n=1.$ Observe $1^3 =1=\frac{4}{4}=\frac{1^2(1+1)^2}{4}.$ Thus, the proposition holds for $n=1.$\\

Inductive Step: Suppose $k \in \NN$, $n\geq 1,$ and $1^3+2^3+3^3 + \cdots + k^3=\frac{k^2(k+1)^2}{4}.$ We must show that $1^3+2^3+3^3 + \cdots + (k+1)^3=\frac{(k+1)^2(k+2)^2}{4}.$ Observe

\begin{align*}
\sum_{i=1}^{k+1} i^3 &=\left(\sum_{i=1}^{k} i^3 \right)+ (k+1)^3& \\
&=\frac{k^2(k+1)^2}{4}+(k+1)^3&\text{ by inductive hypothesis}\\
&=\frac{k^2(k+1)^2}{4}+\frac{4(k+1)^3}{4}&\text{ common denominator}\\
&=\frac{(k+1)^2}{4}\left({k^2+4k+4}\right)&\text{ factor}\\
&=\frac{(k+1)^2}{4}\left(k+2\right)^2,&\\
\end{align*}
which is what we wanted to show. \\

Thus, by the method of induction, $1^3+2^3+3^3 + \cdots + n^3=\frac{n^2(n+1)^2}{4}$ for all $n \in \NN.$\\
\vfill

\newpage
%proof by contrapositive
\item (10 points) Use the method of proof by contrapositive to prove the proposition below.\\
\begin{quote} Suppose $a,b \in \ZZ.$ If $(a+1)b^2$ is even, then $a$ is odd or $b$ is even.\end{quote}

\textbf{Proof:} We will prove that if $a$ is even and $b$ is odd, then $(a+1)b^2$ is odd.\\

Suppose that $a$ is even and $b$ is odd. Thus, there exist integers $m$ and $n$ such that $a=2m$ and $b=2n+1.$ Thus, 
$$ (a+1)b^2=(2m+1)(2n+1)^2 = 2(4mn^2+4mn+m+2n^2+2n)+1,$$ where $4mn^2+4mn+m+2n^2+2n \in \ZZ.$ Thus, $(a+1)b^2$ is odd.\\

\vfill
%proof by contradiciton
\item (10 points) Use the method of proof by contradiction to proof the proposition below.\\
\begin{quote} Suppose $a,b \in \RR.$ If $a$ is rational and $ab$ is irrational, then $b$ is irrational.\end{quote}

\textbf{Proof:} Suppose  $a,b \in \RR,$ $a$ is rational, and $ab$ is irrational. Further, suppose by way of a contradiction that $b$ is rational.\\

Since $a$ and $b$ are rational, there exist integers $m,n,p,q$, $n \not = 0$ and $q \not = 0$ such that $a=\frac{m}{n}$ and $b={p}{q}.$ Thus, $ab=\frac{mp}{nq},$ where $nq \not = 0.$ Thus, $ab$ is rational, which contradicts the assumption that $ab$ is irrational. \\

Thus, by the method of proof by contradiction, if $a$ is rational and $ab$ is irrational, then $b$ is irrational.
\vfill
\newpage
%injective/surjective
\item (15 points) Let the function $f: [0,\infty) \to [6, \infty)$ be defined as $f(x)=3x^2+6.$ Prove that $f$ is a bijection.

\textbf{Proof:} Let $f: [0,\infty) \to [6, \infty)$ be defined as $f(x)=3x^2+6.$\\

(one-to-one) Let $x,x' \in [0,\infty)$ such that $f(x)=f(x').$ Thus, $3x^2+6=3(x')^2+6.$ By subtracting 6 and dividing by 3, we obtain the equation $x^2=(x')^2.$ Since both $x$ and $x'$ are nonnegative, $x=x'.$ Thus, $f$ is injective.\\

(onto) Let $y \in [6, \infty).$ Pick $x=\displaystyle \sqrt{\frac{y-6}{3}}.$ Observe that since $y \geq 6,$ we know $(y-6)/3 \geq 0.$ Thus, $x \in [0,\infty).$ Now, 
$$f(x)=f\left(\displaystyle \sqrt{\frac{y-6}{3}}\right)=3\left(\displaystyle \sqrt{\frac{y-6}{3}}\right)^2+6=y.$$ Thus, $f$ is onto.\\

Since $f$ is one-to-one and onto, $f$ is bijective.\\
\vfill
\newpage
%symmetric, reflexive, transitive
\item (15 points) Let $R$ be a relation on $\mathbb{R}$ such that $xRy$ if $x-y\in \mathbb{Z}.$ 
	\begin{enumerate}
	\item 
	Prove that $R$ is an equivalence relation. \\
	
	\textbf{Proof:} We must show that $R$ is reflexive, symmetric, and transitive.\\
	
	(reflexive). Let $x \in \RR.$ Since $x-x=0$ and $0 \in \ZZ,$ it follows that $xRx.$ Thus, $R$ is reflexive.\\
	
	(symmetric). Let $x,y \in \RR$ such that $xRy.$ By the definition of $R$, it follows that $x-y=n \in \ZZ.$  Thus, $y-x=-n \in \ZZ.$ Thus, $yRx$ and we have shown that $R$ is symmetric.\\
	
	(transitive). Let $x,y,z \in \RR$ such that $xRy$ and $yRz.$ By definition, it follows that $x-y=n \in \ZZ$ and $y-z=m\in \ZZ.$ Now, $x-z=x-y+y-z=n+m \in \ZZ.$ Thus, $xRz$ and we have shown that $R$ is transitive.\\
	
	Since $R$ is reflexive, symmetric and transitive, $R$ is an equivalence relation.\\
	
	\item
       State three distinct elements in $[\pi],$ the equivalence class of $R$ containing $\pi.$\\
       
       There are an infinite number of correct answers. Some include: $-1 + \pi, \pi, 1+\pi, 2+\pi, 3+\pi.$
       
             \end{enumerate}

\newpage
%subset containment
\item (10 points) Let $A,B,$ and $C$ be sets. Suppose that $A \subseteq B,$ $B\subseteq C$, and $C \subseteq A.$ Prove that $A=B.$\\

\textbf{Proof:} Let $A,B,$ and $C$ be sets such that $A \subseteq B,$ $B\subseteq C$, and $C \subseteq A.$\\

To show that $A=B,$ we must show that $A \subseteq B$ and $B \subseteq A.$\\

Observe that $A \subseteq B$ by assumption.\\

(Show $B \subseteq A.$) Let $b \in B.$ Since $b \in B$ and $B \subseteq C,$ it follows that $b \in C.$ Since $b \in C$ and $C \subseteq A,$ it follows that $b \in A.$ Thus, $B \subseteq A.$



\newpage
%sets have same cardinality
\item (8 points) Demonstrate that the sets $\{0,1\} \times \NN$ and $\ZZ$ have the same cardinality.\\

\textbf{Example:} Let $f: \{0,1\} \times \NN \to \ZZ$ be defined as: $f(a,n)=\begin{cases} n-1 & \text{ if } a=0 \\
-n & \text{ if } a=1 \end{cases}.$ \\

\textbf{5 points extra credit} Prove that your answer above is correct.\\

We must show that $f$ is a bijection.\\

(injective) Let $(a,m),(b,n) \in \{0,1\} \times \NN$ such that $f(a,m)=f(b,n).$ There are two possibilities: (i) $a=b=0$ and $m-1=n-1$ or (ii) $a=b=1$ and $-m=-n.$ Both immediately imply not only that $a=b$ but also $m=n.$ Thus, $(a,m) = (b,n)$ and we have shown that the function is injective.

(surjective) Let $n \in \NN.$ We consider two cases: (i) $n<0$ and (ii) $n \geq 0.$ If $n < 0$, pick element $(1,-n) \in \{0,1\} \times \NN.$ Now, $f(1,-n)=-(-n)=n.$ If $n\geq 0,$ pick element $(0,n+1) \in  \{0,1\} \times \NN.$ Then, $f(0,n+1)=(n+1)-1=n.$ Thus, for every $n \in \NN,$ there exists an $(a,m) \in  \{0,1\} \times \NN$ such that $f(a,m) =n.$ Thus, $f$ is surjective.
\vfill	
\end{enumerate}
\vfill


\end{document}