\documentclass[12pt]{article}
\usepackage[margin=.8in]{geometry}
\usepackage{amsmath,amssymb,amsthm, latexsym, mathrsfs, pdfsync, 
fancybox, fancyhdr, 
graphicx, enumerate,
subfig, multicol}
\usepackage{tikz}
\usepackage{pgf}
\usepackage{pgfplots}
\usetikzlibrary{calc}

\newcommand{\blankbox}[2]{\fbox{\rule{#1}{0in}\rule{0in}{#2}}}
%special commands for number sets
\def\RR{{\mathbb R}}
\def\NN{{\mathbb N}}
\def\ZZ{{\mathbb Z}}
\def\QQ{{\mathbb Q}}
\def\CC{{\mathbb C}}
%special commands for formatting: center, enumerate, pmatrix, vector span
\def\bc{\begin{center}}
\def\ec{\end{center}}
\newcommand{\be}{\begin{enumerate}}
\newcommand{\ee}{\end{enumerate}}
\newcommand{\bpm}{\begin{pmatrix}}
\newcommand{\epm}{\end{pmatrix}}
\newcommand{\bv}[1]{\mathbf{#1}}
\newcommand{\spn}[1]{\text{Span}\left\{#1\right\}}
\newcommand{\lra}{\longrightarrow}
\newcommand{\llra}{\longleftrightarrow}

\setlength{\headheight}{30pt}
\setlength{\headsep}{20pt}
\setlength{\fboxsep}{8pt}
\setlength{\fboxrule}{1pt}

\lhead{\sc \quad \\ Math 265}
\chead{\sc Review: Midterm II } 
\rhead{\sc \quad \\ Spring 2022}
\cfoot{}
\pagestyle{fancy}
\pgfplotsset{compat=1.12}
\begin{document}
\thispagestyle{fancy}

\begin{center} Preliminaries \end{center}

The midterm will be given without the use of aids of any kind. You will have one hour to complete the midterm. The midterm will cover Chapters 7-10. All of the problems will be proofs.  For some problems, you will be asked to use a particular technique to prove a statement. For others, it will be your choice what method to use. In all cases, you are expected to write complete, formal, appropriately detailed proofs.\\

\begin{center} Definitions \end{center}

For all of the terms below, you must be able to formally state and use the definition from your textbook. 
\be
\item odd, even, same parity, opposite parity
\item divides, multiple, divisor
\item prime
\item greatest common divisor, least common multiple
\item congruent modulo $n$
\item rational number, irrational number
\item subsets, set equality
\ee

All of the terms below should be familiar to you.
\be
\item element of a set, {cardinality of a set}, set builder notation, {natural numbers, integers, rational numbers, irrational numbers, real numbers}, interval notation
\item ordered pair, {Cartesian product, ordered $n$-tuple}
\item {subset, the power set of a set}
\item {union, intersection and difference of two sets, complement of a set}
\item the mathematical meaning of \emph{and, or} and \emph{not}, conditional statement, biconditional, quantifiers, logically equivalent, contrapositive, negation
\ee

\begin{center} Proof Techniques \end{center}
\be
\item direct proof
\item using cases
\item by contrapositive
\item by contradiction
\item if-and-only-if proofs
\item existence statements
\item proofs involving sets (that is, statements including $A \subseteq B$ or $A=B$)
\item methods of disproving statements
\item proof by induction (both strong and weak)
\ee

\begin{center} Things to Keep in Mind \end{center}
\be
\item If a proof technique is not prescribed, you MUST state the method you are using.
\item You should put in the ``boiler-plate'' language even if you cannot figure out the whole proof.
\item You should expect to \emph{use} all of the hypotheses.
\item I will \emph{not} ask you to prove something that is false.
\ee



\end{document}