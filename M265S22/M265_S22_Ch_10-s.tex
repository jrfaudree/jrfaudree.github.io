\documentclass[11pt]{report}

\usepackage{geometry,amsmath,amssymb,amsthm}

\geometry{margin=1.in}

\theoremstyle{plain}
\newtheorem{thm}{Theorem}
\newtheorem{lem}[thm]{Lemma}
\newtheorem{prop}[thm]{Proposition}
\newtheorem{cor}[thm]{Corollary}

\newcommand{\R}{\mathbb{R}}
\newcommand{\Z}{\mathbb{Z}}
\newcommand{\N}{\mathbb{N}}
\newcommand{\sse}{\subseteq}


\begin{document}
\hfill Math 265

\begin{center}
\Large{\textbf{Chapter 10 Solutions }} \\
\end{center}

\begin{description}
%%%%NEW PROBLEM
\item{\#10.3} Prove that $1^3+2^3+3^3+4^3 + \cdots +n^3=\frac{n^2(n+1)^2}{4}$ for every positive integer $n.$ \\
\begin{proof} We will proceed by induction
\begin{align*}
\sum_{i=1}^{k+1} i  &= 1+2+3+ \cdots + (k-1)+k+(k+1) &\text{expanding summation notation} \\
 &=(1+2+3+ \cdots + (k-1)+k)+(k+1) &\text{associativity of addition} \\
 &=\left(\sum_{i=1}^{k} i \right)+(k+1) &\text{contracting summation notation} \\
  &=\left(\frac{k(k+1)}{2}\right)+(k+1) &\text{contracting summation notation} \\
  &=\left(\frac{k(k+1)}{2}\right)+\frac{2(k+1)}{2} &\text{algebra} \\
   &=\left(\frac{(k+2)(k+1)}{2}\right),&
  \end{align*}
  which is what we wanted to show.\\ \\
\end{proof}

\item{\#10.4} If $n \in N,$ then $1\cdot 2+2 \cdot 3 + 3 \cdot 4 + 4 \cdot 5 + \cdots + n(n+1)=\frac{n(n+1)(n+2)}{3}$. \\
\begin{proof}  Your proof here. \\

\end{proof}

\item{\#10.5} If $n \in N,$ then $2^1+2^2+3^3+\cdots+2^n=2^{n+1}-2$.\\
\begin{proof} Your proof here. \\

\end{proof}

\item{\#10.8} If $n \in \N,$ then $\frac{1}{2!}+\frac{2}{3!}+\frac{3}{4!}+\cdots+\frac{n}{(n+1)!}=1-\frac{1}{(n+1)!}.$\\
\begin{proof}  Your proof here. \\

\end{proof}

\item{\#10.10} Prove that $3 \mid (5^{2n}-1)$ for every integer $n\geq 0.$\\
\begin{proof}   Your proof here. \\

\end{proof}

\item{\#10.13} Prove that $6 \mid (n^3-n)$ for every integer $n \geq 0.$\\
\begin{proof}  Your proof here. \\

\end{proof}

\end{description}

\end{document}