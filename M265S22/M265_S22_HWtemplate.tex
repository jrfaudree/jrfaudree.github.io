\documentclass[11pt]{report}

\usepackage{geometry,amsmath,amssymb,amsthm}

\geometry{margin=1.in}

\theoremstyle{plain}
\newtheorem{thm}{Theorem}
\newtheorem{lem}[thm]{Lemma}
\newtheorem{prop}[thm]{Proposition}
\newtheorem{cor}[thm]{Corollary}


\begin{document}
\hfill Math 265

\hfill N. Bourbaki

\hfill \today

\begin{center}
\Large{\textbf{Homework \#0}} \\
due 1/22/2020
\end{center}
\begin{description}
\item{\S 4.1, \#3}

$$\int_1^2 \frac 1 {x^2} \, dx=\int_1^2 x^{-2} \, dx=-x^{-1} \Big |_1^2=-\frac 12+1=\frac 12$$

\item{\S 4.2, \#17}

\begin{thm} $\sqrt 2$ is irrational.
\end{thm}
\begin{proof}
Suppose, to the contrary, that $\sqrt 2$ is rational. Then
$$\sqrt 2=\frac ab$$
where $a,b\in \mathbb Z$, $b\not = 0$ with $a,b$ having no common factors. Squaring yields
$$2=\frac {a^2}{b^2},$$
so
$$2b^2=a^2.$$
This shows 2 divides $a^2$, and so since 2 is prime by a lemma proved in class, we see 2 divides $a$. Letting $a=2c$ for some $c\in \mathbb Z$, this implies
$$2b^2=4c^2,$$ so
$$b^2=2c^2.$$
Now the same argument as above, but with $b,a$ replaced by $c,b$, shows 2 divides $b$. Therefore 2 divides both $a$ and $b$. But this contradicts that $a,b$ had no common factors.
\end{proof}


\item[\S 99.99, \#99]

\ 

\begin{tabular}{c|c|c|c|c}

$P$ & $Q$ & $P\lor Q$ & $P\Rightarrow Q$ & $P \iff Q$ \\
\hline
$T$  & $T$ & & & $T$ \\
$T$  & $F$ & & & \\
$F$  & $T$ & & & \\
$F$  & $F$ & & & \\
\end{tabular}

\ 

\item[An example of aligned equations:]


%note that "align" automatically puts you in math mode -- no need for $$... $$
\begin{align*}
xy&=(2a+1)(2b+1)\\
&=(2a)(2b)+(2a)(1)+1(2b)+1(1)\\
&=4ab+2a+2b+1\\
&=2k+1,
\end{align*}

\end{description}

\end{document}