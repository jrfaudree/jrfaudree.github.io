\documentclass[11pt]{article}
%{amsart}
%\pagestyle{empty} 
\setlength{\topmargin}{-0.8in} % usually -0.25in
\addtolength{\textheight}{1.4in} % usually 1.25in
\addtolength{\oddsidemargin}{-0.8in}
\addtolength{\evensidemargin}{-0.8in}
\addtolength{\textwidth}{1.6in} %\setlength{\parindent}{0pt}

\newcommand{\normalspacing}{\renewcommand{\baselinestretch}{1.1}\tiny\normalsize}
\normalspacing

% macros
\usepackage{amsmath,amssymb,mathrsfs,amsthm,xspace,alltt,verbatim,fancyhdr,mathtools}
\usepackage[final]{graphicx}
\usepackage[pdftex,colorlinks=true]{hyperref}
\usepackage{fancyvrb}
\usepackage{tikz}

\newtheorem*{lem*}{Lemma}


\newcommand{\be}{\begin{enumerate}}
 \newcommand{\ee}{\end{enumerate}}

\newcommand{\bF}{\mathbf{F}}
\newcommand{\bN}{\mathbf{N}}
\newcommand{\bT}{\mathbf{T}}

\newcommand{\CC}{{\mathbb{C}}}
\newcommand{\RR}{{\mathbb{R}}}
\newcommand{\eps}{\epsilon}
\newcommand{\ZZ}{{\mathbb{Z}}}
\newcommand{\QQ}{{\mathbb{Q}}}
\newcommand{\ZZn}{{\mathbb{Z}}_n}
\newcommand{\NN}{{\mathbb{N}}}
\newcommand{\ip}[2]{\mathrm{\left<#1,#2\right>}}

\renewcommand{\Re}{\operatorname{Re}}
\renewcommand{\Im}{\operatorname{Im}}

\newcommand{\Log}{\operatorname{Log}}

\newcommand{\grad}{\nabla}

\newcommand{\Matlab}{\textsc{Matlab}\xspace}
\newcommand{\Octave}{\textsc{Octave}\xspace}
\newcommand{\pylab}{\textsc{pylab}\xspace}

\newcommand{\prob}[1]{\bigskip\noindent\textbf{#1.} }
\newcommand{\pts}[1]{(\emph{#1 pts})}

\newcommand{\probpts}[2]{\prob{#1} \pts{#2} \quad}
\newcommand{\ppartpts}[2]{\textbf{(#1)} \pts{#2}}
\newcommand{\epartpts}[2]{\medskip\noindent \textbf{(#1)} \pts{#2}}

\lhead{\sc{Math 265 Proofs}}
\chead{\large \sc Midterm II} 
\rhead{\sc Spring 2022}
\cfoot{}
\pagestyle{fancy}
\begin{document}
\thispagestyle{fancy}


\medskip
\large
\vspace{.1in}
\begin{tabular}{l@{\hspace{.4in}}l}
Your Name & Your Signature\\
\framebox(200,30){} & \framebox(200,30){} \\
\end{tabular}

%\bigskip

\vfill
{
\renewcommand{\baselinestretch}{1.8}
\setlength{\tabcolsep}{.2in}
\normalsize
\begin{center}
\begin{tabular}{|c|c|c|}
\hline
Problem&Total Points&\parbox{.8in}{\hfil Score\hfil}\\
\hline
1&20&\\
\hline
2&10&\\
\hline
3&10&\\
\hline
4&10&\\
\hline
5&10&\\
\hline
extra credit&5&\\
\hline
\hline
%\hline
Total&600&\\
\hline
%Current Course Grade&\multicolumn{2}{c|  }{}\\
%\hline

\end{tabular}

\end{center}
}
\vfill
\begin{itemize}
\item 
You have 1 hour.

\item If you have a cell phone with you, it should be turned off and put away. (Not in your pocket)

\item You may not use a calculator, book, notes or aids of any kind.

\item In order to earn partial credit, you must show your work.

\end{itemize}
\newpage
\begin{enumerate}
%disprove
\item (20  points) Disprove the following two statements.
	\begin{enumerate}
	\item For all sets $A$, $B$ and $C,$ if $A \not\subseteq B$ and $B \not\subseteq C$, then $A \not\subseteq C.$
	\vfill
	\item There exists a natural number $n$ such that $3 \mid n$ and $3 \mid (n+1).$
	\vfill	
	\end{enumerate}
\newpage


%plain induction
\item (10 points) Prove that for all integers $n\geq 2,$ 
$$\left(1-\frac{1}{2^2} \right)\left( 1-\frac{1}{3^2} \right)\left( 1-\frac{1}{4^2} \right)\cdots\left(1-\frac{1}{n^2}  \right) =\frac{n+1}{2n}.$$
\vfill
\newpage
%iff
\item (10 points) Suppose $A$, $B$ and $C$ are sets. Prove that $A \subseteq B$ if and only if $A-B=\emptyset.$\\
(Hint: You may not want to use the method of direct proof here.)
\vfill
\newpage
%inequality induction
\item (10 points) Use induction to prove that for every integer $n$ such that $n\geq2,$ $5^n+9 < 6^n.$
%set containment
\vfill
\newpage
\item (10 points) Prove that for all sets $A$ and $B$, $\mathcal{P}(A) \cup \mathcal{P}(B) \subseteq \mathcal{P}(A \cup B).$ (Note $\mathcal{P}(A)$ is the power set of the set $A.$)
\end{enumerate}
\vfill
%extra credit
(5 points extra credit) Suppose $a,b \in \mathbb{N}.$ Then $a=lcm(a,b)$ if and only if $b \mid a.$
\vfill

\end{document}