\documentclass[12pt]{article}
\usepackage[margin=.8in]{geometry}
\usepackage{amsmath,amssymb,amsthm, latexsym, mathrsfs, pdfsync, 
fancybox, fancyhdr, 
graphicx, enumerate,
subfig, multicol}
\usepackage{tikz}
\usepackage{pgf}
\usepackage{pgfplots}
\usetikzlibrary{calc}

\newcommand{\blankbox}[2]{\fbox{\rule{#1}{0in}\rule{0in}{#2}}}
%special commands for number sets
\def\RR{{\mathbb R}}
\def\NN{{\mathbb N}}
\def\ZZ{{\mathbb Z}}
\def\QQ{{\mathbb Q}}
\def\CC{{\mathbb C}}
%special commands for formatting: center, enumerate, pmatrix, vector span
\def\bc{\begin{center}}
\def\ec{\end{center}}
\newcommand{\be}{\begin{enumerate}}
\newcommand{\ee}{\end{enumerate}}
\newcommand{\bpm}{\begin{pmatrix}}
\newcommand{\epm}{\end{pmatrix}}
\newcommand{\bv}[1]{\mathbf{#1}}
\newcommand{\spn}[1]{\text{Span}\left\{#1\right\}}
\newcommand{\lra}{\longrightarrow}
\newcommand{\llra}{\longleftrightarrow}

\setlength{\headheight}{30pt}
\setlength{\headsep}{20pt}
\setlength{\fboxsep}{8pt}
\setlength{\fboxrule}{1pt}

\lhead{\sc \quad \\ Math 265}
\chead{\sc Review: Midterm I } 
\rhead{\sc \quad \\ Spring 2022}
\cfoot{}
\pagestyle{fancy}
\pgfplotsset{compat=1.12}
\begin{document}
\thispagestyle{fancy}

Midterm I will be on Wednesday 23 February during our regular class time. So you will have 1 hour to complete the Midterm. Notes, books and other aid are not allowed.\\

\begin{center} Chapter 1: Sets\end{center}
Section 1
\begin{itemize}
\item Terms to know: element of a set, cardinality of a set, set builder notation, natural numbers, integers, rational numbers, real numbers, interval notation
\item Symbols to know $\NN,\: \ZZ,\: \QQ, \RR, \in$ and basic set notation
\item You should know how to go back and forth between different kinds of set notation.
\end{itemize}
Section 2
\begin{itemize}
\item terms to know: ordered pair, Cartesian product, ordered $n$-tuple
\item symbols to know: $A \times B$
\item You should know how to count the number of elements in $A \times B$ provided $A$ and $B$ are finite.
\item Know how to distinguish between $A \in B$ or $A \subseteq B.$
\end{itemize}
Section 3 and 4
\begin{itemize}
\item terms to know: subset, the power set of a set,
\item symbols to know: $\subseteq$, $\mathcal{P}(A)$
\item Know how to determine the cardinality of the power set of a finite set.
\end{itemize}
Section 5 and 6
\begin{itemize}
\item Know how to find the union, intersection and difference of two sets.
\item symbols to know: $\cup, \: \cap, \: -$ and $\overline{A}$.
\item Know how to find the complement of a set.
\end{itemize}
Section 7
\begin{itemize}
\item Know now to draw and to read a Venn diagram.
\end{itemize}
Section 8
\begin{itemize}
\item terms to know: indexed sets
\item notation to know: $\cup_{i \in I} A_i$, $\cap_{i \in I} A_i$
\end{itemize}
Section 9-10
\begin{itemize}
\item Know the Division Algorithm.
\item You will \emph{not} be asked about the Well-ordering Principle or Russell's Paradox.
\end{itemize}

\begin{center} Chapter 2: Logic \end{center}

\begin{itemize}
\item terms to know: statement, the mathematical meaning of \emph{and, or} and \emph{not}, truth table, conditional statement, biconditional, quantifiers
\item symbols to know: $\vee,\: \wedge,\: \sim,\: \Leftarrow, \: \Leftrightarrow, \: \forall, \: \exists$
\item You need to be familiar with \emph{alternate} formulations of these logical statements in English. (See especially the bottom of page 44.)
\item Know how to decide if a statement with \emph{and}, \emph{or} or \emph{not} is true or false.
\item Know how to decide if a conditional, biconditional, or quantified statement is true or false.
\item Know how to determine if two statements are logically equivalent or not.
\item Know DeMorgan's Laws (page 51).
\end{itemize}


\begin{center} Definitions \end{center}

For all of the terms below, you must be able to formally state the definition from your textbook. 
\be
\item odd, even, same parity, opposite parity
\item divides, multiple, divisor
\item prime
\item greatest common divisor, least common multiple
\item congruent modulo $n$
\item rational number, irrational number
\ee

\begin{center} Proof Techniques \end{center}
\be
\item direct proof
\item using cases
\item by contrapositive
\item by contradiction
\ee

\begin{center} Things to Keep in Mind \end{center}
\be
\item If a proof technique is not prescribed, you MUST state the method you are using.
\item You should put in the ``boiler-plate'' language even if you cannot figure out the whole proof.
\item You should expect to \emph{use} all of the hypotheses.
\item I will \emph{not} ask you to prove something that is false.
\ee
\end{document}