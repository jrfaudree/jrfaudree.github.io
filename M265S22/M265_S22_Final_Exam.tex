\documentclass[11pt]{article}
%{amsart}
%\pagestyle{empty} 
\setlength{\topmargin}{-0.8in} % usually -0.25in
\addtolength{\textheight}{1.4in} % usually 1.25in
\addtolength{\oddsidemargin}{-0.8in}
\addtolength{\evensidemargin}{-0.8in}
\addtolength{\textwidth}{1.6in} %\setlength{\parindent}{0pt}

\newcommand{\normalspacing}{\renewcommand{\baselinestretch}{1.1}\tiny\normalsize}
\normalspacing

% macros
\usepackage{amsmath,amssymb,mathrsfs,amsthm,xspace,alltt,verbatim,fancyhdr,mathtools}
\usepackage[final]{graphicx}
\usepackage[pdftex,colorlinks=true]{hyperref}
\usepackage{fancyvrb}
\usepackage{tikz}

\newtheorem*{lem*}{Lemma}


\newcommand{\be}{\begin{enumerate}}
 \newcommand{\ee}{\end{enumerate}}

\newcommand{\bF}{\mathbf{F}}
\newcommand{\bN}{\mathbf{N}}
\newcommand{\bT}{\mathbf{T}}

\newcommand{\CC}{{\mathbb{C}}}
\newcommand{\RR}{{\mathbb{R}}}
\newcommand{\eps}{\epsilon}
\newcommand{\ZZ}{{\mathbb{Z}}}
\newcommand{\QQ}{{\mathbb{Q}}}
\newcommand{\ZZn}{{\mathbb{Z}}_n}
\newcommand{\NN}{{\mathbb{N}}}
\newcommand{\ip}[2]{\mathrm{\left<#1,#2\right>}}

\renewcommand{\Re}{\operatorname{Re}}
\renewcommand{\Im}{\operatorname{Im}}

\newcommand{\Log}{\operatorname{Log}}

\newcommand{\grad}{\nabla}

\newcommand{\Matlab}{\textsc{Matlab}\xspace}
\newcommand{\Octave}{\textsc{Octave}\xspace}
\newcommand{\pylab}{\textsc{pylab}\xspace}

\newcommand{\prob}[1]{\bigskip\noindent\textbf{#1.} }
\newcommand{\pts}[1]{(\emph{#1 pts})}

\newcommand{\probpts}[2]{\prob{#1} \pts{#2} \quad}
\newcommand{\ppartpts}[2]{\textbf{(#1)} \pts{#2}}
\newcommand{\epartpts}[2]{\medskip\noindent \textbf{(#1)} \pts{#2}}

\lhead{\sc{Math 265 Proofs}}
\chead{\large \sc Final Exam} 
\rhead{\sc Spring 2022}
\cfoot{}
\pagestyle{fancy}
\begin{document}
\thispagestyle{fancy}


\medskip
\large
\vspace{.1in}
\begin{tabular}{l@{\hspace{.4in}}l}
Your Name & Your Signature\\
\framebox(200,30){} & \framebox(200,30){} \\
\end{tabular}

%\bigskip

\vfill
{
\renewcommand{\baselinestretch}{1.8}
\setlength{\tabcolsep}{.2in}
\normalsize
\begin{center}
\begin{tabular}{|c|c|c|}
\hline
Problem&Total Points&\parbox{.8in}{\hfil Score\hfil}\\
\hline
1&20&\\
\hline
2&12&\\
\hline
3&10&\\
\hline
4&10&\\
\hline
5&15&\\
\hline
6&15&\\
\hline
7&10&\\
\hline
8&8&\\
\hline
extra credit&5&\\
\hline
\hline
%\hline
Total&100&\\
\hline
%Current Course Grade&\multicolumn{2}{c|  }{}\\
%\hline

\end{tabular}

\end{center}
}
\vfill
\begin{itemize}
\item 
You have 2 hours.

\item If you have a cell phone with you, it should be turned off and put away. (Not in your pocket)

\item You may not use a calculator, book, notes or aids of any kind.

\item In order to earn partial credit, you must show your work.

\end{itemize}
\newpage
\begin{enumerate}
%miscellany
\item (20 points)
	\begin{enumerate}
	\item State the negation of each statement below.
		\begin{enumerate}
		\item If $n$ is divisible by 14, then $n$ is divisible by $2$ and $n$ is divisible by $7.$
		\vfill
		\item For every real number $r$ there exists a rational number $q$ such that $r < q < r+1.$
		\vfill
		\end{enumerate}
	\vfill
	\item Determine the truth value of the statements below.
		\begin{enumerate}
		\item $2 \in \mathcal{P}(\{0,1,2,3\})$
		\vfill
		\item $\{ \emptyset, \{0,1\}\} \subseteq  \mathcal{P}(\{0,1,2,3\})$
		\vfill
		\end{enumerate}
	\item List three different partitions of the set $S=\{1,2,3\}.$ Label your partitions $P_1, P_2,$ and $P_3.$ Use correct notation.
	\vfill 
	\item Let $R$ be an equivalence relation on $S=\{a,b,c,d\}$ such that $aRb$ and $dRa.$ Circle all of the following statements that \emph{must} also be true.
		\begin{enumerate}
		\item $cRc$
		\item $bRd$
		\item $d \in [a]$
		\end{enumerate}
	\end{enumerate}
\newpage
%plain induction
\item (12 points) Prove that $1^3+2^3+3^3 + \cdots + n^3=\frac{n^2(n+1)^2}{4}$ for all $n \in \NN.$
\vfill

\newpage
%proof by contrapositive
\item (10 points) Use the method of proof by contrapositive to prove the proposition below.\\
\begin{quote} Suppose $a,b \in \ZZ.$ If $(a+1)b^2$ is even, then $a$ is odd or $b$ is even.\end{quote}
\vfill

%proof by contradiciton
\item (10 points) Use the method of proof by contradiction to proof the proposition below.\\
\begin{quote} Suppose $a,b \in \RR.$ If $a$ is rational and $ab$ is irrational, then $b$ is irrational.\end{quote}

\vfill
\newpage
%injective/surjective
\item (15 points) Let the function $f: [0,\infty) \to [6, \infty)$ be defined as $f(x)=3x^2+6.$ Prove that $f$ is a bijection.
\vfill
\newpage
%symmetric, reflexive, transitive
\item (15 points) Let $R$ be a relation on $\mathbb{R}$ such that $xRy$ if $x-y\in \mathbb{Z}.$ 
	\begin{enumerate}
	\item 
	Prove that $R$ is an equivalence relation. 
	\vfill
	\item
       State three distinct elements in $[\pi],$ the equivalence class of $R$ containing $\pi.$
       
       \vspace{1in}
       \end{enumerate}

\newpage
%subset containment
\item (10 points) Let $A,B,$ and $C$ be sets. Suppose that $A \subseteq B,$ $B\subseteq C$, and $C \subseteq A.$ Prove that $A=B.$

\newpage
%sets have same cardinality
\item (8 points) Demonstrate that the sets $\{0,1\} \times \NN$ and $\ZZ$ have the same cardinality.
\vfill
\textbf{5 points extra credit} Prove that your answer above is correct.
\vfill	
\end{enumerate}
\vfill


\end{document}