\documentclass[12pt]{article}
\usepackage[margin=.8in]{geometry}
\usepackage{amsmath,amssymb,amsthm, latexsym, mathrsfs, pdfsync, 
fancybox, fancyhdr, 
graphicx, enumerate,
subfig, multicol}
\usepackage{tikz}
\usepackage{pgf}
\usepackage{pgfplots}
\usetikzlibrary{calc}

\newcommand{\blankbox}[2]{\fbox{\rule{#1}{0in}\rule{0in}{#2}}}
%special commands for number sets
\def\RR{{\mathbb R}}
\def\NN{{\mathbb N}}
\def\ZZ{{\mathbb Z}}
\def\QQ{{\mathbb Q}}
\def\CC{{\mathbb C}}
%special commands for formatting: center, enumerate, pmatrix, vector span
\def\bc{\begin{center}}
\def\ec{\end{center}}
\newcommand{\be}{\begin{enumerate}}
\newcommand{\ee}{\end{enumerate}}
\newcommand{\bpm}{\begin{pmatrix}}
\newcommand{\epm}{\end{pmatrix}}
\newcommand{\bv}[1]{\mathbf{#1}}
\newcommand{\spn}[1]{\text{Span}\left\{#1\right\}}
\newcommand{\lra}{\longrightarrow}
\newcommand{\llra}{\longleftrightarrow}

\setlength{\headheight}{30pt}
\setlength{\headsep}{20pt}
\setlength{\fboxsep}{8pt}
\setlength{\fboxrule}{1pt}

\lhead{\sc \quad \\ Math 265}
\chead{\sc Review: Final Exam } 
\rhead{\sc \quad \\ Spring 2022}
\cfoot{}
\pagestyle{fancy}
\pgfplotsset{compat=1.12}
\begin{document}
\thispagestyle{fancy}

\begin{center} Preliminaries \end{center}

The Final Exam will be given without the use of aids of any kind. You will have two hours to complete the test. The exam will cover Chapters 1-12 and Chapter14 Section 1. The majority of the problems will be proofs.  In some cases you will be asked to use a particular technique to prove a statement. In other instances it will be your choice what method to use. For all proofs you are expected to write complete, formal, appropriately detailed proofs.\\

Generally, a student will be awarded half the points for a proof problem for writing complete, correct boiler-plate language.\\

\begin{center} Step 1 \end{center}

Know the formal definitions for  the terms below. 
\be
\item element of a set, \textbf{cardinality of a set}  including for infinite sets, set builder notation, {natural numbers, integers, rational numbers, irrational numbers, real numbers}, interval notation
\item ordered pair, \textbf{Cartesian product, ordered $n$-tuple}
\item \textbf{subset, the power set of a set}
\item \textbf{union, intersection and difference of two sets, complement of a set}
\item a statement, the mathematical meaning of \emph{and, or} and \emph{not}, truth table, conditional statement, biconditional, quantifiers, logically equivalent, contrapositive, negation
\item \textbf{odd, even, same parity, opposite parity}
\item \textbf{divides, multiple, divisor}
\item \textbf{prime}
\item \textbf{greatest common divisor; least common multiple}
\item \textbf{congruent modulo $n$}
\item \textbf{rational number, irrational number}
\item \textbf{subsets, set equality}
\item \textbf{relation on $A$, reflexive, symmetric, transitive relations, a relation from $A$ to $B$, an equivalence relation, an equivalence class, a partition of a set}
\item \textbf{a function from $A$ to $B$, domain, codomain, range of $f$, injective, surjective, and bijective function, Pigeonhole Principle, identity function, inverse relation, inverse function, image, preimage}
\ee
\newpage

\begin{center} Step 2 \end{center}

Review the boiler-plate wording and standard strategies for proofs of the types below. Double-check in your text if needed. 

\be
\item direct proof
\item using cases
\item by contrapositive (As practice you can go to page 155-6 and re-write each conditional statement in its contrapositive form.)
\item by contradiction (As practice you can go to page 155-6 and write what would be the very first line on a proof by contradiction.)
\item if-and-only-if proofs
\item existence statements
\item proofs involving sets (that is, statements including $A \subseteq B$ or $A=B$)
\item How to disprove different kinds of statements.
\item mathematical induction (strong or weak)
%\item proof by smallest counterexample
\item how to show a relation is or isn't symmetric, reflexive, transitive, an equivalence relation, or a function.
\item how to identify distinct equivalence classes
\item how to prove a function is or isn't injective, surjective, or bijective.
%\item how to apply the Pigeonhole Principle
\item how to find an inverse relation
\item how to prove a function as an inverse and that two functions are or are not inverses
\item how to find the image or preimage of a set given a function
\item how to determine if two infinite sets do or do not have the same cardinality
%\item how to determine if a set is or is not countable
\ee
\newpage
\begin{center} Things to Keep in Mind \end{center}
\be
\item If a proof technique is not prescribed, you are probably going to be more successful if you state the method you are using.
\item You should put in the ``boiler-plate'' language even if you cannot figure out the whole proof.
\item You should expect to \emph{use} all of the hypotheses.
\item I will \emph{not} ask you to prove something that is false.
\item One of the cardinal sins in mathematics is using \emph{faulty logic}. Be vigilant against employing any of the approaches below.
	\be
	\item Assuming what it is you are supposed to be proving.
	\item Using the underlying argument: $p \rightarrow q$ is true and $q$ is true, so $p$ must be true.
	\item The statement holds for this particular example, thus we can conclude the statement holds in general.
	\item Using magic/intimidation to draw an otherwise unwarranted conclusion. Such sentences often begin with ``Clearly...''
	\ee 
\item Be skeptical of yourself! Ask yourself if you really \emph{believe} what you wrote down. This is a way of avoiding writing down an assertion that is obviously false.
\item Two answers with exactly the same amount of correct work will not earn the same grade if one of them also contains incorrect work.\ee


\end{document}