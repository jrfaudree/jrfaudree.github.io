\documentclass[11pt]{report}

\usepackage{geometry,amsmath,amssymb,amsthm}

\geometry{margin=1.in}

\theoremstyle{plain}
\newtheorem{thm}{Theorem}
\newtheorem{lem}[thm]{Lemma}
\newtheorem{prop}[thm]{Proposition}
\newtheorem{cor}[thm]{Corollary}


\begin{document}
\hfill Math 265

\hfill \today

\begin{center}
\Large{\textbf{Ch 8 \& Ch 9}} \\
\end{center}
\begin{enumerate}
\item Recall that we proved the following result in class:\\

Let $a,b \in \mathbb{N}.$ If $a \mid bc$ and $\gcd(a,b)=1,$ then $a \mid c.$\\

\item Let $p$ and $q$ be distinct prime numbers and let $c$ be an integer. Prove that if $p\mid qc$ then $p \mid c.$\\



\item Prove that the previous statement is false without the hypothesis that $p$ and $q$ are distinct prime numbers. \\



\item Prove one of DeMorgan's Laws:\\
Let $A$ and $B$ be sets with universe $U$. Prove $\overline{A \cup B} = \overline{A} \cap \overline{B}.$\\



\item We said in class that we can think of the statement \fbox{$A \subseteq B$} as equivalent to the statement \fbox{If $a \in A$, then $a \in B$.}\\

Write the contrapositive and the negation of the boxed statements.\\



\item Prove the proposition below using the contrapositive and the negation.\\

\textbf{Proposition:} $\{n \in \mathbb{Z} \: : \: 4\mid n \} \subseteq \{n \in \mathbb{Z} \: : \: 2\mid n\}.$\\


\end{enumerate}
\end{document}