\documentclass[11pt]{article}
%{amsart}
%\pagestyle{empty} 
\setlength{\topmargin}{-0.8in} % usually -0.25in
\addtolength{\textheight}{1.4in} % usually 1.25in
\addtolength{\oddsidemargin}{-0.8in}
\addtolength{\evensidemargin}{-0.8in}
\addtolength{\textwidth}{1.6in} %\setlength{\parindent}{0pt}

\newcommand{\normalspacing}{\renewcommand{\baselinestretch}{1.1}\tiny\normalsize}
\normalspacing

% macros
\usepackage{amsmath,amssymb,mathrsfs,amsthm,xspace,alltt,verbatim,fancyhdr,mathtools}
\usepackage[final]{graphicx}
\usepackage[pdftex,colorlinks=true]{hyperref}
\usepackage{fancyvrb}
\usepackage{tikz}

\newtheorem*{lem*}{Lemma}


\newcommand{\be}{\begin{enumerate}}
 \newcommand{\ee}{\end{enumerate}}

\newcommand{\bF}{\mathbf{F}}
\newcommand{\bN}{\mathbf{N}}
\newcommand{\bT}{\mathbf{T}}

\newcommand{\CC}{{\mathbb{C}}}
\newcommand{\RR}{{\mathbb{R}}}
\newcommand{\eps}{\epsilon}
\newcommand{\ZZ}{{\mathbb{Z}}}
\newcommand{\QQ}{{\mathbb{Q}}}
\newcommand{\ZZn}{{\mathbb{Z}}_n}
\newcommand{\NN}{{\mathbb{N}}}
\newcommand{\ip}[2]{\mathrm{\left<#1,#2\right>}}

\renewcommand{\Re}{\operatorname{Re}}
\renewcommand{\Im}{\operatorname{Im}}

\newcommand{\Log}{\operatorname{Log}}

\newcommand{\grad}{\nabla}

\newcommand{\Matlab}{\textsc{Matlab}\xspace}
\newcommand{\Octave}{\textsc{Octave}\xspace}
\newcommand{\pylab}{\textsc{pylab}\xspace}

\newcommand{\prob}[1]{\bigskip\noindent\textbf{#1.} }
\newcommand{\pts}[1]{(\emph{#1 pts})}

\newcommand{\probpts}[2]{\prob{#1} \pts{#2} \quad}
\newcommand{\ppartpts}[2]{\textbf{(#1)} \pts{#2}}
\newcommand{\epartpts}[2]{\medskip\noindent \textbf{(#1)} \pts{#2}}

\lhead{\sc{Math 265 Proofs}}
\chead{\large \sc Midterm I} 
\rhead{\sc Spring 2022}
\cfoot{}
\pagestyle{fancy}
\begin{document}
\thispagestyle{fancy}


\medskip
\large
\vspace{.1in}
\begin{tabular}{l@{\hspace{.4in}}l}
Your Name & Your Signature\\
\framebox(200,30){} & \framebox(200,30){} \\
\end{tabular}

%\bigskip

\vfill
{
\renewcommand{\baselinestretch}{1.8}
\setlength{\tabcolsep}{.2in}
\normalsize
\begin{center}
\begin{tabular}{|c|c|c|}
\hline
Problem&Total Points&\parbox{.8in}{\hfil Score\hfil}\\
\hline
1&15&\\
\hline
2&14&\\
\hline
3&14&\\
\hline
4&15&\\
\hline
5&10&\\
\hline
6&12&\\
\hline
7&10&\\
\hline
8&10&\\
\hline
\hline
%\hline
Total&100&\\
\hline
%Current Course Grade&\multicolumn{2}{c|  }{}\\
%\hline

\end{tabular}

\end{center}
}
\vfill
\begin{itemize}
\item 
You have 1 hour.

\item If you have a cell phone with you, it should be turned off and put away. (Not in your pocket)

\item You may not use a calculator, book, notes or aids of any kind.

\item In order to earn partial credit, you must show your work.

\end{itemize}
\newpage
\begin{enumerate}
%logical equivalence
\item (15 points)
	\begin{enumerate}
	\item Complete the definition below.\\
	
	Given integers $a$ and $b$ and $n \in \mathbb{N}$, we say that $a$ and $b$ are congruent modulo $n$ if\\
	\vspace{.5in}
	\item \textbf{Use the definition and a direct proof} to prove the statement below. Do not use any previous results from the text or in homework.\\
	\begin{quote} If $a \in \mathbb{Z}$ and $a \equiv 1 (\text{mod } 7)$, then $a^2  \equiv 1 (\text{mod } 7).$ \end{quote}
	\vfill
	\end{enumerate}
\newpage
\item (14 points)
	\begin{enumerate}
	\item List the elements in the set $\{x \in \mathbb{Z} \: : \: |3x| \leq 6 \}.$
	\vfill
	\item List the elements in the set $\{X \subseteq \{a,b,c\} \: : \: a \not \in X\}.$
	\vfill
	\item Write the set $\{ \cdots, \frac{- \pi }{4}, \frac{ -\pi }{2},0,\frac{ \pi }{4},\frac{ \pi }{2},\frac{3 \pi }{4},\pi, \cdots  \}$ in set-builder notation. 
	\vfill
	\item Determine the cardinality of the set $\{ \emptyset, \{\emptyset\}, \{ 1,2\}, \{1,2,3\}\}.$
	\vfill
	\end{enumerate}
\item (14 points) Let $A=\{0,1,2,3,4\}$ and $\mathcal{P}(A)$ denote the power set of $A.$
	\begin{enumerate}
	\item Determine $|\mathcal{P}(A)|,$  the cardinality of $\mathcal{P}(A)$.\\
	
	\item List 3 distinct \textbf{elements} of $\mathcal{P}(A)$ such that each element has a different cardinality. Use correct notation.
	\vfill
	\item List 3 distinct \textbf{subsets} of $\mathcal{P}(A)$ such that each subset has different cardinality. Use correct notation.
	\vfill
	\newpage
	\end{enumerate}
\item (15 points) Let $A=\{0,1,2\}, B=\{1,2,3,4\}$ and define the universal set $U=\{0,1,2,3, \cdots, 9\}$. Find:
	\begin{enumerate}
	\item $A \cup B$\\
	\vspace{.3in}
	
	\item $\overline{A \cup B}$\\ 
	\vspace{.3in}
	
	\item $ | A \times B |$\\ 
	\vspace{.3in}
	
	\item $ (A \times A) \cap (B \times B)$\\
	\vspace{.3in}
	
	\item $ (A \times A) - (A \times B)$\\
	\vspace{.3in}
	\end{enumerate}


\item (10 points) Complete the truth table for the statement  $P \Leftrightarrow (Q \vee \sim R).$\\

\begin{center}
\begin{tabular}{c|c|c|p{0.7\textwidth}}
P&Q&R&\\
\hline \hline
T&T&T&\\ \hline
T&T&F&\\ \hline
T&F&T&\\ \hline
T&F&F&\\ \hline
F&T&T&\\ \hline
F&T&F&\\ \hline
F&F&T&\\ \hline
F&F&F&\\ 
\end{tabular}
\end{center}
\newpage
\item (12 points) Negate the two statements below. Your answer should be a complete sentence in English. (You are not asked to determine the truth value of these statements.)
	\begin{enumerate}
	\item There exists a real number $r$ such that $r>1$ and $r^2 < 1.001.$
	\vfill
	\item If $a \in X,$ then $a \not \in Y-X.$
	\vfill
	\end{enumerate}
\item (10 points) Prove the statement below with a contrapositive proof.


\begin{quote} Let $x,y \in \mathbb{Z}.$ If $3x -5y$ is odd, then $x$ and $y$ do not have the same parity. \end{quote}
\vfill
\vspace{5in}
\newpage
\item (10 points) Prove the statement below using a proof by contradiction.

\begin{quote} Let $a,b \in \mathbb{Z}.$ If $4 \mid (a^2+b^2)$, then $a$ is even or $b$ is even. \end{quote}
\vfill
\end{enumerate}
\end{document}