%%% Preamble starts here.
\documentclass{math265}

% Include any special packages you might use.  Uncomment the
% following to use Times as the default font insteand of
% TeX's default font of Computer Modern.
% \usepackage{times,txfonts}

% The following commands set up the material that appears
% in the header.
\doclabel{Math 265: Sample Assignment}
\docauthor{Jill Faudree}
\docdate{February 12, 2019}

% If you want, you can make new commands, e.g. the following
% which uses \Ints to make a blackboard-bold Z.
%
% \newcommand\Ints{\mathbb{Z}}

% I've provided a file (math215extras.tex) with some commonly used extra 
% commands. If you've downloaded it, you can include it in your document
% by uncommenting the line below.  Feel free to make changes to that file.
%
% \input{math215extras.tex}

%%%% Main document starts here.

\begin{document}

\begin{exercise}{1.1} Suppose $p$ is even and $q$ is any integer.
  Then $pq$ is even.
\end{exercise}
\begin{pf}
Since $p$ is even, there exists an integer $k$ such that
$p=2k$.  Hence
\begin{align*}
  pq &= (2k)q\\
     &= 2(kq).
\end{align*}
Since $kq\in\mathbb{Z}$, we conclude that $2 \mid p$ and hence $p$ is even.
\end{pf}

\begin{exercise}{1.2}\label{ex:oddtimesodd} 
  Suppose $p$ is odd and $q$ is odd.
  Then $pq$ is odd.
\end{exercise}
\begin{pf}
Suppose that $pq$ is even.  Then $2\mid pq$.  Since $2$ is prime,
either $2\mid p$ or $2\mid q$.  Suppose $2\mid p$.  Then $p=2k$ for some
integer $k$ and hence $p$ is even.  On the other hand, a
similar argument show that if $2\mid q$ then $q$ is even.  

We have therefore shown that if $pq$ is even, then either $p$ is
even or $q$ is even.  By the contrapositive,  it follows that if
if $p$ and $q$ are odd, then $pq$ is odd.
\end{pf}

\begin{exercise}{1.3} The square root of 2 is irrational.
\end{exercise}
\begin{pf}
Suppose to the contrary that $\sqrt{2}$ is rational. Then
there are integers $a$ and $b$ with no common factors such that 
\begin{equation*}
\sqrt{2} = \frac{a}{b}.
\end{equation*}
Squaring this equation we find that
\begin{equation}\label{step1}
2 b^2 = a^2.
\end{equation}
Hence $2|a^2$ and therefore $a^2$ is even.  
If $a$ were odd, then Exercise \ref{ex:oddtimesodd} would imply
that $a^2$ is even. But $a^2$ is even and hence $a$ is even.
So $a=2k$ for some integer $k$.  It follows from equation
\eqref{step1} that 
\begin{align*}
  2 b^2 &= (2k)^2\\
        &= 4 k^2.
\end{align*}
Hence
\begin{equation*}
b^2 = 2 k^2.
\end{equation*}
Arguing as before we see that $b$ must also be even.  So $2$ is a
common factor of $a$ and $b$, which is a contradiction.
\end{pf}

\end{document}
