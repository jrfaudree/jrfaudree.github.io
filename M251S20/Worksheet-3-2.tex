\documentclass[11pt,fleqn]{article} 
\usepackage[margin=0.8in, head=0.8in]{geometry} 
\usepackage{amsmath, amssymb, amsthm}
\usepackage{fancyhdr} 
\usepackage{palatino, url, multicol}
\usepackage{graphicx} 
\usepackage[all]{xy}
\usepackage{polynom} 
\usepackage{pdfsync}
\usepackage{enumerate}
\usepackage{framed}
\usepackage{setspace, adjustbox}
\usepackage{array%,tikz, pgfplots
}

\usepackage{tikz, pgfplots}
\usetikzlibrary{calc}
%\pgfplotsset{my style/.append style={axis x line=middle, axis y line=
%middle, xlabel={$x$}, ylabel={$y$}, axis equal }}
%
\pagestyle{fancy} 
\lfoot{UAF Calculus I}
\rfoot{3-2 Product and Quotient Rule}


\newcommand{\be}{\begin{enumerate}}
\newcommand{\ee}{\end{enumerate}}

\newcommand{\bi}{\begin{itemize}}
\newcommand{\ei}{\end{itemize}}

\begin{document}
\setlength{\parindent}{0cm}
\renewcommand{\headrulewidth}{0pt}
\newcommand{\blank}[1]{\rule{#1}{0.75pt}}
\renewcommand{\d}{\displaystyle}
\vspace*{-0.7in}
\begin{center}
 {\large{ \sc{Section 3.2 Product Rule and Quotient Rule}}}
\end{center}
\begin{enumerate}

\item Complete \textbf{The Product Rule:} If $f$ and $g$ are differentiable, then 
\begin{quote}  $\d{\frac{d}{dx} \left[f(x) g(x)]\right] =} $ \end{quote}
\vspace{.2in}

\item Complete \textbf{The Quotient Rule:}  If $f$ and $g$ are differentiable, then
\begin{quote} 
$\frac{d}{dx} \left[ \frac{f(x)}{g(x)} \right] = $ \end{quote}
\vspace{.2in}
\item Find the derivatives for each function below. \emph{Do not use the Product Rule or the Quotient Rule  if you don't have to!}
	\begin{enumerate}
	\item $\d{f(x)=5x^3e^x}$
	\vfill
	\item $\d{f(x)=\frac{2x^2-5}{4-x}}$
	\vfill
	\item $\d{f(x)=(1-x^2)(e^x+x)}$
	\vfill
	\item $\d{g(x)=\frac{\sqrt{x}}{8}(1-x\sqrt{x})}$
	\vfill
	
	\newpage
	\item $\d{h(x)=\frac{10x-x^{3/2}}{4x^2}}$
	\vfill
	\item $\d{y=\frac{\sqrt[3]{x}}{2x+1}}$
	\vfill
	\item $\d{v(t)=\frac{2te^t}{t^2+1}}$
	\vfill
	\end{enumerate}
\item   The graphs of $f(x)$ (shown thick) and the graphs of $g(x)$ (shown dashed) are shown below. If $h(x) = f(x)g(x)$, find $h'(0)$.

  \begin{flushleft}
   \begin{tikzpicture}[scale=.7]
\draw[ultra thin] (-3.5,-3.1) grid (6.2,3.1);
\draw[thick,<->] (-3.2,0) -- (3.2,0);
\draw[thick,<->] (0,-3.1) -- (0,3.1);
%%%%% f(x) %%%%
\draw [ultra thick, <->] (-3,1) to  (3,3)  to (6,-1);
\draw (6.2, -1) node[right] {$f(x)$};
%%%%% g(x) %%%%%
\draw[ultra thick, dashed, <->] (-3,2) to (1,-2) to (6, -2);
\draw (6.2,-2) node[right] {$g(x)$};
%%grid marks
\foreach \i in {-3,-2, ..., 6}
{\draw (\i,0) node[below] {\i};
%\draw (\i+.5, -.1) -- (\i+.5, .1);
}
\foreach \i in { -3, -2, -1,1,2, 3}
{\draw (0,\i) node[left] {\i};}
%\foreach \i in { -2, -1,0,1,2}{
%\draw (-.1, \i+.5) -- (.1,\i+.5);}
\end{tikzpicture}

  \end{flushleft}
  
\item Suppose that $f(5) = 1$, $f'(5) = 6$, $g(5) = -3$
and $g'(5) = 2$. Find the following values. 


  \begin{multicols}{3}{
      % make sure you added \usepackage{enumerate}
      \vspace*{-0.45in}
      \begin{enumerate}[(a)]
      \item $(f - g)'(5)$
      \item $(fg)'(5)$
      \item $(g/f)'(5)$
      \end{enumerate}}
  \end{multicols}
\vskip1in
\end{enumerate}
\end{document}