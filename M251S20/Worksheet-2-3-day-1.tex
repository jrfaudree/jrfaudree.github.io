\documentclass[11pt,fleqn]{article} 
\usepackage[margin=0.8in, head=0.8in]{geometry} 
\usepackage{amsmath, amssymb, amsthm}
\usepackage{fancyhdr} 
\usepackage{palatino, url, multicol}
\usepackage{graphicx,latexsym} 
\usepackage[all]{xy}
\usepackage{polynom} 
\usepackage{pdfsync}
\usepackage{enumerate, enumitem}
\usepackage{framed}
\usepackage{setspace}
\usepackage{array,tikz,xcolor, pgfplots}
\pagestyle{fancy} 
\lfoot{UAF Calculus 1}
\rfoot{\S 2.2 }

\usetikzlibrary{calc}
\pgfplotsset{compat=1.6}

\pgfplotsset{soldot/.style={color=black,only marks,mark=*}} \pgfplotsset{holdot/.style={color=black,fill=white,only marks,mark=*}}

\begin{document}
\setlength{\parindent}{0cm}
\renewcommand{\headrulewidth}{0pt}
\newcommand{\blank}[1]{\rule{#1}{0.75pt}}
\renewcommand{\d}{\displaystyle}


\vspace*{-1in}
\begin{center}
   \sc{Worksheet: \S 2.3 }
\end{center}
\begin{enumerate}
\item Fill in the blanks below. Assume $a$ and $c$ are fixed constants. (Note that these are all in your text but not in this order.)\textbf{Assume $\d \lim_{x \to a} f(x) $ and $\d \lim_{x \to a} g(x) $ exist. }
	\begin{enumerate}
	
	\item $\d \lim_{x \to a} c = \underline{\hspace{2in}} $ \\ \vfill
	
	\item $\d \lim_{x \to a} x = \underline{\hspace{2in}} $ \\ \vfill

	\item $\d \lim_{x \to a} \left(f(x) + g(x) \right) = \underline{\hspace{2in}} $ \\ \vfill
	
	\begin{enumerate} \item What do the rules above imply about $\d \lim_{x \to 12} (x+\pi)$? \end{enumerate} 
	 \vfill

	 \item $\d \lim_{x \to a} \left(f(x) - g(x) \right) = \underline{\hspace{2in}}$ \\ \vfill
	 
	  \item $\d \lim_{x \to a} cf(x) = \underline{\hspace{2in}}$ \\ \vfill
	  
	  \begin{enumerate} \item What do the rules above imply about $\d \lim_{x \to 5} 2x+3$? \end{enumerate}  \vfill
	  
	  \item $\d \lim_{x \to a} f(x)g(x)= \underline{\hspace{2in}}$ \\ \vfill
	  
	  \item $\d \lim_{x \to a}x^n= \underline{\hspace{2in}}$ \\ \vfill
	  
	  \item $\d \lim_{x \to a} \left(f(x)\right)^n= \underline{\hspace{2in}}$ \\ \vfill
	  
	  \item $\d \lim_{x \to a} \frac{f(x)}{g(x)}= \underline{\hspace{2in}}$  provided \underline{\hspace{2in}} \\ \vfill
	  
	  \item $\d \lim_{x \to a} \sqrt[n]{x}= \underline{\hspace{2in}}$ \\ \vfill
	  
	  \item $\d \lim_{x \to a} \sqrt[n]{f(x)}= \underline{\hspace{2in}}$ \\ \vfill

	\end{enumerate}
\newpage

\item If $\d \lim_{x \to \sqrt{2}} f(x) = 8 $ and $\d \lim_{x \to \sqrt{2}} g(x) = e^2$, then evaluate $$\d \lim_{x \to \sqrt{2}} \left(\frac{g(x)}{(3-f(x))^2}+2\sqrt{g(x)}\right) \hspace{2in} \quad$$

\vfill

\item Use the previous rules to evaluate (a) and explain why you \emph{cannot} use the rules to evaluate (b).\\
	\begin{enumerate}
	\item $\d \lim_{w \to -\frac{1}{2}} \frac{2w+1}{w^3}$
	\vfill
	\item $\d \lim_{t \to 1} \frac{t^2+t-2}{t^2-1}$
	\vfill
	\end{enumerate}
	
\item (One more super-useful rule!) If \hspace{2in}, then $\d \lim_{x \to a} f(x) = \lim_{x \to a} g(x) $
\vfill
\item Use this rule \emph{and what you know about zeros of polynomials} to evaluate\\

\noindent $\d \lim_{t \to 1} \frac{t^2+t-2}{t^2-1}$


\end{enumerate}
\end{document}