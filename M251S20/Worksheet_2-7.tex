\documentclass[11pt,fleqn]{article} 
\usepackage[margin=0.8in, head=0.8in]{geometry} 
\usepackage{amsmath, amssymb, amsthm}
\usepackage{fancyhdr} 
\usepackage{palatino, url, multicol}
\usepackage{graphicx} 
\usepackage[all]{xy}
\usepackage{polynom} 
%\usepackage{pdfsync}
\usepackage{enumerate}
\usepackage{framed}
\usepackage{setspace}
\usepackage{array,tikz,pgfplots}

\pgfplotsset{compat=1.6}

\pgfplotsset{soldot/.style={color=black,only marks,mark=*}} \pgfplotsset{holdot/.style={color=black,fill=white,only marks,mark=*}}

\pagestyle{fancy} 
\lfoot{UAF Calculus 1}
\rfoot{2-7}

\begin{document}
\setlength{\parindent}{0cm}
\renewcommand{\headrulewidth}{0pt}
\newcommand{\blank}[1]{\rule{#1}{0.75pt}}
\renewcommand{\d}{\displaystyle}
\vspace*{-0.9in}
\begin{center}
  \Large \sc{Section 2-7}
\end{center}
\small
\begin{enumerate}
\item Find the slope of the tangent line to $f(x)=3x^2$ at $x=a$ by taking the limit of the slopes of secant lines. When you are done, check whether or not your solutions seems plausible!\\

\vfill

\item Write the equation of the line tangent to the graph of $f(x)=3x^2$ at $x=\frac{1}{2}.$\\
\vfill

\item The derivative of a function $f$ at $x=a$ is

\vspace{.5in}


\item Assume the tangent line to the graph of $y=f(x)$ at $x=\sqrt{2}$ is $y=\frac{4x-1}{3}.$ Determine:
	\begin{enumerate}
	\item $f(\sqrt{2})$\\
	\vspace{1in}
	\item $f'(\sqrt{2})$\\
	\vspace{.5in}
	\end{enumerate}

\newpage
\item The height in meters of an object is given by the function $s(t)=\frac{2t}{t+1}$ where $t$ is measured in seconds. 
	\begin{enumerate}
	\item Find $s'(a)$ using the definition in \# 3 on this sheet. 
	\item Determine the units of $s'(a).$
	\item Find and interpret in the context of the problem the meaning of $s'(1).$
	\end{enumerate}
	\vfill
\item Let $f(x)=\sqrt{90-x}$
	\begin{enumerate}
	\item Find $f'(a)$ using the definition in \# 3 on this sheet. 
	\item If $f$ is measured in degrees celsius and $x$ is measured in minutes, determine the units of $f'(a).$
	\item Find and interpret $s'(0).$
	\end{enumerate}
	\vfill
\end{enumerate}
 \end{document}