\documentclass[11pt,fleqn]{article} 
\usepackage[margin=0.8in, head=0.8in]{geometry} 
\usepackage{amsmath, amssymb, amsthm}
\usepackage{fancyhdr} 
\usepackage{palatino, url, multicol}
\usepackage{graphicx,latexsym} 
\usepackage[all]{xy}
\usepackage{polynom} 
\usepackage{pdfsync}
\usepackage{enumerate, enumitem}
\usepackage{framed}
\usepackage{setspace}
\usepackage{array,tikz,xcolor, pgfplots}
\pagestyle{fancy} 
\lfoot{UAF Calculus 1}
\rfoot{\S Review Week - day 4 }

\usetikzlibrary{calc}
\pgfplotsset{my style/.append style={axis x line=middle, axis y line=
middle, xlabel={$x$}, ylabel={$y$}, axis equal }}

\begin{document}
\renewcommand{\headrulewidth}{0pt}
\newcommand{\blank}[1]{\rule{#1}{0.75pt}}
\renewcommand{\d}{\displaystyle}


\vspace*{-0.7in}
\begin{center}
  \large \sc{Review Day 4: Inverse Function, Exponential Functions, \& Logarithmic Functions }
\end{center}
\begin{enumerate}
\item In your own words, explain what it means for $f^{-1}(x)$  to be the \emph{inverse} of $f(x)$? You might try explaining it using graphs, algebra, or numerical calculations.
\vfill
\item Without doing a bunch of algebra, find $f^{-1}(x)$ for each function below:
\begin{multicols}{2}
\begin{enumerate}
	\item $f(x)=2x$
	\item $f(x)= x^3$
	\end{enumerate}
	\end{multicols}
	\vfill
\item Without explicitly finding a formula for $f^{-1}(x)$, find $f^{-1}(1)$ for each function below:\\
\begin{multicols}{3}
\begin{enumerate}
	\item $f(x)=x-20$
	\item \begin{tabular}{|c||c|c|c|c|c|c|c|c|c|}
$x$&0&0.25&0.5&0.75&1&1.25&1.5&1.75&2.0\\
\hline
$f(x)$&20&10&5&3&2.5&2&1.5&1&0.25\\
\end{tabular}

	\end{enumerate}
	\end{multicols}
	\vfill
\item Explain why the directions ``Find $f^{-1}(1)$" don't make sense for the following examples:

\begin{multicols}{3}
\begin{enumerate}
	\item $f(x)=x^2-3$
	\item \begin{tabular}{|c||c|c|c|c|c|c|c|c|c|}
$x$&0&1&2&3&4&5&6&7&8\\
\hline
$f(x)$&-3&1&5&8&6&2&3&1&0\\
\end{tabular}

	\end{enumerate}
	\end{multicols}
	\vfill
\newpage
\item Give a not-too-big rough sketch of $f(x)=\sin x$ and ask yourself whether or not it makes since to be asked to find $\sin^{-1}(1)$. (Recall that $\sin^{-1}(1)$ could be written $\arcsin(1)$ or $\text{invsin}(1).$)\\

\tikz{ \draw[<->] (-4,0) -- (4,0); \draw [<->] (0,-2) -- (0,2);}
\vfill


\item Evaluate the following:

	\begin{enumerate}
	\begin{multicols}{2}
	\item $\arcsin(1)$
	\item $\arccos(-\sqrt{3}/2)$
	\end{multicols}
	\vspace{1in}
	\begin{multicols}{2}
	\item $\arctan(1)$
	\item $\arcsin(10)$
	\end{multicols}
	\vspace{1in}
	\end{enumerate}

\newpage
\begin{center} Exponential Functions \& Logarithms \end{center}

\item On the axes below, sketch:
\begin{enumerate}
\begin{multicols}
\item $y=e^x$ and $y=2^x$\\

\tikz{ \draw[<->] (-3,0) -- (2,0); \draw [<->] (0,-1) -- (0,2);}\\
\columnbreak


\item $y=\ln x$ and $y=\log_2(x)$\\

\tikz{ \draw[<->] (-2,0) -- (4,0); \draw [<->] (0,-2) -- (0,2);}\\
\end{multicols}
\end{enumerate}


	
\vfill
\item Find the exact value of each expression.
\begin{multicols}{2}
\begin{enumerate}
	\item $\log_2 16$
	\item $e^{\ln 5}$
	\end{enumerate}
	\end{multicols}
	\vspace{1in}

\newpage
\item Solve each equation below for $x$.
\begin{multicols}{2}
\begin{enumerate}
	\item $10=2e^{x+1}$
	\item $\ln (x^2-1)=1$
	\end{enumerate}
	\end{multicols}
	\vfill
\item Sketch each function. Include domain, range, intercepts and asymptotes.\\
\begin{multicols}{2}
\begin{enumerate}
	\item $f(x)=\ln(x+1)$
	
	\tikz{ \draw[<->] (-4,0) -- (4,0); \draw [<->] (0,-4) -- (0,4);}
	
	\item $f(x)=- \ln x$
	
	\tikz{ \draw[<->] (-4,0) -- (4,0); \draw [<->] (0,-4) -- (0,4);}
	
	\end{enumerate}
	\end{multicols}
	\vfill

\end{enumerate}
\end{document}