
\documentclass[11pt,fleqn]{article} 
\usepackage[margin=0.8in, head=0.8in]{geometry} 
\usepackage{amsmath, amssymb, amsthm}
\usepackage{fancyhdr} 
\usepackage{palatino, url, multicol}
\usepackage{graphicx, pgfplots} 
\usepackage[all]{xy}
\usepackage{polynom} 
%\usepackage{pdfsync} %% I don't know why this messes up tabular column widths
\usepackage{enumerate}
\usepackage{framed}
\usepackage{setspace}
\usepackage{array,tikz}

\pgfplotsset{compat=1.6}

\pgfplotsset{soldot/.style={color=black,only marks,mark=*}} \pgfplotsset{holdot/.style={color=black,fill=white,only marks,mark=*}}
\pgfplotsset{my style/.append style={axis x line=middle, axis y line=
middle, xlabel={$x$}, ylabel={$y$} }}

%axis equal 
\pagestyle{fancy} 
\lfoot{}
\rfoot{Review Midterm II}

\begin{document}
\renewcommand{\headrulewidth}{0pt}
\newcommand{\blank}[1]{\rule{#1}{0.75pt}}
\newcommand{\bc}{\begin{center}}
\newcommand{\ec}{\end{center}}
\renewcommand{\d}{\displaystyle}

\vspace*{-0.7in}

%%%%%%%%%intro page
\begin{center}
  \large
  \sc{Review for Midterm II}\\
\end{center}
\noindent\textbf{Logistics}\\

You will have 90 minutes to take the midterm. \\

All you need to take the exam is a writing utensil and a $3 \times 5$ notecard. Scratch paper or extra paper will be provided for you, if needed. There will be no problems the require (or need) a calculator. \\

\noindent\textbf{Taking a Math Test}

\begin{itemize}
\item Show your work. Use Calculus.
\item Be organized.
\item Check your answers for \emph{plausibility}.
\end{itemize}


\noindent\textbf{Topics}\\

\noindent Since Midterm 1, we have covered several topics from the end of Chapter 3. But all of these topics were about the mechanics of taking derivatives and were tested explicitly on the Derivative Proficiency. Thus, there will not be any problems that simply ask you to take a derivative, however many problems will require that you take a derivative in order to work the problem so you can't just forget all of that material.\\ 

\noindent \fbox{Section 4.1}\\
Related Rate Problems. All of these problems are word problems asking for a rate of change of some quantity with respect to time. 

Example:  An airplane is flying overhead at a constant elevation of 4000 feet as it passes directly over a man standing on the ground. If the plane is flying at a speed of 600 feet per second, how fast is the plane moving away from the man 5 seconds after it passes over his head? Assume the plane is flying in a straight line.\\

\noindent \fbox{Section 4.2}\\
Linear Approximations and Differentials. These problems ask you to find the linear approximation of or differential of a function for particular values and then use these things (the linear approximation or differential) to estimate other things.

Example:  Find the linear approximation of $f(x)=5 \sin (x)$ when $a=0$ and use it to estimate $5 \sin (-0.1)$

Example:  Find the differential of $f(x)=4 \sqrt{x}$ when $x=9$ and use it to estimate  how much $f$ will change if $x$ changes from $9$ to $9.01$\\



\noindent \fbox{Section 4.3}\\
Maxima and minima. Absolute and local. Critical points. These problems are of two types: Finding ABSOLUTE extrema on closed-bounded intervals and finding local extrema in general.

Example: Find the absolute maximum and the absolute minimum of $f(x)=x^2-3x^{2/3}$ on $[0,8].$

Example: Identify any local extrema of $y=x^2- \frac{1}{x^2}.$

Example: Given $f(x)=x^2e^e$, $f'(x)=x(x+1)e^x$, and $f''(x)=(2x^2+4x+1)e^{2x}$. Determine intervals of increase and decrease, the locations and values of all local extrema, intervals of concavity and inflection points, and any asymptotes. Note $2x^2+4x+1=0$ has solutions $x=\frac{-2-\sqrt{2}}{2}$ and $x=\frac{-2+\sqrt{2}}{2}.$\\

\noindent \fbox{Section 4.5}\\
Derivatives and the Shape of a Graph. These problems ask you to use $f'$ and $f''$ to determine when the original function, $f$, is increasing or decreasing, concave up or concave down, has extrema, has inflection points, and to draw sophisticated graphs.

Example: Draw some not-too-complicated graph. Now assume it is $f'$. What can you say about the graph of $f$?

Example: If $f' >0$ for $x>0$, $f' <0$ for $x<0$, $f'' > 0$ for $-2 \leq x \leq 2$ and $f'' < 0$ for $x<-2$ and for $x>2,$ sketch $f.$\\

\noindent \fbox{Section 4.6}\\
Limits at Infinity and Asymptotes. The problems either ask you to evaluate a limit as $x \to \pm \infty$ or ask to find and justify the existence of a  horizontal asymptote.

Example: Determine if the graph of $f(x)=\frac{3x^3-e^x}{2x^3}$ has a horizontal asymptote. Justify your answer.\\

\noindent \fbox{Section 4.7}\\
Optimization Problems. These are word problems where you are asked to maximize or minimize some quantity. Crucial steps here include\\
 (a) identify the quantity to be maximized/minimized, \\
 (b) write the quantity from part (a) as a function of one variable,\\
 (c) identify the domain of the function from part (b), \\
 (d) take derivative and find critical points for function from part (b), \\
 (e) check/justify that your cp actually corresponds to a max/min,\\
 (f) answer the question.
 
 Example: Go work problems from old midterms.\\ 

\noindent \fbox{Section 4.8}\\
L'Hopital's Rule. Be able to use L'Hopital's Rule to evaluate limits of a variety of indeterminate forms.

Example: Evaluate $\lim_{x \to 0^+} x \ln(x^4)$\\

\noindent \fbox{Section 4.10}\\
Antiderivatives and Initial Value Problems 

Example: Evaluate $\int (\frac{3}{sqrt{x}}-\csc^2(x)) \: dx$

Example: If an object as acceleration $a(t) = x + \sin(x),$ find its velocity equation assuming $v(0)=10.$\\


\noindent \fbox{Section 5.1}\\
Approximating areas. Use rectangles with left- or right-hand endpoints to estimate the area under a curve. 

Example: Use $M_3$ (i.e. three rectangles with midpoints to determine heights) to estimate the area under $y=\sqrt{x}$ on the interval $[0,6].$ No need to get a decimal approximation. \\


\noindent \fbox{Section 5.2}\\
The Definite Integral as Signed Area under a Curve.\\

Example: Sketch the graph of $y=10 - 5x.$ Use this graph to evaluate $\int_{1}^6 (10-5x) \:dx$.

\end{document}

