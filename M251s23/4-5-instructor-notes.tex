\documentclass[11pt,fleqn]{article} 
\usepackage[margin=0.8in, head=0.8in]{geometry} 
\usepackage{amsmath, amssymb, amsthm}
\usepackage{fancyhdr} 
\usepackage{palatino, url, multicol}
\usepackage{graphicx} 
\usepackage[all]{xy}
\usepackage{polynom} 
\usepackage{pdfsync}
\usepackage{enumerate}
\usepackage{framed}
\usepackage{setspace, adjustbox}
\usepackage{array%,tikz, pgfplots
}

\usepackage{tikz, pgfplots}
\usetikzlibrary{calc}
%\pgfplotsset{my style/.append style={axis x line=middle, axis y line=
%middle, xlabel={$x$}, ylabel={$y$}, axis equal }}
%
\pagestyle{fancy} 
\lfoot{UAF Calculus I}
\rfoot{4-3}


\newcommand{\be}{\begin{enumerate}}
\newcommand{\ee}{\end{enumerate}}

\newcommand{\bi}{\begin{itemize}}
\newcommand{\ei}{\end{itemize}}

\begin{document}
\setlength{\parindent}{0cm}
\renewcommand{\headrulewidth}{0pt}
\newcommand{\blank}[1]{\rule{#1}{0.75pt}}
\renewcommand{\d}{\displaystyle}
\vspace*{-0.7in}
\begin{center}
 {\large{ \sc{Section 4.5: Increasing/Decreasing, Concavity}}}
 instructor notes
\end{center}
 
There are two Section 4.5 Worksheets.\\
\vfill

Day 1: This goes over how to use $f'$ to determine when a graph is increasing or decreasing and $f''$ for concave up and concave down. \\

This is built as a back and forth. \\
Parts 1,2 and 3 the teacher goes over, followed by having students do 4 individually/groups.\\

Parts 5,6,7 the teacher goes over followed by students doing 8 individually/groups.\\

If there is more time you can\\

 (i) start to foreshadow Day 2 and the second derivative test for extrema OR \\
 
 (ii) give students another problem to work independently, say $y=x-3x^{2/3}$. Have students find incr/decr/ccup/ccdown.\\
 
 OR
 
 (iii) Help students relate 4.5 to 4.3. In 4.3 we don't have a way of systematically determining local extrema.
 
 \vfill
 
 Day 2: Will want to begin by reviewing topics from 4.5 Day 1. \\
 
 Teacher goes over parts 1 and 2. Students do part 3. When done with \#3 talk with students about the value and limitations of the second derivative test for extrema (eg. if all you want to know is what happens at $x=2$, $f''$ makes it easy, but it's useless for $x=0.$\\
 
 Problems 4 and 5 are in groups.\\
 
 

\end{document}