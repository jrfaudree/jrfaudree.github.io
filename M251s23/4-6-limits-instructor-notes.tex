\documentclass[11pt,fleqn]{article} 
\usepackage[margin=0.8in, head=0.8in]{geometry} 
\usepackage{amsmath, amssymb, amsthm}
\usepackage{fancyhdr} 
\usepackage{palatino, url, multicol}
\usepackage{graphicx} 
\usepackage[all]{xy}
\usepackage{polynom} 
\usepackage{pdfsync}
\usepackage{enumerate}
\usepackage{framed}
\usepackage{setspace, adjustbox}
\usepackage{array%,tikz, pgfplots
}

\usepackage{tikz, pgfplots}
\usetikzlibrary{calc}
\pgfplotsset{my style/.append style={axis x line=middle, axis y line=
middle, xlabel={$x$}, ylabel={$y$}, axis equal }}
%
\pagestyle{fancy} 
\lfoot{UAF Calculus I}
\rfoot{4-6}


\newcommand{\be}{\begin{enumerate}}
\newcommand{\ee}{\end{enumerate}}

\newcommand{\bi}{\begin{itemize}}
\newcommand{\ei}{\end{itemize}}

\begin{document}
\setlength{\parindent}{0cm}
\renewcommand{\headrulewidth}{0pt}
\newcommand{\blank}[1]{\rule{#1}{0.75pt}}
\renewcommand{\d}{\displaystyle}
\vspace*{-0.7in}
\begin{center}
 {\large{ \sc{Section 4.6: Limits at Infinity and Asymptotes }}}
 (and sophisticated graphing)
\end{center}

There are two sheets for this section intended to fill 1.5 class hours. \\

\vfill

Day 1:\\
This introduces limits at infinite, algebraic strategies for evaluating them, and their relationship to horizontal asymptotes. \\
The last problems on this page introduces the idea of sophisticated graphing. That is, how to collect all the information you can from limits, first and second derivatives and put them together into a nuanced graph. \\

Problems 1,2 and 3a, I would do at the board with class input. Problems 3bcd, 4, 5 students should do in groups.\\

I would do Problem 6 together at the end. Even with an instructor to keep people from getting too far off track it will take a minimum of 15 minutes.\\

\vfill

Day 2:\\

Students practice sophisticated graphing on their own. Ideally, they are at the board!

\vfill
 \end{document}