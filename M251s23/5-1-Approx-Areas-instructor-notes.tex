\documentclass[11pt,fleqn]{article} 
\usepackage[margin=0.8in, head=0.8in]{geometry} 
\usepackage{amsmath, amssymb, amsthm}
\usepackage{fancyhdr} 
\usepackage{palatino, url, multicol}
\usepackage{graphicx, pgfplots} 
\usepackage[all]{xy}
\usepackage{polynom,tabularx} 
%\usepackage{pdfsync} %% I don't know why this messes up tabular column widths
\usepackage{enumerate}
\usepackage{framed}
\usepackage{setspace}
\usepackage{array}
\usepackage{pgf,tikz}
\usepackage{mathrsfs}

\usepackage[parfill]{parskip}
\usetikzlibrary{arrows}

\usetikzlibrary{calc}

\pgfplotsset{compat=1.6}

\pgfplotsset{soldot/.style={color=blue,only marks,mark=*}} \pgfplotsset{holdot/.style={color=blue,fill=white,only marks,mark=*}}

\renewcommand{\headrulewidth}{0pt}
\newcommand{\blank}[1]{\rule{#1}{0.75pt}}
\newcommand{\bc}{\begin{center}}
\newcommand{\ec}{\end{center}}
\newcommand{\be}{\begin{enumerate}}
\newcommand{\ee}{\end{enumerate}}

\def\ds{\displaystyle}

\renewcommand{\d}{\displaystyle}

\newcommand{\ans}[1][2]{ \ \rule{#1 in}{.5 pt} \ }


\pagestyle{fancy} 
%\lfoot{Uses a calculator}
\rfoot{5-1}

\begin{document}

\vspace*{-0.7in}

\begin{center}
  \Large\sc{Section 5.1: Approximating Areas}\\
  instructor notes
  \end{center}

For students who have not seen Calculus before, this section seems unrelated to anything we have done before \emph{other than} the \textbf{strategy} -- namely the strategy of approximating via an easily refined process.\\

Thus, in this section, I would begin (before starting on the worksheet) by talking through the following ideas.

\begin{itemize}
\item Explain that today we are going to talk about a specific strategy for calculating areas that are weirdly shaped, pointing out that if a shape is a rectangle or circle or polygonal, we can use existing formulas. But what if it's not?
\item Explain that the 10,000-foot view of the strategy is exactly the same one that was used at the beginning of the semester when approximating the slope of the tangent. Recall that we didn't just make an approximation, we used a strategy that was easily refined. That is, we prioritized simplicity of calculation so that refinement (in the form of a limit) would improve the approximation to an arbitrary degree of accuracy. \\

This idea is super-relevant in the very first problem. That is, you should ask students to estimate the area under  $f(x) =\frac{1}{2}x^2+1$ and above the $x$-axis on the interval $[0,2].$ Students will make very reasonable choices typically involving an ad hoc collection of rectangle and trapezoids and triangles. Obviously, this will give a better \emph{initial} approximation than 4 rectangles But it's not easy to refine! \\

I want to point out to students that their common sense is right on track! This is exactly what the greats like Archimedes did (see last bullet). But, we might as well use the work of others and our own work understanding the limit.\\

I remind students that we did this with slopes of tangents. That is, students typically want to take some kind of average of slope on the left or the right, or choose secant lines with points on either side of the point of tangency.\\


\item (This is Jill's thing...) Of the subject that is Calculus, the topic we are covering today is definitely the oldest. Archimedes (200 bc) essentially calculated an area of this type by a sequence of triangles. His method ONLY worked on parabolas. 

\item I would work problem 1a at the board. I would emphasize both the calculation and what is written down and how to simpify. (Hey, we can factor out the width!) \\

Then I would have them do all the others at the board.\\

At the end of this class OR the beginning of the next class I would get the students to state explicitly what the last problems is telling them -- we are given a \emph{rate} of leakage and we estimated a quantity. Or more pragmatically, we were given numbers with units \textbf{liters per hour} and we used it to estimate a quantity measured in \textbf{liters}. *IF* one of these was the derivative of the other -- IF this relationship held -- which one is which??
\end{itemize}
\end{document}
