\documentclass[11pt,fleqn]{article} 
\usepackage[margin=0.8in, head=0.8in]{geometry} 
\usepackage{amsmath, amssymb, amsthm}
\usepackage{fancyhdr} 
\usepackage{palatino, url, multicol}
\usepackage{graphicx, pgfplots} 
\usepackage[all]{xy}
\usepackage{polynom,tabularx} 
%\usepackage{pdfsync} %% I don't know why this messes up tabular column widths
\usepackage{enumerate, adjustbox}
\usepackage{framed}
\usepackage{setspace}
\usepackage{array}
\usepackage{pgf,tikz}
\usepackage{mathrsfs}

\usepackage[parfill]{parskip}
\usetikzlibrary{arrows}

\usetikzlibrary{calc}

\pgfplotsset{compat=1.6}

\pgfplotsset{soldot/.style={color=blue,only marks,mark=*}} \pgfplotsset{holdot/.style={color=blue,fill=white,only marks,mark=*}}

\renewcommand{\headrulewidth}{0pt}
\newcommand{\blank}[1]{\rule{#1}{0.75pt}}
\newcommand{\bc}{\begin{center}}
\newcommand{\ec}{\end{center}}
\newcommand{\be}{\begin{enumerate}}
\newcommand{\ee}{\end{enumerate}}

\def\ds{\displaystyle}

\renewcommand{\d}{\displaystyle}

\newcommand{\ans}[1][2]{ \ \rule{#1 in}{.5 pt} \ }


\pagestyle{fancy} 
\rfoot{5-3}

\begin{document}

\vspace*{-0.7in}

\begin{center}
  \Large\sc{Section 5.6: Integrals Involving Exponentials and Logarithmic Functions}\\
  instructor notes
  \end{center}
  
  This will take at least one day and perhaps longer depending on how much is done by the instructor and how much in groups.
  
  Things to emphasize
  
  \begin{itemize}
  \item \# 1: Get students to remember three principles:
  	\begin{itemize}
	\item When doing $u$-substitution, you replace all the $x$'s with $u$'s, including replacing the $dx.$
	\item In the previous section, we always picked $u$ to be something raised to a power.
	\item Have them build a problem for which $u$-substitution will obviously work and one in which it will not.\\
	e.g. $\int (x^4+1)(x^5+5x)^8 \:dx$ versus $\int (x^5+5x)^8 \:dx$ (The second is do-able, but would require us to expand the expression....)
	
	The goal is for them to see the built-in $du$.
	\end{itemize}

	
\item \#2 Have students complete (a)-(c) independently. Then have them figure out or tell them $d$ and $e$. Tell them that they will not have to memorize the formulas in $b$, $d$ and $e$ but should know that they exist and how to use them.\\

You will have to help them understand why $c$ has absolute value bars.

\item \# 3 I would work this on the board very methodically explicitly writing down how $u$ is being selected: exponent of $e$, the denominator, inside $\ln$ or something for which $du$ is present. \\

For 3d, we are reminding them once more about how to manage a definite integral and substitution simultaneously.

\item \#4 Complete this in groups at the board.

	
	
  \end{itemize}
\end{document}
