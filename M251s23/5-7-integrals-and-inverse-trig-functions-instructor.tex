\documentclass[11pt,fleqn]{article} 
\usepackage[margin=0.8in, head=0.8in]{geometry} 
\usepackage{amsmath, amssymb, amsthm}
\usepackage{fancyhdr} 
\usepackage{palatino, url, multicol}
\usepackage{graphicx, pgfplots} 
\usepackage[all]{xy}
\usepackage{polynom,tabularx} 
%\usepackage{pdfsync} %% I don't know why this messes up tabular column widths
\usepackage{enumerate, adjustbox}
\usepackage{framed}
\usepackage{setspace}
\usepackage{array}
\usepackage{pgf,tikz}
\usepackage{mathrsfs}

\usepackage[parfill]{parskip}
\usetikzlibrary{arrows}

\usetikzlibrary{calc}

\pgfplotsset{compat=1.6}

\pgfplotsset{soldot/.style={color=blue,only marks,mark=*}} \pgfplotsset{holdot/.style={color=blue,fill=white,only marks,mark=*}}

\renewcommand{\headrulewidth}{0pt}
\newcommand{\blank}[1]{\rule{#1}{0.75pt}}
\newcommand{\bc}{\begin{center}}
\newcommand{\ec}{\end{center}}
\newcommand{\be}{\begin{enumerate}}
\newcommand{\ee}{\end{enumerate}}

\def\ds{\displaystyle}

\renewcommand{\d}{\displaystyle}

\newcommand{\ans}[1][2]{ \ \rule{#1 in}{.5 pt} \ }


\pagestyle{fancy} 
\rfoot{5-7}

\begin{document}

\vspace*{-0.7in}

\begin{center}
  \Large\sc{Section 5.7: Integrals Resulting in Inverse Trig Functions}\\
  instructor notes
  \end{center}

This sheet requires so heavy-duty algebra skills  along with creativity. \\

Things to emphasize:
\begin{itemize}
\item Get student to list the ways they would pick $u$. The list should include: something raised to a power, something inside a trig function, the exponent of $e$, the denominator, something whose derivative is also in the integrand.\\

\item The formulas in \#2 should be produced by the students. 

\item \# 3, students should be given a chance to start independently or in groups at their desks. However, plugging into arc trig functions will require teacher help.

\item \# 4 For each of these problems, I would work the problem and then give them a simple analog. So it goes, teacher-student-teacher-student-etc.

\item Then have them work \#5 and get the solution on the board.

\end{itemize}
\end{document}
