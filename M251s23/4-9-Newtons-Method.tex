\documentclass[11pt,fleqn]{article} 
\usepackage[margin=0.8in, head=0.8in]{geometry} 
\usepackage{amsmath, amssymb, amsthm}
\usepackage{fancyhdr} 
\usepackage{palatino, url, multicol}
\usepackage{graphicx, pgfplots} 
\usepackage[all]{xy}
\usepackage{polynom,tabularx} 
%\usepackage{pdfsync} %% I don't know why this messes up tabular column widths
\usepackage{enumerate}
\usepackage{framed}
\usepackage{setspace}
\usepackage{array}
\usepackage{pgf,tikz}
\usepackage{mathrsfs}

\usepackage[parfill]{parskip}
\usetikzlibrary{arrows}

\usetikzlibrary{calc}

\pgfplotsset{compat=1.6}

\pgfplotsset{soldot/.style={color=blue,only marks,mark=*}} \pgfplotsset{holdot/.style={color=blue,fill=white,only marks,mark=*}}

\renewcommand{\headrulewidth}{0pt}
\newcommand{\blank}[1]{\rule{#1}{0.75pt}}
\newcommand{\bc}{\begin{center}}
\newcommand{\ec}{\end{center}}
\newcommand{\be}{\begin{enumerate}}
\newcommand{\ee}{\end{enumerate}}

\renewcommand{\d}{\displaystyle}

\newcommand{\ans}[1][2]{ \ \rule{#1 in}{.5 pt} \ }


\pagestyle{fancy} 
%\lfoot{Uses a calculator}
\rfoot{4-9}

\begin{document}

\vspace*{-0.7in}

\begin{center}
  \Large\sc{Section 4.9 Newton's Method}\\
\end{center}
\begin{enumerate}
\item Why would you want to solve $f(x)=0$?
\vspace{1in}
\item You are going to produce the \emph{iterative} formula that is Newton's Method.
	\begin{enumerate}
	\item Find the equation of the line tangent to $f(x)$ at $x=x_1.$ (Assume $f'(x_1) \not = 0.$)
	\vfill
	\item Determine the $x$-value where the tangent line from part (a) intersects the $x$-axis. Call this $x$-value $x_2$.
	\vfill
	\item Draw a picture of your calculations on the graph below.\\
	\begin{tikzpicture}
\begin{axis}[xticklabels={,,},yticklabels={,,},axis x line=middle, axis y line=
middle,xlabel={$x$}, ylabel={$y$},xmin=-1, xmax=3, ymin=-1, ymax=3]
\addplot[domain=-1:3]{0.5*(x-0.7)*(x+2)};
\end{axis}
\node at (5.2,1){$x_1$};
\end{tikzpicture}
\item Given a guess $x_n$, write the formula for how to get a better guess, $x_{n+1}.$
\vspace{.5in}
\newpage
	\end{enumerate}
\item \noindent{\fbox{\sc{Model Problem:}}} Let $f(x)=x^3-5x.$\\
\be
\item Factor $f(x)$, find its roots algebraically, and sketch its graph.
\vspace{3in}
\item Assume you couldn't factor the function and you wanted to find its positive root. What would be a reasonable first guess and why?
\vspace{1in}
\item Using a first guess of $x_1=3,$ calculate 3 iterations of Newton's method. You must hold onto as many digits as your calculating device will allow. No rounding.
\vfill
\item How close is your estimate of the root, $x_3,$ to the actual root?
\vspace{0.25in}
\ee

\end{enumerate}
\end{document}