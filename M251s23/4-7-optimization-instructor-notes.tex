\documentclass[11pt,fleqn]{article} 
\usepackage[margin=0.8in, head=0.8in]{geometry} 
\usepackage{amsmath, amssymb, amsthm}
\usepackage{fancyhdr} 
\usepackage{palatino, url, multicol}
\usepackage{graphicx, pgfplots} 
\usepackage[all]{xy}
\usepackage{polynom,tabularx} 
%\usepackage{pdfsync} %% I don't know why this messes up tabular column widths
\usepackage{enumerate}
\usepackage{framed}
\usepackage{setspace}
\usepackage{array}
\usepackage{pgf,tikz}
\usepackage{mathrsfs}

\usepackage[parfill]{parskip}
\usetikzlibrary{arrows}

\usetikzlibrary{calc}

\pgfplotsset{compat=1.6}

\pgfplotsset{soldot/.style={color=blue,only marks,mark=*}} \pgfplotsset{holdot/.style={color=blue,fill=white,only marks,mark=*}}

\renewcommand{\headrulewidth}{0pt}
\newcommand{\blank}[1]{\rule{#1}{0.75pt}}
\newcommand{\bc}{\begin{center}}
\newcommand{\ec}{\end{center}}
\newcommand{\be}{\begin{enumerate}}
\newcommand{\ee}{\end{enumerate}}

\renewcommand{\d}{\displaystyle}

\newcommand{\ans}[1][2]{ \ \rule{#1 in}{.5 pt} \ }


\pagestyle{fancy} 
%\lfoot{Uses a calculator}
\rfoot{4-7 }

\begin{document}

\vspace*{-0.7in}

\begin{center}
  \Large\sc{Section 4.7 Optimization }\\
  instructor notes
\end{center}

There are two sheets that should take at least 2 days.\\

Sheet 1 (day 1) \\

This should actually take less than one full day. I would go through these at the board. The purpose to for the students to practice a framework for working problems all of which look different. The goal is for students to see the pattern that goes:\\

identify your goal (ie what is supposed to be maximized or minimized)\\
write that quantity as a function of one variable\\
use calculus to identify extrema -- emphasizing that you will need to check that your critical numbers actually correspond to your goal. That is, emphasize that if we are looking for a minimum, we should check that our critical number actually corresponds to a minimum and not a maximum or neither.\\

Sheet 2 (day 1 and day 2)\\

This has actually challenging problems. They should do this in groups at the board. Ideally everyone gets started on the first problem at the end of day 1 and at the beginning of day 2 you can go back over this. Then on day 2, they work on the remaining three problems at the board. \\

Depending on the personalities, you may need to remind/require them to provide SOLUTIONS on the board, not just answers. You may need to push them to actually use calculus to check the correctness of their answers.
\end{document}
