\documentclass[11pt,fleqn]{article} 
\usepackage[margin=0.8in, head=0.8in]{geometry} 
\usepackage{amsmath, amssymb, amsthm}
\usepackage{fancyhdr} 
\usepackage{palatino, url, multicol}
\usepackage{graphicx, pgfplots} 
\usepackage[all]{xy}
\usepackage{polynom,tabularx} 
%\usepackage{pdfsync} %% I don't know why this messes up tabular column widths
\usepackage{enumerate, adjustbox}
\usepackage{framed}
\usepackage{setspace}
\usepackage{array}
\usepackage{pgf,tikz}
\usepackage{mathrsfs}

\usepackage[parfill]{parskip}
\usetikzlibrary{arrows}

\usetikzlibrary{calc}

\pgfplotsset{compat=1.6}

\pgfplotsset{soldot/.style={color=blue,only marks,mark=*}} \pgfplotsset{holdot/.style={color=blue,fill=white,only marks,mark=*}}

\renewcommand{\headrulewidth}{0pt}
\newcommand{\blank}[1]{\rule{#1}{0.75pt}}
\newcommand{\bc}{\begin{center}}
\newcommand{\ec}{\end{center}}
\newcommand{\be}{\begin{enumerate}}
\newcommand{\ee}{\end{enumerate}}

\def\ds{\displaystyle}

\renewcommand{\d}{\displaystyle}

\newcommand{\ans}[1][2]{ \ \rule{#1 in}{.5 pt} \ }


\pagestyle{fancy} 
%\lfoot{Uses a calculator}
\rfoot{5-2}

\begin{document}

\vspace*{-0.7in}

\begin{center}
  \Large\sc{Section 5.2: The Definite Integral}\\
  instructor notes
  \end{center}
  
  I would expect this to be a 1.5 class worksheet assuming that there the teacher spends 20 minutes at the beginning of class talking about ideas below.

There are two goals for this worksheet:

\begin{itemize}
\item to connect the idea of approximating areas using rectangles (from section 5.1) to the notation associated with the definite integral including specifically the notion of a sum of areas of rectangles inside a limit where the number of rectangles is increasing. (ie the limit of a sum)\\

We do not ask them ever to use this definition. However, understanding the structure is crucial to understanding why something we envisioned as \emph{area} ends up sometimes being negative.\\

You are going to need to tell them that they are not going to have to use the definition so they aren't terrified.\\

\item introduce the technique of evaluating the definite integral as signed area using geometric formulas (ie bypassing the limit of the sum definition) 
\end{itemize}

In summary, you will need to talk about these ideas and state the definition for them. This will take up time at the beginning.\\

I would work problem 2 on the board but slowly. Emphasize that "geometry" means draw the graph. So give them a chance to draw the graph. \\

Then do the calculation.\\

I would tell them they will need to do the same thing on \#3 but before they do that, they should look at the bullet points above \#4.\\

Both the first and the last are ones that the students cannot guess what I meant. For the others, they should try.\\

For the remainder of the sheet, have them work in groups at the board.\\

It is really important to go over the bullet points together the next day so that everyone has the correct answers for the right reasons.\\

If time allows, the next day, go over average value of a function.\\


\end{document}
