
\documentclass[11pt,fleqn]{article} 
\usepackage[margin=0.8in, head=0.8in]{geometry} 
\usepackage{amsmath, amssymb, amsthm}
\usepackage{fancyhdr} 
\usepackage{palatino, url, multicol}
\usepackage{graphicx, pgfplots} 
\usepackage[all]{xy}
\usepackage{polynom} 
%\usepackage{pdfsync} %% I don't know why this messes up tabular column widths
\usepackage{enumerate}
\usepackage{framed}
\usepackage{setspace}
\usepackage{array,tikz}

\pgfplotsset{compat=1.6}

\pgfplotsset{soldot/.style={color=black,only marks,mark=*}} \pgfplotsset{holdot/.style={color=black,fill=white,only marks,mark=*}}
\pgfplotsset{my style/.append style={axis x line=middle, axis y line=
middle, xlabel={$x$}, ylabel={$y$} }}

%axis equal 
\pagestyle{fancy} 
\lfoot{}
\rfoot{3-5 Trig Derivatives}

\begin{document}
\renewcommand{\headrulewidth}{0pt}
\newcommand{\blank}[1]{\rule{#1}{0.75pt}}
\newcommand{\bc}{\begin{center}}
\newcommand{\ec}{\end{center}}
\renewcommand{\d}{\displaystyle}

\vspace*{-0.7in}

%%%%%%%%%intro page
\begin{center}
  \large
  \sc{Section 3-5: Derivatives of Trigonometric Functions}\\
\end{center}
\begin{enumerate}

\item Remind students that these identities follow from the Pythagorean Theorem and basic trig identities so they really don't have to memorize them.\\

\item Student can do this in groups and put their answers on the board. Point out that these rules are easy to develop. 
\item SUMMARY RULES:\\

Remind students that they need to know all of these for the derivative proficiency. However, for the Midterm, they will be able to bring in a 3 by 5 card.
\item Ideally students work this in groups. Before sending students off to work it, it would be helpful to both graph the function $x(t)=8\sin(t)$ and to draw a diagram of what the mass is doing. Emphasize that students should always have a picture of what is going on and should think about the plausibility of their answers.
\end{enumerate}
\end{document}

