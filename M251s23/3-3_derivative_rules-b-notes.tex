
\documentclass[11pt,fleqn]{article} 
\usepackage[margin=0.8in, head=0.8in]{geometry} 
\usepackage{amsmath, amssymb, amsthm}
\usepackage{fancyhdr} 
\usepackage{palatino, url, multicol}
\usepackage{graphicx, pgfplots} 
\usepackage[all]{xy}
\usepackage{polynom} 
%\usepackage{pdfsync} %% I don't know why this messes up tabular column widths
\usepackage{enumerate}
\usepackage{framed}
\usepackage{setspace}
\usepackage{array,tikz}

\pgfplotsset{compat=1.6}

\pgfplotsset{soldot/.style={color=black,only marks,mark=*}} \pgfplotsset{holdot/.style={color=black,fill=white,only marks,mark=*}}
\pgfplotsset{my style/.append style={axis x line=middle, axis y line=
middle, xlabel={$x$}, ylabel={$y$}, axis equal }}


\pagestyle{fancy} 
\lfoot{}
\rfoot{3-3 Derivative Rules (day 2)}

\begin{document}
\renewcommand{\headrulewidth}{0pt}
\newcommand{\blank}[1]{\rule{#1}{0.75pt}}
\newcommand{\bc}{\begin{center}}
\newcommand{\ec}{\end{center}}
\renewcommand{\d}{\displaystyle}

\vspace*{-0.7in}

%%%%%%%%%intro page
\begin{center}
  \large
  \sc{Section 3-3: Derivative Rules (day 2) Teacher Notes}\\
\end{center}
\begin{enumerate}
\item Get students to remember the product and quotient rule. Remind them that they will need to KNOW these for the Quiz on Thursday.
\item If possible, put students in groups to work on these derivatives and get students to put their solutions on the board. If not, get the students to attempt problems independently in pairs (a and b), then (c and d), then (e). Part of the point is to recognize that just because somethings looks like a product or quotient does not mean one should use that rule.

\end{enumerate}

For the remaining problems, get students to complete them in groups and put them on the board, or, have students start them independently and then work them on the board.
\end{document}

