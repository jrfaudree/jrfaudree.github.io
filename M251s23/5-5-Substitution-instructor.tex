\documentclass[11pt,fleqn]{article} 
\usepackage[margin=0.8in, head=0.8in]{geometry} 
\usepackage{amsmath, amssymb, amsthm}
\usepackage{fancyhdr} 
\usepackage{palatino, url, multicol}
\usepackage{graphicx, pgfplots} 
\usepackage[all]{xy}
\usepackage{polynom,tabularx} 
%\usepackage{pdfsync} %% I don't know why this messes up tabular column widths
\usepackage{enumerate, adjustbox}
\usepackage{framed}
\usepackage{setspace}
\usepackage{array}
\usepackage{pgf,tikz}
\usepackage{mathrsfs}

\usepackage[parfill]{parskip}
\usetikzlibrary{arrows}

\usetikzlibrary{calc}

\pgfplotsset{compat=1.6}

\pgfplotsset{soldot/.style={color=blue,only marks,mark=*}} \pgfplotsset{holdot/.style={color=blue,fill=white,only marks,mark=*}}

\renewcommand{\headrulewidth}{0pt}
\newcommand{\blank}[1]{\rule{#1}{0.75pt}}
\newcommand{\bc}{\begin{center}}
\newcommand{\ec}{\end{center}}
\newcommand{\be}{\begin{enumerate}}
\newcommand{\ee}{\end{enumerate}}

\def\ds{\displaystyle}

\renewcommand{\d}{\displaystyle}

\newcommand{\ans}[1][2]{ \ \rule{#1 in}{.5 pt} \ }


\pagestyle{fancy} 
\rfoot{5-3}

\begin{document}

\vspace*{-0.7in}

\begin{center}
  \Large\sc{Section 5.5: Substitution (i.e. undoing the Chain Rule)}\\
  Instructor Notes
  \end{center}
 This sheet is flexible and will take at least one day. It can be longer or shorter depending on how much the instructor does and how much is left exclusively to group work.\\
 
 Things to emphasize.
 
 \begin{itemize}
 
 \item Begin by pointing out that they can always tell if a definite integral is correct via differentiation. Point out that this implies that guess-and-check is always a strategy.
 
 \item Instructor does \#1: Emphasize that you are taking an integral in $x$ and \emph{completely} rewriting it in terms of $u$ including the $du$-part. Observe that it is the $du$-part that is helping you undo the chain rule. 
 
 \item Students do \#3-6 and get complete solutions on the board. Check them.
 
\item Instructor does \#7 in two ways. (way 1: Keep limits for $x$; way 2: Shift limits to $u$) Emphasize that both approaches are important. Point out that you do not want to have only one way to get out of your house in a fire. Same principle in mathematics. 

\item Require that students do \#8 in \emph{both} ways. Get solutions on the board and check them both. 

\item Get them to make an attempt at \#9. Help if needed.

\item Go in with some extra problems if time permits. Some examples:

$\displaystyle \int \: \left(5+x^2+\frac{x^2}{x^3+1}\right)\: dx$

$\displaystyle \int \: \left(\sec^2(x/2) + \cos(\pi x) \sin^2(\pi x)\right)\: dx$

$\displaystyle \int \: (k+rs )\: ds$
\end{itemize}
\end{document}
