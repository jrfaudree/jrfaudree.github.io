\documentclass[11pt,fleqn]{article} 
\usepackage[margin=0.8in, head=0.8in]{geometry} 
\usepackage{amsmath, amssymb, amsthm}
\usepackage{fancyhdr} 
\usepackage{palatino, url, multicol}
\usepackage{graphicx, pgfplots} 
\usepackage[all]{xy}
\usepackage{polynom,tabularx} 
%\usepackage{pdfsync} %% I don't know why this messes up tabular column widths
\usepackage{enumerate}
\usepackage{framed}
\usepackage{setspace}
\usepackage{array}
\usepackage{pgf,tikz}
\usepackage{mathrsfs}

\usepackage[parfill]{parskip}
\usetikzlibrary{arrows}

\usetikzlibrary{calc}

\pgfplotsset{compat=1.6}

\pgfplotsset{soldot/.style={color=blue,only marks,mark=*}} \pgfplotsset{holdot/.style={color=blue,fill=white,only marks,mark=*}}

\renewcommand{\headrulewidth}{0pt}
\newcommand{\blank}[1]{\rule{#1}{0.75pt}}
\newcommand{\bc}{\begin{center}}
\newcommand{\ec}{\end{center}}
\newcommand{\be}{\begin{enumerate}}
\newcommand{\ee}{\end{enumerate}}

\def\ds{\displaystyle}

\renewcommand{\d}{\displaystyle}

\newcommand{\ans}[1][2]{ \ \rule{#1 in}{.5 pt} \ }


\pagestyle{fancy} 
%\lfoot{Uses a calculator}
\rfoot{4-10}

\begin{document}

\vspace*{-0.7in}

\begin{center}
  \Large\sc{Section 4.10 Antiderivatives}
\end{center}

\begin{enumerate}
\item Find the (family of) antiderivatives for the following.
	\begin{enumerate}
	\item $f(x)=4x^3$\\
	\vfill
		
	\item $f(x)=5\sin(x)$	\\
	\vfill
	
	\item $f(x) = \frac{e^x}{4}$	\\
	\vfill
	
	\item $f(x)=\sqrt{2}$\\
	\vfill
	
	\item $f(x)=\frac{1}{x}$\\
	\vfill
	
	\item $f(x) = 1-x+e^x$\\
	\vfill
	\end{enumerate}

\item Is $F(x)=x+xe^x$ is an antiderivative of $f(x)=(x+1)e^x+1$? Show your answer is correct.\\

\vspace{2in}

\doublespacing
\begin{tabular}{|c|c|}\hline
Function & \quad \quad Antiderivative \quad \quad \quad \\ \hline
$\ds x^k$ ($k\neq -1$) &\\  \hline
$\ds x^{-1}$ for all $x$&\\ \hline
$\ds 1$ &\\ \hline
$\ds \sin(x)$ & \\ \hline
$\ds \cos(x)$ &\\ \hline
\end{tabular}
\quad
\begin{tabular}{|c|c|}\hline
Function & \quad \quad Antiderivative \quad \quad \quad \\ \hline
$\ds e^x$ &\\ \hline
$\ds 1/(1+x^2)$  &\\ \hline
$\ds \sec^2(x)$ &\\ \hline
$\ds \sec(x)\tan(x)$ &\\ \hline
$\ds 1/\sqrt{1-x^2}$ &\\ \hline
\end{tabular}
\singlespacing
\newpage
\item Evaluate the integrals.
%\begin{multicols}{3}
	\begin{enumerate}
	\item $\displaystyle{\int( x^{1/2} + x^{-7/4})\: dx}$\\

	\item $\displaystyle{\int( 8e^{x} + \sec^2(x))\: dx}$\\
	\item $\displaystyle{\int \frac{x^2+x^{1/2}+1}{x^{1/2}}\: dx}$
	\vspace{.5in}
	\end{enumerate}
	%\end{multicols}
\item Is the equality in the box true or false? Explain. \hfill \fbox{$\displaystyle \int x \sec^2(x^2+1) \: dx = \tan(x^2+1) +C $}\\
\vspace{.5in}
\item Solve the initial value problem if $f'(x)=x+e^x$ and $f(0)=4.$
\vfill
\item A particle moving along the $x$-axis has acceleration $a(t)=10\sin(t)$ measured in $cm/s^2.$ Assume the particle as initial velocity $v(0)=0$ and initial position $s(0)=0,$ find a function that models its velocity, $v(t),$ and its position $s(t).$
\vfill
\end{enumerate}
\end{document}
