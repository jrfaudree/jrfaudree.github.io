\documentclass[11pt,fleqn]{article} 
\usepackage[margin=0.8in, head=0.8in]{geometry} 
\usepackage{amsmath, amssymb, amsthm}
\usepackage{fancyhdr} 
\usepackage{palatino, url, multicol}
\usepackage{graphicx, pgfplots} 
\usepackage[all]{xy}
\usepackage{polynom,tabularx} 
%\usepackage{pdfsync} %% I don't know why this messes up tabular column widths
\usepackage{enumerate, adjustbox}
\usepackage{framed}
\usepackage{setspace}
\usepackage{array}
\usepackage{pgf,tikz}
\usepackage{mathrsfs}

\usepackage[parfill]{parskip}
\usetikzlibrary{arrows}

\usetikzlibrary{calc}

\pgfplotsset{compat=1.6}

\pgfplotsset{soldot/.style={color=blue,only marks,mark=*}} \pgfplotsset{holdot/.style={color=blue,fill=white,only marks,mark=*}}

\renewcommand{\headrulewidth}{0pt}
\newcommand{\blank}[1]{\rule{#1}{0.75pt}}
\newcommand{\bc}{\begin{center}}
\newcommand{\ec}{\end{center}}
\newcommand{\be}{\begin{enumerate}}
\newcommand{\ee}{\end{enumerate}}

\def\ds{\displaystyle}

\renewcommand{\d}{\displaystyle}

\newcommand{\ans}[1][2]{ \ \rule{#1 in}{.5 pt} \ }


\pagestyle{fancy} 
\rfoot{5-3}

\begin{document}

\vspace*{-0.7in}

\begin{center}
  \Large\sc{Section 5.3: The Fundamental Theorem of Calculus}\\
  instructor notes
  \end{center}
There are two sheets for Section 5.3, each is a day.\\

Day 1 \\

The first page of the first sheet I do using my iPad. I fill out $f(0)$, $f(1)$, have them do it. I fill out $A(0)$ and $A(1)$. Then have them finish the sheet in groups at their desk and then talk about it on the screen and agree we have the right answers for the right reasons.\\

On page 2, I state the FTC and do a baby example. Then they do \#3. \\

Then I state FTC part 2 \emph{without a rationale}. That is, I tell them that first I am going to state the Theorem and practice using it so they understand what is being asserted.  Then I do \# 5a on the board -- obsessing on how to write it. Then have them do b and critique each others' answers and writing.

Day 2\\

The top is a review of FTC part 1. The second leads students to understand why FTC part 2 works. The strategy is to view the definite integral as the evaluation of an accumulated-area-under-a-curve function.\\

Once you get to a restatement of FTC part 2 on page 2, have students work in groups at the board. The remaining problems are practice and foreshadowing of the Net Change Theorem
\end{document}
