\documentclass[11pt,fleqn]{article} 
\usepackage[margin=0.8in, head=0.8in]{geometry} 
\usepackage{amsmath, amssymb, amsthm}
\usepackage{fancyhdr} 
\usepackage{palatino, url, multicol}
\usepackage{graphicx, pgfplots} 
\usepackage[all]{xy}
\usepackage{polynom,tabularx} 
%\usepackage{pdfsync} %% I don't know why this messes up tabular column widths
\usepackage{enumerate, adjustbox}
\usepackage{framed}
\usepackage{setspace}
\usepackage{array}
\usepackage{pgf,tikz}
\usepackage{mathrsfs}

\usepackage[parfill]{parskip}
\usetikzlibrary{arrows}

\usetikzlibrary{calc}

\pgfplotsset{compat=1.6}

\pgfplotsset{soldot/.style={color=blue,only marks,mark=*}} \pgfplotsset{holdot/.style={color=blue,fill=white,only marks,mark=*}}

\renewcommand{\headrulewidth}{0pt}
\newcommand{\blank}[1]{\rule{#1}{0.75pt}}
\newcommand{\bc}{\begin{center}}
\newcommand{\ec}{\end{center}}
\newcommand{\be}{\begin{enumerate}}
\newcommand{\ee}{\end{enumerate}}

\def\ds{\displaystyle}

\renewcommand{\d}{\displaystyle}

\newcommand{\ans}[1][2]{ \ \rule{#1 in}{.5 pt} \ }


\pagestyle{fancy} 
\rfoot{5-3}

\begin{document}

\vspace*{-0.7in}

\begin{center}
  \Large\sc{Section 5.3: The Fundamental Theorem of Calculus (day 2)}
  \end{center}
\begin{enumerate}
\item The Fundamental Theorem of Calculus (part 1):
\vfill
\item Find the derivative of each function below.
	\begin{multicols}{2}
	\begin{enumerate}
	\item $\d g(x)=\int_{-1}^x t^2e^t\: dt$
	\item $\d h(x)=\int_0^{x^2+1} \sin(t) \: dt$
	\end{enumerate}
	\end{multicols}
\vfill

\item Let $f(x)=4x.$ Find two different antiderivatives of $f(x)$. Call them $F_1(x)$ and $F_2(x).$
\vfill
\item Let $G(x)= \int_1^x 4t \: dt.$ 
	\begin{enumerate}
	\item What do $G(x)$, $F_1(x)$ and $F_2(x)$ all have in common?
	\vfill
	\item Find $G(1)$, $F_1(1)$ and $F_2(1).$ You will have to find $G(3)$ by geometry.
	\vfill
	\item Find $G(3)$, $F_1(3)$ and $F_2(3).$ You will have to find $G(3)$ by geometry.
	\vfill
	\item Using your answers above, find $G(3)-G(1)$ and explain what it means geometrically about the curve $y=4x.$\\
	\item Find $F_1(3)-F_1(1)$ and $F_2(3)-F_2(1)$.
	\vfill
	\item What do parts (d) and (e) indicate about how you can calculate the (signed) area under a curve $f(x)$ on an interval $[a,b]$?
	\vfill
	\end{enumerate}
\newpage
\item The Fundamental Theorem of Calculus (part 2):
\vfill
\item Evaluate the integrals.
	\begin{multicols}{2}
	\begin{enumerate}
	\item $\d \int_0^{\pi} \sin (x) \: dx$
	\item $\d \int_{-1}^3 x+e^x \: dx$
	\end{enumerate}
	\end{multicols}
\vfill
\item Find the average value of $f(x)=x^2$ over the interval $[0,3].$
\vfill
\item Assume the velocity of an object thrown directly up into the air is given by $v(t)= 20-9.8t$ where $v$ is measured in meters per second and $t$ is measured in seconds. 
	\begin{enumerate}
	\item Evaluate $\int_0^1 v(t) \: dt$
	\vfill
	\item Evaluate $\int v(t) \: dt$
	\vfill
	\item Explain why you do not have enough information to find the height of the object exactly?
	\vfill
	\item Explain, in the context of the problems what part (a) and part (b) represent.
	\vfill
	\end{enumerate}

\end{enumerate}
\end{document}
