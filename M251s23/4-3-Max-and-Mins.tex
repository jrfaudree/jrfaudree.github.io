\documentclass[11pt,fleqn]{article} 
\usepackage[margin=0.8in, head=0.8in]{geometry} 
\usepackage{amsmath, amssymb, amsthm}
\usepackage{fancyhdr} 
\usepackage{palatino, url, multicol}
\usepackage{graphicx} 
\usepackage[all]{xy}
\usepackage{polynom} 
\usepackage{pdfsync}
\usepackage{enumerate}
\usepackage{framed}
\usepackage{setspace, adjustbox}
\usepackage{array%,tikz, pgfplots
}

\usepackage{tikz, pgfplots}
\usetikzlibrary{calc}
%\pgfplotsset{my style/.append style={axis x line=middle, axis y line=
%middle, xlabel={$x$}, ylabel={$y$}, axis equal }}
%
\pagestyle{fancy} 
\lfoot{UAF Calculus I}
\rfoot{4-3}


\newcommand{\be}{\begin{enumerate}}
\newcommand{\ee}{\end{enumerate}}

\newcommand{\bi}{\begin{itemize}}
\newcommand{\ei}{\end{itemize}}

\begin{document}
\setlength{\parindent}{0cm}
\renewcommand{\headrulewidth}{0pt}
\newcommand{\blank}[1]{\rule{#1}{0.75pt}}
\renewcommand{\d}{\displaystyle}
\vspace*{-0.7in}
\begin{center}
 {\large{ \sc{Section 4.3: Maximums and Minimums}}}
\end{center}
 \begin{enumerate}
 \item local and absolute maximums and minimums: what they are
 \vfill
 \item A variety of examples
 \vfill
 
 \item A critical numberx of $f(x)$ is 
 \vspace{1in}
 \newpage
 \item First, find the domain and all critical numbers. Then, identify all local and/or absolute maxima and minima. Use technology to sketch the graphs and confirm your answers.
 \begin{enumerate}
 \item $f(x)=e^x(x-2)^2$
 \vfill
 \item $f(x)= (x-2)^{2/3}+1$
 \vfill
 \item $f(x)=\frac{x^2}{(x-1)^2}$
 \vfill
 \end{enumerate}
 \end{enumerate}
\end{document}