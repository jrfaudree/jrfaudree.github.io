
\documentclass[11pt,fleqn]{article} 
\usepackage[margin=0.8in, head=0.8in]{geometry} 
\usepackage{amsmath, amssymb, amsthm}
\usepackage{fancyhdr} 
\usepackage{palatino, url, multicol}
\usepackage{graphicx, pgfplots} 
\usepackage[all]{xy}
\usepackage{polynom} 
%\usepackage{pdfsync} %% I don't know why this messes up tabular column widths
\usepackage{enumerate}
\usepackage{framed}
\usepackage{setspace}
\usepackage{array,tikz}

\pgfplotsset{compat=1.6}

\pgfplotsset{soldot/.style={color=black,only marks,mark=*}} \pgfplotsset{holdot/.style={color=black,fill=white,only marks,mark=*}}
\pgfplotsset{my style/.append style={axis x line=middle, axis y line=
middle, xlabel={$x$}, ylabel={$y$} }}

%axis equal 
\pagestyle{fancy} 
\lfoot{}
\rfoot{3-5 Trig Derivatives extra}

\begin{document}
\renewcommand{\headrulewidth}{0pt}
\newcommand{\blank}[1]{\rule{#1}{0.75pt}}
\newcommand{\bc}{\begin{center}}
\newcommand{\ec}{\end{center}}
\renewcommand{\d}{\displaystyle}

\vspace*{-0.7in}

%%%%%%%%%intro page
\begin{center}
  \large
  \sc{Section 3-5: Derivatives of Trigonometric Functions (Extra Practice)}\\
\end{center}
\begin{enumerate}

\item (Revisit the spring problem:)  A mass on a spring vibrates horizontally on a smooth level surface. Its equation of motion is $x(t)=8 \sin(t),$  where $t$ is in seconds and $x$ is in centimeters.\\
	\begin{enumerate}
	\item We found:\\
	$v(t)=x'(t)=8 \cos(t)$ and $a(t)=v'(t)=x''(t)=-8\sin(t)$
	\vspace{0.5in}
	\item We found: \\
	$x(2\pi / 3)= 4\sqrt{3} \: cm$\\
	$x'(2\pi / 3)= -4 \: cm/s$\\
	$x''(2\pi / 3)= -4\sqrt{3} \: cm/s^2$\\
	At $t=2\pi / 3,$ the mass is moving to the left and slowing down.
	\item Draw a picture of the motion of the mass and include the time(s) at which the mass changes direction.
	\vfill
	 \end{enumerate}
	 
\item Higher Order Derivatives. For each function below, find $f'(x),\: f''(x),\:f'''(x),\:f^{(4)}(x),\:f^{(82)}(x)$
	\begin{enumerate}
	\item $f(x)=x^5+2x^2+1$
	\vspace{1in}
	\item $f(x)=2\sin(x)$
	\vspace{1in}
	\end{enumerate}
\newpage
\item Other ways of denoting derivatives.
\end{enumerate}
\end{document}

