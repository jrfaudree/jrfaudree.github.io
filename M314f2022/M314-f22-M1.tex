\documentclass[12pt]{article}
\usepackage[top=1in, bottom=1in, left=.75in, right=.75in]{geometry}
\usepackage{amsmath}
\usepackage{fancyhdr}
\usepackage{graphicx, xcolor}
\usepackage{txfonts}
\usepackage{multicol,coordsys,pgfplots,systeme}
\usepackage[scaled=0.86]{helvet}
\renewcommand{\emph}[1]{\textsf{\textbf{#1}}}
\usepackage{anyfontsize}
% \usepackage{times}
% \usepackage[lf]{MinionPro}
\usepackage{tikz,pgfplots}
%\def\degC{{}^\circ{\rm C}}
\def\ra{\rightarrow}
\usetikzlibrary{calc}
\pgfplotsset{compat = newest}
\newcommand{\blank}[1]{\rule{#1}{0.75pt}}

\pgfplotsset{my style/.append style={axis x line=middle, axis y line=
middle, xlabel={$x$}, ylabel={$y$}}}

\newcommand{\bpm}{\begin{pmatrix}}
\newcommand{\epm}{\end{pmatrix}}

\parindent 0pt
\parskip 4pt
\pagestyle{fancy}
\fancyfoot[C]{\emph{\thepage}}
\fancyhead[L]{\ifnum \value{page} > 1\relax\emph{Math 314: Midterm 1}\fi}
\fancyhead[R]{\ifnum \value{page} > 1\relax\emph{Fall 2022}\fi}
\headheight 15pt
\renewcommand{\headrulewidth}{0pt}
\renewcommand{\footrulewidth}{0pt}
\let\ds\displaystyle
\def\continued{{\emph {Continued....}}}
\def\continuing{{\emph {Problem \arabic{probcount} continued....}}\par\vskip 4pt}


\newcounter{probcount}
\newcounter{subprobcount}
\newcommand{\thesubproblem}{\emph{\alph{subprobcount}.}}
\def\problem#1{\setcounter{subprobcount}{0}%
\addtocounter{probcount}{1}{\emph{\arabic{probcount}.\hskip 1em(#1)}}\par}
\def\subproblem#1{\par\hangindent=1em\hangafter=0{%
\addtocounter{subprobcount}{1}\thesubproblem\emph{#1}\hskip 1em}}
\def\probskip{\vskip 10pt}
\def\medprobskip{\vskip 2in}
\def\subprobskip{\vskip 45pt}
\def\bigprobskip{\vskip 4in}

\begin{document}
{\emph{\fontsize{26}{28}\selectfont Math 314\hfill
{\fontsize{32}{36}\selectfont Midterm 1}
\hfill Fall 2022}}
\vskip 2cm
\strut\vtop{\halign{\emph#\hskip 0.5em\hfil&#\hbox to 2in{\hrulefill}\cr
\emph{\fontsize{18}{22}\selectfont Name:}&\cr
\noalign{\vskip 10pt}
%\emph{\fontsize{18}{22}\selectfont Student Id:}&\cr
%\noalign{\vskip 10pt}
%\emph{\fontsize{18}{22}\selectfont Calculator Model:}&\cr
}}
%\hfill
%\vtop{\halign{\emph{\fontsize{18}{22}\selectfont #}\hfil& \emph{\fontsize{18}{22}\selectfont\hskip 0.5ex $\square$ #}\hfil\cr
%Section: & 001 (Jill Faudree)\cr
%\noalign{\vskip 4pt}
%         & 002 (Ryan Bridges)\cr
%\noalign{\vskip 4pt}
%         & 005 (Leah Berman)\cr}}
%
\vfill
{\fontsize{18}{22}\selectfont\emph{Rules:}}

You have one hour to complete the exam. 

Partial credit will be awarded, but you must show your work.

You may have a single handwritten sheet of notes.

\textbf{Except for problem 1,} you may use technology to find the reduced echelon form of a matrix. 


%Place a box around your  \fbox{FINAL ANSWER} to each question where appropriate.

%If you need extra space, you can use the back sides of the pages.
%Please make it obvious  when you have done so.

Turn off anything that might go beep during the exam.

Good luck!
\vfill
\def\emptybox{\hbox to 2em{\vrule height 16pt depth 8pt width 0pt\hfil}}
\def\tline{\noalign{\hrule}}
\centerline{\vbox{\offinterlineskip
{
\bf\sf\fontsize{18pt}{22pt}\selectfont
\hrule
\halign{
\vrule#&\strut\quad\hfil#\hfil\quad&\vrule#&\quad\hfil#\hfil\quad
&\vrule#&\quad\hfil#\hfil\quad&\vrule#\cr
height 3pt&\omit&&\omit&&\omit&\cr
&Problem&&Possible&&Score&\cr\tline
height 3pt&\omit&&\omit&&\omit&\cr
&1&&15&&\emptybox&\cr\tline
&2&&14&&\emptybox&\cr\tline
&3&&18&&\emptybox&\cr\tline
&4&&18&&\emptybox&\cr\tline
&5&&10&&\emptybox&\cr\tline
&6&&10&&\emptybox&\cr\tline
&7&&15&&\emptybox&\cr\tline
%&9&&15&&\emptybox&\cr\tline
&Extra Credit&&5&&\emptybox&\cr\tline
&Total&&100&&\emptybox&\cr
}\hrule}}}

\newpage
\begin{enumerate}
%%%RREF
%%%S 1.3.1 and 1.1.1
\item (15 points) Use Gauss-Jordan reduction to find the reduced echelon form of the matrix $A$ below. You must show your work and state the row operations you are performing.\\

$A=\bpm 2&-2&2&8\\1&0&2&6\\0&2&0&1 \epm$

\newpage
%%subspaces and span
%%%S 2.1.2
\item (14 points) Let $\displaystyle T=\left\{ a_0+a_1x+a_2x^2+a_3x^3 \: : \: a_0+a_1=a_3 \text{ and } a_2=0 \right\}$ be a subset of, $\mathcal{P}_3,$ the vector space of all polynomials of degree 3 or less.\\

Is $T$ a \emph{subspace} of $\mathcal{P}_3$? Justify your answer.\\

Note that a complete answer is not simply a computation (or computations) but includes and explanation in words indicating how that computation is used to derive your conclusion.\\

\newpage

%%%%% basis column space
%%%%S 2.3.3
\item (18 points) Let $A= \bpm 1&1&0&2&1\\1&0&1&2&1\\0&2&-2&0&0\\1&3&-2&1&-1\\ \epm.$\\
	\begin{enumerate}
	\item Find a basis for the column space of $A$.
	\vfill
	\item What is the rank of the matrix $A.$
	\vspace{.5in}
	\item Give an example of a vector $\vec{v}$ that is not in the column space of $A$ and demonstrate that your example is correct.
	\vspace{2in}
	\end{enumerate}
\newpage
%%%%Solutions to systems of equations
%%%% vector space 1.1.2, 1.1.3, 2,1,1
\item (18 points) Consider the system of linear equations below.
\begin{center} $\systeme{x+y+z+w=5,x-y+z-w=5,x-z+w=1}$ \end{center}

	\begin{enumerate}
	\item Solve the system of linear equations and express your answer in vector form.
	\vfill
	\item In your answer above, identify a particular solution and the solution set of the homogeneous system.
	\vspace{1in}
	\item Show that the solution set you found in part (a) is \emph{not} a vector space under the standard vector addition and scalar multiplication in  $\mathbb{R}^4.$
	\vfill
	\end{enumerate}
\newpage
%span -- just like Q4 re-do
\item (10 points) Do the vectors $\displaystyle \left\{ \bpm 1&0\\0&1 \epm,  \bpm 0&1\\1&0 \epm, \bpm 1&1\\0&1 \epm, \bpm 0&0\\1&0 \epm\right\}$ span the space $\mathcal{M}_{2 \times 2}$? Justify your answer.
\vfill
%basis
\item (10 points) The vectors $\displaystyle B= \langle 1, 1+x, 2x^2 \rangle$ form a basis for $\mathcal{P}_2$ the vector space of all polynomials of degree 2 or less. Write the representation of the vector $2-4x+5x^2$ in terms of the basis $B.$
\vfill
\newpage
%%%linear independence span, theoretical
%%%
\item (15 points) Let $\displaystyle S=\{ \vec{s_1},  \vec{s_2},  \vec{s_3}, \cdots  \vec{s_{10}} \}$ be a subset of vectors from the vector space $V.$ Assume that the the linear equation 
$$c_1\vec{s_1} +c_2\vec{s_2} +c_3\vec{s_3} +\cdots+c_{10}\vec{s_{10}} =\vec{0}$$

has $c_1=2, \: c_2=5, \: c_3=-4, \: c_4=c_5= \cdots = c_{10}=0$ as a solution.\\
	\begin{enumerate}
	\item Can you conclude that $S$ is linearly dependent or linearly independent? Why or why not? Explain your reasoning.
	\vfill
	\item What can you conclude about $\displaystyle [S-\{ \vec{s_1} \}]$ and $\displaystyle [S]$? Are they equal? Is one strictly smaller than the other? Do you have enough information to draw a conclusion? Explain your reasoning.
	\vfill
	\item Is it possible to determine how many solutions to the linear equation exist? Explain your reasoning.
	\vfill
	\end{enumerate}
\end{enumerate}
\newpage
\textbf{Extra Credit:} (5 points) Determine whether the vectors $f(x)=x^2, \: g(x)=2^x,$ and $h(x)=3^x$ are linearly independent in the vector space of functions from $\mathbb{R}$ to $\mathbb{R}.$ Justify your answer.

\end{document}
%%%%ENDDOCUMENT


