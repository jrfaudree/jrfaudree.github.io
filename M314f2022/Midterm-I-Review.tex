\documentclass[11pt,fleqn]{article} 
\usepackage[margin=0.8in, head=0.8in]{geometry} 
\usepackage{amsmath, amssymb, amsthm,systeme,xcolor}
\usepackage{fancyhdr} 
\usepackage{palatino, url, multicol}
\usepackage{graphicx} 
\usepackage[all]{xy}
\usepackage{polynom} 
\usepackage{pdfsync}
\usepackage{enumerate}
\usepackage{framed}
\usepackage{setspace}
\usepackage{array,tikz}

\newcommand{\bpm}{\begin{pmatrix}}
\newcommand{\epm}{\end{pmatrix}}

\pagestyle{fancy} 
\lfoot{UAF Linear}
\rfoot{Dimension }

\begin{document}
\renewcommand{\headrulewidth}{0pt}
\newcommand{\blank}[1]{\rule{#1}{0.75pt}}
\renewcommand{\d}{\displaystyle}

\vspace*{-0.7in}

\begin{center}
  \large \sc{Midterm I Review}
\end{center}

\noindent Logistics:\\
The midterm will be one hour. You may bring in a single sheet of hand written notes. You should bring some form of technology that will allow you to input a matrix and find its reduced echelon form.\\
There will be one problems for which you must describe and perform elementary row operations to transform a matrix into reduced echelon form.\\

\noindent Chapter 1: Linear Systems\\

\noindent Section 1.1.1 Gauss's Method\\

\textbf{Terminology:}  linear combination, elementary row operations, Gauss's Method, echelon form\\

\noindent Section 1.1.2 Describing the Solution Set\\

\textbf{Terminology:} echelon form, leading 1's, parametrized, matrix echelon form, column vector, row vector, components, scalar multiplication\\

\noindent Section 1.1.3 General = Particular + Homogeneous\\

\textbf{Terminology:} homogeneous system, particular solution, homogeneous solution\\

\textbf{Theorems/Lemmas} \\
(3.1) Every solution set can be expressed as the sum of a particular solution and the solution set of a homogeneous system.\\
(3.7) For a linear system and for any particular $p$, the solutions set equals $\{p+h \: | \: h  \text{satisfies the associated homogeneous system}\}$.\\
(3.10) Solutions sets of linear systems are either empty, unique, or have infinitely many elements.\\

\noindent Section 1.3.1 Gauss-Jordan Reduction\\

\textbf{Terminology:} Gauss-Jordan Reduction, reduced row echelon form, row equivalent matrices, \\

\textbf{Theorems/Lemmas} \\
(1.5) Elementary row operations are reversible.\\

\noindent Section 1.3.2 The Linear Combination Lemma\\

\textbf{Theorems/Lemmas:} \\
(2.3) Linear combinations of linear combinations are linear combinations.\\
(2.4) Row equivalent matrices have rows that are linear combinations of each other. That is, if $A'=rref(A),$ then the rows of $A'$ are a linear combinations of the rows of $A.$\\
(2.5) The nonzero rows of a matrix in reduced echelon form are not linear combinations of each other. Note that with the language of Section 2.2.1, we would restate this as: The nonzero rows of a matrix in reduced echelon form are linearly independent.\\
(2.6) The reduced echelon form of a matrix is unique (unlike the echelon form of a matrix).\\


\noindent Chapter 2: Vector Spaces\\

\noindent Section 2.1.1 Definition and Examples\\

\textbf{Terminology:} vector space, trivial vector space\\

\textbf{Theorems/Lemmas:} \\
(1.16) In any vector space $V,$ for any $\vec{v} \in V$ and $r \in \mathbb{R},$ the following are true: $0 \cdot \vec{v}=\vec{0}$ and $r \cdot \vec{0} = \vec{0}$ and $-1\cdot \vec{v} + \vec{v} = \vec{0}.$\\


\noindent Section 2.1.2 Subspaces and Spanning Sets\\

\textbf{Terminology:}  subspace, span\\

\textbf{Theorems/Lemmas:} \\
(2.9) Any set that is closed under $r_1\vec{v_1}+r_2\vec{v_2},$ for every $r_1,r_2 \in \mathbb{R}$ and every $\vec{v_1},\vec{v_2}.$\\
(2.15) In a vector space, the span of any subset of vectors is a subspace.\\

\noindent Section 2.2.1 Linear Independence\\

\textbf{Terminology:} linear dependence, linear independence\\

\textbf{Theorems/Lemmas:}  \\
(1.2) Let $S$ be a subset of the vector space $V.$ The addition of vector $\vec{v}$ to $S$ doesn't change span$(S)$ occurs if and only if $\vec{v}$ is already in span$(S)$. (That is, $\vec{v}$ can be written as a linear combination of vectors in $S.$ That is, $S \cup \{\vec{v}\}$ is linearly dependent)\\
(1.3) The deletion of the vector $\vec{v}$ from $S$ doesn't change span$(S)$ can occur if and only if $\vec{v}$ is already in span$(S)$.\\
(1.5) A subset $S=\{\vec{s_1},\vec{s_2},\cdots,\vec{s_n}\}$ of a vector space is linearly  independent if and only if the only solution to the system $c_1\vec{s_1}+c_2\vec{s_2}+\cdots +c_n\vec{s_n}=\vec{0}$ is $c_1=c_2=\cdots=c_n=0.$\\
(1.14) A set of vectors is linearly independent if and only if the removal of any vector from the set results in a smaller span.\\
(1.15) Let $S$ be a set of vectors and let $\vec{v} \not \in S.$ The set $S \cup \vec{v}$ is linearly independent if and only if $\vec{v} \not \in [S].$\\
(1.20) Any subset of a linearly independent set is also linearly independent. Any superset of a linearly dependent set is also linearly dependent.\\

\noindent Section 2.3.1 Basis\\

\textbf{Terminology:} basis,  representation of $\vec{v}$ with respect to a basis $B.$\\

\textbf{Theorems/Lemmas:}\\
(1.12) In any vector space $V$, a subset $B$ is a basis if and only if every vector of $V$ can be expressed as a linear combination of $B$ in exactly one way.\\

\noindent Section 2.3.2 Dimension\\

\textbf{Terminology:} finite-dimensional vector space, dimension\\

\textbf{Theorems/Lemmas:}\\
(2.3) Given two bases for the same vector space $V$, it is possible to exchange one vector from one basis with a vector from the other basis and still have a basis for $V.$\\
(2.4) If $V$ is finite-dimensional, then all bases have the same number of vectors.\\\\\\
(2.10) No linearly independent set from a finite dimensional vector space $V$ can have more vectors than $dim(V).$\\
(2.12) Any linearly independent set can be expanded to a basis.\\
(2.13) Any set that spans the vector space $V$ can be reduced to a basis.\\
(2.14) If $dim(V)=n$ and $S$ is a subset of $V$ with $n$ vectors, then $S$ spans $V$ if and only if $S$ is linearly independent. (Restate in a practical manner, if  $dim(V)=n$ and $S$ is a subset of $V$ with $n$ vectors, then determining whether $S$ is a bases is reduced to showing only ONE of linear independence OR spanning.)\\
(implied) Every set that spans the finite-dimensional vector space $V$ with dimension $n$ must have at least $n$ vectors.

\noindent Section 2.3.3 Vector Spaces and Linear Spaces\\

\textbf{Terminology}: column space, row space, column rank, row rank, rank of a matrix, transpose of a matrix\\

\textbf{Theorems/Lemmas:}\\
(3.4) The nonzero rows of a matrix in rref are linearly independent.\\
(3.10) Row operations do not change the column rank.\\
(3.11) Row rank equals column rank.\\
NOTE: We are omitting the last two results from this midterm. We will revisit these post midterm 1.\\

\newpage
\begin{center} Sample Problems \end{center}

\begin{enumerate}

\item Determine if the vector $(1,2,3,-2)$ is in the span of the vectors $(1,0,1,0), (0,1,1,1),(0,0,1,2).$

\item Solve each system below. Write your answer in parametrized form. Show your work.
	\begin{enumerate}
	\item $\systeme{2x+y-z=1,4x-y=3}$
	\item $\systeme{x-z=1,y+2z-w=3,x+2y+3z-w=7}$
	\item $\systeme{x-y+z=0,y+w=0,3x-2y+3z+w=0,-y-w=0}$
	\end{enumerate}
	
\item For each system above, describe the solutions as a particular and homogeneous solution.

\item For which values of $k$ are there no solutions, many solutions or a unique solution.\\

\begin{tabular}{rl}
$x-2y$&$=3$\\$2x+ky$&$=6$
\end{tabular}

\item Give examples of two 3 by 3 matrices in reduced echelon form that have their leading ones in the same columns but that are not row equivalent. Explain why your answer is correct.\\

\item Determine whether or not the following are vector spaces.
	\begin{enumerate}
	\item $ \{ a_0+a_1x \: : \: a_0+2a_1=0 \} $ under the usual operations of polynomial addition and scalar multiplication
	\item $ \{ \bpm a& b \\ c& d\\ \epm \: : \: a=b, c+d=1 \} $ under the usual operations of matrix addition and scalar multiplication
	\end{enumerate}
	
\item Determine if the set $\{  \bpm 1\\1\\2\\0\epm,  \bpm 0\\1\\1\\0\epm,  \bpm 2\\0\\1\\1\epm,  \bpm 1\\0\\0\\1\epm,\} $ spans all of $\mathbb{R}^4.$

\item Pick a random 4 by 5 matrix $A$. Find a basis for the row space of $A$. Find a basis for the column space of $A.$ Determine the rank of $A.$

\item Demonstrate that the set $S=\{1, 1+x, x+x^2, 2+x^3, x+2x^3\}$, a subset the vector space $\mathcal{P}_3$, is linearly dependent but that is spans $\mathcal{P}_3$. Find a subset of $S$ that forms a basis of $\mathcal{P}_3,$ call is $B.$ Write the polynomial $1+x-x^2-x^3$ with respect to the basis $B.$

\end{enumerate}
\end{document}