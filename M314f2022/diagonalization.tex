\documentclass[11pt,fleqn]{article} 
\usepackage[margin=0.8in, head=0.8in]{geometry} 
\usepackage{amsmath, amssymb, amsthm,systeme,xcolor}
\usepackage{fancyhdr} 
\usepackage{palatino, url, multicol}
\usepackage{graphicx} 
\usepackage[all]{xy}
\usepackage{polynom} 
\usepackage{pdfsync}
\usepackage{enumerate}
\usepackage{framed}
\usepackage{setspace}
\usepackage{array,tikz}
\pagestyle{fancy} 
\lfoot{UAF Linear}
\rfoot{}

\begin{document}
\renewcommand{\headrulewidth}{0pt}
\newcommand{\blank}[1]{\rule{#1}{0.75pt}}
\renewcommand{\d}{\displaystyle}

\newcommand{\bpm}{\begin{pmatrix}}
\newcommand{\epm}{\end{pmatrix}}
\newcommand{\bbm}{\begin{bmatrix}}
\newcommand{\ebm}{\end{bmatrix}}

\vspace*{-0.7in}

\begin{center}
  \large \sc{Diagonalization} \end{center}
 \begin{enumerate}
 \item \textbf{Definition 1.2 \S 5.2.1:} Let $A$ and $B$ be $n \times n$ matrices. We say $A$ \textbf{is similar to} $B$ if there exists an invertible matrix $P$ such that $P^{-1}AP=B.$

\item \textbf{Example:} Show that $A=\bpm -13&21\\7&1\epm$ and $B=\bpm 8&0\\0&-20\epm$ are similar. (Hint: Use $P=\bpm 1&3\\1&-1\epm.$)

\vfill

\item Think of some alternate ways to write the equation from Definition 1.2.\\
\vspace{1in}

\item  \textbf{Example:}  Show that $I_2=\bpm 1&0\\0&1 \epm$ is not similar to $R=\bpm 0&-1\\1&0 \epm.$
\vfill
 
 \item What would be easier to find by hand,  $A^{10}$ or $B^{10}$?
 \vfill
 \item Let $\vec{v_1}$ and $\vec{v_2}$ be the columns of $P$. Find $A\vec{v_1}$ and $A\vec{v_2}.$
 \vfill
 \item Give two distinct arguments to explain how you know $\{ \vec{v_1}, \vec{v_2}\}$ form a basis for $\mathbb{R}^2.$
 \vfill
 \item \textbf{Definition:} A square matrix is diagonalizable if $A$ is similar to a diagonal matrix.
 \newpage
 \item \textbf{Theorem:} An $n \times n$ matrix $A$ is diagonalizable if and only if $A$ has $n$ linearly independent eigenvectors.\\
 
 Define $P$ to be the $n \times n$ matrix such that its columns consist of $n$ linearly independent eigenvectors of $A.$ Then $$A=PDP^{-1}$$ where $D$ is a diagonal matrix such that the entries on the main diagonal are the eigenvalues associated with the eigenvectors of the columns of $P.$\\
 
 \item \textbf{Example from Worksheet on Monday 7 Nov \S 3.5.1 \& 3.5.2}: \\Define $h:\mathbb{R}^3 \to \mathbb{R}^3$ by $\bpm x\\y\\z \epm \mapsto \bpm y+z \\ x+z \\ x+y \epm.$
 
 \begin{enumerate} 
 \item Its matrix representation of $h$ with respect to the standard basis is: $\bpm 0&1&1\\1&0&1\\1&1&0\epm.$
 
 \item For basis $B=\left\langle \vec{b_1}=\bpm 1\\-1\\0\epm, \vec{b_2}=\bpm1\\1\\-2 \epm, \vec{b_3}=\bpm 1\\1\\1\epm \right\rangle,$ the change of basis matrix from basis $B$ to the standard basis is $\bpm 1&1&1\\-1&1&1\\0&-2&1\epm$
 
 \item The change of basis matrix from the standard basis to $B$ is $\bpm 1/2 & -1/2&0\\1/6&1/6&-1/3\\1/3&1/3&1/3\epm.$
 
 \item The matrix representation of $h$ with respect to basis $B$ is $\bpm -1&0&0\\0&-1&0\\0&0&2\epm.$
 
 \end{enumerate}
 \item Observations?
 \vfill
 \newpage
 \item Return to Example from \#10. \\Define $h:\mathbb{R}^3 \to \mathbb{R}^3$ by $\bpm x\\y\\z \epm \mapsto \bpm y+z \\ x+z \\ x+y \epm.$\\
 \begin{itemize}
 \item Find the matrix of the linear transformation, say $A.$ 
 \item Find the characteristic polynomial of $A$ and use it to find any eigenvalues of $A$. 
 \item For each eigenvalue, find a basis for the corresponding eigenspace. 
 \item Show that $A$ is diagonalizable.
 \end{itemize}

   \end{enumerate}
  \end{document}