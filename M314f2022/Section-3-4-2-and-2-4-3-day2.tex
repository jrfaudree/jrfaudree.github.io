\documentclass[11pt,fleqn]{article} 
\usepackage[margin=0.8in, head=0.8in]{geometry} 
\usepackage{amsmath, amssymb, amsthm,systeme,xcolor}
\usepackage{fancyhdr} 
\usepackage{palatino, url, multicol}
\usepackage{graphicx} 
\usepackage[all]{xy}
\usepackage{polynom} 
\usepackage{pdfsync}
\usepackage{enumerate}
\usepackage{framed}
\usepackage{setspace}
\usepackage{array,tikz}
\pagestyle{fancy} 
\lfoot{UAF Linear}
\rfoot{}

\begin{document}
\renewcommand{\headrulewidth}{0pt}
\newcommand{\blank}[1]{\rule{#1}{0.75pt}}
\renewcommand{\d}{\displaystyle}

\newcommand{\bpm}{\begin{pmatrix}}
\newcommand{\epm}{\end{pmatrix}}
\newcommand{\bbm}{\begin{bmatrix}}
\newcommand{\ebm}{\end{bmatrix}}

\vspace*{-0.7in}

\begin{center}
  \large \sc{Section 3.4.2 and 3.4.3: Composition of Linear Maps and Matrix Multiplication (day 2)}
\end{center}

\begin{enumerate}
\item \textbf{Take-aways from Monday} Let $f: V \to W$ and $g:W \to Y$ be linear maps with matrix representations $A$ and $B$ respectively. Then, \\
\begin{itemize}
\item the matrix representation of $(g \circ f):$\underline{\hspace{1in}} is \underline{\hspace{1in}} with dimension \underline{\hspace{1in}}\\
\item the function $(g \circ f)$ is a \\
\item Function composition - matrix multiplication \underline{\hspace{1in}} commutative.\\
\item Function composition - matrix multiplication \underline{\hspace{1in}} associative.\\
\vspace{1in}
\item Function composition - matrix multiplication \underline{\hspace{1in}} distributive.\\
\vspace{1in}
\end{itemize}
\item Terminology
	\begin{enumerate}
	\item main diagonal
	\vfill
	\item identity matrix
	\vfill
	\newpage
	\item diagonal matrix
	\vfill
	\item permutation matrix
	\vfill
	\item elementary (reduction) matrices
	\vfill
	\end{enumerate}
\end{enumerate}
\end{document}