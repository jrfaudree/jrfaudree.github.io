\documentclass[11pt,fleqn]{article} 
\usepackage[margin=0.8in, head=0.8in]{geometry} 
\usepackage{amsmath, amssymb, amsthm,systeme,xcolor}
\usepackage{fancyhdr} 
\usepackage{palatino, url, multicol}
\usepackage{graphicx} 
\usepackage[all]{xy}
\usepackage{polynom} 
\usepackage{pdfsync}
\usepackage{enumerate}
\usepackage{framed}
\usepackage{setspace}
\usepackage{array,tikz}
\pagestyle{fancy} 
\lfoot{UAF Calculus 1}
\rfoot{review functions }

\begin{document}
\renewcommand{\headrulewidth}{0pt}
\newcommand{\blank}[1]{\rule{#1}{0.75pt}}
\renewcommand{\d}{\displaystyle}
\vspace*{-0.7in}
\begin{center}
  \large \sc{Section One.I.2: Describing the Solutions Set (aka aesthetics)}
\end{center}
Goals: (1) Reframe SoLE (and their solutions) in terms of matrices (vectors), (2) Review elementary vector notation and operations.

\noindent\hrulefill

\noindent How we solved a SoLE in Section One.I.1\\

\noindent $\systeme{x_1-2x_2+x_3=0,2x_2-8x_3=8,5x_1-5x_3=10} \quad \textcolor{blue}{\overrightarrow{\rho_3-5\rho_1 \mapsto \rho_3}} \quad \systeme{x_1-2x_2+x_3=0,2x_2-8x_3=8,10x_2-10x_3=10}\quad \textcolor{blue}{\overrightarrow{\rho_3-5\rho_2 \mapsto \rho_3}} \quad \systeme{x_1-2x_2+x_3=0,2x_2-8x_3=8,30x_3=-30}$

Conclude: $x_3=-1,\: x_2=0,\: x_1=1$ via back substitution.

\noindent\hrulefill

\noindent How we solved a SoLE in Section One.I.2\\

\noindent $\systeme{x_1-2x_2+x_3=0,2x_2-8x_3=8,5x_1-5x_3=10}$ \hspace{2.5in} $\systeme{x_1-2x_2+x_3=0,2x_2-8x_3=8,30x_3=-30}$\quad Solve as before.\\

\quad\hspace{.5in}\textcolor{blue}{$\big\Downarrow$}\hspace{4.5in}\textcolor{blue}{$\big\Uparrow$}\\

$\begin{bmatrix} 1&-2&1&0\\0&2&8&8\\5&0&-5&10 \end{bmatrix} \quad \textcolor{blue}{\overrightarrow{\rho_3-5\rho_1 \mapsto \rho_3}} \quad \begin{bmatrix} 1&-2&1&0\\0&2&8&8\\0&10&-10&10 \end{bmatrix} \quad \textcolor{blue}{\overrightarrow{\rho_3-5\rho_2 \mapsto \rho_3}} \quad 
\begin{bmatrix} 1&-2&1&0\\0&2&8&8\\0&0&30&-30 \end{bmatrix}$\\

\noindent\hrulefill

\quad

Example 1: Solve the SoLE $\systeme{w+y+2z=0,w+2x+y+6z=8,-w+2x+2y+2z=20}$ by converting to matrices.

\newpage
 Vector Review\\
 \begin{itemize}
  \item vector versus scalar?\\
 
\item vector addition:\\
 
 (requires the same dimensions!)\\
 
 \item scalar multiplication:\\
 \vspace{0.6in}
 \end{itemize}

Return to Example 1. Write its solution in vector form.\\

\vspace{2in}

Example 2: Assume that a SoLE has a matrix echelon form $A=\begin{bmatrix} 1&2&0&1&1&5\\ 0&2&-4&0&2&-6\\ 0&0&0&0&1&4\end{bmatrix}$. Find the solution set of the SoLE.
\vfill


\end{document}