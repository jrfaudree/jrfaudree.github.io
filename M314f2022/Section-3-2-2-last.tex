\documentclass[11pt,fleqn]{article} 
\usepackage[margin=0.8in, head=0.8in]{geometry} 
\usepackage{amsmath, amssymb, amsthm,systeme,xcolor}
\usepackage{fancyhdr} 
\usepackage{palatino, url, multicol}
\usepackage{graphicx} 
\usepackage[all]{xy}
\usepackage{polynom} 
\usepackage{pdfsync}
\usepackage{enumerate}
\usepackage{framed}
\usepackage{setspace}
\usepackage{array,tikz}
\pagestyle{fancy} 
\lfoot{UAF Linear}
\rfoot{}

\begin{document}
\renewcommand{\headrulewidth}{0pt}
\newcommand{\blank}[1]{\rule{#1}{0.75pt}}
\renewcommand{\d}{\displaystyle}

\newcommand{\bpm}{\begin{pmatrix}}
\newcommand{\epm}{\end{pmatrix}}
\newcommand{\bbm}{\begin{bmatrix}}
\newcommand{\ebm}{\end{bmatrix}}

\vspace*{-0.7in}

\begin{center}
  \large \sc{Section 3.2.2 Range Space and Null Space (day 3)}
\end{center}

\begin{enumerate}
\item Summary of our 3.2.2 Examples
	\begin{enumerate}
	\item $f: \mathbb{R}^2 \to \mathbb{R}$ defined as $f\left(\bbm x \\y \ebm\right)= x+y$\\
	$\mathcal{R}(f)=\mathbb{R},\:$ $\text{rank}(f)=1, \: \mathcal{N}(f)=f^{-1}(0)=\text{span}\left( \left\{ \bbm 1 \\ -1 \ebm \right\}\right), \:\text{nullity}(f)=1$\\
	dimension of domain $= 2=1+1= \text{rank}(f) + \text{nullity}(f)$\\
	
	\item $f: \mathbb{R}^2 \to \mathbb{R}^3$ defined as $f\left(\bbm x \\y \ebm\right)=\bbm x\\y\\y \ebm$\\
	$\mathcal{R}(f)=\text{span}\left( \left\{ \bbm 1\\0\\0\\ \ebm, \bbm 0\\ 1\\1\\ \ebm \right\} \right),\:$ $\text{rank}(f)=2, \: \mathcal{N}(f)=f^{-1}(\bbm 0\\0\\0 \ebm)= \left\{ \bbm 0 \\ 0 \ebm \right\} \:\text{nullity}(f)=0$\\
	dimension of domain $= 2= 2+0=\text{rank}(f) + \text{nullity}(f)$\\
	\end{enumerate}
\item (Theorem 2.14 and Corollary 2.17) Assume $f:V \to W$ is a linear map between vector spaces $V$ and $W.$
\vspace{0.7in}\\
Why?\\
\vfill
\item (Theorem 2.20) Assume $f:V \to W$ is a linear map between vector spaces $V$ and $W$ and $dim(V)=n.$ The following are equivalent statements.\\

	\begin{enumerate}
	\item The function $f$ is a one-to-one map.\\
	
	\item The rank of $f$ is\\
	
	\item The nullity of $f$ is\\
	
	\item The relation $f^{-1}$ is\\
	
	\item If $\langle \vec{b_1}, \vec{b_2}, \cdots \vec{b_n}\rangle$ forms a basis for $V$, then $\langle f(\vec{b_1}), f(\vec{b_2}), \cdots f(\vec{b_n})\rangle$\\
	
		\end{enumerate}
\end{enumerate}
\end{document}