
\documentclass[12pt]{article}

% Layout.
\usepackage[top=1in, bottom=0.75in, left=1in, right=1in, headheight=1in, headsep=6pt]{geometry}

% Fonts.
\usepackage{mathptmx}
\usepackage[scaled=0.86]{helvet}
\renewcommand{\emph}[1]{\textsf{\textbf{#1}}}

% TiKZ.
\usepackage{tikz, pgfplots}
\usetikzlibrary{calc}
\pgfplotsset{compat = newest}
 
\pgfplotsset{my style/.append style={axis x line=middle, axis y line=
middle, xlabel={$x$}, ylabel={$y$}, axis equal }}

% Misc packages.
\usepackage{amsmath,amssymb,latexsym,systeme}
\usepackage{graphicx}
\usepackage{array}
\usepackage{xcolor}
\usepackage{multicol}

% Commands to set various header/footer components.
\makeatletter
\def\doctitle#1{\gdef\@doctitle{#1}}
\doctitle{Use {\tt\textbackslash doctitle\{MY LABEL\}}.}
\def\docdate#1{\gdef\@docdate{#1}}
\docdate{Use {\tt\textbackslash docdate\{MY DATE\}}.}
\def\doccourse#1{\gdef\@doccourse{#1}}
\let\@doccourse\@empty
\def\docscoring#1{\gdef\@docscoring{#1}}
\let\@docscoring\@empty
\def\docversion#1{\gdef\@docversion{#1}}
\let\@docversion\@empty
\makeatother

% Headers and footers layout.
\makeatletter
\usepackage{fancyhdr}
\pagestyle{fancy}
\fancyhf{} % Clears all headers/footers.
\lhead{\baselineskip 30pt
%\emph{\@doctitle\hfill\@docdate}
\emph{\@docdate\hfill\@doctitle}
\ifnum \value{page} > 1\relax\else\\
\emph{Name: \rule{3.5in}{1pt}\hfill \@docscoring}\fi}
\rfoot{\emph{\@docversion}}
\lfoot{\emph{\@doccourse}}
\cfoot{\emph{\thepage}}
\renewcommand{\headrulewidth}{0pt}%
\makeatother

% Paragraph spacing
\parindent 0pt
\parskip 6pt plus 1pt

% A problem is a section-like command. Use \problem{5} to
% start a problem worth 5 points.
\newcounter{probcount}
\newcounter{subprobcount}
\setcounter{probcount}{0}
\newcommand{\problem}[1]{%
\par
\addvspace{4pt}%
\setcounter{subprobcount}{0}%
\stepcounter{probcount}%
\makebox[0pt][r]{\emph{\arabic{probcount}.}\hskip1ex}\emph{[#1 points]}\hskip1ex}
\newcommand{\thesubproblem}{\emph{\alph{subprobcount}.}}

% Subproblems are an enumerate-like environment with a consistent
% numbering scheme. 
% Use \begin{subproblems}\item...\item...\end{subproblems}
\newenvironment{subproblems}{%
\begin{enumerate}%
\setcounter{enumi}{\value{subprobcount}}%
\renewcommand{\theenumi}{\emph{\alph{enumi}}}}%
{\setcounter{subprobcount}{\value{enumi}}\end{enumerate}}

% Blanks for answers in normal and math mode.
\newcommand{\blank}[1]{\rule{#1}{0.75pt}}
\newcommand{\mblank}[1]{\underline{\hspace{#1}}}
\def\emptybox(#1,#2){\framebox{\parbox[c][#2]{#1}{\rule{0pt}{0pt}}}}

% Misc.
\renewcommand{\d}{\displaystyle}
\newcommand{\ds}{\displaystyle}
\def\bc{\begin{center}}
\def\ec{\end{center}}
\def\be{\begin{enumerate}}
\def\ee{\end{enumerate}}
\def\bpm{\begin{pmatrix}}
\def\epm{\end{pmatrix}}
\def\bbm{\begin{bmatrix}}
\def\ebm{\end{bmatrix}}


\doctitle{Math 314: Quiz 10}
\docdate{Nov 16, 2022}
\doccourse{Linear}
\docversion{v-1}
\docscoring{\blank{0.8in} / 10}
\begin{document}
%\textbf{Please circle your instructor's name:} \hfill Leah Berman  \hfill   Jill Faudree\\

There are 10 points possible on this quiz. You may use technology but you must demonstrate what you are using technology for.

Questions below concern bases of $\mathbb{R}^2$ 

$$\mathcal{E}_2=\left \langle \: \bpm 1\\0  \epm, \bpm 0\\1  \epm \: \right \rangle, \quad \quad
B=\left \langle \:  \bpm 0\\-1  \epm, \bpm 2\\1  \epm \: \right \rangle, \quad \quad
D=\left \langle \: \bpm 1\\1  \epm, \bpm -1\\1  \epm   \: \right \rangle. $$


\begin{enumerate}
\item (2 points) Find \emph{directly} the representation of the vector $\vec{v}$ with respect to $\mathcal{E}_2$ assuming $Rep_B(\vec{v})=\bpm 2\\3\epm_B.$
\vfill
\item (2 points) Find the matrix $A_1=Rep_{B,\mathcal{E}_2}(id).$
\vfill
\item (1 point) Use matrix $A_1$ to find $Rep_{\mathcal{E}_2} (\vec{v}).$
\vfill
\newpage
\item (5 points) Suppose that, with respect to $\mathcal{E}_2$ for both domain and codomain, the transformation $h: \mathbb{R}^2 \to \mathbb{R}^2$ is represented by the matrix $H=\bpm 6&-8\\2&-8 \epm.$ Use change of basis matrices to represent $h$ with respect to the bases 

$$
B=\left \langle \:  \bpm 0\\-1  \epm, \bpm 2\\1  \epm \: \right \rangle \quad  \text{ and } \quad
D=\left \langle \: \bpm 1\\1  \epm, \bpm -1\\1  \epm   \: \right \rangle. $$
where $B$ is the basis for the domain and $D$ is the basis for the range. 

\end{enumerate}	
\end{document}