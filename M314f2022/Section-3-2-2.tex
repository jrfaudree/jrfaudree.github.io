\documentclass[11pt,fleqn]{article} 
\usepackage[margin=0.8in, head=0.8in]{geometry} 
\usepackage{amsmath, amssymb, amsthm,systeme,xcolor}
\usepackage{fancyhdr} 
\usepackage{palatino, url, multicol}
\usepackage{graphicx} 
\usepackage[all]{xy}
\usepackage{polynom} 
\usepackage{pdfsync}
\usepackage{enumerate}
\usepackage{framed}
\usepackage{setspace}
\usepackage{array,tikz}
\pagestyle{fancy} 
\lfoot{UAF Linear}
\rfoot{}

\begin{document}
\renewcommand{\headrulewidth}{0pt}
\newcommand{\blank}[1]{\rule{#1}{0.75pt}}
\renewcommand{\d}{\displaystyle}

\newcommand{\bpm}{\begin{pmatrix}}
\newcommand{\epm}{\end{pmatrix}}
\newcommand{\bbm}{\begin{bmatrix}}
\newcommand{\ebm}{\end{bmatrix}}

\vspace*{-0.7in}

\begin{center}
  \large \sc{Section 3.2.2 Range Space and Null Space}
\end{center}
\begin{enumerate}
\item \textbf{Review} For vector spaces $V$ and $W$, the function $f:V \to W$ is called a \emph{homomorphism} or \emph{linear map} if
\vspace{0.5in}

\item (Some preliminary terminology) Assume $f:V \to W$, $S \subseteq V$, and $T \subseteq W.$ \\

\textbf{image:}\\

\vspace{0.25in}

\textbf{inverse image:}\\

\vspace{0.25in}

\textbf{inverse function:}\\

\vspace{0.25in}

\item (Lemma 2.1, Definition 2.2, and some notation) Assume $f:V \to W$ is a linear map between vector spaces $V$ and $W$ and assume $V'$ is a subspace of $V$.  
\vfill
\newpage
\item (Lemma 2.10, Definition 2.11, and some notation) Assume $f:V \to W$ is a linear map between vector spaces $V$ and $W$ and assume $W'$ is a subspace of $W$.  
\vfill
\item (Theorem 2.14 and Corollary 2.17) Assume $f:V \to W$ is a linear map between vector spaces $V$ and $W.$
\vfill
\item (Theorem 2.20) Assume $f:V \to W$ is a linear map between vector spaces $V$ and $W$ and $dim(V)=n.$ The following are equivalent statements.
\vfill
\end{enumerate}
\newpage
\textbf{Examples:}
	\begin{enumerate}
	%example 1
	\item $f: \mathbb{R}^2 \to \mathbb{R}$ defined as $f\left(\bbm x \\y \ebm\right)= x+y$
	\begin{enumerate}
	\item Find the image of $\bbm 1\\2 \ebm.$\\
	
	\item Find the image of $S$ where $S=\{ \bbm x\\10-x \ebm \: : \: x \in \mathbb{R}\}$
	\vfill
	\item Find the inverse image of 
	\vfill
	\item Find the range space of $f$ and determine its rank.
	\vfill
	\item Find the null space and nullity of $f.$
	\vfill
	\end{enumerate}
\newpage
%example 2
	\item $f: \mathbb{R}^2 \to \mathbb{R}^3$ defined as $f\left(\bbm x \\y \ebm\right)=\bbm x\\y\\y \ebm$
	\begin{enumerate}
	\item Find the image of $\bbm 1\\2 \ebm.$\\
	
	\item Find the image of $S$ where $S=\{ \bbm 0\\y \ebm \: : \: y \in \mathbb{R}\}$
	\vfill
	\item Find the inverse image of 
	\vfill
	\item Find the range space of $f$ and determine its rank.
	\vfill
	\item Find the null space and nullity of $f.$
	\vfill
	\end{enumerate}
	\end{enumerate}

\end{document}