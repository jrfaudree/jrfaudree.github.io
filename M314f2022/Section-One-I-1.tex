\documentclass[11pt,fleqn]{article} 
\usepackage[margin=0.8in, head=0.8in]{geometry} 
\usepackage{amsmath, amssymb, amsthm}
\usepackage{fancyhdr} 
\usepackage{palatino, url, multicol}
\usepackage{graphicx} 
\usepackage[all]{xy}
\usepackage{polynom} 
\usepackage{pdfsync}
\usepackage{enumerate}
\usepackage{framed}
\usepackage{setspace}
\usepackage{array,tikz}
\pagestyle{fancy} 
\lfoot{UAF Calculus 1}
\rfoot{review functions }

\begin{document}
\renewcommand{\headrulewidth}{0pt}
\newcommand{\blank}[1]{\rule{#1}{0.75pt}}
\renewcommand{\d}{\displaystyle}
\vspace*{-0.7in}
\begin{center}
  \large \sc{Section One.I.1: Gauss's Method}
\end{center}
Goals:
\begin{itemize}
\item Know Terminology: linear combination, linear equation, coefficients, constant, a system of linear equations, a solution to a system of linear equations, elementary row operations
\item Understand an algorithm: Gauss's Method (or Gaussian Elimination). Understanding an algorithm means knowing \emph{when} to apply it, \emph{how} to apply it and correctly \emph{interpreting} the results. 
\end{itemize}
\begin{enumerate}
\item linear combination\\
\vspace{.5in}
\item linear equation\\
\vspace{.5in}
\item system of linear equations\\
\vspace{1in}
\item a solutions to a system of linear equations\\
\vspace{.5in}
\item elementary row operations\\
\vspace{.5in}
\item Gauss's Method \\
\vfill
\newpage
\item Example 1: \quad \quad $\begin{matrix}
x_1&-&2x_2&+&x_3&=&0\\
&&2x_2&-&8x_3&=&8\\
5x_1&&&-&5x_3&=&10
\end{matrix}$
\vfill
\item Example 2: \quad \quad $\begin{matrix}
&&x_2&-&4x_3&=&8\\
2x_1&-&3x_2&+&2x_3&=&1\\
4x_1&-&8x_2&+&12x_3&=&1
\end{matrix}$

\vfill
\item Example 3: \quad \quad $\begin{matrix}
&&x_2&-&4x_3&=&8\\
2x_1&-&3x_2&+&2x_3&=&1\\
2x_1&-&2x_2&-&2x_3&=&9
\end{matrix}$

\vfill
\end{enumerate}

\end{document}