\documentclass[11pt,fleqn]{article} 
\usepackage[margin=0.8in, head=0.8in]{geometry} 
\usepackage{amsmath, amssymb, amsthm,systeme,xcolor}
\usepackage{fancyhdr} 
\usepackage{palatino, url, multicol}
\usepackage{graphicx} 
\usepackage[all]{xy}
\usepackage{polynom} 
\usepackage{pdfsync}
\usepackage{enumerate}
\usepackage{framed}
\usepackage{setspace}
\usepackage{array,tikz}

\newcommand{\bpm}{\begin{pmatrix}}
\newcommand{\epm}{\end{pmatrix}}

\pagestyle{fancy} 
\lfoot{UAF Linear}
\rfoot{Dimension }

\begin{document}
\renewcommand{\headrulewidth}{0pt}
\newcommand{\blank}[1]{\rule{#1}{0.75pt}}
\renewcommand{\d}{\displaystyle}

\vspace*{-0.7in}

\begin{center}
  \large \sc{Midterm II Review}
\end{center}

\noindent Logistics:\\
The midterm will be one hour. You may bring in a single sheet of hand written notes. You should bring some form of technology that will allow you to: put a matrix into reduced row echelon form and to multiply matrices.\\
There will be one problem for which you must demonstrate that you know how to multiply matrices by hand.\\

%%%SECTION
\noindent Section 3.1.1: Definition of Isomorphisms\\

\textbf{Terminology:} isomorphism, image\\

\textbf{Lemmas/Theorems:}  \\
Lemma 1.10: An isomorphism maps the zero vector of the domain to the zero vector of the codomain.\\
Lemma 1.11: This give alternate ways to demonstrate a map is linear.\\

\textbf{Sample Problems:} \\
Given a map, find the image of elements in the domain (1.13). \\
Verify that a given map is (or is not) an isomorphism (1.16, 1.17).\\

%%%SECTION
\noindent Section 3.1.2: Dimension Characterizes Isomorphism\\

%\textbf{Terminology:} \\

\textbf{Lemmas/Theorems:} \\
Theorem 2.3: Vector spaces are isomorphic if and only if they have the same dimension.\\

\textbf{Sample Problems:} \\
Observe that you can determine if two vector spaces are or are not isomorphic based on their dimension. (2.10)\\

%%%SECTION
\noindent Section 3.2.1: Definition of Homomorphism \\

\textbf{Terminology:} homomorphism, zero homomorphism, linear extension of a map\\

\textbf{Lemmas/Theorems:} \\
Lemma 1.6: A linear map send the zero vector of the domain to the zero vector of the range.\\
Theorem 1.9: A homomorphism is determined by the action on a basis. \\

\textbf{Sample Problems:} \\
Determine whether or not a given map is linear (1.18, 1.19)\\
Verify the a particular map is a  homomorphism (1.22)\\

%%%SECTION
\noindent Section 3.2.2:  Range Space and Null Space\\

\textbf{Terminology:} range space, rank of a homomorphism, null space, nullity of a homomorphism, image (again), inverse image\\

\textbf{Lemmas/Theorems:} \\
Lemma 2.1: Under a homomorphism, the image of a subspace of the domain is a subspace of the codomain. In particular, the range of the homomorphism is a subspace of the codomain.\\
Lemma 2.10: For any homomorphism, the inverse image of a subspace of the range is a subspace of the domain. In particular, the inverse image of the zero vector in the range is a subspace (ie null space) of the domain.\\
Theorem 2.14: For any linear map, the rank of the map plus the nullity of the map must equal the dimension of the domain.\\
Corollary 2.17: The rank of a linear map is less than or equal to the dimension of the domain and equality holds if and only if the nullity is zero.\\
Lemma 2.18: Under a linear map, the image of a set of linearly dependent vectors must be linearly dependent.\\
Theorem 2.20: This is a ``The following are equivalent..." list starting with the statement ``$h: V \to W$ is one to one" where $dim(V)=n.$\\


\textbf{Sample Problems:} \\
Determine whether or not a vector is in the range or null space of a linear map. (2.21)\\
Determine the range space or null space of a linear map (and thus the rank and nullity of the map). (2.23, 2.24, 2.25)\\
Find the inverse image of a vector (2.30)\\

%%%SECTION
\noindent Section 3.3.1: Representing Linear Maps with Matrices\\

\textbf{Terminology:} the matrix representation of a linear map with respect to bases for domain and codomain ($Rep_{B,D}(h)$), the matrix-vector product (which we now interpret as matrix multiplication...) \\

%\textbf{Lemmas/Theorems:} \\

\textbf{Sample Problems:} \\
How to use the matrix representation to find the image of a vector. (1.13)\\
How to use the image of basis elements to find the image of a vector (1.17)\\
Find the matrix representation of a linear map with respect to given bases (1.19, 1.21, 1,27*) . \\
*Note that because of sections 3.5.1 and 3.5.2, we now have a different way of thinking about 1.27.\\

%%%SECTION
\noindent Section 3.3.2: Any Matrix Can Represent a Linear Map \\

\textbf{Terminology:} a nonsingular linear map\\

\textbf{Lemmas/Theorems:} \\
Theorem 2.2: Any matrix can be interpreted as a representation of a linear map. The representation is not unique as it depends on the bases.\\
Quick Reminder: If $H$ is an $m \times n$ matrix that represents a linear map from $V$ to $W$, then $dim(W)=m$ and $dim(V)=n.$\\
Theorem 2.4: The rank of a matrix equal the rank of any linear map it represents.\\
Corollary 2.6: Let $h$ be a linear map represented by matrix $H$. The map $h$ is onto if and only if rank of $H$ equal the number of rows of $H.$ The map $h$ is one to one if and only if rank of $H$ equal the number of columns of $H.$ \\

\textbf{Sample Problems:} \\
Determine if a map is singular or nonsingular from its matrix representation (2.13)\\
Determine the image of a vector given a matrix representation and corresponding bases. (2.14, 2.15, 2.16)\\

%%%SECTION
\noindent Section 3.4.1: Sums and Scalar Products of Matrices \\

\textbf{Terminology:} sum of two matrices, scalar multiple of two matrices \\

\textbf{Lemmas/Theorems:} \\
Theorem 1.4: This states the expected relationship between matrix addition/scalar multiplication of matrices to that of linear maps.\\

\textbf{Sample Problems:} \\
Be able to perform matrix addition and scalar multiplication of matrices.\\

%%%SECTION
\noindent Section 3.4.2: Matrix Multiplication\\

\textbf{Terminology:} matrix multiplication \\

\textbf{Lemmas/Theorems:} \\
Theorem 2.7: A composition of linear maps is represented by matrix multiplication of their respective representations.\\
Theorem 2.12: Matrix multiplication (if defined) is associative and has the expected distributive laws.\\

\textbf{Sample Problems:} \\
Be able to perform matrix multiplication by hand. (2.14)\\
Be able to find the matrix of a composition of linear functions using matrix representations. (2.19)\\

%%%SECTION
\noindent Section 3.4.3: Mechanics of Matrix Multiplication\\

\textbf{Terminology:} main diagonal, identity matrix, diagonal matrix, permutation matrix, elementary reduction matrices \\

\textbf{Lemmas/Theorems:} \\
Corollary 3.23: Elementary row and column operations can be performed with a sequence of products of elementary reduction matrices\\

\textbf{Sample Problems:} \\
Find the elementary reduction matrix such that appropriate multiplication performs a specific row or column operation. (3.27)\\


%%%SECTION
\noindent Section 3.4.4: Inverses \\

\textbf{Terminology:} invertible matrix, \\

\textbf{Lemmas/Theorems:} \\
Theorem 4.3: A matrix is invertible if and only if it is nonsingular.\\
Lemma 4.7: A matrix is invertible if and only if it can be written as a product of elementary reduction matrices.\\

\textbf{Sample Problems:} \\
Be able to find an inverse of a matrix \emph{using only the reduced row echelon operator} and be able to determine that no inverse exists (4.15).\\
Be able to use matrix algebra (and inverses) to solve matrix problems. (4.18)\\

%%%SECTION
\noindent Section 3.5.1 \& 3.5.2: Changing Representations of Vectors and Linear Maps\\

\textbf{Terminology:} change of basis matrix ($Rep_{B,D}(id)$), matrix representation of a linear map with respect to given bases (again) $Rep_{B,D}(id)$\\

\textbf{Lemmas/Theorems:} \\
Lemma 1.5: A matrix is a change of basis matrix if and only if it is nonsingular.\\

\textbf{Sample Problems:} \\
Find the change of basis matrix ($Rep_{B,D}(id)$) given $B$ and $D$. (S 3.5.1 \#1.9)\\
Be able to use a change of basis matrix (S 3.5.1 \#1.12)\\
Be able to change the matrix representation of a linear map from one set of bases to another (S 3.5.2 \#2.14, 2.17)



\end{document}