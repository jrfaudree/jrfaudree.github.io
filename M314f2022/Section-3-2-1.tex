\documentclass[11pt,fleqn]{article} 
\usepackage[margin=0.8in, head=0.8in]{geometry} 
\usepackage{amsmath, amssymb, amsthm,systeme,xcolor}
\usepackage{fancyhdr} 
\usepackage{palatino, url, multicol}
\usepackage{graphicx} 
\usepackage[all]{xy}
\usepackage{polynom} 
\usepackage{pdfsync}
\usepackage{enumerate}
\usepackage{framed}
\usepackage{setspace}
\usepackage{array,tikz}
\pagestyle{fancy} 
\lfoot{UAF Linear}
\rfoot{Dimension }

\begin{document}
\renewcommand{\headrulewidth}{0pt}
\newcommand{\blank}[1]{\rule{#1}{0.75pt}}
\renewcommand{\d}{\displaystyle}

\newcommand{\bpm}{\begin{pmatrix}}
\newcommand{\epm}{\end{pmatrix}}

\vspace*{-0.7in}

\begin{center}
  \large \sc{Section 3.2.1 Homomorphisms/Linear Maps}
\end{center}
\begin{enumerate}
\item \textbf{Review} For vector spaces $V$ and $W$, the function $f:V \to W$ is called an \emph{isomorphism} if 
\vfill
\item \textbf{Review} For vector spaces $V$ and $W$, the function $f:V \to W$ is called a \emph{homomorphism} or \emph{linear map} if
\vfill

\item \textbf{Examples:}
	\begin{enumerate}
	\item $f: \mathcal{P}_2 \to \mathbb{R}^2$ defined as  $f(ax^2+bx+c)=\bpm a \\ b+c \epm$ (or $ax^2+bx+c \mapsto \bpm a \\ b+c \epm$)
	\vspace{2in}
	\item The \emph{projection} $\pi: \mathbb{R}^3 \to \mathbb{R}^2$ defined by $\bpm x \\ y \\ z \epm \mapsto \bpm x \\ y \epm$
	\vspace{2in}
	\item The zero homomorphism $z: \mathbb{R}^4 \to \mathcal{M}_{2 \times 2}$
	\vfill
	\end{enumerate}
\newpage
\item Lemma 1.6: \\
\vspace{.5in}
\item Theorem 1.9:\\
\vspace{1in}
\item Consequence of Theorem 1.9: Define a homomorphism $h$ from $\mathcal{P}_2$ to $\mathcal{M}_{2 \times 2}$ by defining $h$ on a basis of $\mathcal{P}_2$ and demonstrating how $h$ operates on an arbitrary element of $\mathcal{P}_2$.
\vfill
\item Definition 1.12\\
\vfill
\end{enumerate}
\end{document}