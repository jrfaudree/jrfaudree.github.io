
\documentclass[12pt]{article}

% Layout.
\usepackage[top=1in, bottom=0.75in, left=1in, right=1in, headheight=1in, headsep=6pt]{geometry}

% Fonts.
\usepackage{mathptmx}
\usepackage[scaled=0.86]{helvet}
\renewcommand{\emph}[1]{\textsf{\textbf{#1}}}

% TiKZ.
\usepackage{tikz, pgfplots}
\usetikzlibrary{calc}
\pgfplotsset{compat = newest}
 
\pgfplotsset{my style/.append style={axis x line=middle, axis y line=
middle, xlabel={$x$}, ylabel={$y$}, axis equal }}

% Misc packages.
\usepackage{amsmath,amssymb,latexsym,systeme}
\usepackage{graphicx}
\usepackage{array}
\usepackage{xcolor}
\usepackage{multicol}

% Commands to set various header/footer components.
\makeatletter
\def\doctitle#1{\gdef\@doctitle{#1}}
\doctitle{Use {\tt\textbackslash doctitle\{MY LABEL\}}.}
\def\docdate#1{\gdef\@docdate{#1}}
\docdate{Use {\tt\textbackslash docdate\{MY DATE\}}.}
\def\doccourse#1{\gdef\@doccourse{#1}}
\let\@doccourse\@empty
\def\docscoring#1{\gdef\@docscoring{#1}}
\let\@docscoring\@empty
\def\docversion#1{\gdef\@docversion{#1}}
\let\@docversion\@empty
\makeatother

% Headers and footers layout.
\makeatletter
\usepackage{fancyhdr}
\pagestyle{fancy}
\fancyhf{} % Clears all headers/footers.
\lhead{\baselineskip 30pt
%\emph{\@doctitle\hfill\@docdate}
\emph{\@docdate\hfill\@doctitle}
\ifnum \value{page} > 1\relax\else\\
\emph{Name: \rule{3.5in}{1pt}\hfill \@docscoring}\fi}
\rfoot{\emph{\@docversion}}
\lfoot{\emph{\@doccourse}}
\cfoot{\emph{\thepage}}
\renewcommand{\headrulewidth}{0pt}%
\makeatother

% Paragraph spacing
\parindent 0pt
\parskip 6pt plus 1pt

% A problem is a section-like command. Use \problem{5} to
% start a problem worth 5 points.
\newcounter{probcount}
\newcounter{subprobcount}
\setcounter{probcount}{0}
\newcommand{\problem}[1]{%
\par
\addvspace{4pt}%
\setcounter{subprobcount}{0}%
\stepcounter{probcount}%
\makebox[0pt][r]{\emph{\arabic{probcount}.}\hskip1ex}\emph{[#1 points]}\hskip1ex}
\newcommand{\thesubproblem}{\emph{\alph{subprobcount}.}}

% Subproblems are an enumerate-like environment with a consistent
% numbering scheme. 
% Use \begin{subproblems}\item...\item...\end{subproblems}
\newenvironment{subproblems}{%
\begin{enumerate}%
\setcounter{enumi}{\value{subprobcount}}%
\renewcommand{\theenumi}{\emph{\alph{enumi}}}}%
{\setcounter{subprobcount}{\value{enumi}}\end{enumerate}}

% Blanks for answers in normal and math mode.
\newcommand{\blank}[1]{\rule{#1}{0.75pt}}
\newcommand{\mblank}[1]{\underline{\hspace{#1}}}
\def\emptybox(#1,#2){\framebox{\parbox[c][#2]{#1}{\rule{0pt}{0pt}}}}

% Misc.
\renewcommand{\d}{\displaystyle}
\newcommand{\ds}{\displaystyle}
\def\bc{\begin{center}}
\def\ec{\end{center}}
\def\be{\begin{enumerate}}
\def\ee{\end{enumerate}}
\def\bpm{\begin{pmatrix}}
\def\epm{\end{pmatrix}}
\def\bbm{\begin{bmatrix}}
\def\ebm{\end{bmatrix}}


\doctitle{Math 314: Quiz 8}
\docdate{Nov 2, 2022}
\doccourse{Linear}
\docversion{v-1}
\docscoring{\blank{0.8in} / 10}
\begin{document}
%\textbf{Please circle your instructor's name:} \hfill Leah Berman  \hfill   Jill Faudree\\

There are 10 points possible on this quiz. No aids (book, calculator, etc.)
are permitted.  {\bf This is a short-answer quiz.}

\begin{enumerate}
\item (2 points) 
	\begin{enumerate}
	\item What is the null space of the differentiation transformation $d/dx: \mathcal{P}_n \to \mathcal{P}_n$?\\
	\vspace{1in}
	\item What is the rank of the differentiation transformation $d/dx: \mathcal{P}_n \to \mathcal{P}_n$?\\
	\vspace{1in}
	\end{enumerate} 
\item (4 points) Multiply the matrix $M=\bpm 1&-1 \\ 2&0 \\ 0& -3 \epm$ by each vector below or state that the operation is not defined.
	\begin{enumerate}
	\item $\vec{v}=\bpm 2\\ -1\\0 \epm$ \\
	\vspace{1in}
	\item $\vec{v}=\bpm 2 \\ -3 \epm$ \\
	\vspace{1in}
	\end{enumerate}
\newpage
\item (4 points) Consider the linear map $h: V \to W$ represented with respect to some bases $B, D$ by the matrix $M=\bpm 1&1&0&-2 \\ 2&3&1&-1 \\ 1&2&1&1\epm.$ Observe that the reduced echelon form of $M$ is $\bpm 1&0&-1&-5\\0&1&1&3\\0&0&0&0\epm .$
	\begin{enumerate}
	\item What is the dimension of the domain of $h$?\\
	\vspace{1in}
	
	\item What is the dimension of the codomain of $h$?\\
	\vspace{1in}
	
	\item What is the dimension of the range of $h$?\\
	\vspace{1in}
	
	\item What is the dimension of the nullity of $h$?\\
	
	\end{enumerate}
\end{enumerate}	
\end{document}