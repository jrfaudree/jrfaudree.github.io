\documentclass[11pt,fleqn]{article} 
\usepackage[margin=0.8in, head=0.8in]{geometry} 
\usepackage{amsmath, amssymb, amsthm,systeme,xcolor}
\usepackage{fancyhdr} 
\usepackage{palatino, url, multicol}
\usepackage{graphicx} 
\usepackage[all]{xy}
\usepackage{polynom} 
\usepackage{pdfsync}
\usepackage{enumerate}
\usepackage{framed}
\usepackage{setspace}
\usepackage{array,tikz}
\pagestyle{fancy} 
\lfoot{UAF Calculus 1}
\rfoot{review functions }

\begin{document}
\renewcommand{\headrulewidth}{0pt}
\newcommand{\blank}[1]{\rule{#1}{0.75pt}}
\renewcommand{\d}{\displaystyle}

\vspace*{-0.7in}

\begin{center}
  \large \sc{Section Two.III.1: Basis}
\end{center}
\begin{enumerate}
\item (Warm-up) Let $S \subset V$ where $V$ is a vector space and $S=\{\vec{s_1}, \vec{s_2},\cdots \vec{s_n}\}.$ 
	\begin{enumerate}
	\item What can you say about the relationship between the objects $[ S-\{\vec{s_i}\}]$, $[S]$, and $V$?
	\vspace{1in}
	\item What can you conclude if  $[ S-\{\vec{s_i}\}]\not =[S]$ for every $\vec{s_i} \in S$?
	\vspace{1.5in}
	\end{enumerate}
\item \textbf{Definition:}
\vspace{1in}
\item Which of the following sets form a \emph{basis} for $\mathbb{R}^3$? (Note: These are the same sets of vectors from Monday's sheet.)
	\begin{enumerate}
	\item $A=\langle\:(1,1,0),(0,1,0) \:\rangle$
	\vfill
	\item $B=\langle\:(1,0,0),(0,1,0),(0,0,1) \:\rangle$
	\vfill
	\item $C=\langle\:(1,0,0),(0,1,0),(0,0,1),(1,1,0) \:\rangle$
	\vfill
	\item $D=\langle\:(0,1,0),(0,0,1),(1,1,0) \:\rangle$
	\vfill
	\end{enumerate}
\newpage
\item Write the vector with coordinates $(1,-2,3)$ using each basis below:
	\begin{enumerate}
	\item $B_1=\langle\:(1,0,0),(0,1,0),(0,0,1) \:\rangle$
	\vfill
	\item $B_2=\langle\:(0,1,0),(0,0,1),(1,1,0) \:\rangle$
	\vfill
	\item $B_3=\langle\:(1,1,0) ,(0,1,0),(0,0,1)\:\rangle$
	\vfill
	\end{enumerate}
\item Write the vector $1-2x+3x^2$ with respect to the basis $B=\langle \: 1,x,x-x^2 \: \rangle.$
\vfill


 \end{enumerate}
\end{document}