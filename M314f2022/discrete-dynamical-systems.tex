\documentclass[11pt,fleqn]{article} 
\usepackage[margin=0.8in, head=0.8in]{geometry} 
\usepackage{amsmath, amssymb, amsthm,systeme,xcolor}
\usepackage{fancyhdr} 
\usepackage{palatino, url, multicol}
\usepackage{graphicx} 
\usepackage[all]{xy}
\usepackage{polynom} 
\usepackage{pdfsync}
\usepackage{enumerate}
\usepackage{framed}
\usepackage{setspace}
\usepackage{array,tikz}
\pagestyle{fancy} 
\lfoot{UAF Linear}
\rfoot{}

\begin{document}
\renewcommand{\headrulewidth}{0pt}
\newcommand{\blank}[1]{\rule{#1}{0.75pt}}
\renewcommand{\d}{\displaystyle}

\newcommand{\bpm}{\begin{pmatrix}}
\newcommand{\epm}{\end{pmatrix}}
\newcommand{\bbm}{\begin{bmatrix}}
\newcommand{\ebm}{\end{bmatrix}}

\vspace*{-0.7in}

\begin{center}
  \large \sc{Applications to Discrete Dynamical Systems} \end{center}
 \begin{enumerate}
 \item \textbf{Definition:} The vector $\vec{v}$ is an eigenvector of matrix $A$ with associated eigenvalue $\lambda$ means \\
 
 \vspace{0.5in}
 
 \item \textbf{Observation:} If  $\vec{v}$ is an eigenvector of matrix $A$ with associated eigenvalue $\lambda$ and $k$ is an positive integer, then $$A^k\vec{v}=$$\\
 
 \item \textbf{Observation:} If $A$ is an $3 \times  3$ matrix with $3$ linearly  independent eigenvectors, $\vec{v_1}, \: \vec{v_2},\:\vec{v_3}$ associated with eigenvalues $\lambda_1, \:\lambda_2, \:\lambda_3,$ and $k$ is a positive integer, find an easy way to write $A^k \vec{x}$ for any $\vec{x} \in \mathbb{R}^3.$\\
 
 \vspace{1.5in}

 \item  Let $O_k$ and $R_k$ denote the owl and rat populations at month $k$ where $O_k$ counts the number of owls and $R_k$ is measured in thousands of rats. Suppose a model describing these populations is below: 
 $$O_{k+1}=(0.5)O_{k}+(0.4)R_k \text{   and   } R_{k+1}=(-0.104)O_k+(1.1)R_k.$$
 \begin{enumerate}
 \item Assume a population begins with 10 owls and $10,000$ rats in month 0, determine how many owls and rats the model indicates in month 1.
 
 \vfill
 \newpage
 \item In month 2? In month 3?
 \vspace{1in}
 \item What are the eigenvalues and associated eigenvectors associated with the matrix in part (a)?
 \vspace{1.5in}
 \item Write an arbitrary initial population vector, $\vec{x_0}$, with respect to the eigenvectors from part $c$ and use this to determine $\vec{x}_{k+1}$,the population vector in month $k+1.$
 \vfill
 \item What happens as $k \to \infty$?
 \vfill
 \end{enumerate}
 \newpage
 \item A different owl population is modeled by the discrete dynamical system $$\bpm j_{k+1}\\s_{k+1}\\a_{k+1} \epm = \bpm 0&0&0.33\\ 0.18&0&0\\0&0.71&0.94\epm \bpm j_{k}\\s_{k}\\a_{k}\epm$$ where $k$ is measured in years and $\vec{x_k}=\bpm j_{k}\\s_{k}\\a_{k}\epm$ represents the number of female juvenile, subadult and adult owls. 
 \begin{enumerate}
 \item Assume the matrix $ \bpm 0&0&0.33\\ 0.18&0&0\\0&0.71&0.94\epm $ has three distinct eigenvalues each with magnitude less than 1. What can you conclude about the long term trajectory of the owl population?
 \vfill
 \item On the other hand, the matrix $ \bpm 0&0&0.33\\ 0.3&0&0\\0&0.71&0.94\epm $ has three distinct eigenvalues one of which is $1.01$ with eigenvector $\bpm 10\\3\\31 \epm$. The other two eigenvalues still have magnitude less than 1. What can you conclude about the long term trajectory of the owl population?
\vfill
\end{enumerate}
   \end{enumerate}
  \end{document}