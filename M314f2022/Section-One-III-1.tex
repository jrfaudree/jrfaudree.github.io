\documentclass[11pt,fleqn]{article} 
\usepackage[margin=0.8in, head=0.8in]{geometry} 
\usepackage{amsmath, amssymb, amsthm,systeme,xcolor}
\usepackage{fancyhdr} 
\usepackage{palatino, url, multicol}
\usepackage{graphicx} 
\usepackage[all]{xy}
\usepackage{polynom} 
\usepackage{pdfsync}
\usepackage{enumerate}
\usepackage{framed}
\usepackage{setspace}
\usepackage{array,tikz}
\pagestyle{fancy} 
\lfoot{UAF Calculus 1}
\rfoot{review functions }

\begin{document}
\renewcommand{\headrulewidth}{0pt}
\newcommand{\blank}[1]{\rule{#1}{0.75pt}}
\renewcommand{\d}{\displaystyle}
\vspace*{-0.7in}
\begin{center}
  \large \sc{Section One.III.1: Gauss-Jordan Reduction}
\end{center}

\textbf{Fact:} There are many different echelon forms of a SoLE (matrix).\\

We observed this as a class on Friday's quiz!!!\\

\textbf{Today's Ideas:} Gauss-Jordan Reduction, which consists of adding some steps after achieving echelon form, results in a unique matrix. Moreover, the reversible nature of elementary row operations can be used to establish that the nature of solution sets of SoLE are not dependent upon its reduced form.

\noindent \hrulefill

\noindent Example of Gauss-Jordan Reduction\\

Given a SoLe: $\systeme{x_1-2x_2+x_3=0,2x_2-8x_3=8,5x_1-5x_3=10}$ in matrix form $\begin{bmatrix} 1&-2&1&0\\0&2&-8&8\\5&0&-5&10 \end{bmatrix}$ \\

\vfill

\noindent \textbf{Step 1:} Put the matrix into echelon form. Recall this is a left-to-right, top-to-bottom process.\\

$\begin{bmatrix} 1&-2&1&0\\0&2&-8&8\\5&0&-5&10 \end{bmatrix} \quad \textcolor{blue}{\overrightarrow{\rho_3-5\rho_1 \mapsto \rho_3}} \quad \begin{bmatrix} 1&-2&1&0\\0&2&-8&8\\0&10&-10&10 \end{bmatrix} \quad \textcolor{blue}{\overrightarrow{\rho_3-5\rho_2 \mapsto \rho_3}} \quad 
\begin{bmatrix} 1&-2&1&0\\0&2&-8&8\\0&0&30&-30 \end{bmatrix}$\\

\vfill

\noindent\textbf{Step 2:} Make all leading coefficients 1. (Done in one step.) \\

$\begin{bmatrix} 1&-2&1&0\\0&2&-8&8\\0&0&30&-30 \end{bmatrix}$ \quad 
\begin{tabular}{c} 
\textcolor{blue}{$\frac{1}{2}\rho_2\mapsto \rho_2$}\\ 
\textcolor{blue}{$\overrightarrow{\frac{1}{30}\rho_3\mapsto \rho_3}$} 
\end{tabular}
\quad
$\begin{bmatrix} 1&-2&1&0\\0&1&-4&4\\0&0&1&-1 \end{bmatrix}$\\

\vfill

\noindent\textbf{Step 3:}  Use leading 1's to eliminate \emph{all} nonzero entries in that column. This is a right-to-left, bottom-to-top process. \\

$\begin{bmatrix} 1&-2&1&0\\0&1&-4&4\\0&0&1&-1 \end{bmatrix}$
\quad 
\begin{tabular}{c} 
\textcolor{blue}{$\rho_1-\rho_3\mapsto \rho_1$}\\ 
\textcolor{blue}{$\overrightarrow{\rho_2-4\rho_3\mapsto \rho_2}$}
\end{tabular}
\quad
$\begin{bmatrix} 1&-2&0&1\\0&1&0&0\\0&0&1&-1 \end{bmatrix}$
\quad 
\begin{tabular}{c} 
\textcolor{blue}{$\overrightarrow{\rho_1+2\rho_2\mapsto \rho_1}$}
\end{tabular}
\quad
$\begin{bmatrix} 1&0&0&1\\0&1&0&0\\0&0&1&-1 \end{bmatrix}$

\vfill

\emph{pivots} and \emph{pivoting}:

\vspace{1in}

\emph{definition:} A matrix or SoLe is in \textbf{reduced row echelon form} if 

\vspace{1in}
\newpage
\noindent Example: Use Gauss-Jordan Reduction to put the SoLE $\systeme{w-3x+z=5,-w+x+5z=2,x+y+z=0}$ into reduced row echelon form and solve. 
\vfill

\textbf{Notes:}
\vfill
\end{document}