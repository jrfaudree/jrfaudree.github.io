\documentstyle[11pt]{article}
%\documentclass{amsart}
%\usepackage{amsmath}
%\usepackage{amssymb}
\pagestyle{empty} \setlength{\topmargin}{-.5in}
\addtolength{\textheight}{2in} \addtolength{\oddsidemargin}{-1in}
\addtolength{\textwidth}{2in} \setlength{\parindent}{0pt}
\newcommand{\real}{\mathbb{R}}
\newcommand{\un}{\mathcal{U}}
\newcommand{\z}{\mathbb{Z}}
\begin{document}
\Large{
\phantom{space}
\vfill
\center{Math 307    \\
Discrete Mathematics\\
Exam IV\\
December 15, 2003\\
\bigskip
Name:\underline{\hspace{2in}}}
\vfill
\newpage

\begin{enumerate}

\item (15 points) Solve the recurrence relation $a_n=-8a_{n-1}-16a_{n-2}, $ for $n \geq 2,$
with initial conditions: $a_0=2, \: a_1= -20$
\vfill

\item (15 points)  Solve the recurrence relation $a_n = a_{n-1} +3,$ for $n \geq 2$
with initial condition $a_1=2.$ \vspace{3in}
\newpage
\item (25 points) Define a relation $R$ on $Z^+,$ the set of positive integers,
 as follows: $a \: R \: b$ if $a|b.$
\begin{enumerate}\item Is $R$ reflexive? Explain your answer.
\vfill
\item Is $R$ symmetric? Explain your answer.
\vfill
\item Is $R$ antisymmetric? Explain your answer.
\vfill
\item Is $R$ transitive? Explain your answer.\vfill
\item Is $R$ a partial order? Explain your answer.
\vfill
\end{enumerate}
\newpage

\item (10 points) Let $f(x)=3n^2-2n \log_2 n$ and $g(x)=n^2.$
Use the definition to show that $g$ dominates $f.$\vfill

\item (15 points) Answer the following questions using the algorithm below.\\
\bigskip

 for i:= 1 to n do \\
 \hspace{.2in} for j:= 1 to n do \\
 \hspace{.4in} for k:= 1 to i do \\
 \hspace{.6in} print (i,j,k) \\
 \hspace{.4in} end \\
 \hspace{.2in} end \\
 end \\

\begin{enumerate}\item Determine exactly how many times the print statement is executed.
\vfill
\item Determine the best Big Oh notation for the expression you found in part (a) from the following list.(Note that even if you can't get an expression for part (a), you may still be able to guess what order of magnitude it should be.)\\
$O(1), \; O( \log n), \; O(n), \; O(n \log n),\; O(n^2), \; O(n^3),
\; O(n^4), \: O(2^n) $ \vspace{1in}
\end{enumerate}
\newpage

\item (10 points) Professor Euclid is paid every other week on Friday. Use the Pigeon Hole Principle to show that in some month she is paid three times.

\vfill


\item (10 points) Suppose that we have $n$ dollars and that each day we buy either orange juice (\$1), milk (\$2), or soda (\$2). If $a_n$ is the number of ways of spending all the money, find a recurrence relation and initial conditions for the sequence $a_1, \: a_2, \: a_3, \cdots.$\\
Assume order is taken into account. For example, there are 11 ways we can spend \$4: JJJJ,JJM,JJS,JMJ,JSJ,MJJ,SJJ,MM,SS,MS,SM.\\

\vfill
\end{enumerate}
}
\end{document}
