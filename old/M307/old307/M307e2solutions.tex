\documentclass{amsart}
%\documentstyle[11pt]{article}
\pagestyle{empty} \setlength{\topmargin}{-.5in}
\addtolength{\textheight}{2in} \addtolength{\evensidemargin}{-1in}
\addtolength{\oddsidemargin}{-1in} \addtolength{\textwidth}{1in}
%\setlength{\parindent}{0pt}

\usepackage{amsmath}
\usepackage{amssymb}
\newcommand{\real}{\mathbb{R}}
\newcommand{\un}{\mathcal{U}}
\newcommand{\z}{\mathbb{Z}}
\begin{document}


\Large{
\begin{center}Math 307\\
Exam II - Solutions\\
\begin{enumerate}
\item (20 points) Let $A=\{1,2,3\}, \: B=\{x \in \real | 0 \leq x \leq 2 \},$ and
$C= \{x \in \real | -1 \leq x \leq 3\}.$ Assume the universe is the
set of real numbers.The answers to the first four questions are not
unique.
\begin{enumerate}
\item Give an example of a proper, nonempty subset of $A.$ \\ $\{1\}$
\item List two distinct elements of the set $B.$ \\ 1,2
\item List two distinct elements of the set $\mathcal{P}(B),$ the
power set of $B.$ \\ $\{1\},\{2\}$
\item Give an example of a nonempty set disjoint from $B.$ \\
$\{3\}$
\item Find $B \triangle C.$ \\ $[-1,0) \cup (2, 3]$
\item Find $B \cap A.$ \\ $\{1,2\}$
\item Find $B \cup A.$ \\ $[0,2] \cup \{3\}$
\item Find $\overline{B}.$ \\ $(- \infty, \infty )$
\item Find $A - C.$ \\ $ \emptyset $
\item Find $\overline{A} \cup \overline{B}.$\\ Using the fact that $\overline{A} \cup \overline{B}
= \overline{ A \cap B},$ we get $\{ x \in \real | x \not = 1 \wedge
x \not = 2 \}$
\end{enumerate}
\item (20 points) Let $A=\{1,2,3,4, \cdots, 29,30\}.$ \begin{enumerate}
\item How many distinct subsets of $A$ are possible? \\ $2^{30}$
\item How many distinct subsets of $A$ have exactly eight elements?
\\ $ 30 \choose 8$
\item Assuming all subsets of $A$ are equally likely, what is the
probability of picking a subset with exactly eight elements?
\\ ${30 \choose 8}/ 2^{30}$
\item How many eight element subsets contain both 1 and 2 or contain
both 3 and 4? \\ ${28 \choose 6}+{28 \choose 6} - {26 \choose 4}$
\end{enumerate}

\item(10 points) Use the laws of set theory to simplify
$$\overline{(\overline{A-B}) \cap A}.$$
\\ $ \overline{(\overline{A-B}) \cap A}= \overline{(\overline{A-B})} \cup \overline{A}
= (A-B) \cup \overline{A}= (A \cap \overline{B}) \cup \overline{A}
 = (A \cup \overline{A}) \cap (\overline{B} \cup \overline{A}) =
\un \cap  (\overline{B} \cup \overline{A})= (\overline{B} \cup
\overline{A})$
\item (10 points) Write and label the converse, inverse, and contrapositive
of the implication:

If it is Saturday and there is snow,  Tom goes skiing. \\
converse: If Tom goes skiing, then it's Saturday and there's snow.
\\
inverse: If it is not Saturday or there isn't snow, then Tom doesn't
ski \\
contrapositive: If Tom isn't skiing, then it isn't Saturday or there
isn't snow.\\

\item (15 points) Assume the universe is the set of real numbers.
Determine whether the following propositions are true or false.
Carefully explain your answers using complete sentences.
\begin{enumerate}
\item $\forall x \hspace{.1in}\exists y \hspace{.1in} 0 < x-y < 1$
\\Answer: True \\
For any $x,$ pick $y=x-0.5.$ Then, $x-y=0.5$ which is strictly
between 0 and 1. So for all $x$ we found a $y$ that makes the
inequality true.\\
\item $ \exists x \hspace{.1in} \forall y  \hspace{.1in}
\left[ (x<y) \rightarrow (y^2 > 3) \right]$\\ Answer: True \\
I'll pick $x = \sqrt{3}.$ (You could pick $x$ to be any number
greater than or equal to $\sqrt{3}.$) Then for any $y \leq
\sqrt{3}$, the hypothesis is false, thus the implication is true.
For any $y > \sqrt{3}$, $y^2 > 3$ and the implication is true. Thus
we have found
an $x$ such that the implication is true for all $y$.\\
\end{enumerate}
\item (10 points) Negate and simplify the following:
$$ \forall x \hspace{.1in}\left[(x >0) \rightarrow (
\exists y \hspace{.1in} 0 < y < \sqrt{x}) \right]$$ \\

$ \neg \left[\forall x \hspace{.1in}\left[(x >0) \rightarrow (
\exists y \hspace{.1in} 0 < y < \sqrt{x}) \right]\right]$ \\ $
\Leftrightarrow \exists x \hspace{.1in}\neg\left[\left[(x >0)
\rightarrow ( \exists y \hspace{.1in} 0 < y < \sqrt{x})
\right]\right]$ \\ $ \Leftrightarrow \exists x \hspace{.1in}\left[(x
>0) \vee \neg( \exists y \hspace{.1in} 0 < y < \sqrt{x}) \right]
$ \\ $\Leftrightarrow \exists x \hspace{.1in}\left[(x >0) \vee (
\forall y \hspace{.1in} 0 \geq y \vee y \geq \sqrt{x}) \right]$


\item (15 points) Establish the validity of the argument below by
listing a series of numbered steps and reason for the steps.\\
$a \rightarrow (b \rightarrow c)$ \\
$d \vee a $\\
$\neg d \rightarrow b $\\
$\underline{\neg d \hspace{.4in} } $\\
$\therefore c$
\end{enumerate}

\bigskip

\begin{tabular}{ll}
\underline{Steps} &  \underline{Reasons} \\
(1) $d \vee a $ & premise \\
(2) $\neg d$ & premise \\
(3) $a$ & Rule of Disjunctive Syllogism applied to (1) and (2) \\
(4) $\neg d \rightarrow b $ & premise \\
(5) $b$ & Rule of Detachment applied to (2) and (4) \\
(6) $a \wedge b$ & Rule of Conjunction applied to (5) and (3) \\
(7) $a \rightarrow (b \rightarrow c)$ & premise \\
(8) $c$ & Rule of Conditional Proof applied to (7) and (6)

\end{tabular}

\end{document}
