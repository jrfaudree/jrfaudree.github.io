\documentclass[12pt]{article}
% Time-stamp: <homework-02.tex, saved on Fri, Sep 14, 2007 at 12:46pm>
\usepackage[margin=1in, head=1in]{geometry}
\usepackage{amsmath, amssymb, amsthm}
\usepackage{fancyhdr}
\usepackage{graphicx}
%\usepackage{pdfsync}
\addtolength{\textwidth}{.5in}
\addtolength{\leftmargin}{-1in}
\addtolength{\textheight}{.5in}
\addtolength{\topmargin}{-0.5in}

%\pagestyle{fancy}
%\lhead{MATH 200X }
%\chead{Fall 2007}
%\rhead{FINAL EXAM}
%\lfoot{}
%\cfoot{\thepage}
%\rfoot{}

\setcounter{secnumdepth}{0}
%\renewcommand{\theenumi}{\alph{enumi}}
%\renewcommand{\emptyset}{\varnothing}
\newcommand{\R}{\mathbb{R}}
\newcommand{\N}{\mathbb{N}}
\newcommand{\Z}{\mathbb{Z}}
\newcommand{\clm}{\par\textit{Claim:}\par}
\newcommand{\diam}{\mathrm{diam}}

\parindent=0in
\parskip=0.5\baselineskip

\begin{document}
\begin{center}MATH 307  \: FALL 2008 \\ \textsc{MIDTERM I -- Selected Solutions}
\end{center}

\begin{enumerate}
\item Let $P$ be the proposition:
 \emph{A sufficient condition for the stock market to fall is for winter to arrive early}.  \\
\begin{enumerate}
\item State $P$ as a conditional proposition. (That is, rewrite $P$ as an If-then statement.)\\
If winter arrives early, then the stock market will fall.
\item Write the converse of $P$.\\
If the stock markets falls, then winter arrives early.
\item Write the  contrapositive of $P$.\\
If the stock market doesn't fall, then winter does not arrive early. 
\item Write the negation of $P.$ (Do not use the words ``It is not the case that...")\\
Winter arrives early and the market doesn't fall.
\item Which, if any, of the statements in parts $b$, $c$, and $d$, logically equivalent to $P$?
The contrapositive (part c) is equivalent to $P.$
\end{enumerate}

\item (If you have questions, ask me.)

\item (If you have questions, ask me.)

\item  Use Theorem 1.1.1 Logical Equivalences to verify the logical equivalence: \\
$[\sim(q \vee \sim p)] \vee (q \wedge p) \equiv p. $ Supply a reason for each step.\\

\smallskip

$[\sim(q \vee \sim p)] \vee (q \wedge p) \equiv (\sim q \wedge p) \vee (q \wedge p)$ by DeMorgan's Law\\

\hspace{1.6in} $\equiv (p \wedge \sim q) \vee (q \wedge p)$ by commutativity\\

\hspace{1.6in} $\equiv p \wedge (\sim q \vee q)$ by the distributive law\\

\hspace{1.6in} $\equiv p \wedge {\bf{t}}$ by the negation law\\

\hspace{1.6in} $\equiv p$ by the identity law.\\

\item Negate each of the following propositions.\\
\begin{enumerate}
\item $\forall x \in \mathbb{R} \: \exists y \in \mathbb{Q}$ such that $\frac{x}{100} < y < x.$\\
$\exists x \in \mathbb{R} \: \forall y \in \mathbb{Q}$,  $\frac{x}{100} \geq y$ or $y \geq x.$\\
\item $\forall x \in \mathbb{Z},$ if $x \geq 10$ and $x$ is prime, then $x+2$ is not prime or $x+4$ is not prime.\\
$\exists x \in \mathbb{Z},$ such that $x \geq 10$ and $x$ is prime and $x+2$ is prime and $x+4$ is prime.\\
\item $\forall x \in \mathbb{R}$, $|x|<1$ if and only if $x^2<1.$\\
$\exists x \in \mathbb{R}$, ($|x|<1$ and $x^2 \geq 1$) or ($x\geq 1$ and $x^2<1.$)
\end{enumerate}

\item Determine the truth value for each of the following and justify your answer.
\begin{enumerate}
\item For every composite number $c$, $c^2 \geq 16.$\\
True. By definition, the first composite number is 4. That is, for every composite number $c$, $c \geq 4.$ Now for every real number, if $c\geq 4,$ then $c^2 \geq 16.$\\

\item $\forall x \in \mathbb{R}$ if $x^2$ is even, then $x$ is even.\\
False. Let $x=\sqrt{2}.$ Then $x \in \mathbb{R}$ and $x^2=2$ which is even and $x$ is not even since it is not an integer. So, $x=\sqrt{2}$ is a counterexample.
\item $\forall x \in \mathbb{R}$ such that $x \not = 0, \: \exists y \in \mathbb{R}$ such that $xy > 0.$\\
True. Given any real number $x$ not equal to zero, choose $y=x.$ Then $xy=x^2 \geq 0$ because any real number squared is nonnegative. Furthermore, since $x \not =0,$ $x^2 \not = 0,$ by the zero property. Thus, for any given $x$, we have a choice of $y$ such that $xy$ is always positive.
\end{enumerate}
\item \begin{enumerate}
 \item  Define what it means for the integer $a$ to be divisible by the integer $b.$\\
 Look in your book.

 \item  Use the definitions (of divisibility and odd) to prove that, for any two consecutive odd integers, the difference of their squares is a multiple of 8. (Note: for any two numbers $n$ and $m$ the \emph{difference of their squares} means $n^2-m^2.$)
     \end{enumerate}
     
     Proof: Let $n$ and $m$ be consecutive odd integers. So, $n=m+2.$ Also, from the definition of odd,  we know there exists an integer $k$ such that $m=2k+1.$ Thus, by substitution, $n=2k+3.$ Now, $n^2-m^2=(2k+3)^2-(2k+1)^2=8k+8=8(k+1).$ Let $k_1=k+1.$ Since $k$ is an integer, $k_1$ is an integer. Thus, $n^2-m^2=8k_1$ where $k_1$ is an integer. Thus, by the definition of divides, we have shown that 8 divides $n^2-m^2.$ Or, equivalently, we have shown that $n^2-m^2$ is a multiple of 8 for any pair of consecutive odd integers $m$ and $n$.


\end{enumerate}
\end{document}
