\documentclass[12pt]{article}
\usepackage[margin=.8in]{geometry}
\usepackage{amsmath,amssymb,amsthm, latexsym, mathrsfs, pdfsync, 
fancybox, fancyhdr, 
graphicx, enumerate,
subfig, multicol}
\usepackage{tikz}
\usepackage{pgf}
\usepackage{pgfplots}
\usetikzlibrary{calc}

\newcommand{\blankbox}[2]{\fbox{\rule{#1}{0in}\rule{0in}{#2}}}
%special commands for number sets
\def\RR{{\mathbb R}}
\def\NN{{\mathbb N}}
\def\ZZ{{\mathbb Z}}
\def\QQ{{\mathbb Q}}
\def\CC{{\mathbb C}}
%special commands for formatting: center, enumerate, pmatrix, vector span
\def\bc{\begin{center}}
\def\ec{\end{center}}
\newcommand{\be}{\begin{enumerate}}
\newcommand{\ee}{\end{enumerate}}
\newcommand{\bpm}{\begin{pmatrix}}
\newcommand{\epm}{\end{pmatrix}}
\newcommand{\bv}[1]{\mathbf{#1}}
\newcommand{\spn}[1]{\text{Span}\left\{#1\right\}}

\setlength{\headheight}{22pt}
\setlength{\headsep}{20pt}

\lhead{\sc Math 307\\ Discrete Math}
\chead{Quiz \#8\\ \S 6.1-6.2} 
\rhead{\sc Spring 2016}
\cfoot{}
\pagestyle{fancy}

\begin{document}
\thispagestyle{fancy}


\noindent {\Large{NAME:\underline{\hspace{3in}}}}\\

\noindent This quiz contains 4 problems worth 30 points. You may not use books, notes, or a calculator. You have 30 minutes to take the quiz.\\

NOTE: As we discussed in class on  Monday, Problem 1 on the quiz requires you to give  simplified numerical answers (for example $102$ or $17/15$). For all other problems, you may give an unsimplified numerical answer (for example $12! \cdot 7! / 4!$ or $12\cdot P(10,6)\cdot C(18,6).$\\
\noindent\hrulefill

\be
\item (2 points each) Calculate the following. Your answers must be in simplified numerical form. Any fractions must be in lowest terms.\\
\be
\item $P(8,3)$
\vspace{.2in}
\item $P(5,5)$ \vspace{.2in}
\item $C(10,7)$ \vspace{.2in}
\item $C(14,1)$ \vspace{.2in}
\ee

\item (2 points each) The eight letters in the set $X=\{A,B,C,D,E,F,G,H\}$ are used to form strings of length 5. Assume you are allowed to repeat letters when forming a string. So, for example, $ABFFA$ is an allowable string.\\
\be
\item How many strings can be formed? \vfill
\item How many strings begin with the letter $A$ \vfill
\item How many strings contain the letter $A$? (This questions could be rephrased as: How many strings contain at least one $A$?) \vfill
\ee
\newpage
\item (2 points each) The eight letters in the set $X=\{A,B,C,D,E,F,G,H\}$ are used to form strings of length 5. Assume you are NOT allowed to repeat letters when forming a string. So, for example, $CDGHA$ is an allowable string but $AABBC$ is not.\\
\be
\item How many strings can be formed? \vfill
\item How many strings contain the substring $AB$? \vfill
\item How many strings contain the substring $AB$ or the substring $CDE$? \vfill
\ee
\item (2 points each) A local bookstore has a ``freebie" table holding a total of 21 books, all distinct. Six of the books are math books, seven are history books, and eight are computer science books. You are going to select 6 books from the table. Assume the order in which you select the books does not matter.\be
\item In how many ways can you select 6 books?  \vfill
\item How many selections contain exactly 3 math books? \vfill
\item How many selections have at most 2 history books? \vfill
\item How many selections have at least two of the three subjects represented? \vfill
\ee
\item (2 points) How many binary strings of length $20$ contain exactly $6$ ones. (Recall, \emph{binary} means strings of $0$'s and $1$'s.) \vfill
\ee

\end{document}