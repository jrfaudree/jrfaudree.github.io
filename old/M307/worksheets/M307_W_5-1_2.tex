\documentclass[12pt]{article}
\usepackage[margin=.8in]{geometry}
\usepackage{amsmath,amssymb,amsthm, latexsym, mathrsfs, pdfsync, 
fancybox, fancyhdr, 
graphicx, enumerate,
subfig, multicol}
\usepackage{tikz}
\usepackage{pgf}
\usepackage{pgfplots}
\usetikzlibrary{calc}

\newcommand{\blankbox}[2]{\fbox{\rule{#1}{0in}\rule{0in}{#2}}}
%special commands for number sets
\def\RR{{\mathbb R}}
\def\NN{{\mathbb N}}
\def\ZZ{{\mathbb Z}}
\def\QQ{{\mathbb Q}}
\def\CC{{\mathbb C}}
%special commands for formatting: center, enumerate, pmatrix, vector span
\def\bc{\begin{center}}
\def\ec{\end{center}}
\newcommand{\be}{\begin{enumerate}}
\newcommand{\ee}{\end{enumerate}}
\newcommand{\bpm}{\begin{pmatrix}}
\newcommand{\epm}{\end{pmatrix}}
\newcommand{\bv}[1]{\mathbf{#1}}
\newcommand{\spn}[1]{\text{Span}\left\{#1\right\}}
\newcommand{\lra}{\longrightarrow}
\newcommand{\llra}{\longleftrightarrow}

\setlength{\headheight}{30pt}
\setlength{\headsep}{20pt}
\setlength{\fboxsep}{8pt}
\setlength{\fboxrule}{1pt}

\lhead{\sc Math 307\\ Discrete Math}
\chead{\sc Worksheet \S 5.1-5.2} 
\rhead{\sc Spring 2016 \\21 Mar 2016}
\cfoot{}
\pagestyle{fancy}
\pgfplotsset{compat=1.12}
\begin{document}
\thispagestyle{fancy}

\quad
Name: \\

\quad
\bc Section 5.1 \ec
\be
\item For each integer below (i) trace the standard algorithm (Algorithm 5.1.8 page 226) to determine if it is prime and (ii) find its prime factorization.
\be
\item $n=966$
\vfill
\item $n=127$
\ee
\item For each pair of integers find (i) the greatest common divisor of the pair and (ii) the least common multiple of the pair.
\be
\item $n=30,~m=120$
\vfill
\item $n=104,~m=363$
\vfill
\item $n=72,~m=306$
\vfill
\item $n=2^2\cdot 3 \cdot 5^4,~m=2^3 \cdot 5^3 \cdot 7$
\vfill
\ee
\item For \#2d, write  $n$ and $m$ as products of the \emph{same} set of prime factors.
\vfill
\newpage
\item Let $m=p_1^{a_1}p_2^{a_2}p_3^{a_3}\cdots p_n^{a_n}$ and $n=p_1^{b_1}p_2^{b_2}p_3^{b_3}\cdots p_n^{b_n}$ where $a_i,~b_i \in \ZZ^{nonneg}.$
\be
\item Is $p_1^{a_1}p_2^{a_2}p_3^{a_3}\cdots p_n^{a_n} $ necessarily the prime factorization of $m$? Explain.
\vfill
\item Give  formulas for the greatest common divisor and least common multiple of  $m$ and $n.$
\vfill
\ee
\item Write a formal, direct proof of the following:
\begin{quote}
Let $n,~c,$ and $d$ be integers. If $dc\: |\: nc,$ then $d\:|\:n.$
\end{quote}
\vfill
\ee
\newpage
\bc Section 5.2 \ec
\be
\item When a number is represented in 
\begin{itemize}
\item \emph{decimal} form, digits are selected from the set \underline{$\{\hspace{2in}\}$} and each position represents a power of \underline{\hspace{.5in}}\\

So the expansion of the symbols: $8032$ is \underline{\hspace{2in}}\\

\item \emph{binary} form, digits are selected from the set \underline{$\{\hspace{2in}\}$} and each position represents a power of \underline{\hspace{.5in}}\\

So the expansion of the symbols: $1101$ is \underline{\hspace{2in}}\\

\item \emph{hexadecimal} form, digits are selected from the set \underline{$\{\hspace{2in}\}$} and each position represents a power of \underline{\hspace{.5in}}\\

So the expansion of the symbols: $20AF$ is \underline{\hspace{2in}}\\
\end{itemize}

\item Express the binary number $1101010$ in decimal.
\vfill
\item Express the decimal number $357$ in binary.
\vfill
\item Express the hexadecimal number $A105$ in decimal.
\vfill
\item Express the decimal number $10400$ in hexadecimal.
\vfill
\newpage
\item Assume you are given a decimal integer $n,$ how many bits (digits) would you  need to represent $n$ in binary? (If you don't immediately know the answer, return to \#3 and think about how you calculated it.)
\vfill
\item Without actually finding the binary representation, determine the number of bits needed to represent the decimal number $2,500,230.$
\vfill

\ee
\end{document}
