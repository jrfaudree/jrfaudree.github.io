\documentclass[12pt]{article}
\usepackage[margin=.8in]{geometry}
\usepackage{amsmath,amssymb,amsthm, latexsym, mathrsfs, pdfsync, 
fancybox, fancyhdr, 
graphicx, enumerate,
subfig, multicol}
\usepackage{tikz}
\usepackage{pgf}
\usepackage{pgfplots}
\usetikzlibrary{calc}

\newcommand{\blankbox}[2]{\fbox{\rule{#1}{0in}\rule{0in}{#2}}}
%special commands for number sets
\def\RR{{\mathbb R}}
\def\NN{{\mathbb N}}
\def\ZZ{{\mathbb Z}}
\def\QQ{{\mathbb Q}}
\def\CC{{\mathbb C}}
%special commands for formatting: center, enumerate, pmatrix, vector span
\def\bc{\begin{center}}
\def\ec{\end{center}}
\newcommand{\be}{\begin{enumerate}}
\newcommand{\ee}{\end{enumerate}}
\newcommand{\bpm}{\begin{pmatrix}}
\newcommand{\epm}{\end{pmatrix}}
\newcommand{\bv}[1]{\mathbf{#1}}
\newcommand{\spn}[1]{\text{Span}\left\{#1\right\}}
\newcommand{\lra}{\longrightarrow}
\newcommand{\llra}{\longleftrightarrow}

\setlength{\headheight}{30pt}
\setlength{\headsep}{20pt}
\setlength{\fboxsep}{8pt}
\setlength{\fboxrule}{1pt}

\lhead{\sc Math 307\\ Discrete Math}
\chead{\sc Worksheet \sect 1.3 } 
\rhead{\sc Spring 2016 \\22 Jan 16}
\cfoot{}
\pagestyle{fancy}
\pgfplotsset{compat=1.12}
\begin{document}
\thispagestyle{fancy}

Section 1.3 contains the bread and butter of symbolic logic, and thus very useful in math, in computer science, and if life. Key words here are: conditional proposition, converse, biconditional proposition, DeMorgan's Laws of Logic, negation of conditional propositions, contrapositive, and logical equivalence.\\

\noindent\hrulefill

\be
\item ( Fill in the blanks. ) Let $p$ and $q$ be propositions. Then the proposition of the form \\

\vspace{.1in}
\fbox{$p~\longrightarrow~q$} is called a $\underline{\hspace{2in}}$ proposition \\

\vspace{.1in}
where $p$ is called the  $\underline{\hspace{2in}},$  $q$ is called the $\underline{\hspace{2in}}$, and its truth table is:\\

\begin{tabular}{ll}
{\Large {
\begin{tabular}[t]{c|c||c}
$p$ & $q$ & $p~\longrightarrow~q$ \\
\hline \hline
T&T&\\
\hline
T&F&\\
\hline
F&T&\\
\hline
F&F&\\
\end{tabular}}}
&
Explain in your own words why $p~\longrightarrow~q$ is true when $p=F.$
\end{tabular}

\item Let $p=F,$ $q=T$, and $r=T.$ Determine the truth values of the propositions below.
\be
\item $p \vee q \lra r$
\item $p \lra \neg(q \wedge r) \vee p$
\item $ p \lra q$
\ee
\item Let $p:$ \emph{The bird is a raven.} and $q:$ {\emph {The bird is black.} }The following table lists sentences in English that are equivalent to $p \lra q.$\\

\begin{tabular}{l|l}
short-hand & example\\
\hline
\hline
if - then & If the bird is a raven, then the bird is black.\\
\hline
only if & The bird is a raven only if the bird is black.\\
\hline
when & When a bird is a raven, the bird is black.\\
\hline
necessary condition&A necessary condition for a bird to be a raven is that the bird be black.\\
\hline
sufficient condition& A sufficient condition for a bird to be black is that the bird is a raven.\\
\hline
\end{tabular}

Rewrite the sentences below in the form of an \textbf{If-then} sentence. 
\be
\item Today is Friday only if we have a quiz.
\newpage
\item When it is cold, my car won't start.
\vfill
\item A necessary condition to enroll at Hogwarts is that you are a witch or wizard.
\vfill
\item A sufficient condition to have experienced frequent earth quakes is to be a resident of Oklahoma.
\vfill
\ee

\item ( Fill in the blanks. ) Let $p$ and $q$ be propositions. Then the proposition of the form \\

\vspace{.1in}
\fbox{$p~\llra~q$} is called a $\underline{\hspace{2in}}$ proposition. \\

with truth table:\\

\begin{tabular}{p{6cm} p{8cm}}
{\Large {
\begin{tabular}[t]{c|c||c}
$p$ & $q$ & $p~\llra~q$ \\
\hline \hline
T&T&\\
\hline
T&F&\\
\hline
F&T&\\
\hline
F&F&\\
\end{tabular}}}
&
Give an equivalent formulation of the biconditional proposition in terms of the conditional proposition.\\
\end{tabular}

\item State De Morgan's Laws for Logic.
\vfill
\newpage
\item Use De Morgan's Laws for Logic to write a sentence in English equivalent to $\neg(p \vee q)$ if $p:$ \emph{Hermione studies a lot.} and $q:$ \emph{Ron isn't serious.}
\vfill

\item State the negation of $p \lra q$ symbolically (using $\wedge$) and explain how you know you are correct.
\vfill
\item Write the negation of the statement: $r:$ \emph{If Donald Trump is elected President, David Brooks will eat his shoe.}
\vfill
\item Write the {\emph{converse}} of the proposition $r$ above. 
\vfill
\item Write the {\emph{contrapositive}} of the proposition $r$ above. 
\vfill
\item How would you convince another student that a conditional proposition is equivalent to its contrapositive and not equivalent to its converse with appealing to a truth table.
\vfill
\ee
\end{document}
