\documentclass[12pt]{article}
\usepackage[margin=.8in]{geometry}
\usepackage{amsmath,amssymb,amsthm, latexsym, mathrsfs, pdfsync, 
fancybox, fancyhdr, 
graphicx, enumerate,
subfig, multicol}
\usepackage{tikz}
\usepackage{pgf}
\usepackage{pgfplots}
\usetikzlibrary{calc}

\newcommand{\blankbox}[2]{\fbox{\rule{#1}{0in}\rule{0in}{#2}}}
%special commands for number sets
\def\RR{{\mathbb R}}
\def\NN{{\mathbb N}}
\def\ZZ{{\mathbb Z}}
\def\QQ{{\mathbb Q}}
\def\CC{{\mathbb C}}
%special commands for formatting: center, enumerate, pmatrix, vector span
\def\bc{\begin{center}}
\def\ec{\end{center}}
\newcommand{\be}{\begin{enumerate}}
\newcommand{\ee}{\end{enumerate}}
\newcommand{\bpm}{\begin{pmatrix}}
\newcommand{\epm}{\end{pmatrix}}
\newcommand{\bv}[1]{\mathbf{#1}}
\newcommand{\spn}[1]{\text{Span}\left\{#1\right\}}
\newcommand{\lra}{\longrightarrow}
\newcommand{\llra}{\longleftrightarrow}

\setlength{\headheight}{30pt}
\setlength{\headsep}{20pt}
\setlength{\fboxsep}{8pt}
\setlength{\fboxrule}{1pt}

\lhead{\sc Math 307\\ Discrete Math}
\chead{\sc Worksheet \S 5.2-5.3} 
\rhead{\sc Spring 2016 \\21 Mar 2016}
\cfoot{}
\pagestyle{fancy}
\pgfplotsset{compat=1.12}
\begin{document}
\thispagestyle{fancy}

\quad
Name: \\

\quad
\bc Section 5.2 \ec
\be
\item \emph{Without converting to decimal} add the binary numbers: $1110101+110101$
\vfill
\item Describe the method you are using to add the binary numbers above.
\vfill
\item \emph{Without converting to decimal} add the hexadecimal numbers: $48F9+D62$
\vfill
\item Describe the method you are using to add the hexadecimal numbers above.
\vfill
\item Take the binary number $10010101101$ and convert it to hexadecimal by:
\be
\item converting from binary to decimal and decimal to hexadecimal.
\vfill
\item converting directly from binary to hexadecimal.
\vfill
\ee
\ee
\newpage
\bc Section 5.3 \ec

This section has two main ideas: (a) the Euclidean Algorithm (how to run it)\\
\indent and \hspace{2.02in} (b) how to use the Euclidean (what it tells you)\\

You will find this algorithm in pseudo code on page 249. Here is the algorithm in plain English. The input consists of two nonnegative integers $a$ and $b$ and without loss of generality, assume $a \geq b$. Apply the Quotient-Remainder Theorem (page 111) to $a$ and $b$ to obtain a remainder $r.$ Now repeat with $b$ and $r.$ Continue until obtaining the remainder $0.$\\

Here is the trace of the Euclidean Algorithm on $a=225$ and $b=84.$\\

\begin{tabular}{|c|c|c|c|c|c|}
\hline
iteration&$a$&$b$&Quotient-Remainder Thm&$r$&comments\\
&&&$a=q \cdot b+r;~0\leq r <b$&&\\
\hline\hline
1&225&84&$225=2 \cdot 84 +57$ & 57& $r \not=0$ so repeat\\
\hline
2&84&57&$84=1 \cdot 57 + 27$ & 27& $r \not=0$ so repeat\\
\hline
3&57&27&$57=2 \cdot 27 +3 $&3&$r \not = 0$ so repeat \\
\hline
4&27&3&$27=9 \cdot 3 +0$&0&$r=0$ so return \emph{previous} $r$-value\\
\hline
\end{tabular}\\
The algorithm would return the number \fbox{3}.
\be
\item Apply the Euclidean Algorithm to each pair below. Show your work by including the Quotient-Remainder Thm calculation for each iteration.
\be
\item $m=2310,~n=805$
\vfill
\item $n=18,~m=305$
\vfill
\ee
\newpage
\item Let $a,b,q,r,d \in \ZZ^+$ and assume $a=q \cdot b +r.$ If $d$ divides $a$ and $d$ divides $b$, does that mean $d$ divides $r$? Explain your answer.
\vfill
\item (Read carefully! This is different from \#2.) Let $a,b,q,r,d \in \ZZ^+$ and assume $a=q \cdot b +r.$ If $d$ divides $r$ and $d$ divides $b$, does that mean $d$ divides $a$? Explain your answer.
\vfill
\item Now use your answers to \#2 and \#3 above to explain why the Euclidean Algorithm returns the greatest common divisor of its two inputs.
\vfill
\item For 1a and 1b above, find the prime factorization of each integer and confirm that the Euclidean Algorithm returns the greatest common divisor of $m$ and $n.$
\vfill
\newpage

One of the other useful results of the Euclidean Algorithm is that the calculations used to find the GCD and be reversed to obtain the GCD of two integers \emph{in terms of a linear combination of the two integers.} For example, we found that $\gcd{225,84}=3.$ By reversing the calculations, we can obtain the equation: \fbox{$3=3\cdot 225-8\cdot 84.$}\\

In the table below, columns 1 and 2 are copied from the table on page 1. Column 3 is obtained by solving each equation for $r.$ Column 4 is back substitutions \emph{starting at the last row and working up.}\\

\begin{tabular}{|c|c|c|c|p{2cm}|}
\hline
iter-&QR Thm &Solve for $r$&back substitute and simplify&comments\\
ation&(copied)&&&\\
\hline\hline
1&$225=2 \cdot 84 +57$ & $225-2 \cdot 84 =57$& $3 \cdot \Big(225-2 \cdot 84\Big)-2\cdot 84=3$&Replace 57; \\
&&&$3\cdot 225-8\cdot 84=3$&re-group\\
\hline
2&$84=1 \cdot 57 + 27$ & $84-1 \cdot 57 = 27$&$57-2\cdot$ \fbox{$84-1 \cdot 57$}$=3$ &Replace {27};  \\
&&&$3 \cdot 57-2\cdot 84=3$& re-group\\
\hline
3&$57=2 \cdot 27 +3 $&$57-2 \cdot 27 =3 $&$57-2 \cdot \fbox{27} =3 $&START HERE \\
&&&&work up\\
\hline
\hline
\end{tabular}\\


\item For each pair of numbers below, write their GCD as a linear combination of $m$ and $n$.
\be
\item $m=2310,~n=805$
\vfill
\item $n=18,~m=305$
\vfill
\ee
\ee

\end{document}

Answer:
2310 - (805 x 2) = 700
805 - (700 x 1) = 105
700 - (105 x 6) = 70
105 - (70 x 1) = 35
70 - (35 x 2) = 0

For the values 805 and 2310
GCF = 35

Answer:
305 - (18 x 16) = 17
18 - (17 x 1) = 1
17 - (1 x 17) = 0

For the values 305 and 18
GCF = 1
