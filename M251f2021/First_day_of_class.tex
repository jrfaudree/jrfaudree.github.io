\documentclass[11pt,fleqn]{article} 
\usepackage[margin=0.8in, head=0.8in]{geometry} 
\usepackage{amsmath, amssymb, amsthm}
\usepackage{fancyhdr} 
\usepackage{palatino, url, multicol}
\usepackage{graphicx} 
\usepackage[all]{xy}
\usepackage{polynom} 
\usepackage{pdfsync}
\usepackage{enumerate}
\usepackage{framed}
\usepackage{setspace}
\usepackage{array,tikz}
\pagestyle{fancy} 
\lfoot{UAF Calculus 1}
\rfoot{first day }

\begin{document}
\renewcommand{\headrulewidth}{0pt}
\newcommand{\blank}[1]{\rule{#1}{0.75pt}}
\renewcommand{\d}{\displaystyle}
\vspace*{-0.7in}
\begin{center}
  \large \sc{First Day of Class}
\end{center}


\begin{enumerate}
\item My teacher's name: \underline{\hspace{2in}}\\ office:  \underline{\hspace{2in}}\\ email:  \underline{\hspace{2in}}\\

%My teacher's name: \underline{Jill Faudree or James Gossell}\\ office:  \underline{(Jill) 306B, (James) 301C}\\ email:  \underline{(\hspace{2in}(Jill) jrfaudree@alaska.edu, (James) jegossell@alaska.edu}\\

\item Where can I get information about this class?\\
%\begin{itemize}
%\item by attending class
%\item from the course webpage: \textit{https://uaf-math251.github.io/} \\(or just google: Calculus I at UAF)
%\item from Canvas
%\end{itemize}

\vfill
\item Where is my textbook?\\
%\begin{itemize}
%\item Our text -- OpenStax Calculus Volume 1, authors Herman \& Strang -- is available free online. \\
%(google \emph{openstax calculus I})
%\item general info can be found here: \textit{https://openstax.org/details/books/calculus-volume-1} 
%\item Note that it is possible to buy a hard copy. 
%\item There are student resources online including a free student solutions manual.
%\end{itemize}
\vfill
\item How will I be graded? \\
%Based on attendance, homework completion, quizzes and their corrections, and some tests
\vfill

\item Where are my grades?\\
%grades live in Canvas
\vfill

\item Where is the Math Lab?\\
%Chapman 305. Notably on the same floor as James' and Jill's offices. Also worth noting that James, Jill and our TA's have weekly hours in the math lab!
\vfill

\item Where is the Chapman Computer Lab? The Rasmuson Computer Lab?\\
%Chapman lab on the first floor right across the hall from our Tuesday/Thursday classroom.\\
%The Rasmuson Lab is in the Library. The third floor is down one level from ground level.
\vfill

\item What do I need to do in the next 24 hours for this class?\\
%\begin{itemize}
%\item Go to the Calculus I webpage and find (i) syllabus (ii) weekly schedule (iii) instructor info
%\item Read through the whole syllabus.
%\item Make sure you have Canvas set up to notify you of class announcements.
%\item Find your ALEKS access code on Canvas.
%\end{itemize}
\vfill
\item What will I need to do this week for this class?\\
%begin{itemize}
%\item Attend class every day. Remember that MWF and T Th are in different rooms!
%\item Bring your laptop to class tomorrow if possible.
%\item Complete a practice ALEKS for extra credit.
%\item Do 5 hours in learning mode or reach 90% of your pie.
%\end{itemize}
\vfill
\item Where will I go tomorrow?
%Go to Chapman 106. After a brief intro, those with laptops will get started and those without will go across the hall to the computer lab.
\vfill
\item How will my teacher contact me? \\
%In class. Canvas. Email.
\vfill
%\item What will a typical day in Calc I look like?
\newpage
\item List the names of the members of your group:
\vfill
\item Describe the things you dislike about in-class group work.
\vfill
\item Describe some ways in which in-class group work is beneficial.
\vfill
\item Make a list of guidelines (or principles or ground rules) for group interactions. 
\vfill
\newpage
\item Without using too much jargon, give \textbf{short} answers the following questions.
	\begin{enumerate}
	\item  What is the difference between a \emph{function} and an \emph{equation} or are these the same thing?\\
	
	%An equation is an expression with an equal sign. A function is a rule that assigns exactly one output for every valid input.\\
	
	\item What is the \emph{domain} of a function? \\
	
	%The domain of a function is the set of allowable inputs.\\
	
	\item What is the \emph{range} of a function? \\
	
	%The range of a function is the set of all outputs when considering all allowable inputs.\\
	
	\item How can you tell if a function is \emph{linear}? \emph{quadratic}? \emph{exponential}? \\
	
	%Their graphs look different. Their algebraic expressions look different.
	\vfill
	\end{enumerate}
\end{enumerate}
\end{document}