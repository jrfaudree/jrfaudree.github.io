\documentclass[11pt,fleqn]{article} 
\usepackage[margin=0.8in, head=0.8in]{geometry} 
\usepackage{amsmath, amssymb, amsthm}
\usepackage{fancyhdr} 
\usepackage{palatino, url, multicol}
\usepackage{graphicx} 
\usepackage[all]{xy}
\usepackage{polynom} 
\usepackage{pdfsync}
\usepackage{enumerate}
\usepackage{framed}
\usepackage{setspace}
\usepackage{array,tikz}
\pagestyle{fancy} 
\lfoot{UAF Calculus 1}
\rfoot{Sect 2.1 }

\begin{document}
\renewcommand{\headrulewidth}{0pt}
\newcommand{\blank}[1]{\rule{#1}{0.75pt}}
\renewcommand{\d}{\displaystyle}
\vspace*{-0.7in}
\begin{center}
  \large \sc{Section 2.1: Preview of Calculuis}
\end{center}



\begin{enumerate}
\item The point $P(2,3)$ lies on the graph of $f(x)=x+\frac{2}{x}.$
	\begin{enumerate}
	\item If possible, find the slope of the secant line between the point $P$ and each of the points with $x$ values listed below. For each estimate the slope to 4 decimal places. NOTE: You do not need the graph of the function to answer this numerical question.\\
		{\LARGE{\begin{center}
		\begin{tabular}{l | l | c}
		\multicolumn{2}{c}{point $Q$}& slope of secant line $PQ$\\
		$x$-value&\quad$y$-value \quad \quad& $PQ$\\
		\hline
		$x=4$&&\\
		\hline
		$x=3$&&\\
		\hline
		$x=2.5$&&\\
		\hline
		$x=2.25$&&\\
		\hline
		$x=2.1$&&\\
		\hline
		$x=0$&&\\
		\hline
		$x=1$&&\\
		\hline
		$x=1.5$&&\\
		\hline
		$x=1.75$&&\\
		\hline
		$x=1.9$&&\\
		\hline
		\end{tabular}
		\end{center}}}
	\item  Now, use technology to sketch a rough graph $f(x)$ on the interval $(0,5]$ and add the secant lines from part $a$. (Your graph may be messy...It's ok.) Label the secant lines with their respective slopes. What can you conclude about the slope of the tangent line to $f(x)$ at $x=2$?
	\vfill
	\item Write a best guess for the equation of the line tangent to $f(x)$ at point $P$. Is your equation plausible?
	\vspace{.5in}
	\end{enumerate}
	\newpage

\item The table shows the position of a cyclist after accelerating from rest.\\

\begin{tabular}{|c||c|c|c|c|c|c|c|c|c|}
$t$ (minutes) &0&30&60&90&120&150&180&210&240\\
\hline
$d$ (miles) &0&9.2&18.7&23.1&38.1&46.6&59.7&72.6&80\\
\end{tabular}
\begin{enumerate}
\item Estimate the cyclist's average velocity in miles per hour  during:
\begin{enumerate}
\item the first hour\\ \vfill
\item the second hour\\ \vfill
\end{enumerate}
\item Estimate how fast the cyclist was going 1.5 hours into the ride.\\  \vspace{2.5in}

\item What does any this have to do with secant lines and tangent lines?
\vfill
\end{enumerate}
\end{enumerate}
\end{document}