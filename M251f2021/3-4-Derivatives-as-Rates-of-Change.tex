
\documentclass[11pt,fleqn]{article} 
\usepackage[margin=0.8in, head=0.8in]{geometry} 
\usepackage{amsmath, amssymb, amsthm}
\usepackage{fancyhdr} 
\usepackage{palatino, url, multicol}
\usepackage{graphicx, pgfplots} 
\usepackage[all]{xy}
\usepackage{polynom} 
%\usepackage{pdfsync} %% I don't know why this messes up tabular column widths
\usepackage{enumerate}
\usepackage{framed}
\usepackage{setspace}
\usepackage{array,tikz}

\pgfplotsset{compat=1.6}

\pgfplotsset{soldot/.style={color=black,only marks,mark=*}} \pgfplotsset{holdot/.style={color=black,fill=white,only marks,mark=*}}
\pgfplotsset{my style/.append style={axis x line=middle, axis y line=
middle, xlabel={$x$}, ylabel={$y$}, axis equal }}


\pagestyle{fancy} 
\lfoot{}
\rfoot{3-4 Derivatives as Rates of Change}

\begin{document}
\renewcommand{\headrulewidth}{0pt}
\newcommand{\blank}[1]{\rule{#1}{0.75pt}}
\newcommand{\bc}{\begin{center}}
\newcommand{\ec}{\end{center}}
\renewcommand{\d}{\displaystyle}

\vspace*{-0.7in}

%%%%%%%%%intro page
\begin{center}
  \large
  \sc{Section 3-4: Derivatives as Rates of Change}\\
\end{center}
Read Section 3.4. Work the embedded problems. \\
\begin{enumerate}
\item A potato is launched vertically upward from a platform 20 feet off the ground. The distance in feet that the potato travels from the ground after $t$ seconds is given by $s(t)=-16t^2+64t+20.$
	\begin{enumerate}
	\item Find the initial velocity of the potato.
	\vfill
	\item Find the velocity and the acceleration of the potato when $t=3.$ 	\vfill
	\item Is the potato speeding up or slowing down? Why?
	\vfill
	\item What is the velocity of the potato when it reaches its maximum height and why?
	\vfill
	\item What is the maximum height of the potato?
	\vfill
	\item Assume the potato lands on the ground (not the platform). How long is the potato in the air?
	\vfill
	\item What is the velocity of the potato when it hits the ground?
	\vfill
	\item You should have observed in part (b) that the acceleration is constant. What does this number represent? 
	\vfill
	\end{enumerate}
\end{enumerate}
\end{document}

