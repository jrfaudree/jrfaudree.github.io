\documentclass[11pt,fleqn]{article} 
\usepackage[margin=0.8in, head=0.8in]{geometry} 
\usepackage{amsmath, amssymb, amsthm}
\usepackage{fancyhdr} 
\usepackage{palatino, url, multicol}
\usepackage{graphicx, pgfplots} 
\usepackage[all]{xy}
\usepackage{polynom,tabularx} 
%\usepackage{pdfsync} %% I don't know why this messes up tabular column widths
\usepackage{enumerate, adjustbox}
\usepackage{framed}
\usepackage{setspace}
\usepackage{array}
\usepackage{pgf,tikz}
\usepackage{mathrsfs}

\usepackage[parfill]{parskip}
\usetikzlibrary{arrows}

\usetikzlibrary{calc}

\pgfplotsset{compat=1.6}

\pgfplotsset{soldot/.style={color=blue,only marks,mark=*}} \pgfplotsset{holdot/.style={color=blue,fill=white,only marks,mark=*}}

\renewcommand{\headrulewidth}{0pt}
\newcommand{\blank}[1]{\rule{#1}{0.75pt}}
\newcommand{\bc}{\begin{center}}
\newcommand{\ec}{\end{center}}
\newcommand{\be}{\begin{enumerate}}
\newcommand{\ee}{\end{enumerate}}

\def\ds{\displaystyle}

\renewcommand{\d}{\displaystyle}

\newcommand{\ans}[1][2]{ \ \rule{#1 in}{.5 pt} \ }


\pagestyle{fancy} 
\rfoot{5-3}

\begin{document}

\vspace*{-0.7in}

\begin{center}
  \Large\sc{Section 5.4: The Net Change Theorem}
  \end{center}
\begin{enumerate}
\item Quick Review: Evaluate the following.
\begin{multicols}{3}
\begin{enumerate}
\item $\displaystyle{\int \left(\frac{x}{3}-\sin(x) \right) \: dx}$
\item $\displaystyle{\int_0^5 \left(3-e^x \right) \: dx}$
\item $\displaystyle{\frac{d}{dx} \left(\int_1^{x^2} \left(\ln(t)\right) \: dt\right)}$
\end{enumerate}
\end{multicols}

\vfill

\item Assume $P'(t)$ gives the rate of change in a population of ants over time where time $t$ is measured in days and $P'(t)$ is measured in hundreds of ants per day. Use the table below to answer the questions.\\

\begin{tabular}{l|c|c|c|c|c|c}
$t$&0&7&14&21&28&35\\
\hline
P'(t)&0&1.9&2.4&2.7&3.0&3.2\\
\end{tabular}
\begin{enumerate}
\item Interpret $P'(14)=2.4.$\\

\item Estimate how much the ant population increased in the first three weeks. Include units with your answer.\\
\vfill

\item What would $\displaystyle{\int_0^{21} P'(t)\: dt}$ represent? (There are many ways to answer this question. Think of as many as you can. Include units this that is appropriate)
\vfill

\item What would $P(t)$ represent? What is $P(14)$?
\vfill
\end{enumerate}
\item The Net Change Theorem: 
\vspace{1in}
\newpage
\item Snow is falling on my garden at a rate of
\[
A(t) = 10 e^{-2 t}
\]
kilograms per hour for $0\le t\le 2$, where $t$ is measured in hours.
\begin{enumerate}
\item Find $A(1)$ and interpret in the context of the problem.
\vfill
\item If $m(t)$ is the total mass of snow on my garden, how are $m(t)$
and $A(t)$ related to each other?
\vfill
\item What does $m(2)-m(0)$ represent?
\vfill
\item Find an antiderivative of $A(t)$.
\vfill
\item Compute the total amount of snow accumulation from $t=0$ to $t=1$.
\vfill
\item Compute the total amount of snow accumulation from $t=0$ to $t=2$.
\vfill

\item From the information given so far, can you compute $m(2)$?
\vfill
\item Suppose $m(0)=9$.  Compute $m(1)$ and $m(2)$.
\vfill
\end{enumerate}
\end{enumerate}
\end{document}
\newpage
\item A airplane is descending.  Its rate of change of height
is $\d r(t) = -4 t + \frac{t^2}{10}$ meters per second.  
\begin{enumerate}
\item If $A(t)$ is the altitude of the airplane in meters, 
how are $A(t)$ and $r(t)$ related?
\vfill
\item What physical quantity
does $\d \int_1^3 r(t)\; dt$ represent?
\vfill
\item  Compute $A(3)-A(1)$.
\vfill
\item What is the height of the plane when $t=3$?
\vfill
\end{enumerate}


