% !TEX TS-program = pdflatexmk
\documentclass[11pt]{article}

\usepackage[margin=1in]{geometry}\headsep = 0.2 in
%\parskip = 0.2in
%\parindent = 0.0in

\usepackage{amsmath,amssymb,latexsym,graphicx,amsthm,enumerate}
\usepackage{palatino, url, multicol,tikz}
\newtheorem{theorem}{Theorem}
\newtheorem{corollary}[theorem]{Corollary}
\newtheorem{conjecture}[theorem]{Conjecture}
\newtheorem{lemma}[theorem]{Lemma}
\newtheorem{proposition}[theorem]{Proposition}
\newtheorem{definition}[theorem]{Definition}
\newtheorem{example}[theorem]{Example}
\newtheorem{axiom}{Axiom}
\theoremstyle{remark}
\newtheorem{remark}{Remark}
\newtheorem{exercise}{Exercise}%[section]
\def\RR{{\mathbb R}}
\def\NN{{\mathbb N}}
\def\ZZ{{\mathbb Z}}
\def\QQ{{\mathbb Q}}
\def\CC{{\mathbb C}}
\def\bc{\begin{center}}
\def\ec{\end{center}}
\def\be{\begin{enumerate}}
\def\ee{\end{enumerate}}
\def\bi{\begin{itemize}}
\def\ei{\end{itemize}}
\def\bs{\begin{slide}}
\def\es{\end{slide}}
%\def\bx{\begin{exercise}}
\newcommand{\bx}[1]{\begin{exercise}({#1} pts.)}
\def\ex{\end{exercise}}
\def\t{\times}
%\def\[{\left[}
%\def\]{\right]}
%\def\({\left(}
%\def\){\right)}
\newcommand{\ol}[1]{\overline{#1}}
\newcommand{\oimp}[1]{\overset{#1}{\Longleftrightarrow}}
\newcommand{\bv}[1]{\ensuremath{ \mathbf{\vec{#1}}} }
\renewcommand{\d}{\displaystyle}
\newcommand{\bcd}{\boldsymbol{\cdot}}

\begin{document}
{\bf Math 253 Calculus III Fall 2018 \hfill Quiz \# 2,  12 Sept 2018 }\\
\\
{\bf Name: \rule{3.5in}{1pt}}\\
\\
\noindent There are 20 points possible on this quiz. This is a closed
book quiz and closed note quiz. Calculators are not allowed. If you have any questions, please
raise your hand.

\begin{enumerate}
% section 12.4 # 29
\item (6 points) Given points $P(0,-2,0), \: Q(4,1,2),$ and $R(5,3,1)$ in $\mathbb{R}^3.$ Answer the questions below.\\
\begin{enumerate}
\item Find a nonzero vector orthogonal to the plane through points $P$, $Q$, and $R.$
\vfill
\item Find the area of triangle $PQR.$
\vspace{1in}
\end{enumerate}
%section 12.5 #6,7,15,16
\item (6 points) Find equations (parametric, vector, or symmetric) for the line through the point $P(-2,5,8)$ and parallel to line $L_2$ with parametric equations: $x=3-2t, \; y=4t, \: z=9.$
\vspace{2in}
\newpage
%section 12.5 #30
\item (6 points) Find an equation of the plane that contains the line $\vec{r}(t)=\langle -1,1,0 \rangle + t \langle 3,2,-2 \rangle$ and is parallel to the plane $z=3-6x+y.$
\vfill
%section 12.4 #13
\item (2 points) State whether each expression is meaningful. If not, explain why. If so, state whether it is a vector or a scalar.
\begin{enumerate}
\item  $(\vec{a} \times \vec{b}) \times \vec{c}$
\vspace{1in}
\item $( \vec{a} \bcd \vec{b}) \times (\vec{c} \bcd \vec{d})$
\vspace{1in}
\end{enumerate}
\end{enumerate}
\end{document}