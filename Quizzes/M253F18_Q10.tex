% !TEX TS-program = pdflatexmk
\documentclass[11pt]{article}

\usepackage[margin=1in]{geometry}\headsep = 0.2 in
%\parskip = 0.2in
%\parindent = 0.0in

\usepackage{amsmath,amssymb,latexsym,graphicx,amsthm,enumerate}
\usepackage{palatino, url, multicol,tikz}
\newtheorem{theorem}{Theorem}
\newtheorem{corollary}[theorem]{Corollary}
\newtheorem{conjecture}[theorem]{Conjecture}
\newtheorem{lemma}[theorem]{Lemma}
\newtheorem{proposition}[theorem]{Proposition}
\newtheorem{definition}[theorem]{Definition}
\newtheorem{example}[theorem]{Example}
\newtheorem{axiom}{Axiom}
\theoremstyle{remark}
\newtheorem{remark}{Remark}
\newtheorem{exercise}{Exercise}%[section]
\def\RR{{\mathbb R}}
\def\NN{{\mathbb N}}
\def\ZZ{{\mathbb Z}}
\def\QQ{{\mathbb Q}}
\def\CC{{\mathbb C}}
\def\bc{\begin{center}}
\def\ec{\end{center}}
\def\be{\begin{enumerate}}
\def\ee{\end{enumerate}}
\def\bi{\begin{itemize}}
\def\ei{\end{itemize}}
\def\bs{\begin{slide}}
\def\es{\end{slide}}
%\def\bx{\begin{exercise}}
\newcommand{\bx}[1]{\begin{exercise}({#1} pts.)}
\def\ex{\end{exercise}}
\def\t{\times}
%\def\[{\left[}
%\def\]{\right]}
%\def\({\left(}
%\def\){\right)}
\newcommand{\ol}[1]{\overline{#1}}
\newcommand{\oimp}[1]{\overset{#1}{\Longleftrightarrow}}
\newcommand{\bv}[1]{\ensuremath{ \mathbf{\vec{#1}}} }
\renewcommand{\d}{\displaystyle}
\newcommand{\bcd}{\boldsymbol{\cdot}}

\begin{document}
{\bf Math 253 Calculus III Fall 2018 \hfill Quiz \# 10,  28 November 2018 }\\
\\
{\bf Name: \rule{3.5in}{1pt}}\\
\\
\noindent There are 20 points possible on this quiz. This is a closed
book quiz and closed note quiz. Calculators are not allowed. If you have any questions, please
raise your hand.

\begin{enumerate}
\item Assume $C$ is the upper half of the unit circle $x^2+y^2=1.$
	\begin{enumerate}
	\item (2 points) Give a complete parametrization of $C.$
	\vfill
	\item (2 points) Assume $\d{\int_C (2+x^2y) \, ds=2 \pi +\frac{2}{3}.}$ Explain what this means geometrically. Be specific.
	\vfill
	\end{enumerate}
\item (8 points) Evaluate the line integral $\d{\int_C yz\cos x \, ds}$ where $C$ is the curve parametrized by $x=t, \, y= 3 \cos t$ and $z=3 \sin t$ for $0 \leq t \leq \pi/2.$
	\vspace{4in}
	\newpage
\item \begin{enumerate} \item (6 points) Evaluate the line integral $\d{\int_C \textbf{F} \cdot d\textbf{r}}$ where $\textbf{F}(x,y)=e^{2x}\,\textbf{i}+xy\,\textbf{j}$ and $C$ is given by $\textbf{r}(t)=t^3 \,\textbf{i} +(1+t) \, \textbf{j}$ for $0 \leq t \leq 1.$
\vfill
\item (2 points) Interpret your answer from part (a).
\vspace{2in}
\end{enumerate}
\end{enumerate}
\end{document}