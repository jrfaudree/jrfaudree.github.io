% !TEX TS-program = pdflatexmk
\documentclass[11pt]{article}

\usepackage[margin=1in]{geometry}\headsep = 0.2 in
%\parskip = 0.2in
%\parindent = 0.0in

\usepackage{amsmath,amssymb,latexsym,graphicx,amsthm,enumerate}
\usepackage{palatino, url, multicol,tikz}
\newtheorem{theorem}{Theorem}
\newtheorem{corollary}[theorem]{Corollary}
\newtheorem{conjecture}[theorem]{Conjecture}
\newtheorem{lemma}[theorem]{Lemma}
\newtheorem{proposition}[theorem]{Proposition}
\newtheorem{definition}[theorem]{Definition}
\newtheorem{example}[theorem]{Example}
\newtheorem{axiom}{Axiom}
\theoremstyle{remark}
\newtheorem{remark}{Remark}
\newtheorem{exercise}{Exercise}%[section]
\def\RR{{\mathbb R}}
\def\NN{{\mathbb N}}
\def\ZZ{{\mathbb Z}}
\def\QQ{{\mathbb Q}}
\def\CC{{\mathbb C}}
\def\bc{\begin{center}}
\def\ec{\end{center}}
\def\be{\begin{enumerate}}
\def\ee{\end{enumerate}}
\def\bi{\begin{itemize}}
\def\ei{\end{itemize}}
\def\bs{\begin{slide}}
\def\es{\end{slide}}
%\def\bx{\begin{exercise}}
\newcommand{\bx}[1]{\begin{exercise}({#1} pts.)}
\def\ex{\end{exercise}}
\def\t{\times}
%\def\[{\left[}
%\def\]{\right]}
%\def\({\left(}
%\def\){\right)}
\newcommand{\ol}[1]{\overline{#1}}
\newcommand{\oimp}[1]{\overset{#1}{\Longleftrightarrow}}
\newcommand{\bv}[1]{\ensuremath{ \mathbf{\vec{#1}}} }
\renewcommand{\d}{\displaystyle}
\newcommand{\bcd}{\boldsymbol{\cdot}}

\begin{document}
{\bf Math 253 Calculus III Fall 2018 \hfill Quiz \# 11,  5 December 2018 }\\
\\
{\bf Name: \rule{3.5in}{1pt}}\\
\\
\noindent There are 20 points possible on this quiz. This is a closed
book quiz and closed note quiz. Calculators are not allowed. If you have any questions, please
raise your hand.

\begin{enumerate}
\item (7 points) Let $\textbf{F}(x,y,z)=2x\sin y\;\textbf{i} +(x^2 \cos y +e^z)\;\textbf{j}+(ye^z+1)\;\textbf{k}.$
\be
\item  Find a potential for  $\textbf{F}.$
\vspace{1.5in}

\item Use part (a) to evaluate $\int_C \textbf{F} \cdot d\textbf{r}$ where  $C$ is the curve: $x=t+1,\: y=\pi t,$ and $z=t^2$ for $0 \leq t \leq 1.$
\vfill
\ee
\newpage
\item (7 points) Evaluate $\oint_C xy \;dx +x^2y \; dy$ where $C$ consists of the curve $y=\sqrt{x}$ from $(0,0)$ to $(4,2)$ followed by the line segments from $(4,2)$ to $(0,2)$ and from $(0,2)$ to $(0,0).$ (You do not need to simplify your answer.)
\vfill

\item (6 points) Let $\textbf{F}= ye^x\;\textbf{i} +e^x\;\textbf{j}+xz\;\textbf{k}.$
	\be
	\item Find $\text{curl}\; \textbf{F}.$
	\vspace{2in}
	\item Find $\text{div}\; \textbf{F}.$
	\vspace{1.5in}
	\ee
	
\end{enumerate}
\end{document}