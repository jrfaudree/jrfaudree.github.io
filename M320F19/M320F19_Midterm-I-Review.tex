\documentclass[11pt]{article}
\usepackage[margin=1in, head=1in]{geometry}
\usepackage{amsmath, amssymb, amsthm}
\usepackage{fancyhdr}
\usepackage{graphicx}

\addtolength{\textwidth}{.5in}
\addtolength{\leftmargin}{-1in}
\addtolength{\textheight}{.5in}
\addtolength{\topmargin}{-0.5in}

%command for double parentheses
\newcommand{\textmultiset}[2]{\bigl(\!{\binom{#1}{#2}}\!\bigr)}
\newcommand{\displaymultiset}[2]{\left(\!{\binom{#1}{#2}}\!\right)}
\newcommand\multiset[2]{\mathchoice{\displaymultiset{#1}{#2}}
                                {\textmultiset{#1}{#2}}
                                {\textmultiset{#1}{#2}}
                                {\textmultiset{#1}{#2}}}

\setcounter{secnumdepth}{0}
\newcommand{\R}{\mathbb{R}}
\newcommand{\N}{\mathbb{N}}
\newcommand{\Z}{\mathbb{Z}}
\newcommand{\clm}{\par\textit{Claim:}\par}
\newcommand{\diam}{\mathrm{diam}}
\newcommand{\sect}{\textsection}

\parindent=0in
\parskip=0.5\baselineskip

\begin{document}
\begin{center}MATH 320: Topics in Combinatorics  \\ Fall 2019 \\ Midterm I Review\end{center}

\hrulefill

\noindent\textbf{Logistics:} The Midterm will be one hour long and will include material from Chapters 1 and 2. No books, notes, or other aides allowed. \\

\hrulefill

\noindent\textbf{Chapters 1 and 2:} 

\begin{itemize}
\item vocabulary: lists, words, passwords, binary number, ternary number, repetition allowed, repetition not allowed, power set, set of subsets, permutation, $k$-permutation, combination, committee, multiset, Cartesian product of two sets, relation from set $A$ to set $B$, function from set $A$ to set $B$, bijection, one-to-one correspondence, one-to-one, onto, domain, codomain, range, function composition, inverse relation/function, equivalence relation, equivalence classes, congruence modulo $n$, divisibility, partition of a set, blocks of a partition, circular arrangements, $k$-to-one function, Stirling numbers of the second kind, Bell numbers,  integer partitions and parts of an integer partition

\item notation: $[n]$, $(n)_k$, $\displaystyle{n \choose k}$, $\displaystyle{\multiset{n}{k}}$, $\text{rng}(f)$, $\text{dom}(f)$, $\text{co}(f)$, $S(n,k)$, $P(n,k)$

\item useful theorems/results: product principle, sum principle, the bijection principle, inherited properties, equivalence principle, pigeonhole principle (recall the most general versions Theorem 1.5.4 and Theorem 1.5.6), the Binomial Theorem,,

\item tasks/problems:
	\begin{itemize}
	\item Know the denominations and suits of a standard deck of 52 cards.
	\item Counting the complement.
	\item "Best of $2n-1$" series.
	\item Checking that a function is well-defined.
	\item How to determine equivalence classes.
	\item The relationship between equivalence classes on the set $A$ and partitions of the set $A.$
	\item Know how to give a \emph{bijective} proof or a \emph{combinatorial} proof.
	\item Counting using the language of distributions.
	\item Be able to fill out the chart on page 81.
	\end{itemize}

\item things you won't be asked
	\begin{itemize}
	\item to recall the great number of combinatorial identities
	\item the formulas on pages 68 and 69 for how to calculate Bell numbers and  Stirling numbers.
	\end{itemize}
\end{itemize}

\end{document}