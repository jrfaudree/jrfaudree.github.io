% !TEX TS-program = pdflatexmk
\documentclass[11pt]{article}
\usepackage[margin=.8in]{geometry}
\usepackage{amsmath,amssymb,amsthm, latexsym, mathrsfs, pdfsync, multicol,
%setspace,
%graphics, 
fancybox, fancyhdr,
graphicx, enumerate,
subfig, tikz, pgfplots,array}

%\singlespacing
\def\RR{{\mathbb R}}
\def\NN{{\mathbb N}}
\def\ZZ{{\mathbb Z}}
\def\QQ{{\mathbb Q}}
\def\CC{{\mathbb C}}
\def\bc{\begin{center}}
\def\ec{\end{center}}
\def\be{\begin{enumerate}}
\def\ee{\end{enumerate}}
\def\bi{\begin{itemize}}
\def\ei{\end{itemize}}
\def\bs{\begin{slide}}
\def\es{\end{slide}}
\def\bx{\begin{exercise}}
\def\ex{\end{exercise}}
\def\t{\times}
\newcommand{\ol}[1]{\overline{#1}}
\newcommand{\oimp}[1]{\overset{#1}{\Longleftrightarrow}}
\newcommand{\bv}[1]{\ensuremath{ \mathbf{\vec{#1}}} }
\renewcommand{\d}{\displaystyle}
\newcommand{\blank}[1]{\rule{#1}{0.75pt}}

\usetikzlibrary{calc}

%for tikz pictures
\pgfplotsset{compat=1.6}

\pgfplotsset{soldot/.style={color=black,only marks,mark=*}} \pgfplotsset{holdot/.style={color=black,fill=white,only marks,mark=*}}


%
% Answerbox:
%
%\newcommand\answerbox[3]{#3 \fbox{\rule{#1}{0cm}\rule{0cm}{#2}}}
%
%\setlength{\headsep}{2pt}

\lhead{\sc{Math 320 Combinatorics}}
\chead{\large \sc Midterm I} 
\rhead{\sc Fall 2019}
\cfoot{}
\pagestyle{fancy}
%
\begin{document}
\thispagestyle{fancy}

\vspace{0.1in}
Your Name \\

\vspace{0.051in}

\framebox(200,30){}\\

%\vspace{.1in}
%\begin{tabular}{l@{\hspace{.4in}}l}
%Your Name & Your Signature\\
%\framebox(200,30){} & \framebox(200,30){} \\
%\end{tabular}
%
%%\bigskip
%
%\begin{tabular}{l@{\hspace{.4in}}l}
%Instructor Name & \\
%\framebox(200,30){}&  \\
%\end{tabular}
{
\renewcommand{\baselinestretch}{1.8}
\setlength{\tabcolsep}{.2in}
\normalsize
\begin{center}
\begin{tabular}{|c|c|c|}
\hline
Problem&Total Points&\parbox{.8in}{\hfil Score\hfil}\\
\hline
1&24&\\
\hline
2&12&\\
\hline
3&18&\\
\hline
4&15&\\
\hline
5&15&\\
\hline
6&16&\\
\hline
\hline
Extra Credit & (5) & \\
%\hline
\hline
Total&100&\\
\hline
%Current Course Grade&\multicolumn{2}{c|  }{}\\
%\hline

\end{tabular}

\end{center}
}
\begin{itemize}
\item 
This test is closed notes and closed book.

\item No calculator is needed since all answers can be left in ``choose" form. (So \: \fbox{$\d{6^8{10 \choose 4}}$} \: is an acceptable answer.)

\item
In order to receive full credit, you must {\bf show your work.}  


%\item
%\textbf{PLACE A BOX AROUND \fbox{YOUR FINAL ANSWER} to each question} where appropriate. 
\item
Raise your hand if you have a question.

\end{itemize}

\newpage
\vspace*{-0.3in}
\begin{enumerate}
%%%Problem 1: Simple Counting
\item (24 points) Let $S=\{a,b,c,d,e,f,g,h,i,j\}$ be a set of 10 letters. (Answers without work are acceptable.)
	\begin{enumerate}
	\item How many permutations of $S$ have all of the vowels before all of the consonants?
	\vfill
	\item How many 6-character passwords are possible using only letters from S?
	\vfill
	\item How many 6-character passwords have at least one repeated letter?
	\vfill
	\item How many 6-character passwords have all characters  in alphabetical order? (So the passwords \: \textbf{bcfhij} and \textbf{acdehi} would be counted but password \: \textbf{bcaefg} would not, since the letter $a$ is out of order. Indeed the letters in the password must all be distinct.)
	\vfill
	\end{enumerate}
	

%%%Problem 2: Counting  and functions
\item (12 points)
	\begin{enumerate}
	\item How many functions $f:[n] \to [k]$ are onto?
	\vfill
	\item How large does $n$ need to be to guarantee that, for every function $f:[n] \to [20]$, there exists some $b \in [20]$ such that $|f^{-1}(b)| \geq 3$?
	\vfill 
	\end{enumerate}
\newpage
%%%%%Problem 3: Counting Questions
\item (18 points) There are 100 pieces of candy and 35 children. Find the number of ways to distribute the candy to the children in each of the following situations.
	\begin{enumerate}
	\item The pieces of candy are indistinguishable and each child gets at least one piece. (The children are considered to be different from each other.)
	\vfill
	\item The pieces of candy are all different and each child gets exactly one piece of candy. (So some candy is left over.)
	\vfill
	\item The pieces of candy are all different but you distribute them among 35 identical paper bags assuming no bag is left empty.
	\vfill
	\end{enumerate}
	%%%integer partitions
\item (15 points)
	\begin{enumerate}
	\item By enumeration, determine $P(4,2)$, the number of 2-partitions of 4 and $P(5,3)$ the number of $3$-partitions of $5.$ \\
	\vfill

	\item For $n \geq 4,$ determine a formula for $P(n,n-2)$, the number of $(n-2)$-partitions of $n,$ and explain why your answer is correct.
	\end{enumerate}
\vspace{4in}

\newpage	
%%%Combinatorial proof
\item (15 points) Give a \textbf{combinatorial} proof of the identity below.\\

\begin{quote} For any positive integers $m$ and $n$, $\d{{m+n \choose k} = \sum_{j=0}^k{m \choose j}{n \choose k-j}}$ \end{quote}
\newpage
%%%Bijective Proof & partitions of [n]
\item (16 points) The following question concerns the proposition below: 

\begin{quote} \textbf{Proposition:} If $n \geq 1,$ then $S(n,2)=2^{n-1}-1.$ \end{quote}

	\begin{enumerate}
	\item List all 2-partitions of $[4]$ and show that the proposition holds for $n=4.$\\
	\vspace{1.5in}
	\item Give a \textbf{bijective} proof of the proposition. Hint: Create a bijection between the $2$-partitions of $[n]$ and the nonempty subsets of $[n-1].$
	\end{enumerate}
	\newpage
	
\end{enumerate}
\textbf{Extra Credit:} (5 points) Consider any 5 points in the $xy$-plane with integer coordinates. (That is, point $A(-3,18)$ had integer coordinates but point $B(2,4/3)$ does not.) Prove that there must exist two of the five points such that the midpoint of the line segment joining those two points also has integer coordinates.
\vfill
\end{document}