% !TEX TS-program = pdflatexmk
\documentclass[11pt]{article}
\usepackage[margin=.8in]{geometry}
\usepackage{amsmath,amssymb,amsthm, latexsym, mathrsfs, pdfsync, multicol,
%setspace,
%graphics, 
fancybox, fancyhdr,
graphicx, enumerate,
subfig, tikz, pgfplots,array}
\usepackage{ stmaryrd }
%% The next line says how the "vertex" style of nodes should look: drawn as small circles.
\tikzstyle{vertex}=[circle, draw, inner sep=0pt, minimum size=6pt]
%%
%% Next, we make a \vertex command as a shorthand in place of \node[vertex} to get that style.
\newcommand{\vertex}{\node[vertex]}

%\singlespacing
\def\RR{{\mathbb R}}
\def\NN{{\mathbb N}}
\def\ZZ{{\mathbb Z}}
\def\QQ{{\mathbb Q}}
\def\CC{{\mathbb C}}
\def\bc{\begin{center}}
\def\ec{\end{center}}
\def\be{\begin{enumerate}}
\def\ee{\end{enumerate}}
\def\bi{\begin{itemize}}
\def\ei{\end{itemize}}
\def\bs{\begin{slide}}
\def\es{\end{slide}}
\def\bx{\begin{exercise}}
\def\ex{\end{exercise}}
\def\t{\times}
\newcommand{\ol}[1]{\overline{#1}}
\newcommand{\oimp}[1]{\overset{#1}{\Longleftrightarrow}}
\newcommand{\bv}[1]{\ensuremath{ \mathbf{\vec{#1}}} }
\renewcommand{\d}{\displaystyle}
\newcommand{\blank}[1]{\rule{#1}{0.75pt}}
\def\ldb{\llbracket}
\def\rdb{\rrbracket}

\newcommand{\textmultiset}[2]{\bigl(\!{\binom{#1}{#2}}\!\bigr)}
\newcommand{\displaymultiset}[2]{\left(\!{\binom{#1}{#2}}\!\right)}
\newcommand\multiset[2]{\mathchoice{\displaymultiset{#1}{#2}}
                                {\textmultiset{#1}{#2}}
                                {\textmultiset{#1}{#2}}
                                {\textmultiset{#1}{#2}}}


\usetikzlibrary{calc}

%for tikz pictures
\pgfplotsset{compat=1.6}

\pgfplotsset{soldot/.style={color=black,only marks,mark=*}} \pgfplotsset{holdot/.style={color=black,fill=white,only marks,mark=*}}


%
% Answerbox:
%
%\newcommand\answerbox[3]{#3 \fbox{\rule{#1}{0cm}\rule{0cm}{#2}}}
%
%\setlength{\headsep}{2pt}

\lhead{\sc{Math 320 Combinatorics}}
\chead{\large \sc Midterm II} 
\rhead{\sc Fall 2019}
\cfoot{}
\pagestyle{fancy}
%
\begin{document}
\thispagestyle{fancy}

\vspace{0.1in}
Your Name \\

\vspace{0.051in}

\framebox(200,30){  }\\

%\vspace{.1in}
%\begin{tabular}{l@{\hspace{.4in}}l}
%Your Name & Your Signature\\
%\framebox(200,30){} & \framebox(200,30){} \\
%\end{tabular}
%
%%\bigskip
%
%\begin{tabular}{l@{\hspace{.4in}}l}
%Instructor Name & \\
%\framebox(200,30){}&  \\
%\end{tabular}
{
\renewcommand{\baselinestretch}{1.8}
\setlength{\tabcolsep}{.2in}
\normalsize
\begin{center}
\begin{tabular}{|c|c|c|}
\hline
Problem&Total Points&\parbox{.8in}{\hfil Score\hfil}\\
\hline
1&15&\\
\hline
2&12&\\
\hline
3&16&\\
\hline
4&18&\\
\hline
5&12&\\
\hline
6&12&\\
\hline
6&15&\\
\hline
\hline
Extra Credit & (6) & \\
%\hline
\hline
Total&100&\\
\hline
%Current Course Grade&\multicolumn{2}{c|  }{}\\
%\hline

\end{tabular}

\end{center}
}
\begin{itemize}
\item 
This test is closed book.

\item A student may bring one $8 \times 11$ sheet of paper with writing on the front.

\item No calculator is needed since all answers can be left in ``choose" form. (So \: \fbox{$\d{6^8{10 \choose 4}}$} \: is an acceptable answer.)

\item
In order to receive full credit, you must {\bf show your work.}  


%\item
%\textbf{PLACE A BOX AROUND \fbox{YOUR FINAL ANSWER} to each question} where appropriate. 
\item
Raise your hand if you have a question.

\end{itemize}

\newpage
\vspace*{-0.3in}
\begin{enumerate}
%% inclusion-exclusion
\item (15 points) After a day of skiing, a family of 4 throws their mittens into a bin. The next day, each member grabs two mittens (one right-hand and one left-hand). Count the number of ways this can happen such that not a single member of the family has both of their own mittens. (Use Inclusion-Exclusion.)\\

\item (12 points) 
	\begin{enumerate}
	\item Find the coefficient of $x^{20}$ in $(x^3+x^4+x^5+x^6+ \cdots)^4$\\
	\vfill
	\item Find $\displaystyle{\left\ldb \frac{a}{b+cx}\right \rdb_{x^k/k!}}$
	\vfill
	\end{enumerate}
\item (16 points) In each case, find a concise ordinary generating function for answering the question and also identify what coefficient you need.
	\begin{enumerate}
	\item How many solutions to $z_1+z_2+z_3=20$ are possible such that each $z_i$ is an integer satisfying $1 \leq z_1 \leq 5,$ $0 \leq z_2$ and $0 \leq z_3$?
	\vfill
	\item How many ways to make change for a dollar using only nickels, dimes and quarters?
	\vfill
	\end{enumerate}
\item (18 points) Solve the recurrence relation below using the generating function technique.\\
$$ a_0=1 \quad \quad \text{  and  } \quad \quad a_n=3a_{n-1}+2^n, \text{  for  } n \geq 1$$

\textbf{Answer:} Let $\displaystyle{f(x)=\sum_{i=0}^\infty a_nx^n}$ be the ordinary generating function for the sequence. \\

\vfill
\item (12 points) Prove that if $\delta(G)=k,$ then $G$ contains a cycle on at least $k+1$ vertices.\\

\vfill

\item (12 points) Draw the tree with Pr\"{u}fer sequence $(7,5,10,5,1,8,10,7).$

\vfill

\item (15 points) Let $G$ be the graph pictured below.\\
\begin{tikzpicture}[scale=1.3]
\vertex[fill=white] (a) at (0,0)[label=above:$a$]{};
\vertex[fill=white] (b) at (1,1)[label=above:$b$]{};
\vertex (c) at (1,-1)[label=below:$c$]{};
\vertex[fill=white] (d) at (2,0)[label=above:$d$]{};
\vertex (e) at (3,0)[label=above:$e$]{};
\vertex[fill=red] (f) at (4,0)[label=above:$f$]{};
\draw (c)--(a)--(b)--(c)--(d)--(e)--(f)(b)--(d);
\end{tikzpicture}
	\begin{enumerate}
	\item Determine the chromatic  number of $G$. Explain your answer.\\
	
	\item Find $p(G,k),$  the chromatic polynomial of $G$. Explain your answer. \\
	\end{enumerate}
\end{enumerate}	
\textbf{Extra Credit:} (6 points) 
\begin{enumerate}
	\item Give an example of a graph $G$ such that $\chi(G)=4$ but $G$ has no subgraph isomorphic to $K_4.$\\
	
	\item Use your answer from part $a$ to construct a graph $G$ such that $\chi(G)=5$ but $G$ has no subgraph isomorphic to $K_4.$ (Hint: You might want to start with 4 disjoint copies of your graph from part (a).)\\
		
\end{enumerate}
\vfill
\end{document}