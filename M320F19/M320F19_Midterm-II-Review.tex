\documentclass[11pt]{article}
\usepackage[margin=1in, head=1in]{geometry}
\usepackage{amsmath, amssymb, amsthm}
\usepackage{fancyhdr}
\usepackage{graphicx}

\addtolength{\textwidth}{.5in}
\addtolength{\leftmargin}{-1in}
\addtolength{\textheight}{.5in}
\addtolength{\topmargin}{-0.5in}

%command for double parentheses
\newcommand{\textmultiset}[2]{\bigl(\!{\binom{#1}{#2}}\!\bigr)}
\newcommand{\displaymultiset}[2]{\left(\!{\binom{#1}{#2}}\!\right)}
\newcommand\multiset[2]{\mathchoice{\displaymultiset{#1}{#2}}
                                {\textmultiset{#1}{#2}}
                                {\textmultiset{#1}{#2}}
                                {\textmultiset{#1}{#2}}}

\setcounter{secnumdepth}{0}
\newcommand{\R}{\mathbb{R}}
\newcommand{\N}{\mathbb{N}}
\newcommand{\Z}{\mathbb{Z}}
\newcommand{\clm}{\par\textit{Claim:}\par}
\newcommand{\diam}{\mathrm{diam}}
\newcommand{\sect}{\textsection}

\parindent=0in
\parskip=0.5\baselineskip

\begin{document}
\begin{center}MATH 320: Topics in Combinatorics  \\ Fall 2019 \\ Midterm II Review\end{center}

\hrulefill

\noindent\textbf{Logistics:} The Midterm will be one hour long and will include material from Chapters 3 and 6 Sections 1-3. You may bring in one page of notes with writing on the front. \\

\hrulefill

\noindent\textbf{Chapter 3:} 

\begin{itemize}
\item Section 1: Inclusion-Exclusion\\
We did a short review of this on Monday in the context of Section 6.3. You will want to remind yourself of the notation and the sort of problems that lend themselves to this technique.\\
\item Section 2: Mathematical Induction\\
This section is largely review from Proofs with the exception of an increased emphasis on proving inequalities and the use of this technique to solving recurrence relations.\\
\item Section 3 and 4: Intro to Generating Functions\\
Remind yourself what a generating function is and how it is useful. Recall the difference between OGFs and EGFs. Remind yourself if the elementary formulas (e.g. $1+ax+a^2x^2 + a^3x^3+ \cdots$ and  $1+ax+a^2 x^2 + a^3x^3+ \cdots a^nx^n$). Ideally you are thinking about these formulas and not simply copying them onto your note sheet. There is also notation to recall about ``coefficient extraction." There are convolution formulas. We had to recall the method of partial fractions.\\
\item Sections 5 and 6: Generating Functions and Solutions to Recurrence Relations\\
Recall that these two sections comprise a story that goes: We can use OGFs and EGFs to find solutions to recurrence relations and in some particular instances, we can use these techniques to develop plug-n-chug formulas for solutions.\\
\end{itemize}

\noindent\textbf{Chapter 6:}

\begin{itemize}
\item Section 6.1 Introduction to Graph Theory\\
This section was an introduction to notation, terminology, examples, and several elementary theorems. You will want to carefully read this section and make sure you are familiar with all of these.\\
\item Section 6.2 Trees\\
This section introduced acyclic graphs and Cayley's Formula (Theorem 6.2.5) which involved Pr\"{u}fer Codes. It ended with a discussion of binary trees.\\
\item Section 6.3 Coloring\\
This sections discussed vertex coloring in graphs and introduces much new terminology (e.g. \emph{proper $k$-coloring}) and notation (e.g. $\chi(G)$ and $p(G,k)$).
\end{itemize}
\end{document}