% !TEX TS-program = pdflatexmk
\documentclass[11pt]{article}
\usepackage[margin=.8in]{geometry}
\usepackage{amsmath,amssymb,amsthm, latexsym, mathrsfs, pdfsync, multicol,
%setspace,
%graphics, 
fancybox, fancyhdr,
graphicx, enumerate,
subfig, tikz, pgfplots,array}
\usepackage{ stmaryrd }
%% The next line says how the "vertex" style of nodes should look: drawn as small circles.
\tikzstyle{vertex}=[circle, draw, inner sep=0pt, minimum size=6pt]
%%
%% Next, we make a \vertex command as a shorthand in place of \node[vertex} to get that style.
\newcommand{\vertex}{\node[vertex]}

%\singlespacing
\def\RR{{\mathbb R}}
\def\NN{{\mathbb N}}
\def\ZZ{{\mathbb Z}}
\def\QQ{{\mathbb Q}}
\def\CC{{\mathbb C}}
\def\bc{\begin{center}}
\def\ec{\end{center}}
\def\be{\begin{enumerate}}
\def\ee{\end{enumerate}}
\def\bi{\begin{itemize}}
\def\ei{\end{itemize}}
\def\bs{\begin{slide}}
\def\es{\end{slide}}
\def\bx{\begin{exercise}}
\def\ex{\end{exercise}}
\def\t{\times}
\newcommand{\ol}[1]{\overline{#1}}
\newcommand{\oimp}[1]{\overset{#1}{\Longleftrightarrow}}
\newcommand{\bv}[1]{\ensuremath{ \mathbf{\vec{#1}}} }
\renewcommand{\d}{\displaystyle}
\newcommand{\blank}[1]{\rule{#1}{0.75pt}}
\def\ldb{\llbracket}
\def\rdb{\rrbracket}

\newcommand{\textmultiset}[2]{\bigl(\!{\binom{#1}{#2}}\!\bigr)}
\newcommand{\displaymultiset}[2]{\left(\!{\binom{#1}{#2}}\!\right)}
\newcommand\multiset[2]{\mathchoice{\displaymultiset{#1}{#2}}
                                {\textmultiset{#1}{#2}}
                                {\textmultiset{#1}{#2}}
                                {\textmultiset{#1}{#2}}}


\usetikzlibrary{calc}

%for tikz pictures
\pgfplotsset{compat=1.6}

\pgfplotsset{soldot/.style={color=black,only marks,mark=*}} \pgfplotsset{holdot/.style={color=black,fill=white,only marks,mark=*}}


%
% Answerbox:
%
%\newcommand\answerbox[3]{#3 \fbox{\rule{#1}{0cm}\rule{0cm}{#2}}}
%
%\setlength{\headsep}{2pt}

\lhead{\sc{Math 320 Combinatorics}}
\chead{\large \sc Final Exam} 
\rhead{\sc Fall 2019}
\cfoot{}
\pagestyle{fancy}
%
\begin{document}
\thispagestyle{fancy}

\vspace{0.1in}
Your Name \\

\vspace{0.051in}

\framebox(200,30){  }\\

%\vspace{.1in}
%\begin{tabular}{l@{\hspace{.4in}}l}
%Your Name & Your Signature\\
%\framebox(200,30){} & \framebox(200,30){} \\
%\end{tabular}
%
%%\bigskip
%
%\begin{tabular}{l@{\hspace{.4in}}l}
%Instructor Name & \\
%\framebox(200,30){}&  \\
%\end{tabular}
{
\renewcommand{\baselinestretch}{1.8}
\setlength{\tabcolsep}{.2in}
\normalsize
\begin{center}
\begin{tabular}{|c|c|c|}
\hline
Problem&Total Points&\parbox{.8in}{\hfil Score\hfil}\\
\hline
1&16&\\
\hline
2&10&\\
\hline
3&15&\\
\hline
4&16&\\
\hline
5&15&\\
\hline
6&12&\\
\hline
7&16&\\
\hline
\hline
Total&100&\\
\hline
%Current Course Grade&\multicolumn{2}{c|  }{}\\
%\hline

\end{tabular}

\end{center}
}
\begin{itemize}
\item 
This test is closed book.

\item A student may bring one $8 \times 11$ sheet of paper with writing on the front.

\item No calculator is needed since all answers can be left in ``choose" form. (So \: \fbox{$\d{6^8{10 \choose 4}}$} \: is an acceptable answer.)

\item
In order to receive full credit, you must {\bf show your work.}  


%\item
%\textbf{PLACE A BOX AROUND \fbox{YOUR FINAL ANSWER} to each question} where appropriate. 
\item
Raise your hand if you have a question.

\end{itemize}

\newpage
\vspace*{-0.3in}
\begin{enumerate}
%counting, stirline numbers,
\item (16 points) Let $S(n,k)$ denote the Stirling numbers of the second kind.
	\be
	\item Use complete enumeration of appropriate set partitions to determine $S(4,2).$
	\vfill
	\item Give a combinatorial justification that $S(n,2)=2^{n-1}-1.$ (That is, we were given a formula for $S(n,k)$. You cannot use this formula. Essentially, you are asked to show that this formula is correct when $k=2.$)
	\vspace{3.5in}
	\item How  many different functions with domain $[n]$ and \textbf{codomain} $[k]$ are possible?
	\vfill
	\item How many different functions with domain $[n]$ and \textbf{range} $[k]$ are possible?
	\vfill
	\ee 
\newpage
%inclusion-exclusion
\item (10 points) Use inclusion-exclusion to determine how many integers in $[100]$ are not divisible by $4$, $6$, or $7$. A calculation is sufficient. You do not need to simplify your answer.
\vspace{2in}
%partition numbers
\item (15 points) Let $P(n,k)$ count the number of integer partitions of $n$ into $k$ parts.
	\be
	\item Use complete enumeration to determine $P(6,3).$
	\vspace{1in}
	\item Give a combinatorial proof of the identity below:
	\begin{quote} If $n\geq1$ and $k \geq 1$, then $P(n,k)=P(n-1,k-1)+P(n-k,k).$ \end{quote}
	\ee
\newpage
%generating functions
\item (16 points)
	\be
	\item Write an ordinary generating function to count the number of ways to distribute 20 identical pieces of candy to two adults and three children. Assume the adults receive at most one piece of candy and the children receive at least one piece of candy. Identify what coefficient you need.
	\vfill
	\item Let $g(x)=(1+x^2+x^4)(x+x^2+x^3+\cdots)(1+x+x^2+x^3+\cdots)^2.$
	\be 
	\item Write a concise form of $g(x).$
	\vfill
	\item Find $\displaystyle{\left\ldb g(x) \right\rdb_{x^6}}.$ Simplify your answer.
	\vfill
	\item Given an example of a problem for which $\displaystyle{\left\ldb g(x) \right\rdb_{x^k}}$ is an answer.
	\vfill
	\ee
	\ee
\newpage
%%recurrence relaiton
%\item (10 points) Use generating functions to solve the first order recurrence relation:
%$$a_0=4 \text{     and     } a_n=2a_{n-1}+1, \text{ for } n\geq 1$$
%\newpage
%
%%simple graph theory
%\item (10 points) Let $G$ be a graph on $n$ vertices such that $\delta(G) \geq \left\lfloor \frac{n}{2} \right\rfloor.$ Prove that for every pair of vertices $u$ and $v,$ there exists a $uv$-path in $G$ of length at most 2. (Recall that the \emph{length} of a path counts the number of edges, not the number of vertices.)
%\vfill
%\newpage
%% ramsey theory
\item (15 points) Prove that $R(K_{1,3},K_3)=7.$
	\be
	\item Draw and label the graphs $K_{1,3}$ and $K_3.$
	\vspace{1in}
	\item Demonstrate that $R(K_{1,3},K_3)>6.$ (Note that an example is not sufficient. You must explain how that example implies the lower bound.)
	\vspace{2in}
	\item Demonstrate that $R(K_{1,3},K_3)=7.$ 	\ee
\newpage
%designs
\item (12 points) Let $\mathcal{V} =[n]$ for an integer $n\geq 3$ and let $k$ be an integer such that $2 \leq k \leq n-1.$
	\be
	\item Let $\mathcal{B}$ be the set of all $k$-subsets of $\mathcal{V}.$ Does $(\mathcal{V},\mathcal{B})$ form a balanced incomplete block design? Prove your answer is correct.
	\vfill
	\item Let $B=\{1,2,3\} \subseteq \mathcal{V}$ and assume $n \geq 5.$ Can $B$ be the base block of a cyclic design? Prove your answer is correct.
	\vfill
	\ee

\vfill
\newpage
%basic counting
\item (16 points) 
	\be
	\item A store has 20 varieties of doughnuts. In how many ways can you fill an order for a dozen doughnuts assuming there are no restrictions on the types of doughnuts. (That is, you may choose several of the same variety. Assume the store has a least 12 of each variety.)
	\vfill
	\item A store has 20 varieties of doughnuts. In how many ways can you pick a different variety of doughnut for each for your 7 best friends.
	\vfill
	
	\item A nice teacher brings a box of 12 \textbf{identical} glazed doughnuts for her class of 8 (nonidentical) students. In how many ways can the doughnuts be given to the students assuming each student gets at least one doughnut but at most two doughnuts. (Assume the doughnuts are not divided.)
	\vfill
	\item A nice teacher brings a box of 12 \textbf{different} doughnuts for her class of 8 (nonidentical) students. In how many ways can the doughnuts be given to the students assuming each student gets at least one but there is no other restriction. (So some student could get 5 doughnuts. Assume that the order in which a student gets his/her doughnuts does not matter.)
	\vfill
	\ee
	\end{enumerate}
\end{document}