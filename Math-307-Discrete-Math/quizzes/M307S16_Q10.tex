\documentclass[12pt]{article}
\usepackage[margin=.8in]{geometry}
\usepackage{amsmath,amssymb,amsthm, latexsym, mathrsfs, pdfsync, 
fancybox, fancyhdr, 
graphicx, enumerate,
subfig, multicol}
\usepackage{tikz}
\usepackage{pgf}
\usepackage{pgfplots}
\usetikzlibrary{calc}

\newcommand{\blankbox}[2]{\fbox{\rule{#1}{0in}\rule{0in}{#2}}}
%special commands for number sets
\def\RR{{\mathbb R}}
\def\NN{{\mathbb N}}
\def\ZZ{{\mathbb Z}}
\def\QQ{{\mathbb Q}}
\def\CC{{\mathbb C}}
%special commands for formatting: center, enumerate, pmatrix, vector span
\def\bc{\begin{center}}
\def\ec{\end{center}}
\newcommand{\be}{\begin{enumerate}}
\newcommand{\ee}{\end{enumerate}}
\newcommand{\bpm}{\begin{pmatrix}}
\newcommand{\epm}{\end{pmatrix}}
\newcommand{\bv}[1]{\mathbf{#1}}
\newcommand{\spn}[1]{\text{Span}\left\{#1\right\}}

\setlength{\headheight}{22pt}
\setlength{\headsep}{20pt}

\lhead{\sc Math 307\\ Discrete Math}
\chead{Quiz \#10\\ \S 6.8, 7.1} 
\rhead{\sc Spring 2016}
\cfoot{}
\pagestyle{fancy}

\begin{document}
\thispagestyle{fancy}


\noindent {\Large{NAME:\underline{\hspace{3in}}}}\\

\noindent This quiz contains 4 problems worth 30 points. You may not use books, notes, or a calculator. You do not have to simplify your answers. You have 30 minutes to take the quiz.\\

\noindent\hrulefill

\be
\item (8 points) Suppose that eleven distinct integers are selected from the set $\{1,2,3,\cdots, 19,20\}.$ Prove that at least two of the eleven have a sum equal to 21.
\vfill
\item (8 points) An inventory consists of a list of 200 items, each marked ``available" or ``unavailable." There are 103 available items. Show that there are at least two available items in the list exactly 5 items apart.
\vfill
\newpage

\item Assume a person deposits \$200  into an account at the beginning of \emph{each year} and that the account earns 10\% interest compounded annually. Assume no money is withdrawn from the account. Let $A_i$ denote the amount in the account at the end of $i$ years.
\be
\item (3 points) Find $A_i$ for $i=1,2,3.$ (Actually do the calculation. It isn't hard.)
\vfill
\item (4 points) Find a recurrence relation for $A_n.$
\vspace{1in}
\ee
\item (7 points) Suppose that we have $n$ dollars and that each day we buy either coffee (\$1), tea (\$1), a cookie (\$2), a bagel (\$3), or a burrito (\$3). Let $R_n$ be the number of ways of spending all the money. Derive a recurrence relation for the sequence $R_1,\: R_2,\: R_3, \cdots$ [Assume order is taken into account. So the \$4 purchase (coffee, coffee, cookie) is different from the purchase (coffee, cookie, coffee). Also make sure you include appropriate and complete initial conditions.]
\vfill

\ee
\end{document}