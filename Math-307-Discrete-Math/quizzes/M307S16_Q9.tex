\documentclass[12pt]{article}
\usepackage[margin=.8in]{geometry}
\usepackage{amsmath,amssymb,amsthm, latexsym, mathrsfs, pdfsync, 
fancybox, fancyhdr, 
graphicx, enumerate,
subfig, multicol}
\usepackage{tikz}
\usepackage{pgf}
\usepackage{pgfplots}
\usetikzlibrary{calc}

\newcommand{\blankbox}[2]{\fbox{\rule{#1}{0in}\rule{0in}{#2}}}
%special commands for number sets
\def\RR{{\mathbb R}}
\def\NN{{\mathbb N}}
\def\ZZ{{\mathbb Z}}
\def\QQ{{\mathbb Q}}
\def\CC{{\mathbb C}}
%special commands for formatting: center, enumerate, pmatrix, vector span
\def\bc{\begin{center}}
\def\ec{\end{center}}
\newcommand{\be}{\begin{enumerate}}
\newcommand{\ee}{\end{enumerate}}
\newcommand{\bpm}{\begin{pmatrix}}
\newcommand{\epm}{\end{pmatrix}}
\newcommand{\bv}[1]{\mathbf{#1}}
\newcommand{\spn}[1]{\text{Span}\left\{#1\right\}}

\setlength{\headheight}{22pt}
\setlength{\headsep}{20pt}

\lhead{\sc Math 307\\ Discrete Math}
\chead{Quiz \#9\\ \S 6.3, 6.4, 6.7} 
\rhead{\sc Spring 2016}
\cfoot{}
\pagestyle{fancy}

\begin{document}
\thispagestyle{fancy}


\noindent {\Large{NAME:\underline{\hspace{3in}}}}\\

\noindent This quiz contains 7 problems worth 30 points. You may not use books, notes, or a calculator. You do not have to simplify your answers. You have 30 minutes to take the quiz.\\

\noindent\hrulefill

\be
\item (3 points each)
\be
\item Determine the  number of strings that can be formed by ordering the letters in the word ENGINEER.
\vfill
\item Determine the  number of strings that can be formed by ordering the letters in the word ENGINEER if no two E's are allowed to be consecutive.
\vfill
\ee
\item (4 points) Find the number of integer solutions to $x_1+x_2+x_3+x_4 = 30$ subject to the conditions $x_1 \geq 0,$ $x_2 \geq 0$, $x_3 \geq 1$ and $x_4 \geq 2.$\\
\vfill
\item (2 points each) For each 5-combination of $X=\{1,2,3,4,5,6,7,8,9\}$, give the 4-combination that is next lexicographically:
\be
\item 34678
\vfill
\item 24789
\vfill
\ee
\item (2 points) Explain why the algorithm we described in class (i.e. Algorithm 6.4.9 in your book) that generates all $r$-combinations of a given $n$-set would never produce the following output: 4286.
\vfill 
\newpage
\item (2 points) For each permutation of $X=\{1,2,3,4,5,6,7,8,9\}$, give the permutation that is next lexicographically:
\be
\item 873261945
\vfill
\item 957341268
\vfill
\ee
\item (2 points each) Find the coefficient of the term when the expression is expanded:
\be
\item $x^5y^2z^3;\: (x+y+z)^{10}$ \vfill
\item $x^2y^3;\:(5x-y)^5$ \vfill
\ee
\item (3 points each)
\be 
\item Fill in the box below in the statement of the Binomial Theorem:\\
\begin{center} If $a$ and $b$ are real numbers and $n$ is a positive integer, then \\ \framebox(300,50){} \end{center}
\item Use the Binomial Theorem to prove that $2^n=C(n,0)+C(n,1)+C(n,2)+ \cdots + C(n,n).$
\vspace{2in}
\ee
\ee
\end{document}