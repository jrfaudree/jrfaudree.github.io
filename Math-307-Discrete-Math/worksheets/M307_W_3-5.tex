\documentclass[12pt]{article}
\usepackage[margin=.8in]{geometry}
\usepackage{amsmath,amssymb,amsthm, latexsym, mathrsfs, pdfsync, 
fancybox, fancyhdr, 
graphicx, enumerate,
subfig, multicol}
\usepackage{tikz}
\usepackage{pgf}
\usepackage{pgfplots}
\usetikzlibrary{calc}

\newcommand{\blankbox}[2]{\fbox{\rule{#1}{0in}\rule{0in}{#2}}}
%special commands for number sets
\def\RR{{\mathbb R}}
\def\NN{{\mathbb N}}
\def\ZZ{{\mathbb Z}}
\def\QQ{{\mathbb Q}}
\def\CC{{\mathbb C}}
%special commands for formatting: center, enumerate, pmatrix, vector span
\def\bc{\begin{center}}
\def\ec{\end{center}}
\newcommand{\be}{\begin{enumerate}}
\newcommand{\ee}{\end{enumerate}}
\newcommand{\bpm}{\begin{pmatrix}}
\newcommand{\epm}{\end{pmatrix}}
\newcommand{\bv}[1]{\mathbf{#1}}
\newcommand{\spn}[1]{\text{Span}\left\{#1\right\}}
\newcommand{\lra}{\longrightarrow}
\newcommand{\llra}{\longleftrightarrow}

\setlength{\headheight}{30pt}
\setlength{\headsep}{20pt}
\setlength{\fboxsep}{8pt}
\setlength{\fboxrule}{1pt}

\lhead{\sc Math 307\\ Discrete Math}
\chead{\sc Worksheet \S 3.5} 
\rhead{\sc Spring 2016 \\2 Mar 2016}
\cfoot{}
\pagestyle{fancy}
\pgfplotsset{compat=1.12}
\begin{document}
\thispagestyle{fancy}

\quad
Name: \\

\quad
Let $X=\{0,1,2\},~Y=\{a,b\},$ and $Z=\{d,e,f,g\}$ Let $R_1$ be a relation from $X$ to $Y$ defined as $R_1=\{(0,a),(0,b),(1,a),(2,b)\}$ and $R_2=\{(a,d),(b,d),(b,f)\}.$
\be
\item Write the matrix of each relation above using the ordering of sets $X$, $Y$, and $Z$ as they are listed above.\\
\vfill
\item Take a moment to determine whether each relation is a function, one to one, onto.
\vfill
\item Why weren't you asked to determine if the relations were symmetric, reflexive, or antisymmetric?
\vfill
\item Using the definitions, determine the ordered pairs in $R_2 \circ R_1.$
\vfill
\newpage
\item Now multiply the two matrices from part (1) to show it is (essentially) the matrix of the composition of the two relations. 
\vfill
\item Why is the word \emph{essentially} needed here?
\vfill
\item Can you explain why you multiplied the matrices in the order you did?
\vfill
\ee

\end{document}
