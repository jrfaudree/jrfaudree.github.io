\documentclass[11pt]{article}
% Time-stamp: <homework-02.tex, saved on Fri, Sep 14, 2007 at 12:46pm>
\usepackage[margin=1in, head=1in]{geometry}
\usepackage{amsmath, amssymb, amsthm}
\usepackage{fancyhdr}
\usepackage{graphicx,color}

%\usepackage{pdfsync}
\addtolength{\textwidth}{.5in}
\addtolength{\leftmargin}{-1in}
\addtolength{\textheight}{.5in}
\addtolength{\topmargin}{-0.5in}

%\pagestyle{fancy}
%\lhead{MATH 200X }
%\chead{Fall 2007}
%\rhead{FINAL EXAM}
%\lfoot{}
%\cfoot{\thepage}
%\rfoot{}

\setcounter{secnumdepth}{0}
%\renewcommand{\theenumi}{\alph{enumi}}
%\renewcommand{\emptyset}{\varnothing}
\newcommand{\R}{\mathbb{R}}
\newcommand{\N}{\mathbb{N}}
\newcommand{\Z}{\mathbb{Z}}
\newcommand{\clm}{\par\textit{Claim:}\par}
\newcommand{\diam}{\mathrm{diam}}
\newcommand{\sect}{\textsection}

\parindent=0in
\parskip=0.5\baselineskip

\begin{document}
\begin{center}MATH/CS 307:  Discrete Mathematics \\ Spring 2016 \\ Problem List\end{center}

\hrulefill

\begin{tabular}{|p{1.6cm}|p{12cm}|p{3cm}|}
\hline
Section & Problems & Quiz Date\\
\hline \hline
1.1 & \# 1,4,7,10,13,16,20,24,28,32,36,37,47,53,57,64,68,76,77,80,83,87& Friday 22 Jan \\
\hline
1.2 & \#1,7,10,12*,15*,16,19,22,25,28,33,36,39,40,42,44,45,55-59,66,67,74 & Friday 29 Jan \\
&*Give a proper negation of the proposition. That is, do not use some version of ``It is not the case that..."&\\
\hline
1.3 & \#1,3-8,11,12,13,16,19,21,24,27,30,31,34,43,44-49,52,53,59,68*,70,73&\\
&&\\
& \underline{Problem A}: \textbf{SHOW} whether or not the propositions $P=p \wedge(q\vee r)$ and $Q=(p \wedge q) \vee (p \wedge r)$ are logically equivalent.&Friday 29 Jan\\
&&\\
& * For \# 68, use the directions from Problem A. That is, it is \emph{not sufficient} to simple \emph{state} whether the two propositions are equivalent. You must give a sound explanation of your conclusion.&\\
\hline
1.4 & \# 1-5, 6,9,11-15,18,21,24& Friday 29 Jan\\
\hline
1.5 & \# 12-20,21,24,27,28,31,34,35,38,41,43,44,47,48,49-54,55*&Friday 5 Feb\\
& *Only negate symbolically. &\\
& \#'s 57-66 are amusing, but not required. & \\
\hline
1.6 & \#37-60,64-66*&Friday 5 Feb\\
&*You don't have to use the Logic Game to make your argument.&\\
\hline
2.1 & \#7,10,13,19,22,25,31,33,37& Friday 12 Feb \\
\hline
\end{tabular}



\end{document}
