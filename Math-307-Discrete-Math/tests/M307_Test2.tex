\documentclass[12pt]{article}
\usepackage[margin=.8in]{geometry}
\usepackage{amsmath,amssymb,amsthm, latexsym, mathrsfs, pdfsync, 
fancybox, fancyhdr, 
graphicx, enumerate,
subfig, multicol}
\usepackage{tikz}
\usepackage{pgf}
\usepackage{pgfplots}
\usetikzlibrary{calc}

\newcommand{\blankbox}[2]{\fbox{\rule{#1}{0in}\rule{0in}{#2}}}
%		Problem and Part
\newcounter{problemnumber}
\newcounter{partnumber}
\newcommand{\Problem}{\stepcounter{problemnumber}\setcounter{partnumber}{0}
       \item[\fbox{\parbox{.18in}{\hfil\theproblemnumber\hfil}}]}
\newcommand{\Part}{\stepcounter{partnumber}\item[(\alph{partnumber})]}
\newcommand{\Points}[1]{(#1 points) \quad}

\def\RR{{\mathbb R}}
\def\NN{{\mathbb N}}
\def\ZZ{{\mathbb Z}}
\def\QQ{{\mathbb Q}}
\def\CC{{\mathbb C}}
\def\bc{\begin{center}}
\def\ec{\end{center}}

\newcommand{\be}{\begin{enumerate}}
\newcommand{\ee}{\end{enumerate}}
\newcommand{\bpm}{\begin{pmatrix}}
\newcommand{\epm}{\end{pmatrix}}
\newcommand{\bv}[1]{\mathbf{#1}}
\newcommand{\spn}[1]{\text{Span}\left\{#1\right\}}

\newcommand\answerbox[3]{#3 \fbox{\rule{#1}{0cm}\rule{0cm}{#2}}}

\setlength{\headheight}{22pt}
\setlength{\headsep}{2pt}

\lhead{\sc Math F307}
\chead{\Large \sc Test 2} 
\rhead{\sc Spring 2016}
\cfoot{}
\pagestyle{fancy}
%

\begin{document}
\thispagestyle{fancy}

\begin{tabular}{l@{\hspace{.075\linewidth}}  l}
Your Name (print clearly) &\\
%Your Instructor\\
\blankbox{.6\linewidth}{.45in} & Monday 28 March 2016\\
%\blankbox{.3\linewidth}{.45in}\\
\end{tabular}
\bigskip

\bigskip
\bigskip

{
\renewcommand{\baselinestretch}{1.8}
\setlength{\tabcolsep}{.2in}
\normalsize
\begin{center}
\begin{tabular}{|c|c|c|}
\hline
Page&Total Points&\parbox{.8in}{\hfil Score\hfil}\\
\hline
1&19&\\
\hline
2&29&\\
\hline
3&19&\\
\hline
4&23&\\
\hline
5&10&\\
\hline
\hline
extra credit&5&\\
\hline
\hline
Total&100&\\
\hline
\end{tabular}

\end{center}
}

\bigskip

\begin{center}
\begin{Large}
Instructions and information:
\end{Large}
\end{center}

\begin{itemize}
\item Please turn off cell phones or any other thing that will go BEEP.

\item
Scientific calculators are allowed on this test. You may not use a cell phone or a laptop.
\item
Read the directions for each problem. You must always show your work to receive partial credit.  

\item Be wary of doing computations in your head. Instead, write out your
computations on the exam paper.

\item
If you need more room, use the backs of the pages and indicate to the
grader where to look.

\item
Raise your hand (or come up to the front) if you have a question.
\item Formulas from Calculus:\\
\be 
\item $1+a+a^2+\cdots+ a^{k}=\frac{a^{k+1}-1}{a-1}$\\
\item $\log_a n = \log_b n / \log_b a$\\
\item $1+2+3+\cdots+n=\frac{n(n+1)}{2}$
\ee

\end{itemize}

\newpage
%%%%%%%%%%%
%BEGIN TEST
%%%%%%%%%%%
\begin{enumerate}
\item  
\be
\item (4 points) Fill in the blanks below:
\be
\item A function $f$ from $X$ to $Y$ is \emph{one-to-one} if \underline{\hspace{3in}}\\ 

\underline{\hspace{5.5in}}.\\

\item A function $f$ from $X$ to $Y$ is \emph{onto} if \underline{\hspace{3in}}\\

\underline{\hspace{5.5in}}.\\
\ee
\item (4 points) Assume that the domain and the codomain of the function $f(n)=\lfloor n/4 \rfloor$ is the set of all integers. Determine if $f(n)$ is one-to-one, onto, or both.\\
\vfill
\item (3 points) Given an example of a function $g(x)$ from $\RR$ to $\RR$ that is one-to-one but not onto and show that your example is not onto. (You do not need to justify the one-to-one property.)
\vspace{1in}
\ee 
\item (8 points) Let $a$ be a sequence defined by $a_n=3n+1,$ for $n=1,2,3,\cdots.$
\be
\item Find $a_6.$
\vspace{.4in}
\item Find $\sum_{i=1}^3a_i$
\vspace{.6in}
\item Find a formula for the subsequence of $a$ obtained by selecting every other term of $a$ starting with the first.
\vspace{.6in}
\ee
\newpage
\item (14 points) Let $R$ be a relation on the set of positive integers defined by $(x,y) \in R$ if 2 divides $x+y$.
\be 
\item Find an example of an ordered pair in $R$.\\

\item Find an example of an ordered pair not in $R$.\\

\item Is $R$ reflexive?\\
\vfill
\item Is $R$ antisymmetric?\\
\vfill
\item Is $R$ transitive?\\
\vfill
\ee
\item (15 points) Let $S$ be the set of all strings on the set $\{0,1,2\}$ of length 3 or less. Let $R$ be the equivalence relation on $S$ defined by $s_1Rs_2$ if the strings $s_1$ and $s_2$ have the same number of zeros. 
\be
\item Explain why the string 12 is related to the string 211.\\
\vspace{.3in}
\item Explain why the string 120 is not related to the string 001.\\
\vspace{.3in}
\item Find all elements in $[001]$, the equivalence class containing the string 001.\\
\vfill
\item How many equivalence classes does $R$ have?\\
\vfill
\item List one member of each equivalence class.\\
\vfill
\ee
\newpage
\item 
\be
\item (7 points) Fill in the blank below in the definition:
\begin{quote}
 For $f(n)$ and $g(n)$ be functions with domain $\{1,2,3,\cdots\},$ we write $f(n)=O(g(n))$\\ 
 \quad\\
if \quad \underline{\hspace{4in}} \quad for all but finitely many $n \in \ZZ^+$.
\end{quote}
\item Use the definition above to show $1+2^k+3^k+\cdots+n^k$ is $O(n^{k+1}).$
\vfill
\ee
\item (12 points) For each expression below, select a $\theta$ notation from the table  and justify your answer.

\begin{tabular}{l|l|c|l|l}
Theta Form & Name &&Theta Form & Name \\
\hline
\hline
$\theta(1)$& Constant &&$\theta(n^2)$&Quadratic\\
\hline
$\theta(lg(lg(n)))$& Log log&& $\theta(n^3)$&Cubic\\
\hline 
$\theta(lg(n))$& Log && $\theta(n^k),$ $k \geq 1$&Polynomial\\
\hline 
$\theta(n)$&Linear&& $\theta(c^n),$ $c>1$&Exponential\\
\hline 
$\theta(n~lg(n))$& $n$ log $n$&& $\theta(n!)$&Factorial\\
\hline 
\end{tabular}
\be
\item $\frac{n^3+5n\lg n}{4n+8}$
\vfill
\item $3+9+27+\cdots+3^n.$
\vfill
\ee
\newpage
\item (8 points) Let $m=2^3\cdot 5\cdot 11^2\cdot 17^3$ and $n=2^2\cdot 7 \cdot 11^5.$
\be
\item Find the greatest common divisor of $m$ and $n.$\\
\vfill
\item Find the least common multiple of $m$ and $n.$\\
\vfill
\ee
\item (15 points) Let $m=159$ and $n=509.$
\be
\item Trace the Euclidean Algorithm for inputs $m$ and $n$ above. (You need to show your steps, at least in abbreviated form and state the output explicitly.)
\vspace{3in}
\item What is the significance of the number returned by the Euclidean Algorithm?
\vspace{.3in}
\item Write the greatest common divisor of $m$ and $n$ as a linear combination of $m$ and $n.$
\vspace{2in}
\ee
\newpage
\item (10 points)
\be
\item Write the decimal number 900 in binary.
\vfill
\item Write the decimal number 900 in hexadecimal.
\vfill
\ee
\end{enumerate}
Extra Credit (5 points): Prove that $\lg (n!) = \theta(n \lg n).$
\vfill
\end{document}
