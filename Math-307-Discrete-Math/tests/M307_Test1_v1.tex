\documentclass[12pt]{article}
\usepackage[margin=.8in]{geometry}
\usepackage{amsmath,amssymb,amsthm, latexsym, mathrsfs, pdfsync, 
fancybox, fancyhdr, 
graphicx, enumerate,
subfig, multicol}
\usepackage{tikz}
\usepackage{pgf}
\usepackage{pgfplots}
\usetikzlibrary{calc}

\newcommand{\blankbox}[2]{\fbox{\rule{#1}{0in}\rule{0in}{#2}}}
%		Problem and Part
\newcounter{problemnumber}
\newcounter{partnumber}
\newcommand{\Problem}{\stepcounter{problemnumber}\setcounter{partnumber}{0}
       \item[\fbox{\parbox{.18in}{\hfil\theproblemnumber\hfil}}]}
\newcommand{\Part}{\stepcounter{partnumber}\item[(\alph{partnumber})]}
\newcommand{\Points}[1]{(#1 points) \quad}

\def\RR{{\mathbb R}}
\def\NN{{\mathbb N}}
\def\ZZ{{\mathbb Z}}
\def\QQ{{\mathbb Q}}
\def\CC{{\mathbb C}}
\def\bc{\begin{center}}
\def\ec{\end{center}}

\newcommand{\be}{\begin{enumerate}}
\newcommand{\ee}{\end{enumerate}}
\newcommand{\bpm}{\begin{pmatrix}}
\newcommand{\epm}{\end{pmatrix}}
\newcommand{\bv}[1]{\mathbf{#1}}
\newcommand{\spn}[1]{\text{Span}\left\{#1\right\}}

\newcommand\answerbox[3]{#3 \fbox{\rule{#1}{0cm}\rule{0cm}{#2}}}

\setlength{\headheight}{22pt}
\setlength{\headsep}{2pt}

\lhead{\sc Math F307}
\chead{\Large \sc Test 1 -- Version ONE} 
\rhead{\sc Spring 2016}
\cfoot{}
\pagestyle{fancy}
%

\begin{document}
\thispagestyle{fancy}

\begin{tabular}{l@{\hspace{.075\linewidth}}  l}
Your Name (print clearly) &\\
%Your Instructor\\
\blankbox{.6\linewidth}{.45in} & Wednesday, February 17\\
%\blankbox{.3\linewidth}{.45in}\\
\end{tabular}
\bigskip

\bigskip
\bigskip

{
\renewcommand{\baselinestretch}{1.8}
\setlength{\tabcolsep}{.2in}
\normalsize
\begin{center}
\begin{tabular}{|c|c|c|}
\hline
Page&Total Points&\parbox{.8in}{\hfil Score\hfil}\\
\hline
1&30&\\
\hline
2&30&\\
\hline
3&20&\\
\hline
4&10&\\
\hline
5&10&\\
\hline
\hline
extra credit&5&\\
\hline
\hline
Total&100&\\
\hline
\end{tabular}

\end{center}
}

\bigskip

\begin{center}
\begin{Large}
Instructions and information:
\end{Large}
\end{center}

\begin{itemize}
\item Please turn off cell phones or any other thing that will go BEEP.

\item
Calculators are {\bf not} allowed on this test.
\item
Read the directions for each problem. You must always show your work to receive partial credit.  

\item Be wary of doing computations in your head. Instead, write out your
computations on the exam paper.

\item
If you need more room, use the backs of the pages and indicate to the
grader where to look.

\item
Raise your hand (or come up to the front) if you have a question.

\end{itemize}

\newpage
%%%%%%%%%%%
%BEGIN TEST
%%%%%%%%%%%
\begin{enumerate}
\item (10 points) Use $A=\{1,3\},$ $B=\{2,3,4\},$ and universal set is $U=\{0,1,2,3,4,5\}$ to find the quantities below.
\be
\item $\mathcal{P}(A),$ the power set of $A$ 
\vfill
\item $B - A$
\vfill
\item $\overline{(\overline{A} \cup {B})}$ 
\vfill
\ee
\item (10 points) Use $X=\{1,2,\emptyset, \{1,2,3\},\ZZ\}$ to answer the following questions.
\be
\item Find $|X|.$\\
\vspace{.5in}
\item Is $1 \in X$? Explain.\\
\vspace{.5in}
\item Is $3 \in X$? Explain.\\
\vspace{.5in}
\item Is $\{1,2,3\} \subseteq X$? Explain.\\
\vspace{.5in}
\ee
\item (10 points) For the propositions below, propositions $p$ and $q$ are true, but the truth value of proposition $r$ is unknown. Determine whether each proposition below is True, False, or has unknown status at this time. Justify your answer.
\be
\item $p \leftrightarrow (q \rightarrow r)$
\vfill
\item $\neg(p \vee r) \wedge q$
\vfill
\ee
\newpage

\item (10 points) Determine if the argument below is valid. Justify your answer. (Note that should you choose to use the truth table, you must explain what about your table shows that the argument of valid or not.)\\

\begin{tabular}{c}
$p\vee q$\\
$(p\wedge q) \rightarrow r$\\
$q \wedge \neg r $\\
\hline
$\therefore \neg p$\\
\end{tabular}
\hspace{.5in}
\begin{tabular}{|c|c|c|p{10cm}|}
p&q&r&\\
\hline \hline
T&T&T&\\
\hline
T&T&F&\\
\hline
T&F&T&\\
\hline
T&F&F&\\
\hline
F&T&T&\\
\hline
F&T&F&\\
\hline
F&F&T&\\
\hline
F&F&F&\\
\hline
\end{tabular}
\vfill

\item  (10 points) Write the converse and the contrapositive of the following: \\
\begin{center}\emph{The team wins if the quarterback can run.}\end{center}

Converse:\\
\vspace{.2in}

Contrapositive:\\
\vspace{.2in}


\item (10 points) Write the negation of the proposition $\forall x \Big( (P(x) \vee Q(x)) \rightarrow \neg R(x)\Big)$
\vfill
\newpage

\item (10 points) Write down two distinct propositions which are logically equivalent to $\neg p \rightarrow q.$
\vspace{2in}

\item (10 points) Determine the truth value of each proposition below and justify your answer.
\begin{enumerate}
\item $\forall x \in \ZZ ,~\forall y \in \ZZ ~ \big( (x < y) \rightarrow (x^2 < y^2) \big)$
\vfill
\item $\forall v \in \RR^+,~ \exists u \in \RR^+$ such that $\frac{u+1}{v} > 2$
\vfill
\end{enumerate}
\newpage
\item (10 points) Prove the following statement. For any integer $n$, if $n^3$
is an even integer then $n$ is even. (You must state the method of proof you are using: direct, contradiction, contrapositive.)
\vfill
\newpage

\item (10 points) Prove using mathematical induction that for all $n \geq 1$,
$$1 + 4 + 7 + \cdots+ (3n- 2) = \frac{n(3n- 1)}{2}.$$
\newpage
\vfill
Extra Credit (5 points)
Prove that for any real number $x > -1$ and any positive integer $n$, $(1 + x)^n\geq 1 + nx.$ (Full credit will be given only for proofs that use the assumption $x > -1.)$
\vfill
\end{enumerate}
\end{document}
