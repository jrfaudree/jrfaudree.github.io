\documentclass{amsart}
\usepackage{amsmath}
\usepackage{amssymb}
\addtolength\topmargin{-1in}
\addtolength\textheight{1in}
\addtolength{\oddsidemargin}{-.75in}
\addtolength{\evensidemargin}{-.75in}
\addtolength{\textwidth}{1.5in}
\newcommand{\real}{\mathbb{R}}
\newcommand{\un}{\mathcal{U}}
\newcommand{\z}{\mathbb{Z}}
\begin{document}
\noindent Math 307 Discrete Mathematics\\
Exam II \\
7 November 2005 \hfill Name:{\underline{\hspace{2in}}} \\

This exam contains seven questions worth at total of 100 points.
Books, notes, and calculators are not allowed.

\begin{enumerate}
\item (15 points)\begin{enumerate}
\item Use the Euclidean algorithm to find the greatest common
divisor of 53 and 36.\vfill
\item Find integers $s$ and $t$ such that
$53s+36t=\gcd(53,36).$\vfill
\item Find the inverse, $s,$ of 36 mod 53, if it exists. If it does
not exists, explain why.\vfill\end{enumerate}
\newpage

\item (5 points) Calculate $C(11,8).$\vfill


\item (15 points) Problems (a) and (b) refer to the algorithm
below:\\


\noindent for $i=1$ to $n$\\
\indent for $j=1$ to $i$\\
\indent \indent for $k=1$ to $i$\\
\indent \indent \indent $x=x+1$\\

\begin{enumerate}
\item How many times is the statement $x=x+1$ executed if $n=8.$
\vfill
\item Select a theta notation from among $$ \theta(1), \theta(\lg n), \theta(n),
\theta(n\lg n), \theta(n^2), \theta(n^3), \theta(2^n), \theta(n!) $$
for the number of times the statement $x=x+1$ is executed.\vfill
\end{enumerate}\newpage

\item (15 points) Use the definition to show that
$f(n)=2n^2-50n+159n \lg n + 57$ is $O(n^2).$ \vfill

\item (15 points) Let $R$ be a relation on $\real$, the set of real
numbers defined as $x R y$ if $x+1 \leq y.$
\begin{enumerate}
\item Is $R$ reflexive? Explain your answer. \vfill
\item Is $R$ antisymmetric? Explain your answer. \vfill
\end{enumerate} \newpage

\item (15 points) Let $R$ be a relation on $\z \times \z$ defined as
$(a,b) R (c,d)$ if $a+c$ is even.
\begin{enumerate}
\item Give an example of an ordered pair $((a,b),(c,d))$ that is in
the relation $R.$ \vfill

\item Give an example of an ordered pair $((a,b),(c,d))$ that is NOT in
the relation $R.$ \vfill

\item Note that $R$ is an equivalence relation. Show that $R$ is
transitive.\vfill

\item Describe $[(1,2)],$ the equivalence class of $(1,2).$ \vfill

\item How many distinct equivalence classes does $R$ have? \vfill
\end{enumerate} \newpage

\end{enumerate}

\end{document}
