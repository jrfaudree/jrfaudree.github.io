\documentclass{amsart}
%\documentstyle[11pt]{article}
\pagestyle{empty} \setlength{\topmargin}{-.5in}
\addtolength{\textheight}{2in} \addtolength{\evensidemargin}{-1in}
\addtolength{\oddsidemargin}{-1in} \addtolength{\textwidth}{1in}
%\setlength{\parindent}{0pt}

\usepackage{amsmath}
\usepackage{amssymb}
\newcommand{\real}{\mathbb{R}}
\newcommand{\un}{\mathcal{U}}
\newcommand{\z}{\mathbb{Z}}
\begin{document}

* \vfill
\Large{
\begin{center}Math 307\\
Discrete Mathematics\\
Exam II\\
24 October 2003\\
\vspace{1in} Name:{\underline{\hspace{2in}}}\\\end{center}

\vspace{1in} There are seven questions worth a total of 100 points.
Use standard notation (like what is used in class and in the text).
This exam is closed note and closed book. No calculators are
allowed.

\vfill *

\newpage

\begin{enumerate}
\item (20 points) Let $A=\{1,2,3\}, \: B=\{x \in \real | 0 \leq x \leq 2 \},$ and
$C= \{x \in \real | -1 \leq x \leq 3\}.$ Assume the universe is the
set of real numbers.\begin{enumerate}
\item Give an example of a proper, nonempty subset of $A.$ \vfill
\item List two distinct elements of the set $B.$ \vfill
\item List two distinct elements of the set $\mathcal{P}(B),$ the
power set of $B.$ \vfill
\item Give an example of a nonempty set disjoint from $B.$ \vfill
\item Find $B \triangle C.$ \vfill
\item Find $B \cap A.$ \vfill
\item Find $B \cup A.$ \vfill
\item Find $\overline{B}.$ \vfill
\item Find $A - C.$ \vfill
\item Find $\overline{A} \cup \overline{B}.$\vfill
\end{enumerate}
\newpage

\item (20 points) Let $A=\{1,2,3,4, \cdots, 29,30\}.$ \begin{enumerate}
\item How many distinct subsets of $A$ are possible? \vspace{1in}
\item How many distinct subsets of $A$ have exactly eight elements?
\vspace{1in}
\item Assuming all subsets of $A$ are equally likely, what is the
probability of picking a subset with exactly eight elements?
\vspace{1in}
\item How many eight element subsets contain both 1 and 2 or contain
both 3 and 4? \vspace{1.5in} \end{enumerate}

\item(10 points) Use the laws of set theory to simplify
$$\overline{(\overline{A-B}) \cap A}.$$
\vfill
\newpage

\item (10 points) Write and label the converse, inverse, and contrapositive
of the implication:

If it is Saturday and there is snow,  Tom goes skiing. \vspace{3in}

\item (15 points) Assume the universe is the set of real numbers.
Determine whether the following propositions are true or false.
Carefully explain your answers using complete sentences.
\begin{enumerate}
\item $\forall x \hspace{.1in}\exists y \hspace{.1in} 0 < x-y < 1$ \vfill
\item $ \exists x \hspace{.1in} \forall y  \hspace{.1in}
\left[ (x<y) \rightarrow (y^2 > 3) \right]$\vfill
\end{enumerate}

\newpage
\item (10 points) Negate and simplify the following:
$$ \forall x \hspace{.1in}\left[(x >0) \rightarrow (
\exists y \hspace{.1in} 0 < y < \sqrt{x}) \right]$$ \vspace{2in}



\item (15 points) Establish the validity of the argument below by
listing a series of numbered steps and reason for the steps.\\

\noindent $a \rightarrow (b \rightarrow c) \\
d \vee a \\
\neg d \rightarrow b \\
\underline{\neg d \hspace{.4in} }\\
\therefore c$
\end{enumerate}

\bigskip
\underline{Steps} \hspace{2in} \underline{Reasons} \vfill
\end{document}
