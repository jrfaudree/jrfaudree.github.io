\documentclass[12pt]{article}

% Layout.
\usepackage[top=1in, bottom=0.75in, left=1in, right=1in, headheight=1in, headsep=6pt]{geometry}

% Fonts.
\usepackage{mathptmx}
\usepackage[scaled=0.86]{helvet}
\renewcommand{\emph}[1]{\textsf{\textbf{#1}}}


% Misc packages.
\usepackage{amsmath,amssymb,latexsym}
\usepackage{graphicx,hyperref}
\usepackage{array}
\usepackage{xcolor}
\usepackage{multicol}
\usepackage{tabularx,colortbl}
\usepackage{enumitem}
\usepackage{soul}

\newcommand{\blankbox}[2]{\fbox{\rule{#1}{0in}\rule{0in}{#2}}}
%special commands for number sets
\def\RR{{\mathbb R}}
\def\NN{{\mathbb N}}
\def\ZZ{{\mathbb Z}}
\def\QQ{{\mathbb Q}}
\def\CC{{\mathbb C}}
%special commands for formatting: center, enumerate, pmatrix, vector span
\def\bc{\begin{center}}
\def\ec{\end{center}}
\newcommand{\be}{\begin{enumerate}}
\newcommand{\ee}{\end{enumerate}}
\newcommand{\bpm}{\begin{pmatrix}}
\newcommand{\epm}{\end{pmatrix}}
\newcommand{\bv}[1]{\mathbf{#1}}
\newcommand{\spn}[1]{\text{Span}\left\{#1\right\}}
\newcommand{\lra}{\longrightarrow}
\newcommand{\llra}{\longleftrightarrow}
\newcommand{\sse}{\subseteq}

\setlength{\headheight}{30pt}
\setlength{\headsep}{20pt}
\setlength{\fboxsep}{8pt}
\setlength{\fboxrule}{1pt}

% Paragraph spacing
\parindent 0pt
\parskip 6pt plus 1pt
\def\tableindent{\hskip 0.5 in}
\def\ts{\hskip 1.5 em}

%header
\usepackage{fancyhdr}
\pagestyle{fancy} 
\lhead{\large\sf\textbf{MATH 265: Exam 2 Review}}
\rhead{\large\sf\textbf{Spring 2026}}

\newcommand{\localhead}[1]{\par\smallskip\textbf{#1}\nobreak\\}%
\def\heading#1{\localhead{\large\emph{#1}}}
\def\subheading#1{\localhead{\emph{#1}}}
\begin{document}
\thispagestyle{fancy}

Exam 2 will be on Friday 6 March during our regular class time. It will include Chapters 4, 5, 6, 7, 8.  So you will have one hour to complete the exam. You may bring a single 3 in by 5 in notecard with writing on both sides.\\

\begin{center} Definitions \end{center}
For all of the terms below, you must be able to formally state the definition from your textbook. 
\be
\item odd, even, same parity, opposite parity
\item divides, multiple, divisor
\item prime
\item greatest common divisor, least common multiple
\item congruent modulo $n$
\item rational number, irrational number
\ee

\begin{center} Proof Techniques \end{center}
\be
\item direct proof
\item using cases
\item by contrapositive
\item by contradiction
\item if and only if proofs
\item existence proofs
\item proofs involving sets, such as proving $A \sse B$ and $A=B.$
\ee

\begin{center} Things to Keep in Mind \end{center}
\be
\item If a proof technique is not prescribed, you must state the method you are using.
\item You should put in the ``boiler-plate'' language even if you cannot figure out the whole proof.
\item You should expect to \emph{use} all of the hypotheses.
\item I will \emph{not} ask you to prove something that is false.
\ee

\end{document}