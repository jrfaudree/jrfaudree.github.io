\documentclass[11pt]{article}
\setlength{\topmargin}{-0.8in} % usually -0.25in
\addtolength{\textheight}{1.4in} % usually 1.25in
\addtolength{\oddsidemargin}{-0.8in}
\addtolength{\evensidemargin}{-0.8in}
\addtolength{\textwidth}{1.6in} %\setlength{\parindent}{0pt}

\usepackage[T1]{fontenc} % Recommended for better encoding
\renewcommand{\familydefault}{\sfdefault}
\usepackage{helvet}

% macros
\usepackage{amsmath,
amssymb,
mathrsfs,
amsthm,
fancyhdr,
tikz,
multicol,
enumitem
}

\newcommand{\be}{\begin{enumerate}}
 \newcommand{\ee}{\end{enumerate}}


\newcommand{\C}{{\mathbb{C}}}
\newcommand{\R}{{\mathbb{R}}}
\newcommand{\eps}{\epsilon}
\newcommand{\Z}{{\mathbb{Z}}}
\newcommand{\Q}{{\mathbb{Q}}}
\newcommand{\N}{{\mathbb{N}}}



\lhead{{Math 265 Proofs}}
\chead{\large \ Midterm I} 
\rhead{Spring 2026}
\cfoot{}
\pagestyle{fancy}
\begin{document}
\thispagestyle{fancy}


\medskip
\large
\vspace{.1in}
\begin{tabular}{l@{\hspace{.4in}}l}
Your Name & \\
\framebox(200,30){} &  \\
\end{tabular}

%\bigskip

\vfill
{
\renewcommand{\baselinestretch}{1.8}
\setlength{\tabcolsep}{.2in}
\normalsize
\begin{center}
\begin{tabular}{|c|c|c|}
\hline
Problem&Total Points&\parbox{.8in}{\hfil Score\hfil}\\
\hline
1&8&\\
\hline
2&10&\\
\hline
3&10&\\
\hline
4&10&\\
\hline
5&12&\\
\hline
6&12&\\
\hline
7&15&\\
\hline
8&8&\\
\hline
9&15&\\
\hline
\hline
%\hline
Total&100&\\
\hline
%Current Course Grade&\multicolumn{2}{c|  }{}\\
%\hline

\end{tabular}

\end{center}
}
\vfill
\begin{itemize}
\item 
You have 1 hour.

\item If you have a cell phone with you, it should be turned off and put away. (Not in your pocket)

\item You may not use a calculator, book, notes or aids other than a single 3 by 5 notecard.

\item In order to earn partial credit, you must show your work.

\end{itemize}
\newpage
\begin{enumerate}
%Set builder notation
\item (8 pts) Write the set using \textbf{set builder} notation.
		\be
		\item $\{ \dots, -10, -6, -2, 2,6,10, 14, 18, \dots \}$ (Assume the pattern continues.)
		\vfill
		\item The set of real numbers whose squares are integers.
		\vfill
		\ee
% element subset distinction
\item (10 pts) Consider the set $A=\big\{ 1, \, 2, \, 3, \, \{2\}, \, \{2,3\},\,  \{ 1, \{1\}\}\,  \big\}$
	\be
	\item The cardinality of $A$ is \underline{\hspace{1in}}\\
	
	
	\item Determine if the statements below are true or false.
		
		\begin{multicols}{2}
		\be[leftmargin=0cm]
		\item $\emptyset \in A$\\
		\vspace{.2in}
		
		\item $\emptyset \subseteq A$\\
		\vspace{.2in}

		\item $1 \in A$\\
		\vspace{.2in}

		
		\item $1 \subseteq A$\\
		\vspace{.2in}

		
		\item $\{1\} \in A$\\
		\vspace{.2in}
		
		\item $\{1\} \subseteq A$\\
		\vspace{.2in}
		
		\item $\{1,\{1\}\} \subseteq A$\\
		\vspace{.2in}
		
		\item $\{2,\{2\}\} \subseteq A$\\
		\vspace{.2in}

		
		\ee
		\end{multicols}
	\ee

%power set and cartesian product
\item (10 pts) Consider the set $B=\{ n^2 \, : \,  n \in \Z \text{ and } |n| \leq 3 \}$
	\begin{enumerate}
	\item Rewrite the set $B$ by listing its elements between braces.\\ 	\vfill
	\item  $\big\vert\mathcal{P}(B)\big\vert = $ \underline{\hspace{1in}} (where $\mathcal{P}(B)$ is the power set of $B.$)
	\item Show that the following statement is true: $$\exists X_1\in \mathcal{P}(B), \exists X_2 \in \mathcal{P}(B), \,0 < |X_1| < |X_2| \text{ and } X_1 \cap X_2 = \emptyset.$$
	\vfill
	\item $\big\vert B \times B \big\vert=$ \underline{\hspace{1in}}	
	\item Let $D=\{ (a,b) \in B \times B \, : \, a+b=1\}.$ Rewrite the set $D$ by listing its elements between braces.
	\vfill
	\ee
\newpage
%union and complement
\item (10 pts) Let $A=\{ (x,y) \in \R \times \R \, : \, y \leq 1\}$, $B =\{ (x,y) \in \R \times \R \, : \, y \geq x^2\}$, and the universal set, $U=\R \times \R,$ the entire $xy$-plane. Sketch each set below. Clearly label your pictures.
	\begin{multicols}{2}
	\be
	\item $A \cap B$
	\item $ \overline{A} \cup B$
	\ee
	\end{multicols}
	\vfill
	\vfill
\item (12 pts) For any number $n,$ let $A_n = [0,\frac{n}{n+1}].$ Determine the following subsets of the real line.
	\begin{multicols}{2}
	\be
	\item $A_1=$ \underline{\hspace{1in}}\\
	
	\item $A_2=$ \underline{\hspace{1in}}\\
	
	\item $A_3=$ \underline{\hspace{1in}}\\
	
	\item $\displaystyle\bigcap_{n \in \N} A_n=$ \underline{\hspace{1in}}\\
	\vspace{.2in}
	\item $\displaystyle\bigcup_{n \in \N}  A_n=$ \underline{\hspace{1in}}\\
	\vspace{.2in}
	\ee
	\end{multicols}

%% Write If-Thens
\item (12 pts) Rewrite each sentence below in the form ``If $P$, then $Q.$"
	\be
	\item In order for Rachel to wear black, it is necessary that it is a Tuesday.\\
	\vspace{.5in}
	
	\item The presence of a full moon is sufficient for the flower to bloom.\\
	\vspace{.5in}
	\item The card is an ace only if the table is flat.\\
	\vspace{.5in}
	\item The blueberries are ripe or the cranberries are ripe.\\
	\vspace{.5in}
	\ee
\newpage
%% english to logic, negations, back to english
\item (15 pts)  For parts (a) and (b): (i) rewrite the given statement symbolically and, then,  (ii) negate it. 
	\be
	\item For every subset $X$ of $\N$ there is always another subset $Y$ of $\N$ such that $X\not = Y$ and $ X \subseteq Y.$\\
		\be[leftmargin=0cm]
		\item symbolic form: \\
		\vspace{.5in}
		
		\item negation: \\
		\vspace{.5in}
		\ee
	
	\item For every $\epsilon > 0$, there is a $\delta >0$ such that if $|x-a| < \delta$ and $x \not = a,$ then $|f(x)-L| < \epsilon.$ (Note: The function $f(x)$ is fixed and $a$ and $L$ represent fixed constants.)
		\be[leftmargin=0cm]
		\item symbolic form: \\
		\vspace{.5in}
		
		\item negation: \\
		\vspace{.5in}
		\ee

	\item Determine if the statement in part (a) is true or false. Justify your answer.\\
	\vfill
	\ee
\item(8 pts) Show that the statements \fbox{$(P \Rightarrow Q) \Rightarrow R$} and \fbox{$(P \land Q) \lor R$} are \textbf{not} logically equivalent.
\vfill
\newpage
%% Truth table + valid argument
\item (15 pts) \textbf{Use a truth table} to determine whether the argument below is valid or invalid. \textbf{Your answer must include:}
\be
\item[(a)] A complete truth table with clearly labeled columns.
\item[(b)] An explanation of how the truth table demonstrates whether the argument is valid or invalid
\ee

\textbf{Note:} Organize your work clearly. Points will be deducted for poor organization.\\

\begin{tabular}{c}
$P \: \Rightarrow \: R$ \\
$Q \: \Rightarrow \: R $ \\
\hline
$\therefore \: P \lor Q \Rightarrow \: R$ \\
\end{tabular}
\newpage

\end{enumerate}
\end{document}