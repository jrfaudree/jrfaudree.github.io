\documentclass[12pt]{article}

% Layout.
\usepackage[top=1in, bottom=0.75in, left=1in, right=1in, headheight=1in, headsep=6pt]{geometry}

% Fonts.
\usepackage{mathptmx}
\usepackage[scaled=0.86]{helvet}
\renewcommand{\emph}[1]{\textsf{\textbf{#1}}}

% Misc packages.
\usepackage{amsmath,amssymb,latexsym}
\usepackage{graphicx,hyperref}
\usepackage{array}
\usepackage{xcolor}
\usepackage{multicol}
\usepackage{tabularx,colortbl}
\usepackage{enumitem}
\usepackage{soul}

\hypersetup{
    colorlinks=true,
    linkcolor=blue,
    filecolor=magenta,      
    urlcolor=blue,
    pdftitle={Proofs Homework},
    pdfpagemode=FullScreen,
    }

\def\mailto#1{\href{mailto:#1}{#1}}

%% special math symbols
\newcommand{\N}{\mathbb{N}}
\newcommand{\Z}{\mathbb{Z}}
\newcommand{\R}{\mathbb{R}}
\newcommand{\Q}{\mathbb{Q}}
\newcommand{\sse}{\subseteq}
\newcommand{\mcp}{\mathcal{P}}
\newcommand{\ol}[1]{\overline{#1}}

%other specials
\newcommand{\be}{\begin{enumerate}}
\newcommand{\ee}{\end{enumerate}}
\newcommand{\ds}{\displaystyle}


% Paragraph spacing
\parindent 0pt
\parskip 6pt plus 1pt
\def\tableindent{\hskip 0.5 in}
\def\ts{\hskip 1.5 em}

%header
\usepackage{fancyhdr}
\pagestyle{fancy} 
\lhead{\large\sf\textbf{MATH 265: Introduction to Mathematical Proofs}}
\rhead{\large\sf\textbf{HW 4}}

\newcommand{\localhead}[1]{\par\smallskip\textbf{#1}\nobreak\\}%
\def\heading#1{\localhead{\large\emph{#1}}}
\def\subheading#1{\localhead{\emph{#1}}}

\newenvironment{clist}%
{\bgroup\parskip 0pt\begin{list}{$\bullet$}{\partopsep 4pt\topsep 0pt\itemsep -2pt}}%
{\end{list}\egroup}%

\begin{document}
\quad \\

\begin{center} {\huge{Homework 4}} \end{center}

\hrulefill

Section 2.7 Quantifiers: \#1-10\\

\hrulefill

Section 2.9 \# 1-7, 9,10\\

\hrulefill

Section 2.10 \# 1-6, 8-10\\

\hrulefill

Section 2.11 \# A, B, C, D\\

\textbf{Directions:} For several problems you are asked to \emph{write an argument in English.} In all cases, you are expected to use only complete sentences. You should use correct grammar and punctuation. Moreover, every sentence must begin with an English word and cannot begin with a symbol. So, a sentence that begins, ``There exists ..." is OK. A sentence that begins ``$\exists$..." is not. A sentence that begins "Observe the function $f(x) >0,$..." is OK. A sentence that begins ``$f(x) >0$..." is not.

\be
\item[A.] Use a truth table to show that the argument below is valid.

\begin{tabular}{l}
$P \Rightarrow Q$\\
$Q \Rightarrow R$\\
\hline
$P \Rightarrow R$\\
\end{tabular} 
\item[B.]
	\be
	\item[i.] Show that the argument below is an invalid argument by finding truth values for $P$ and $Q$ that demonstrate this.\\
	
	\begin{tabular}{l} $P \vee Q$\\ $P$ \\ \hline $\sim Q$ \end{tabular} \\
	\item[ii.] Write an argument in English with the logical structure in part (i.) and then show that the conclusion is false. \\
	Example English Argument (that you cannot use): Every student in Introduction to Proofs has also passed Calculus 2 or Linear Algebra. Jane Doe passed Calculus 2. Thus, Jane Doe did not pass Linear Algebra. 
	\ee
\item[C.] 
	\be
	\item[i.] Construct an \textbf{open, conditional statement}, $P(x) \Rightarrow Q(x)$, that you know to be true. This statement should be written as a sentence in English, but its content should be mathematical. \\
	Example (that you cannot use): If $f(x)$ is differentiable on $\R$, then $f(x)$ is continuous on $\R.$\\
	\item[ii.] Using your statement from part (i.), make an argument in English using Modus Ponens.\\
	Example: If $f(x)$ is differentiable on $\R$, then $f(x)$ is continuous on $\R.$ We know the function $f(x)=x^2$ is differentiable on $\R$ since its derivative is $f'(x)=2x$ is defined on all of $\R.$ Thus, we an conclude that the function $f(x)=x^2$ is continuous on $\R.$\\
	\item[iii.] Using your statement from part (i.), make an argument in English using Modus Tollens. \\
	Example: If $f(x)$ is differentiable on $\R$, then $f(x)$ is continuous on $\R.$ We know the function $h(x)=\lfloor x \rfloor$ fails to be continuous at every integer. (FYI $h(x)$ is called the \emph{floor} function or the ``always round down" function. Its graph looks like a ``stair steps.") Thus, we can conclude that $h(x)$ is not differentiable on $\R.$
	\ee
\item[D.] Make an argument in English that the number 20 is composite using the logical structure of Elimination and exactly two facts.\\
Fact 1: Every integer larger than 1 is either prime or composite.\\
Fact 2: An integer larger than 1 is prime if its only divisors are 1 and itself.\\
Observe that the definition of \emph{composite} is not given.\\

\ee


\end{document}  















