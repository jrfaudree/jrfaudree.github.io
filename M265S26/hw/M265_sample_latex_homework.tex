% Header and Set up
\documentclass[11pt]{report}

% List of packages
\usepackage{geometry,
amsmath,
amssymb,
amsthm,
xcolor,
setspace}

% Set page formatting
\geometry{margin=1.in}
\doublespacing

% newcommands
\newcommand{\ans}{\textit{Answer:}}

%Document Starts
\begin{document}

%Your Name
\hfill Your Name Here

%Today's Date
\hfill \today

% Homework Information
% Centered
\begin{center}
Math 265\\
 Homework \#500\\
 due 1/11/2026
\end{center}

% Problems Start
\begin{description}
% First Problem Statement
\item{\fbox{\textbf{\S 4.1, \#3:}}\:} Evaluate $\int_1^2 \frac 1 {x^2} \, dx.$

% Solution to First Problem Statement
\ans

$$\int_1^2 \frac 1 {x^2} \, dx=\int_1^2 x^{-2} \, dx=-x^{-1} \Big |_1^2=-\frac 12+1=\frac 12$$

% Space between problems
\vspace{1in}

% Second Problem Statement
\item{\fbox{\textbf{\S 4.2, \#17}}\: } Prove $\sqrt 2$ is irrational.

% Proof for Second Problem
\begin{proof}
Suppose, to the contrary, that $\sqrt 2$ is rational. Then
$$\sqrt 2=\frac ab$$
where $a,b\in \mathbb Z$, $b\not = 0$ with $a,b$ having no common factors. Squaring yields
$$2=\frac {a^2}{b^2},$$
so
$$2b^2=a^2.$$
This shows 2 divides $a^2$, and so since 2 is prime by a lemma proved in class, we see 2 divides $a$. Letting $a=2c$ for some $c\in \mathbb Z$, this implies
$$2b^2=4c^2,$$ so
$$b^2=2c^2.$$
Now the same argument as above, but with $b,a$ replaced by $c,b$, shows 2 divides $b$. Therefore 2 divides both $a$ and $b$. But this contradicts that $a,b$ had no common factors.
\end{proof}

% Space between problems
\vspace{1in}



% Third Problem
\item[{\fbox{\textbf{\S 4.2, \#18}}\: }] Find the product of $x$ and $y$ supposing that both are odd. \\
\textcolor{red}{This shows you an aligned string of equations.}


%note that "align" automatically puts you in math mode -- no need for $$... $$
\begin{align*}
xy&=(2a+1)(2b+1)\\
&=(2a)(2b)+(2a)(1)+1(2b)+1(1)\\
&=4ab+2a+2b+1\\
&=2(2ab+a+b)+1\\
&=2k+1,
\end{align*}

\textcolor{red}{This shows you an aligned string of equations with justifications.}


%note that "align" automatically puts you in math mode -- no need for $$... $$
\begin{align*}
xy&=(2a+1)(2b+1) && \text{$a$, $b$ integers, (by definition of odd)}\\
&=(2a)(2b)+(2a)(1)+1(2b)+1(1)&& \text{(expanding binomial multiplication)}\\
&=4ab+2a+2b+1&& \text{(simplifying)}\\
&=2(2ab+a+b)+1&& \text{(factoring out a 2)}\\
&=2k+1,&& \text{($k=2ab+a+b$)}\\
\end{align*}

% Space between problems
\vspace{1in}

% Fourth Problem
\item[{\fbox{\textbf{\S 4.2, \#19}}\: }] Make a table of useful \LaTeX symbols.

% An example table
\begin{tabular}{ | c || c | c |c|}
\hline
words& what you type into \LaTeX & what appears in the PDF & example\\
\hline \hline
is an element of & \textbackslash in & $\in$ & $x \in \mathbb{R}$ \\
\hline
is not an element of & \textbackslash not \textbackslash in & $\not\in$ & $x \not\in \mathbb{R}$ \\
\hline
is a subset of & \textbackslash subseteq & $\subseteq$ & $\mathbb{Q} \in \mathbb{R}$ \\
\hline
for all & \textbackslash forall & $\forall$ & $\forall x \in \mathbb{R}$\\
\hline
there exists & \textbackslash exists &$\exists$&\\
\hline
logical and& \textbackslash land &$\land$&\\
\hline
logical or& \textbackslash lor &$\lor$&\\
\hline
logical negation & \textbackslash sim &$\sim$&\\
\hline
subscript& A\_2 &$A_2$&\\
\hline
superscript& x\^{}2&$x^2$&\\
\hline
number sign & \textbackslash\# &$\#$&\\
\hline
ampersand& \textbackslash\& &$\&$&\\
\hline
new line & \textbackslash cr \: or \: \textbackslash\textbackslash&&\\
\hline
backslash & \textbackslash textbackslash &\textbackslash&\\
\hline
\end{tabular}

\end{description}

\end{document}