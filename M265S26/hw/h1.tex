\documentclass[12pt]{article}

% Layout.
\usepackage[top=1in, bottom=0.75in, left=1in, right=1in, headheight=1in, headsep=6pt]{geometry}

% Fonts.
\usepackage{mathptmx}
\usepackage[scaled=0.86]{helvet}
\renewcommand{\emph}[1]{\textsf{\textbf{#1}}}

% Misc packages.
\usepackage{amsmath,amssymb,latexsym}
\usepackage{graphicx,hyperref}
\usepackage{array}
\usepackage{xcolor}
\usepackage{multicol}
\usepackage{tabularx,colortbl}
\usepackage{enumitem}
\usepackage{soul}

\hypersetup{
    colorlinks=true,
    linkcolor=blue,
    filecolor=magenta,      
    urlcolor=blue,
    pdftitle={Overleaf Example},
    pdfpagemode=FullScreen,
    }

\def\mailto#1{\href{mailto:#1}{#1}}

%% special math symbols
\newcommand{\N}{\mathbb{N}}
\newcommand{\Z}{\mathbb{Z}}
\newcommand{\sse}{\subseteq}


% Paragraph spacing
\parindent 0pt
\parskip 6pt plus 1pt
\def\tableindent{\hskip 0.5 in}
\def\ts{\hskip 1.5 em}

%header
\usepackage{fancyhdr}
\pagestyle{fancy} 
\lhead{\large\sf\textbf{MATH 265: Introduction to Mathematical Proofs}}
\rhead{\large\sf\textbf{Spring 2026 Syllabus}}

\newcommand{\localhead}[1]{\par\smallskip\textbf{#1}\nobreak\\}%
\def\heading#1{\localhead{\large\emph{#1}}}
\def\subheading#1{\localhead{\emph{#1}}}

\newenvironment{clist}%
{\bgroup\parskip 0pt\begin{list}{$\bullet$}{\partopsep 4pt\topsep 0pt\itemsep -2pt}}%
{\end{list}\egroup}%

\begin{document}
\quad \\

\begin{center} {\huge{Homework 1}} \end{center}

\hrulefill

Section 1.1 Introduction to Sets \# E, F, 2, 3, 4, 9, 12, 14, 16, 18, 28, 32, 33, 38, 40, 46, 50, G

\begin{enumerate}
\item[E.] Write each statement in English entirely in symbols.
	\begin{enumerate}
	\item[a.] The set $S$ is the set of all nonnegative, even integers.
	\item[b.] The set of polynomials of degree 2 with real coefficients.
	\item[c.] The set of real-valued functions of a real variable that contain the point $(3, \pi).$
	\item[d.] The set of integers is a subset of the set of rational numbers.
	\end{enumerate}
\item[F.] Write each mathematical statement as a sentence in English. 
	\begin{enumerate}
	\item[a.] $A=\{ \emptyset \}$
	\item[b.] $\Z \sse \N$
	\end{enumerate}
\item[G.] Construct a set $A$ such that $\{1,2\}$ is \emph{both} an element of $A$ and a subset of $A.$

\vspace{.3in}

(This section has 18 problems. Each part of each problems is worth 1 points for a total of 22 points.)\\

\vspace{.3in}

\hspace*{-.3in}\hrulefill

\hspace*{-.3in}Section 1.2 The Cartesian Product \# 2, 8, 10, 12, 14, 18, C\\

\hspace*{-.3in} C. Determine the cardinality or each set below.
	\begin{enumerate}
	\item[a.] $ \{0,1,2,3\} \times \{a,\: b,\: \{c,d\},\: e, \emptyset \} $
	\item[b.] $ ( \: \{1,2,3,4 \} \times \{1,2,3\} \: ) \times \{1,2 \}$
	\end{enumerate}

\vspace{.3in}

(This section has 7 problems. Each part of each problems is worth 1 points for a total of 15 points.)\\

\end{enumerate}

\end{document}  