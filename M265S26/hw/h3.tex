\documentclass[12pt]{article}

% Layout.
\usepackage[top=1in, bottom=0.75in, left=1in, right=1in, headheight=1in, headsep=6pt]{geometry}

% Fonts.
\usepackage{mathptmx}
\usepackage[scaled=0.86]{helvet}
\renewcommand{\emph}[1]{\textsf{\textbf{#1}}}

% Misc packages.
\usepackage{amsmath,amssymb,latexsym}
\usepackage{graphicx,hyperref}
\usepackage{array}
\usepackage{xcolor}
\usepackage{multicol}
\usepackage{tabularx,colortbl}
\usepackage{enumitem}
\usepackage{soul}

\hypersetup{
    colorlinks=true,
    linkcolor=blue,
    filecolor=magenta,      
    urlcolor=blue,
    pdftitle={Proofs Homework},
    pdfpagemode=FullScreen,
    }

\def\mailto#1{\href{mailto:#1}{#1}}

%% special math symbols
\newcommand{\N}{\mathbb{N}}
\newcommand{\Z}{\mathbb{Z}}
\newcommand{\R}{\mathbb{R}}
\newcommand{\Q}{\mathbb{Q}}
\newcommand{\sse}{\subseteq}
\newcommand{\mcp}{\mathcal{P}}
\newcommand{\ol}[1]{\overline{#1}}

%other specials
\newcommand{\be}{\begin{enumerate}}
\newcommand{\ee}{\end{enumerate}}
\newcommand{\ds}{\displaystyle}


% Paragraph spacing
\parindent 0pt
\parskip 6pt plus 1pt
\def\tableindent{\hskip 0.5 in}
\def\ts{\hskip 1.5 em}

%header
\usepackage{fancyhdr}
\pagestyle{fancy} 
\lhead{\large\sf\textbf{MATH 265: Introduction to Mathematical Proofs}}
\rhead{\large\sf\textbf{HW 3}}

\newcommand{\localhead}[1]{\par\smallskip\textbf{#1}\nobreak\\}%
\def\heading#1{\localhead{\large\emph{#1}}}
\def\subheading#1{\localhead{\emph{#1}}}

\newenvironment{clist}%
{\bgroup\parskip 0pt\begin{list}{$\bullet$}{\partopsep 4pt\topsep 0pt\itemsep -2pt}}%
{\end{list}\egroup}%

\begin{document}
\quad \\

\begin{center} {\huge{Homework 3}} \end{center}

\hrulefill

Section 1.7 Venn Diagrams: \#4, A, 12 \\
For 12, try to make it simple!

\be
\item[A.]
	\be
	\item[a.] Draw a Venn diagram for the set $(A-B) \cup C$
	\item[b.] Draw a Venn diagram for the set $A-(B \cup C)$
	\item[c.] Explain what the Venn diagrams in \#4 (above) and parts (a) and (b) indicate.
	\ee
\ee

\hrulefill

Section 1.8 \#1, 4, ,6, 8, 9, 10, A, B

\be
\item[A.] Let $A_n= \left( \frac{-1}{n}, \frac{1}{n} \right) \sse \R$ for $n \in \N.$ (For clarity, $A_n$ is an \textbf{interval} on the real line, not a point in the $xy$-plane.) Determine $\ds \bigcup_{n=1}^\infty A_n $ and $\ds \bigcap_{n=1}^\infty A_n $

\item[B.] Let $A_\alpha= \R - \alpha$ for $\alpha \in [0,1].$ Determine $\ds \bigcup_{\alpha \in [0,1]} A_\alpha $ and $\ds \bigcap_{\alpha \in [0,1]} A_\alpha $ 

\ee

\hrulefill

Section 2.2 And, Or, or Not

Translate each sentence to logical symbols by reducing chunks of the language to symbolic statements (like $P$, $Q$, $R$) and using $\vee,$ $\wedge$ or $\sim.$ An example is below.

\begin{quote} \textbf{Sentence:} The integer $n$ is divisible by the first three primes.\\
\textbf{Answer:} Let $P(n)= n$ is divisible by 2, $Q(n)= n$ is divisible by 3, and $R(n)= n$ is divisible by 5. Now, the statement becomes $P(n) \wedge Q(n) \wedge R(n).$
\end{quote} 
	\be
	\item The function $f(x)$ is continuous but not differentiable.
	\item At least one of $x$ or $y$ is equal to zero.
	\item Each of the functions $f(x), g(x)$ and $h(x)$ contains the point $(1,2).$
	\ee

\hrulefill

Section 2.3 Conditional Statements \#1-7, A, B, C, D

The directions for problems 1-7 apply for problems A, B, C, D

\be
\item[A.] For the integer to be even, it is sufficient that the integer is greater than 5.
\item[B.] For the bird to be black, it is necessary that the bird is a raven.
\item[C.] Whenever a series converges, the ratio test will give a value greater than 1. 
\item[D.] Luna will eat a treat only if today is Tuesday.
\ee

\hrulefill

Section 2.4 Biconditional Statements \#3,4

\hrulefill

Section 2.5 Truth Tables for Statements \#1,2,3,4,7,10,11

\hrulefill

Section 2.6 Logical Equivalence \# 5,7, 10*, 11*, 12*

* Note: The directions use the word "Decide..." The expectation is that you determine whether or not the statements are logically equivalent \emph{and} rigorously justify your conclusion.


\end{document}  















