\documentclass[12pt]{article}

% Layout.
\usepackage[top=1in, bottom=0.75in, left=0.75in, right=0.75in, headheight=1in, headsep=6pt]{geometry}

% Fonts.
\usepackage{mathptmx}
\usepackage[scaled=0.86]{helvet}
\renewcommand{\emph}[1]{\textsf{\textbf{#1}}}

% Misc packages.
\usepackage{amsmath,amssymb,latexsym}
\usepackage{graphicx,hyperref}
\usepackage{array}
\usepackage{xcolor}
\usepackage{multicol}
\usepackage{tabularx,colortbl}
\usepackage{enumitem}
\usepackage{soul,multicol}

\hypersetup{
    colorlinks=true,
    linkcolor=blue,
    filecolor=magenta,      
    urlcolor=blue,
    pdftitle={Proofs Worksheet},
    pdfpagemode=FullScreen,
    }

\def\mailto#1{\href{mailto:#1}{#1}}

%% special math symbols
\newcommand{\N}{\mathbb{N}}
\newcommand{\Z}{\mathbb{Z}}
\newcommand{\R}{\mathbb{R}}
\newcommand{\Q}{\mathbb{Q}}
\newcommand{\sse}{\subseteq}
\newcommand{\mcp}{\mathcal{P}}
\newcommand{\ol}[1]{\overline{#1}}

%other specials
\newcommand{\be}{\begin{enumerate}}
\newcommand{\ee}{\end{enumerate}}
\newcommand{\se}{\subseteq}
\newcommand{\ds}{\displaystyle}


% Paragraph spacing
\parindent 0pt
\parskip 6pt plus 1pt
\def\tableindent{\hskip 0.5 in}
\def\ts{\hskip 1.5 em}

%header
\usepackage{fancyhdr}
\pagestyle{fancy} 
\lhead{\large\sf\textbf{MATH 265: Introduction to Mathematical Proofs}}
\rhead{\large\sf\textbf{Worksheet 4: \S 2.1-2.3}}

\newcommand{\localhead}[1]{\par\smallskip\textbf{#1}\nobreak\\}%
\def\heading#1{\localhead{\large\emph{#1}}}
\def\subheading#1{\localhead{\emph{#1}}}

\newenvironment{clist}%
{\bgroup\parskip 0pt\begin{list}{$\bullet$}{\partopsep 4pt\topsep 0pt\itemsep -2pt}}%
{\end{list}\egroup}%

\begin{document}
\be
\item Suppose $P$ and $Q$ are true and $R$ and $S$ are false. Determine the truth value of each logical statement below. Think about how you can write down your reasoning or your work.
	\be
	\item $(P \vee Q) \wedge (R \vee S)$\\
	
	%$=(T \vee T) \wedge (F \vee F)\\ = T \wedge F\\=F$
	\vfill
	\item $(P \vee R) \Rightarrow (Q \wedge S)$\\
	
	%$=(T \vee F)\Rightarrow (T \wedge F)\\=T \Rightarrow F\\ = F$
	\vfill
	\item $((P \wedge \sim P) \Rightarrow S) \Rightarrow Q$\\
	
	%$=((T \wedge F) \Rightarrow F) \Rightarrow T \\= (F \Rightarrow F) \Rightarrow T \\= T \Rightarrow T \\= T$
	\vfill
	\ee
\newpage
\item For each sentence below, write its logical structure in symbols. Make sure to clarify which words are associated with which letters. Hint: All but (b) and (c) are conditional statements.
	\be
	\item Differentiability is sufficient for continuity. \\
	
%	Differentiability implies continuity or 
%	
%	\fbox{$P \Rightarrow Q$ where $P$ is differentiability and $Q$ is continuity.}\\
%	
%	\textbf{Sufficient conditions are hypotheses.}
%	
	\vfill
	\item At least one of $a$ or $b$ is an integer.\\
	
%	$a \in \Z$ or $b \in \Z$\\
%	
%	\fbox{$P \vee Q$ where $P$ is $a \in \Z$ and $Q$ is $b \in \Z$}\\
	
	\vfill
	\item Both $A$ and $B$ are subsets of $C.$
	
	%$P \wedge Q$ where $P$ is $A \sse C$ and $Q$ is $B \sse C.$\\
	\vfill
	\item The grass is green whenever the sky is blue.\\
	
	%Equivalently: If the sky is blue, then the grass is green. \\
	
	%\fbox{$P \Rightarrow Q$ where $P$ is the sky is blue and $Q$ is grass is green.}\\

	\vfill
	\item My car turns on only if it's a leap year.\\
	
	%Equivalently: If my car turns on, then it's a leap year.\\
	%$P \Rightarrow Q$
	\vfill
	\item It is a Monday provided the door is open.\\
	
	%Equivalently: If the door is open, then it is a Monday. 
	%$P \Rightarrow Q$
	\vfill
	\item Warm bread is necessary for cold water.\\
	
	%Equivalently: If the water is cold then the bread is warm.\\
	%$P \Rightarrow Q$
	\vfill
	\ee 
\ee

\newpage
\begin{center} Jill's Notes \end{center}
The purpose of Chapter 2 is to rigorously define and understand the underlying logical structure of sentences and arguments written in words. We want to be able to:
\begin{itemize}
\item Go back and forth between English and logical symbols
\item How to determine the truth value of very complicated statements
\item How to determine whether an argument is logically sound and, if not, identify the error.
\end{itemize}

\be
\item (2.1) A \textbf{statement} is an assertion (sentence, mathematical expression) that is true or false, typically denoted with capital letters, $P$, $Q$, $R$, etc.\\
Some examples and non-examples below. 
	\be
	\item $2 \geq 1$ \textcolor{red}{A statement that is true.}
	
	\item $f(x) = 1/x$ is continuous on $(-\infty, \infty)$ \textcolor{red}{A statement that is false.}
	
	\item $ax^2+bx +c$ \textcolor{red}{Not a statement.}
	
	\item If $f(x)$ is differentiable, then $f(x)$ is continuous. \textcolor{red}{A statement that is true.}
	
	\item $\Z \sse \N.$ \textcolor{red}{A statement that is false.}

	\ee
\item (2.2 and 2.3) New Statements from Old
	\begin{itemize}
	\item Consider some ways to combine statements into more complicated statements.
	\item Determine truth value of the combined statement based on the truth values of its (simpler, not compound) statements.
	\item How to use a truth table to define/communicate truth values.
	\end{itemize}
\item Some examples to think about intuitively. The goal here is to see that the definitions we are about to state are intuitive.
	\be
	\item \textbf{OR}: \\
	P or Q \\
	$P \vee Q$\\
	Examples (\textcolor{red}{Should $R, S, T, U$ be true or false?})\\
	 R: $\Z \sse \N$ or $\N \sse \Z.$ (true) \\
	 S: $2$ is odd or $2$ is negative. (false)\\
	 T: $2 \in \N$ or $2 \in \R$. (true) \\
	 U: Today I will ski or go to the gym.\\
	 You are intuiting a definition. 
	\begin{center}
	\begin{tabular}{|c|c|c|}
	\hline
	$P$ & $Q$ & $P \vee Q$ \\
	\hline
	T & T & T  \\
	T & F & T \\
	F & T & T  \\
	F & F & F \\
	\hline
	\end{tabular}
	
	\end{center}
	
	\vfill
	
	\item \textbf{AND}: P and Q , $P \wedge Q$\\
	Examples (\textcolor{red}{Should $R, S, T, U$ be true or false?})\\
	 R: $\Z \sse \N$ and $\N \sse \Z.$ (false) \\
	 S: $2$ is odd and $2$ is negative. (false)\\
	 T: $2 \in \N$ and $2 \in \R$. (true) \\
	 U: Today I will ski and go to the gym.\\
	 You are intuiting a definition. 
	\begin{center}
	\begin{tabular}{|c|c|c|}
	\hline
	$P$ & $Q$ & $P \wedge Q$ \\
	\hline
	T & T & T  \\
	T & F & F \\
	F & T & F \\
	F & F & F \\
	\hline
	\end{tabular}
	
	\end{center}
	
	\vfill

	\item \textbf{NOT}: not P , $\sim P $\\
	Examples (\textcolor{red}{Should $R, S, T, U$ be true or false?})\\
	 R: $\sim(\N \sse \Z)$ which could also be $ \N \not\sse \Z$  (false) \\
	 S: $\sim$($2$ is odd) which could be written $2$ is even (true)\\
	 U: ''It is not the case that today I will ski." or, less awkwardly, ``I will not ski today."\\
	 You are intuiting a definition. 
	\begin{center}
	\begin{tabular}{|c|c|}
	\hline
	$P$ & $\sim P$ \\
	\hline
	T & F  \\
	F & T  \\
	\hline
	\end{tabular}
	
	\end{center}
	Observe this means that if statement $P$ is not true, it must be false; and if $P$ is not false, it must be true.\\
	\vfill

	\item \textbf{Conditional}: \\
	 If P, then Q \\
	 $P \Rightarrow Q$\\
	Examples (\textcolor{red}{Should $R, S, T$ be true or false?})\\
	 R: If $x \leq 5,$ then $x \leq 10.$ (true)\\
	 S: If $x \in \R,$ then $x \leq x^2.$ (false, take $x= 0.5$)\\
	 U: If I walk, then I'll be late.\\
	 U': If I have wings, then I'll fly. (vacuously true)\\ 
	 T: If $ \pi \in \Z$, then $2 \pi$ is even. (vacuously true)\\
	 You are intuiting a definition. 
	\begin{center}
	\begin{tabular}{|c|c|c|}
	\hline
	$P$ & $Q$ & $P \Rightarrow Q$ \\
	\hline
	T & T & T  \\
	T & F & F \\
	F & T & T  \\
	F & F & T \\
	\hline
	\end{tabular}
	
	\end{center}
	
	\vfill
 
	\ee

\end{enumerate}
\end{document}  















