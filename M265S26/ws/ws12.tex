\documentclass[12pt]{article}

% Layout.
\usepackage[top=1in, bottom=0.75in, left=0.75in, right=0.75in, headheight=1in, headsep=6pt]{geometry}

% Fonts.
\usepackage{mathptmx}
\usepackage[scaled=0.86]{helvet}
\renewcommand{\emph}[1]{\textsf{\textbf{#1}}}

% Misc packages.
\usepackage{amsmath,amssymb,latexsym,amsthm}
\usepackage{graphicx,hyperref}
\usepackage{array}
\usepackage{xcolor}
\usepackage{multicol}
\usepackage{tabularx,colortbl}
\usepackage{enumitem}
\usepackage{soul,multicol}

\hypersetup{
    colorlinks=true,
    linkcolor=blue,
    filecolor=magenta,      
    urlcolor=blue,
    pdftitle={Proofs Worksheet},
    pdfpagemode=FullScreen,
    }

\def\mailto#1{\href{mailto:#1}{#1}}

%% special math symbols
\newcommand{\N}{\mathbb{N}}
\newcommand{\Z}{\mathbb{Z}}
\newcommand{\R}{\mathbb{R}}
\newcommand{\Q}{\mathbb{Q}}
\newcommand{\sse}{\subseteq}
\newcommand{\mcp}{\mathcal{P}}
\newcommand{\ol}[1]{\overline{#1}}

%other specials
\newcommand{\be}{\begin{enumerate}}
\newcommand{\ee}{\end{enumerate}}
\newcommand{\se}{\subseteq}
\newcommand{\ds}{\displaystyle}
\newcommand{\lcm}{\text{lcm}}
%\newcommand{\gcd}{\text{gcd}}

% Paragraph spacing
\parindent 0pt
\parskip 6pt plus 1pt
\def\tableindent{\hskip 0.5 in}
\def\ts{\hskip 1.5 em}

%header
\usepackage{fancyhdr}
\pagestyle{fancy} 
\lhead{\large\sf\textbf{MATH 265: Introduction to Mathematical Proofs}}
\rhead{\large\sf\textbf{Worksheet 12: Ch 6}}

\newcommand{\localhead}[1]{\par\smallskip\textbf{#1}\nobreak\\}%
\def\heading#1{\localhead{\large\emph{#1}}}
\def\subheading#1{\localhead{\emph{#1}}}

\newenvironment{clist}%
{\bgroup\parskip 0pt\begin{list}{$\bullet$}{\partopsep 4pt\topsep 0pt\itemsep -2pt}}%
{\end{list}\egroup}%

\begin{document}
\be
\item Prove there are an infinite number of primes.
\vfill
\item Prove that any two rational numbers $x$ and $y$ can be written in the form $x=\frac{a}{n}$ and $y=\frac{b}{n}$ such that $\gcd(a,b,n)=1$ but that it is not possible to conclude that $\gcd(a,n)=1$ or $\gcd(b,n)=1.$
\vfill
\item Statements about even and odd numbers could be rewritten in the language of integers modulo 2. For example, the statement:\\
\fbox{If $n$ is odd, then $n^2$ is odd} could be rewritten as
\fbox{If $n \equiv 1 \: (\text{mod } 2)$, then $n^2 \equiv 1 \: (\text{mod } 2).$} 

Make \textbf{conjectures} about what happens when you consider the squares of integers modulo 3 and then prove that you are correct.
\vfill
\newpage
\item Make a conjecture about when the sum of two integers can be congruent to $0$ modulo 3 and prove that you are correct. Your proposition will look something like the one below:

\textbf{Proposition:} Let $a,b \in \Z$ such that $a+b \equiv 0 \: (\text{mod } 3),$ then [ \textit{something here about the nature of $a$ and $b$ modulo 3}].
\vspace{2in}
\item Describe the set of points in the $xy$-plane that satisfy $x^2+y^2-3=0.$
\vspace{1in}
\item Prove that $x^2+y^2-3=0$ contains no rational points. (That is, for every $(x,y) \in \Q \times \Q,$  $x^2+y^2-3 \not =0.$ Also, the previous propositions should help and please pause to think at least momentarily about how interesting this result is.)

\vfill
\end{enumerate}
\end{document}  















