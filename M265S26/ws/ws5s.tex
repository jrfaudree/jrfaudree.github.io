\documentclass[12pt]{article}

% Layout.
\usepackage[top=1in, bottom=0.75in, left=0.75in, right=0.75in, headheight=1in, headsep=6pt]{geometry}

% Fonts.
\usepackage{mathptmx}
\usepackage[scaled=0.86]{helvet}
\renewcommand{\emph}[1]{\textsf{\textbf{#1}}}

% Misc packages.
\usepackage{amsmath,amssymb,latexsym}
\usepackage{graphicx,hyperref}
\usepackage{array}
\usepackage{xcolor}
\usepackage{multicol}
\usepackage{tabularx,colortbl}
\usepackage{enumitem}
\usepackage{soul,multicol}

\hypersetup{
    colorlinks=true,
    linkcolor=blue,
    filecolor=magenta,      
    urlcolor=blue,
    pdftitle={Proofs Worksheet},
    pdfpagemode=FullScreen,
    }

\def\mailto#1{\href{mailto:#1}{#1}}

%% special math symbols
\newcommand{\N}{\mathbb{N}}
\newcommand{\Z}{\mathbb{Z}}
\newcommand{\R}{\mathbb{R}}
\newcommand{\Q}{\mathbb{Q}}
\newcommand{\sse}{\subseteq}
\newcommand{\mcp}{\mathcal{P}}
\newcommand{\ol}[1]{\overline{#1}}

%other specials
\newcommand{\be}{\begin{enumerate}}
\newcommand{\ee}{\end{enumerate}}
\newcommand{\se}{\subseteq}
\newcommand{\ds}{\displaystyle}


% Paragraph spacing
\parindent 0pt
\parskip 6pt plus 1pt
\def\tableindent{\hskip 0.5 in}
\def\ts{\hskip 1.5 em}

%header
\usepackage{fancyhdr}
\pagestyle{fancy} 
\lhead{\large\sf\textbf{MATH 265: Introduction to Mathematical Proofs}}
\rhead{\large\sf\textbf{Worksheet 5: \S 2.4-2.6}}

\newcommand{\localhead}[1]{\par\smallskip\textbf{#1}\nobreak\\}%
\def\heading#1{\localhead{\large\emph{#1}}}
\def\subheading#1{\localhead{\emph{#1}}}

\newenvironment{clist}%
{\bgroup\parskip 0pt\begin{list}{$\bullet$}{\partopsep 4pt\topsep 0pt\itemsep -2pt}}%
{\end{list}\egroup}%

\begin{document}
Jill's Solutions
\be
\item Review:
	\be
	\item Fill out the truth table for the biconditional statement.
	\begin{center}
	\begin{tabular}{|c|c|c|}
	\hline
	$P$ & $Q$ & $P \Leftrightarrow Q$ \\
	\hline
	T & T &  T \\
	T & F &  F\\
	F & T & F  \\
	F & F & T \\
	\hline
	\end{tabular}
	\end{center}
	
	\item For \emph{objects} $A$ and $B$, how could you show that the statement below was \emph{false}.
	\begin{quote} $AB=0$ if and only if $A=0$ or $B=0.$ \end{quote}
	\vfill
	\textbf{ans} Find two objects $A$ and $B$ such that $AB=0$ and $A \not 0$ and $B \not 0$. Thus, $P: AB=0$ is true and $Q: A=0 \vee B=0$ is false.\\
	
	FYI: The biconditional statement is true for real numbers but false, in general, for matrices, modular arithmetic, and many other mathematical objects.
	
	\item Fill out DeMorgan's Laws\\
	\begin{quote} 
	$\sim (P \vee Q)$ is equivalent to $\sim P \: \wedge \: \sim Q$ \\
	
	\vspace{.2in}
	
	$\sim (P \wedge Q)$ is equivalent to $\sim P \: \vee \: \sim Q$ \\
	\end{quote}
	
	\ee
 \item Use a truth table to demonstrate that \fbox{$P \Rightarrow Q$} is equivalent to \fbox{$\sim P \vee Q$}.
\vfill
\begin{center}
	\begin{tabular}{|c|c|c|c|c|}
	\hline
	1&2&3&4&5\\
	$P$ & $Q$ & $P \Rightarrow Q$ &  $\sim P$ &  $\sim P \vee  Q$\ \\
	\hline
	T & T & T &F& T\\
	T & F & F&F& F\\
	F & T & T &T& T\\
	F & F & T&T&T\\
	\hline
	\end{tabular}
\end{center}	
We see that \fbox{$P \Rightarrow Q$} is equivalent to \fbox{$\sim P \vee Q$} because columns 3 and 5 are the same.\\
\item Use the equivalence above to rewrite the conditional statement in an equivalent form.

\begin{quote}
If $f'(a)=0$, then $f(a)$ is a maximum.\\

$f'(a) \not = 0$ or $f(a)$ is a maximum.\\

\end{quote}

 \vfill
 \item Prove that  \fbox{$\sim(P \Rightarrow Q)$} is equivalent to \fbox{$P \wedge \sim Q$} by constructing a string of logical equivalences that start with $\sim(P \Rightarrow Q)$ and end with $P \wedge \sim Q.$ Each step must be justified by a specific, already established and referenced, logical equivalence.\\
 

 
 \textbf{Proof:} \\
 
 \begin{tabular}{rlr}
 $\sim(P \Rightarrow Q)$ & $=\quad \sim( \sim P \vee Q)$ & by previous problem (\# 2 above)\\
 & $=\quad \sim(\sim P) \wedge \sim Q$ & by DeMorgan's Laws (see \#1c above)\\
 & $=\quad P \wedge \sim Q$ & by the definition of negation\\
 \end{tabular}

\vfill

\item Think up your own favorite conditional statement that you know is \textbf{false.} Call this statement $R. $ (So, $R: P \Rightarrow Q.$ )
	\begin{enumerate}
	\item Write $R$ as a sentence.\\
	
	\textbf{ans:} If $x^2 \geq 0$, then $x \geq 0.$
	\vfill
	\item Write \fbox{$\sim R$} as a sentence using both logical structures: \fbox{$\sim(P \Rightarrow Q)$} and \fbox{$P \wedge \sim Q$}.\\
	
	\textbf{ans:} \\
	
	$\sim (P \Rightarrow Q) :$ It is not the case that if $x^2 \geq 0$, then $x \geq 0.$\\
	
	$P \wedge \sim Q:$ It is possible for $x^2 \geq 0$ and $x < 0.$
	
	\vfill
	\end{enumerate}
 

\item What ideas/concepts/skills do you think this worksheet was supposed to teach you?\\
\begin{itemize}
\item There are two ways to show two statements are logically equivalent: (1) a truth table or (2) an argument. If you use option (2) you need to use a rigid and carefully justified argument.
\item The facts: \fbox{$P \Rightarrow Q=\sim P \vee Q$} and \fbox{$\sim(P \Rightarrow Q)=P \wedge \sim Q$} are facts you should know because they get used a lot. One of the reasons is because it is sometimes easier to see (prove) a statement in its alternative form.
\end{itemize}
\end{enumerate}
\end{document}  















