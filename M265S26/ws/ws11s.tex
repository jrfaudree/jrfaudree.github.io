\documentclass[12pt]{article}

% Layout.
\usepackage[top=1in, bottom=0.75in, left=0.75in, right=0.75in, headheight=1in, headsep=6pt]{geometry}

% Fonts.
\usepackage{mathptmx}
\usepackage[scaled=0.86]{helvet}
\renewcommand{\emph}[1]{\textsf{\textbf{#1}}}

% Misc packages.
\usepackage{amsmath,amssymb,latexsym,amsthm}
\usepackage{graphicx,hyperref}
\usepackage{array}
\usepackage{xcolor}
\usepackage{multicol}
\usepackage{tabularx,colortbl}
\usepackage{enumitem}
\usepackage{soul,multicol}

\hypersetup{
    colorlinks=true,
    linkcolor=blue,
    filecolor=magenta,      
    urlcolor=blue,
    pdftitle={Proofs Worksheet},
    pdfpagemode=FullScreen,
    }

\def\mailto#1{\href{mailto:#1}{#1}}

%% special math symbols
\newcommand{\N}{\mathbb{N}}
\newcommand{\Z}{\mathbb{Z}}
\newcommand{\R}{\mathbb{R}}
\newcommand{\Q}{\mathbb{Q}}
\newcommand{\sse}{\subseteq}
\newcommand{\mcp}{\mathcal{P}}
\newcommand{\ol}[1]{\overline{#1}}

%other specials
\newcommand{\be}{\begin{enumerate}}
\newcommand{\ee}{\end{enumerate}}
\newcommand{\se}{\subseteq}
\newcommand{\ds}{\displaystyle}
\newcommand{\lcm}{\text{lcm}}
%\newcommand{\gcd}{\text{gcd}}

% Paragraph spacing
\parindent 0pt
\parskip 6pt plus 1pt
\def\tableindent{\hskip 0.5 in}
\def\ts{\hskip 1.5 em}

%header
\usepackage{fancyhdr}
\pagestyle{fancy} 
\lhead{\large\sf\textbf{MATH 265: Introduction to Mathematical Proofs}}
\rhead{\large\sf\textbf{Worksheet 11: Ch 6}}

\newcommand{\localhead}[1]{\par\smallskip\textbf{#1}\nobreak\\}%
\def\heading#1{\localhead{\large\emph{#1}}}
\def\subheading#1{\localhead{\emph{#1}}}

\newenvironment{clist}%
{\bgroup\parskip 0pt\begin{list}{$\bullet$}{\partopsep 4pt\topsep 0pt\itemsep -2pt}}%
{\end{list}\egroup}%

\begin{document}
\be
\item Tautologies and Contradictions\\

Always true: $P \lor \sim P$, $P \Rightarrow P$\\

Always false: $P \land \sim P$, $(P \Leftrightarrow Q ) \land (P \land \sim Q)$ \\

\item Proof by Contradiction\\
\fbox{
\begin{tabular}{l}
\textbf{Proposition:} $P$ is true.\\
\textbf{Proof:} (by contradiction) Suppose $\sim P.$ \\
\quad\\
$\vdots$\\
\quad \\
Thus, $C \land \sim C$. \hfill $\square$\\
\end{tabular}
}
\hfill
\fbox{
\begin{tabular}{l}
\textbf{Proposition:} If  $P$, then $Q$.\\
\textbf{Proof:} (by contradiction) \\
Suppose $P$ and $\sim Q.$
\quad\\
$\vdots$\\
\quad \\
Thus, $C \land \sim C$. \hfill $\square$\\
\end{tabular}
}
\item Is this a valid argument? \\
When $P=\text{true},$ $\sim P \Rightarrow (C \land \sim C)$ is true. When $P=\text{false},$ $\sim P \Rightarrow (C \land \sim C)$ is false. So, yes, it is a valid argument. \\

\item Prove that $\sqrt{2}$ is irrational.
\begin{proof} (by contradiction) Suppose $\sqrt{2}$ is rational. By the definition of rational, this implies that there exist integers $a$ and $b$ such that $\sqrt{2}=\frac{a}{b}.$ We choose a representation, $\frac{a}{b},$ that is in \textit{lowest terms.} 

By squaring both sides of  $\sqrt{2}=\frac{a}{b},$ we obtain $2=\frac{a^2}{b^2},$ or, equivalently $2b^2=a^2.$

The last equality implies that $a^2$ is even. Since the square of odd numbers is odd, it follows that $a$ is even. Thus, there is an integer $k$ such that $a=2k.$ Returning to the expression $2b^2=a^2$ and replacing $a$ with $2k,$ we obtain $2b^2=4k^2.$ Dividing the previous equality by 2 gives the equation $b^2=2k^2$ which implies that $b^2$ is even. Thus, we conclude that $b$ is even. Thus, $\frac{a}{b}$ is a ratio of even numbers and not in lowest terms. 

Now we have the contradiction that the expression $\frac{a}{b}$ is in lowest terms  and not in lowest terms. So $\sqrt{2}$ is not rational. Thus, we conclude $\sqrt{2}$ is irrational.
\end{proof}

\vfill
\newpage
\item Use proof by contradiction.
	\begin{enumerate}
	\item Prove if $a,b \in \Z,$ then $a^2-4b-3 \not = 0.$
	
	\begin{proof} (by contradiction) Suppose $a,b \in \Z$ and $a^2-4b-3 = 0.$ Thus, $a^2=4b+3.$ The last equality implies that $a^2$ is odd and therefore $a$ is odd. Thus, there exists an integer $k$ such that $a=2k+1.$ 
	
	Now, substituting $a=2k+1$ into $a^2=4b+3,$ we obtain $4k^2+4k+1=4b+3.$  Let $n = 4k^2+4k+1=4b+3.$ Observe that $4k^2+4k+1 \equiv 1 \: ( \text{mod} 4 )$ but $ 4b+3 \equiv 3 \: ( \text{mod} 4 ).$ Thus, we have the contradiction that, when divided by 4, the integer $n$ has a remainder of 1 and a remainder of 3. 
	
	Thus, it is not possible for two integers to satisfy the expression $a^2-4b-3 = 0.$ Thus,  if $a,b \in \Z,$ then $a^2-4b-3 \not = 0.$ \end{proof}
	\vfill
	\item Prove that for every $x \in [\frac{\pi}{2}, \pi],$ $\sin(x) - \cos(x) \geq 1.$
	
	\begin{proof} (by contradiction) Suppose that there exists an $x \in [\frac{\pi}{2}, \pi]$ such that $\sin(x) - \cos(x) < 1.$
	
	In the interval $ [\frac{\pi}{2}, \pi],$ we know $\sin(x) \geq 0$ and $\cos(x) \leq 0.$ Thus, we know that $\sin(x) - \cos(x)\geq 0.$ Since $0 \leq \sin(x) - \cos(x) < 1,$ we also know that $0 \leq( \sin(x) - \cos(x))^2 < 1.$
	
	On the other hand, we know that $\sin^2(x) + \cos^2(x) =1$ from trigonometry and $-2\sin(x)\cos(x) \geq 0$ because $\sin(x) \geq 0$ and $\cos(x) \leq 0.$ Now, we observe that 
	$$( \sin(x) - \cos(x))^2 = \sin^2(x) + \cos^2(x)-2\sin(x)\cos(x)\geq 1.$$
	
	This lead to the contradiction that $( \sin(x) - \cos(x))^2$ is both strictly less than 1 and greater than or equal to 1.  Thus, it is not possible for there to be an $x$-value in $[\frac{\pi}{2}, \pi]$ such that $\sin(x) - \cos(x) < 1.$ Thus, for every $x \in [\frac{\pi}{2}, \pi],$ $\sin(x) - \cos(x) \geq 1.$


	
	\end{proof}
	\vfill
	\end{enumerate}

\end{enumerate}
\end{document}  















