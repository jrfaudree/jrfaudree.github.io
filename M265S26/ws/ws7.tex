\documentclass[12pt]{article}

% Layout.
\usepackage[top=1in, bottom=0.75in, left=0.75in, right=0.75in, headheight=1in, headsep=6pt]{geometry}

% Fonts.
\usepackage{mathptmx}
\usepackage[scaled=0.86]{helvet}
\renewcommand{\emph}[1]{\textsf{\textbf{#1}}}

% Misc packages.
\usepackage{amsmath,amssymb,latexsym}
\usepackage{graphicx,hyperref}
\usepackage{array}
\usepackage{xcolor}
\usepackage{multicol}
\usepackage{tabularx,colortbl}
\usepackage{enumitem}
\usepackage{soul,multicol}

\hypersetup{
    colorlinks=true,
    linkcolor=blue,
    filecolor=magenta,      
    urlcolor=blue,
    pdftitle={Proofs Worksheet},
    pdfpagemode=FullScreen,
    }

\def\mailto#1{\href{mailto:#1}{#1}}

%% special math symbols
\newcommand{\N}{\mathbb{N}}
\newcommand{\Z}{\mathbb{Z}}
\newcommand{\R}{\mathbb{R}}
\newcommand{\Q}{\mathbb{Q}}
\newcommand{\sse}{\subseteq}
\newcommand{\mcp}{\mathcal{P}}
\newcommand{\ol}[1]{\overline{#1}}

%other specials
\newcommand{\be}{\begin{enumerate}}
\newcommand{\ee}{\end{enumerate}}
\newcommand{\se}{\subseteq}
\newcommand{\ds}{\displaystyle}


% Paragraph spacing
\parindent 0pt
\parskip 6pt plus 1pt
\def\tableindent{\hskip 0.5 in}
\def\ts{\hskip 1.5 em}

%header
\usepackage{fancyhdr}
\pagestyle{fancy} 
\lhead{\large\sf\textbf{MATH 265: Introduction to Mathematical Proofs}}
\rhead{\large\sf\textbf{Worksheet 7: \S 2.9-2.10}}

\newcommand{\localhead}[1]{\par\smallskip\textbf{#1}\nobreak\\}%
\def\heading#1{\localhead{\large\emph{#1}}}
\def\subheading#1{\localhead{\emph{#1}}}

\newenvironment{clist}%
{\bgroup\parskip 0pt\begin{list}{$\bullet$}{\partopsep 4pt\topsep 0pt\itemsep -2pt}}%
{\end{list}\egroup}%

\begin{document}
\be
\item Review
	\be
	\item $\sim \big( \forall n \in \N, \: 2n^2-n \geq 1 \big)$
	\vfill
	\item $\sim \big( \exists n \in \N, \: 2n^2-n > 10 \big)$
	\vfill	
	\ee
\item From the previous sheet:
	\be[leftmargin=0.5cm]
	\item[d.] There are squares with integer values for the sides and the diagonals.
	
	$$ \exists \text{ squares with side } s \text{ and  diagonal } d, \big( s \in \Z \land d \in \Z \big)$$
	We concluded it was false. What do we need to show?\\
	\vfill
	\vfill
	\item[e.] Every integer that is not positive must be negative.
	$$ \forall n \in \N, \: \sim \big( n>0)\: \Rightarrow \: (n<0)$$
	We concluded it was false. What do we need to show?\\
	\vfill
	\item[g.] For every quadratic polynomial $p(x)$, there is some real number $a$, where $a$ is a root of $p(x).$
	$$ \forall p(x) \in \mathbb{P}_2(x),\: \exists a \in \R,\: p(a)=0, \quad \text{ where } \mathbb{P}_2(x) \text{ is the set of degree 2 polynomials} $$
	We concluded it was false. What do we need to show?\\
	\vfill
	\ee
\newpage
\item Logical Inference: \\
\vfill
\item Modus Ponens\\
\vfill
\item The most common fallacy (invalid argument): 
\vfill
\newpage
\item For each argument below, (a) determine whether it is valid or invalid, (b) write an argument in English that models the logical structure of the argument.
	\be
	\item \begin{tabular}{l} $ P \: \Rightarrow \: Q$ \\ $\sim P$ \\ \hline $\sim Q$ \end{tabular}
	
	\vfill
	\item \begin{tabular}{l} $ P \: \Rightarrow \: Q$ \\ $\sim Q$ \\ \hline $\sim P$ \end{tabular}
	
	\vfill
	\item \begin{tabular}{l} $ P \: \vee \: Q$ \\ $\sim P$ \\ \hline $ Q$ \end{tabular}
	\vfill
	\ee
\newpage
\item Show that \fbox{$ P\: \Rightarrow \: Q$} is logically equivalent to \fbox{$\sim Q\: \Rightarrow \:\sim P.$} (Note: $\sim Q \Rightarrow \sim P$ is called the \textbf{contrapositive} of $ P \Rightarrow Q.$)
\vfill
\item Rewrite each theorem below with its equivalent contrapositive statement. Note that the ``Let..."  sentence does not change.
	\be
	\item If two sides of a triangle are congruent (aka of equal length), then the two angles opposite those sides are congruent (aka are equal in measure).
	\vfill
	\item Let $f(x)$ be defined on the interval $[a,b].$ If $f(x)$ is continuous on $[a,b]$, then for every $y$-value, $y_0,$ strictly between $f(a)$ and $f(b)$ there exists an $x$-value, $x_0,$ in $(a,b)$ such that $f(x_0)=y_0.$\\
	\vfill
	\ee
\end{enumerate}
\end{document}  















