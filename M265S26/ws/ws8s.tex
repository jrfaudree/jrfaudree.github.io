\documentclass[12pt]{article}

% Layout.
\usepackage[top=1in, bottom=0.75in, left=0.75in, right=0.75in, headheight=1in, headsep=6pt]{geometry}

% Fonts.
\usepackage{mathptmx}
\usepackage[scaled=0.86]{helvet}
\renewcommand{\emph}[1]{\textsf{\textbf{#1}}}

% Misc packages.
\usepackage{amsmath,amssymb,amsthm,latexsym}
\usepackage{graphicx,hyperref}
\usepackage{array}
\usepackage{xcolor}
\usepackage{multicol}
\usepackage{tabularx,colortbl}
\usepackage{enumitem}
\usepackage{soul,multicol}

\hypersetup{
    colorlinks=true,
    linkcolor=blue,
    filecolor=magenta,      
    urlcolor=blue,
    pdftitle={Proofs Worksheet},
    pdfpagemode=FullScreen,
    }

\def\mailto#1{\href{mailto:#1}{#1}}

%% special math symbols
\newcommand{\N}{\mathbb{N}}
\newcommand{\Z}{\mathbb{Z}}
\newcommand{\R}{\mathbb{R}}
\newcommand{\Q}{\mathbb{Q}}
\newcommand{\sse}{\subseteq}
\newcommand{\mcp}{\mathcal{P}}
\newcommand{\ol}[1]{\overline{#1}}

%other specials
\newcommand{\be}{\begin{enumerate}}
\newcommand{\ee}{\end{enumerate}}
\newcommand{\se}{\subseteq}
\newcommand{\ds}{\displaystyle}
\newcommand{\lcm}{\text{lcm}}

% Paragraph spacing
\parindent 0pt
\parskip 6pt plus 1pt
\def\tableindent{\hskip 0.5 in}
\def\ts{\hskip 1.5 em}

%header
\usepackage{fancyhdr}
\pagestyle{fancy} 
\lhead{\large\sf\textbf{MATH 265: Introduction to Mathematical Proofs}}
\rhead{\large\sf\textbf{Worksheet : Ch 4}}

\newcommand{\localhead}[1]{\par\smallskip\textbf{#1}\nobreak\\}%
\def\heading#1{\localhead{\large\emph{#1}}}
\def\subheading#1{\localhead{\emph{#1}}}

\newenvironment{clist}%
{\bgroup\parskip 0pt\begin{list}{$\bullet$}{\partopsep 4pt\topsep 0pt\itemsep -2pt}}%
{\end{list}\egroup}%

\begin{document}
Solutions
\be
\item Definitions and Facts
	\begin{enumerate}
	\item An integer $n$ is \emph{even} if $\exists k \in \Z$, $n=2k.$ \\ \vfill
	\item An integer $n$  is \emph{odd} if $\exists k \in \Z$, $n=2k+1.$\\ \vfill
	\item Let $a,b \in \Z.$ We say $a$ \emph{divides} $b$ if $\exists k \in \Z$, $ak=b.$ \\ \vfill
	Alternate wording: $a$ \emph{is a divisor of} $b$ OR $b$ is \emph{a multiple of} $a$ \\ \vfill
	
	\item A number $n \in \N$ is \emph{prime} if  it has exactly two distinct divisors.\\ \vfill
	A number $n \in \N$ is \emph{composite} if it has more than two distinct divisors. \\ \vfill
	\item Let $a,b \in \Z.$ The \emph{greatest common division of $a$ and $b$} ( denoted $\gcd(a,b)$) is the largest integer $n$ such that $n|a$ and $n|b.$\\ \vfill
	\item $a,b \in \Z -\{0\}.$ The \emph{least common multiple of $a$ and $b$} (denoted $\lcm(a,b)$) is the smallest positive integer $n$ such that $a|n$ and $b|n$\\ \vfill
	\item \emph{Fact 4.1:} If $a,b, \in \Z,$ then $a+b$, $a-b$, and $ab$ are also in $\Z.$ \\ \vfill
	Alternate wording: The integers are closed under addition and multiplication. \\ \vfill
	\quad \\
	\item \emph{The Division Algorithm} For every $a \in \Z$ and $ b \in \N -\{0\},$ there exists unique integers $q$ and $r$ such that
	$$ a=qb+r, \quad \text{where} \quad 0 \leq r < b.$$  \vfill
	
	
	\vfill 
	\end{enumerate}
\item Outline for a \emph{Direct Proof}\\

\fbox{
\begin{tabular}{l}
\emph{Proposition:} If $P$, then $Q.$ \\
\emph{Proof:} (direct) Suppose $P$ (is true). \\
\quad\\
$\vdots$\\
\quad \\
Thus, $Q$ (is true). \hspace{3cm}$\square$\\
\end{tabular}
}
\newpage
\item Prove that for every integer $m$, if $n$ is even, then $3n^2 -5mn-8$ is also even.

\begin{proof} (direct) Let $m\in \Z$ and suppose $n$ is even. Then, by the definition of even, there exists an integer $k$ such that $n=2k.$  Let $\ell =  3k^2-5km-4.$\\
Now, 
\begin{align*}
3n^2-5mn-8 & = 3(2k)^2-5(2k)m-8 && \text{by substituting $n=2k$}\\
 &= 6k^2-10km-8 && \text{by rules of multiplication}\\
 &=2(3k^2-5km-4) && \text{by factoring out a 2}\\
 &=2\ell&&  \text{by substituting $\ell =  3k^2-5km-4.$}\\
\end{align*}
 By Fact 4.1, since $k$ and $m$ are integers, we know $ 3k^2-5km-4$ is also an integer.  
 Thus, $3n^2-5mn-8=2 \ell,$ where $\ell \in \Z.$ Thus, $3n^2-5mn-8$ is even by definition.
\end{proof}

\vfill

%\item Prove that for all $a,b,c \in \Z,$ $\lcm(ca,cb)=c\cdot \lcm(a,b).$
\item Let $x,y \in \R^+.$ Prove that if $x \leq y,$ then $\sqrt{x} \leq \sqrt{y}.$

Q: Do you think all of the hypotheses are needed?

\begin{proof} (direct) Let $x, y \in \R^+$ such that $x \leq y.$ By subtracting $x$ from both sides, we obtain $0\leq y-x.$ Since $x$ and $y$ are both positive, we know that $\sqrt{x}$ and $\sqrt{y}$ are defined. Thus, we can factor $y-x$ as a difference of squares to get $y-x= (\sqrt{y}+\sqrt{x})(\sqrt{y}-\sqrt{x}).$ 

Using $0\leq y-x$ and $y-x= (\sqrt{y}+\sqrt{x})(\sqrt{y}-\sqrt{x}),$ we conclude $0 \leq (\sqrt{y}+\sqrt{x})(\sqrt{y}-\sqrt{x}).$ 

Since $\sqrt{y}+\sqrt{x}>0,$ we can divide $0 \leq (\sqrt{y}+\sqrt{x})(\sqrt{y}-\sqrt{x})$ by $\sqrt{y}+\sqrt{x}$ to obtain $0 \leq \sqrt{y}-\sqrt{x}.$ By adding $\sqrt{2}$ to both sides of the previous inequality, we obtain $\sqrt{x} \leq \sqrt{y},$ which is what we wanted to prove.  
\end{proof}
\vfill

Q: Did we use all the hypotheses?

\vfill
\item Rigid and Unforgiving Rules
	\begin{enumerate}
	\item All parts of all proofs are complete sentences which begin with a word in English and end with a period. No sentence fragments.
	\item The following symbols never appear: $\forall,$ $\exists$, $\Rightarrow$, $\lor$, $\land$. 
	\item \emph{All} strings of equalities are aligned vertically, with justification.
	\item Don't use a fact if you haven't proved it.
	\end{enumerate}
	
\newpage
\item Let $a,b \in \Z.$ Prove that if $a \big\vert b,$ then $a^2 \big\vert b^2.$

\begin{proof} (direct) Let $a,b \in \Z$ such that $a \big\vert b.$ Then, by the definition of \emph{divides}, there is an integer $k$ such that $ak=b.$ Let  $\ell=k^2.$ Squaring both sides of the previous equation, we obtain  

\begin{align*}
b^2&=(ak)^2 && \\
&=a^2(k^2) && \text{ factoring out $a^2$}\\
&=a^2\ell&&\text{by substituting$\ell=k^2$.} \\
\end{align*}

Since $b^2 = a^2\ell$ for $\ell \in \Z,$ $a^2 \big\vert b^2$ by the definition of \emph{divides}.

\end{proof} 
\vfill
\item Let $x,y \in \R^+.$ Prove that $2\sqrt{xy} \leq{x+y}.$ 

\begin{proof} (direct)  Let $x,y \in \R^+.$ Since $x$ and $y$ are real numbers, we know

$$0 \leq (x-y)^2=x^2-2xy+y^2.$$

If we add $4xy$ to both ends of the inequality above, we obtain

$$4xy \leq x^2+2xy+y^2.$$

Factoring the right-hand side yields

$$4xy \leq (x+y)^2.$$

Since $x$ and $y$ are both positive real numbers, we know that both $4xy$ and $(x+y)^2$ are positive. Thus, we can apply the result from \#4 above and use the fact that $4xy \leq (x+y)^2$ to conclude that $2 \sqrt{xy} \leq x+y,$ which is what we wanted to show.

\end{proof}
\vfill
{\scriptsize{Start with what you know about $(x-y)^2.$}}
\end{enumerate}
\end{document}  















