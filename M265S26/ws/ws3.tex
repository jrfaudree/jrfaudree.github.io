\documentclass[12pt]{article}

% Layout.
\usepackage[top=1in, bottom=0.75in, left=0.75in, right=0.75in, headheight=1in, headsep=6pt]{geometry}

% Fonts.
\usepackage{mathptmx}
\usepackage[scaled=0.86]{helvet}
\renewcommand{\emph}[1]{\textsf{\textbf{#1}}}

% Misc packages.
\usepackage{amsmath,amssymb,latexsym}
\usepackage{graphicx,hyperref}
\usepackage{array}
\usepackage{xcolor}
\usepackage{multicol}
\usepackage{tabularx,colortbl}
\usepackage{enumitem}
\usepackage{soul,multicol}

\hypersetup{
    colorlinks=true,
    linkcolor=blue,
    filecolor=magenta,      
    urlcolor=blue,
    pdftitle={Overleaf Example},
    pdfpagemode=FullScreen,
    }

\def\mailto#1{\href{mailto:#1}{#1}}

%% special math symbols
\newcommand{\N}{\mathbb{N}}
\newcommand{\Z}{\mathbb{Z}}
\newcommand{\R}{\mathbb{R}}
\newcommand{\Q}{\mathbb{Q}}
\newcommand{\sse}{\subseteq}
\newcommand{\mcp}{\mathcal{P}}
\newcommand{\ol}[1]{\overline{#1}}

%other specials
\newcommand{\be}{\begin{enumerate}}
\newcommand{\ee}{\end{enumerate}}
\newcommand{\se}{\subseteq}
\newcommand{\ds}{\displaystyle}


% Paragraph spacing
\parindent 0pt
\parskip 6pt plus 1pt
\def\tableindent{\hskip 0.5 in}
\def\ts{\hskip 1.5 em}

%header
\usepackage{fancyhdr}
\pagestyle{fancy} 
\lhead{\large\sf\textbf{MATH 265: Introduction to Mathematical Proofs}}
\rhead{\large\sf\textbf{Worksheet 3: \S 1.7-1.8}}

\newcommand{\localhead}[1]{\par\smallskip\textbf{#1}\nobreak\\}%
\def\heading#1{\localhead{\large\emph{#1}}}
\def\subheading#1{\localhead{\emph{#1}}}

\newenvironment{clist}%
{\bgroup\parskip 0pt\begin{list}{$\bullet$}{\partopsep 4pt\topsep 0pt\itemsep -2pt}}%
{\end{list}\egroup}%

\begin{document}
\begin{center} Venn Diagrams and Indexed Sets \end{center}
\begin{enumerate}
\item Venn Diagrams
\vfill
\item Indexed Sets
	\begin{enumerate}
	\item Finite Examples and Definitions
	\vfill
\newpage
	\item Infinite and More General Examples and Definitions
	\vfill
	\ee


\newpage
\item Draw a Venn Diagram for each set and then answer the questions.
	\begin{multicols}{2}
	\be[itemsep=3cm]
	\item $A \cap B$
	\vfill
	\item $\ol{A \cap B}$
	\vfill
	\item $\ol{A}$
	\vfill
	\item $\ol{B}$
	\vfill
	\item $\ol{A} \cup \ol{B}$
	\vfill
	\item Use the work above to make a conjecture.
	\vfill
	\item Make a conjecture about $A \cup B$ and check it with a Venn Diagram.
	\vfill
	\ee
	\end{multicols}
\vfill
\item  Suppose $A_n = \{n, n+1,n+2,\dots,2n\}$ for $n \in \N.$
	\be
	\item Determine the sets $A_1,$ $A_2,$ and $A_3$ by writing out their elements.
	\vfill
	\item $\ds \bigcup_{n \in \N} A_n \: = $
	\vfill
	\item $\ds \bigcap_{n \in \N} A_n \: = $
	\vfill
	\ee
\newpage
\item  Suppose $B_\alpha = [1,3-\alpha] \subseteq \R$ for $\alpha \in [0,1).$
	\be
	\item Determine the set $B_\alpha$ for four different values of $\alpha$.
	\vfill
	\item $\ds \bigcup_{\alpha \in [0,1]} A_\alpha \: = $
	\vfill
	\item $\ds \bigcap_{\alpha \in [0,1]} A_\alpha \: = $
	\vfill
	\ee
\item For each $i \in I,$ $A_i$ is a set. Suppose $J \subseteq I.$
	\begin{enumerate}
	\item Draw a Venn diagram of sets $I$ and $J.$
	\vfill
	\item Is it possible to determine the relationship between:
		\be
		\item $\ds \bigcup_{i \in I} A_i $ and $\ds \bigcup_{j \in J} A_j$? Explain.
		\vfill 
		\item $\ds \bigcap_{i \in I} A_i $ and $\ds \bigcap_{j \in J} A_j$? Explain.
		\vfill
		\ee
	\ee
\end{enumerate}
\end{document}  















