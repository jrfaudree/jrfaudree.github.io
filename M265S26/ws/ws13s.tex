\documentclass[12pt]{article}

% Layout.
\usepackage[top=1in, bottom=0.75in, left=0.75in, right=0.75in, headheight=1in, headsep=6pt]{geometry}

% Fonts.
\usepackage{mathptmx}
\usepackage[scaled=0.86]{helvet}
\renewcommand{\emph}[1]{\textsf{\textbf{#1}}}

% Misc packages.
\usepackage{amsmath,amssymb,latexsym,amsthm}
\usepackage{graphicx,hyperref}
\usepackage{array}
\usepackage{xcolor}
\usepackage{multicol}
\usepackage{tabularx,colortbl}
\usepackage{enumitem}
\usepackage{soul,multicol}

\hypersetup{
    colorlinks=true,
    linkcolor=blue,
    filecolor=magenta,      
    urlcolor=blue,
    pdftitle={Proofs Worksheet},
    pdfpagemode=FullScreen,
    }

\def\mailto#1{\href{mailto:#1}{#1}}

%% special math symbols
\newcommand{\N}{\mathbb{N}}
\newcommand{\Z}{\mathbb{Z}}
\newcommand{\R}{\mathbb{R}}
\newcommand{\Q}{\mathbb{Q}}
\newcommand{\sse}{\subseteq}
\newcommand{\mcp}{\mathcal{P}}
\newcommand{\ol}[1]{\overline{#1}}

%other specials
\newcommand{\be}{\begin{enumerate}}
\newcommand{\ee}{\end{enumerate}}
\newcommand{\se}{\subseteq}
\newcommand{\ds}{\displaystyle}
\newcommand{\lcm}{\text{lcm}}
%\newcommand{\gcd}{\text{gcd}}

% Paragraph spacing
\parindent 0pt
\parskip 6pt plus 1pt
\def\tableindent{\hskip 0.5 in}
\def\ts{\hskip 1.5 em}

%header
\usepackage{fancyhdr}
\pagestyle{fancy} 
\lhead{\large\sf\textbf{MATH 265: Introduction to Mathematical Proofs}}
\rhead{\large\sf\textbf{Worksheet 13: Ch 7}}

\newcommand{\localhead}[1]{\par\smallskip\textbf{#1}\nobreak\\}%
\def\heading#1{\localhead{\large\emph{#1}}}
\def\subheading#1{\localhead{\emph{#1}}}

\newenvironment{clist}%
{\bgroup\parskip 0pt\begin{list}{$\bullet$}{\partopsep 4pt\topsep 0pt\itemsep -2pt}}%
{\end{list}\egroup}%

\begin{document}
Prove the following statements. Use any method you like, but follow directions.
\be
\item Given an integer $a$, then $a^2+4a+7$ is odd if and only if $a$ is even.

\begin{proof}
$(\Rightarrow:)$ (by contrapositive) Suppose $a$ is odd. Then by the definition of odd, we know there exist an integer $k$ such that $a=2k+1.$ Substituting into $a^2+4a+7$, we obtain
$$a^2+4a+7=(2k+1)^2+4(2k+1)+7=4k^2+4k+1+8k+4+7=4k^2+12k+12=2(2k^2+6k+6),$$ where $2k^2+6k+6$ is an integer. Thus, $a^2+4a+7$ is even. 

We have shown that if $a$ is odd, then $a^2+4a+7$ is even which this is equivalent to the statement that if $a^2+4a+7$ is odd, then $a$ is even.

$(\Leftarrow:)$ (direct) Suppose $a$ is even. Then $a=2k,$ for some integer $k.$ Substituting into $a^2+4a+7$, we obtain
$$(2k)^2+4(2k)+7=4k^2+8k+2\cdot 3 +1 =2(2k^2+4k+3)+1,$$
where $2k^2+4k+3$ is an integer. Thus, $a^2+4a+7$ is odd.
\end{proof}
\vfill
\item There exists a set $X$ such that $\N \in X$ and $N \subseteq X.$

\begin{proof} Let $X = \N \cup \{ \N\} = \big\{ \{\N\},1,2,3,\dots\big\}.$\\
We can see that $\N \in X$ since in the ``list" form of $X$ on the right, $\N$ is the first element in the list. We can see that $\N \subseteq X$ because we see that for every $n \in \N,$ $n \in X.$

\end{proof}
\vfill
\item Suppose $x,y \in \R.$ Then $(x+y)^2=x^2+y^2$ if and only if $x=0$ or $y=0.$\\

\begin{proof} Let $x,y \in \R.$\\
($\Rightarrow:$) (direct) Suppose $(x+y)^2=x^2+y^2.$ By expanding  and rearranging $(x+y)^2=x^2+y^2$ we obtain $2xy=0.$ Now, we apply a property of real numbers that if a product is zero, at least one of its terms is zero. Thus, $2$, $x$ or $y$ is zero. But $2$ is not zero. So either $x=0$ or $y=0.$\\

($\Leftarrow:$) (direct) Suppose $x=0$ or $y=0.$\\
If $x=0,$ then by substitution the expression $(x+y)^2=x^2+y^2$ becomes $y^2=y^2,$ which is always true. The argument of $y=0$ is the same. Thus, if $x=0$ or $y=0,$ then $(x+y)^2=x^2+y^2.$
\end{proof}
\vfill


\item Suppose $a,b,c \in \N.$ Use the proposition we proved in class to show that if $a \big\vert bc$ and $\gcd(a,b)=1,$ then $a \big\vert c.$

\begin{proof} Suppose $a,b,c \in \N,$ $a \big\vert bc$ and $\gcd(a,b)=1.$\\
Since $a \big\vert bc$, there exists an integer $k$ such that $ak=bc.$ Since $\gcd(a,b)=1,$ by Prop. 7.1, there exist integers $m$ and $n$ such that $an+bm=1.$ \\

Multiply the previous equation by $c$ to obtain $anc+bmc=c.$ Using the fact that $ak=bc$, we can plug in for $bc$ into $amc+bmc=c$ to obtain $c=anc+akm=a(nc+km).$\\

Now, $c=a(nc+km)$ where $nc+km$ is an integer. Thus, $a \big\vert c$ by definition, which is what we wanted to prove. 

\end{proof}
\vfill
\end{enumerate}
\end{document}  















