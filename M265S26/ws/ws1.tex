\documentclass[12pt]{article}

% Layout.
\usepackage[top=1in, bottom=0.75in, left=1in, right=1in, headheight=1in, headsep=6pt]{geometry}

% Fonts.
\usepackage{mathptmx}
\usepackage[scaled=0.86]{helvet}
\renewcommand{\emph}[1]{\textsf{\textbf{#1}}}

% Misc packages.
\usepackage{amsmath,amssymb,latexsym}
\usepackage{graphicx,hyperref}
\usepackage{array}
\usepackage{xcolor}
\usepackage{multicol}
\usepackage{tabularx,colortbl}
\usepackage{enumitem}
\usepackage{soul}

\hypersetup{
    colorlinks=true,
    linkcolor=blue,
    filecolor=magenta,      
    urlcolor=blue,
    pdftitle={Overleaf Example},
    pdfpagemode=FullScreen,
    }

\def\mailto#1{\href{mailto:#1}{#1}}

%% special math symbols
\newcommand{\N}{\mathbb{N}}
\newcommand{\Z}{\mathbb{Z}}
\newcommand{\R}{\mathbb{R}}
\newcommand{\Q}{\mathbb{Q}}
\newcommand{\sse}{\subseteq}
\newcommand{\mcp}{\mathcal{P}}



% Paragraph spacing
\parindent 0pt
\parskip 6pt plus 1pt
\def\tableindent{\hskip 0.5 in}
\def\ts{\hskip 1.5 em}

%header
\usepackage{fancyhdr}
\pagestyle{fancy} 
\lhead{\large\sf\textbf{MATH 265: Introduction to Mathematical Proofs}}
\rhead{\large\sf\textbf{Worksheet 1: \S 1.1, 1.2}}

\newcommand{\localhead}[1]{\par\smallskip\textbf{#1}\nobreak\\}%
\def\heading#1{\localhead{\large\emph{#1}}}
\def\subheading#1{\localhead{\emph{#1}}}

\newenvironment{clist}%
{\bgroup\parskip 0pt\begin{list}{$\bullet$}{\partopsep 4pt\topsep 0pt\itemsep -2pt}}%
{\end{list}\egroup}%

\begin{document}
\begin{center} Introduction to Sets \& Cartesian Products \end{center}
\begin{enumerate}
\item A \textbf{set} is\\

\vspace{1in}

\item Some examples and typical notation\\

\vspace{2in}

\item The \textbf{Cartesian Product} of two sets $A$ and $B$ is\\

\vspace{1in}

\item Some examples and typical notation\\

\newpage
\item List the \textbf{elements} in each set below and determine its \emph{cardinality}.
	\begin{enumerate}
	\item $A=\{ 1, 2, \{a,b,c\}, \emptyset \}$
	\vfill
	\item $A=\{ x \in R \: : \: x^3-x^2=6x\}$
	\vfill
	\item $A=\{ x \in Z \: : \: x^3-x^2=6x\}$
	\vfill
	\end{enumerate}
\item Write each set in \textbf{set-builder} notation.
	\begin{enumerate}
	\item The half-open interval of the real line: $[2,8).$
	\vfill
	\item $\{-6, -3, 0, 3, 6, 9,12, \dots\}$ (Assume the pattern continues.)
	\vfill
	\item The set of points in the $xy$-plane that lie on the graph $y=x^2 +1.$
	\vfill
	\end{enumerate}
\item Sketch $[0,1]^3$
\vfill
\end{enumerate}
\end{document}  















