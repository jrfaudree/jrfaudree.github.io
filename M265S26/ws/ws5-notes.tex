\documentclass[12pt]{article}

% Layout.
\usepackage[top=1in, bottom=0.75in, left=0.75in, right=0.75in, headheight=1in, headsep=6pt]{geometry}

% Fonts.
\usepackage{mathptmx}
\usepackage[scaled=0.86]{helvet}
\renewcommand{\emph}[1]{\textsf{\textbf{#1}}}

% Misc packages.
\usepackage{amsmath,amssymb,latexsym}
\usepackage{graphicx,hyperref}
\usepackage{array}
\usepackage{xcolor}
\usepackage{multicol}
\usepackage{tabularx,colortbl}
\usepackage{enumitem}
\usepackage{soul,multicol}

\hypersetup{
    colorlinks=true,
    linkcolor=blue,
    filecolor=magenta,      
    urlcolor=blue,
    pdftitle={Proofs Worksheet},
    pdfpagemode=FullScreen,
    }

\def\mailto#1{\href{mailto:#1}{#1}}

%% special math symbols
\newcommand{\N}{\mathbb{N}}
\newcommand{\Z}{\mathbb{Z}}
\newcommand{\R}{\mathbb{R}}
\newcommand{\Q}{\mathbb{Q}}
\newcommand{\sse}{\subseteq}
\newcommand{\mcp}{\mathcal{P}}
\newcommand{\ol}[1]{\overline{#1}}

%other specials
\newcommand{\be}{\begin{enumerate}}
\newcommand{\ee}{\end{enumerate}}
\newcommand{\se}{\subseteq}
\newcommand{\ds}{\displaystyle}


% Paragraph spacing
\parindent 0pt
\parskip 6pt plus 1pt
\def\tableindent{\hskip 0.5 in}
\def\ts{\hskip 1.5 em}

%header
\usepackage{fancyhdr}
\pagestyle{fancy} 
\lhead{\large\sf\textbf{MATH 265: Introduction to Mathematical Proofs}}
\rhead{\large\sf\textbf{Notes for Worksheet 5: \S 2.4-2.6}}

\newcommand{\localhead}[1]{\par\smallskip\textbf{#1}\nobreak\\}%
\def\heading#1{\localhead{\large\emph{#1}}}
\def\subheading#1{\localhead{\emph{#1}}}

\newenvironment{clist}%
{\bgroup\parskip 0pt\begin{list}{$\bullet$}{\partopsep 4pt\topsep 0pt\itemsep -2pt}}%
{\end{list}\egroup}%

\begin{document}
\be
\item The Biconditional\\

Suppose $P$ and $Q$ are statements. Then the statement $P$ if and only if $Q$ (symbolically $P \Leftrightarrow Q$ should be true if
\begin{itemize}
\item (in words) $P$ and $Q$ have the same truth value -- so both true or both false.
\item (truth table)
\begin{center}
	\begin{tabular}{|c|c|c|}
	\hline
	$P$ & $Q$ & $P \Leftrightarrow Q$ \\
	\hline
	T & T & T  \\
	T & F & F \\
	F & T & F  \\
	F & F & T \\
	\hline
	\end{tabular}
\end{center}
\end{itemize}
\item Examples: \\
\begin{enumerate}
\item R: $ab=0$ if and only if $a=0$ or $b=0$ (true for real numbers)
\item S: $f$ is differentiable if and only if $f$ is continuous (false, in general)
\end{enumerate}
\item Logical Equivalence

Two (compound) statements are (logically) equivalent if they have the same truth value for all possible truth values of the ``input" statements. 

\item Claim: $\sim ( P \vee Q)$ is equivalent to $\sim P \wedge \sim Q$\\

\textbf{Proof:} Construct a truth table.

\begin{center}
	\begin{tabular}{|c|c|c|c|c|c|c|}
	\hline
	1&2&3&4&5&6&7\\
	$P$ & $Q$ & $P \vee Q$ & $\sim ( P \vee Q)$ & $\sim P$ & $\sim Q$ & $\sim P \wedge \sim Q$\ \\
	\hline
	T & T & T &F&F& F&F\\
	T & F & T&F&F&T&F \\
	F & T & T &F &T&F&F\\
	F & F & F&T &T&T&T\\
	\hline
	\end{tabular}
\end{center}
 Since 4 and 7 have the same truth values, $\sim ( P \vee Q)$ is equivalent to $\sim P \wedge \sim Q.$\\
 
 \textbf{Think about what this means in words}: The truth value of the three statements below are the same.\\

\begin{quote}
(1) It is not the case that $x \geq 2$ or that $x \leq 5.$\\

(2) $x \not\geq 2$ and $x \not\leq 5.$\\

(3) $x < 2$ and $x > 5.$\\

\end{quote}

\item Claim: $P \vee (Q \wedge R)$ is \emph{not} equivalent to $(P\vee Q) \wedge R$

Option 1: Make a truth table and show that the columns are different in some particular row.\\

Option 2: Find (via intuition or a truth table) some particular truth values for which the expressions are different. 

\textbf{Proof:} Suppose $P$ is true but $Q$ and $R$ are both false. Then the statement $P \vee (Q \wedge R)$ is true because $P$ is true. But the statement $(P\vee Q) \wedge R$ is false because $R$ is false. Since the expressions do not have the same truth values, they cannot be logically equivalent.

 \ee
 \end{document} 
 \item Claim: $P \Rightarrow Q$ is equivalent to $\sim P \vee Q$
 

\textbf{Proof:} Construct a truth table.

\begin{center}
	\begin{tabular}{|c|c|c|c|c|}
	\hline
	1&2&3&4&5\\
	$P$ & $Q$ & $P \Rightarrow Q$ &  $\sim P$ &  $\sim P \vee  Q$\ \\
	\hline
	T & T & T &F& T\\
	T & F & F&F& F\\
	F & T & T &T& T\\
	F & F & T&T&T\\
	\hline
	\end{tabular}
\end{center}	

 Since 3 and 5 have the same truth values, $P \Rightarrow Q$ is equivalent to $\sim P \vee Q$\\
 
 \textbf{Think about what this means in words}: The truth value of the two sentences below are the same.\\

\begin{quote}
If $f'(a)=0$, then $f(a)$ is a maximum.\\

$f'(a) \not = 0$ or $f(a)$ is a maximum.\\

\end{quote}
 
 \item Claim: $\sim(P \Rightarrow Q)$ is equivalent to $P \wedge \sim Q$
 
 Option 1: A truth table.\\
 
 Option 2: An direct argument structured by starting with one and, using and stating established rules, manipulate it to become the other.
 
 \textbf{Proof:}
 
 \begin{tabular}{rlr}
 $\sim(P \Rightarrow Q)$ & $=\sim( \sim P \vee Q)$ & by previous problem (\# 6 above)\\
 & $= \sim(\sim P) \wedge \sim Q$ & by DeMorgan's Laws (see \#5 above)\\
 & $= P \wedge \sim Q$ & by the definition of negation\\
 \end{tabular}
\end{enumerate}
\end{document}  















