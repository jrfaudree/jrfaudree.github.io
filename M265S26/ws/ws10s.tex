\documentclass[12pt]{article}

% Layout.
\usepackage[top=1in, bottom=0.75in, left=0.75in, right=0.75in, headheight=1in, headsep=6pt]{geometry}

% Fonts.
\usepackage{mathptmx}
\usepackage[scaled=0.86]{helvet}
\renewcommand{\emph}[1]{\textsf{\textbf{#1}}}

% Misc packages.
\usepackage{amsmath,amssymb,latexsym,amsthm}
\usepackage{graphicx,hyperref}
\usepackage{array}
\usepackage{xcolor}
\usepackage{multicol}
\usepackage{tabularx,colortbl}
\usepackage{enumitem}
\usepackage{soul,multicol}

\hypersetup{
    colorlinks=true,
    linkcolor=blue,
    filecolor=magenta,      
    urlcolor=blue,
    pdftitle={Proofs Worksheet},
    pdfpagemode=FullScreen,
    }

\def\mailto#1{\href{mailto:#1}{#1}}

%% special math symbols
\newcommand{\N}{\mathbb{N}}
\newcommand{\Z}{\mathbb{Z}}
\newcommand{\R}{\mathbb{R}}
\newcommand{\Q}{\mathbb{Q}}
\newcommand{\sse}{\subseteq}
\newcommand{\mcp}{\mathcal{P}}
\newcommand{\ol}[1]{\overline{#1}}

%other specials
\newcommand{\be}{\begin{enumerate}}
\newcommand{\ee}{\end{enumerate}}
\newcommand{\se}{\subseteq}
\newcommand{\ds}{\displaystyle}
\newcommand{\lcm}{\text{lcm}}
%\newcommand{\gcd}{\text{gcd}}

% Paragraph spacing
\parindent 0pt
\parskip 6pt plus 1pt
\def\tableindent{\hskip 0.5 in}
\def\ts{\hskip 1.5 em}

%header
\usepackage{fancyhdr}
\pagestyle{fancy} 
\lhead{\large\sf\textbf{MATH 265: Introduction to Mathematical Proofs}}
\rhead{\large\sf\textbf{Worksheet 10: Ch 5 (1)}}

\newcommand{\localhead}[1]{\par\smallskip\textbf{#1}\nobreak\\}%
\def\heading#1{\localhead{\large\emph{#1}}}
\def\subheading#1{\localhead{\emph{#1}}}

\newenvironment{clist}%
{\bgroup\parskip 0pt\begin{list}{$\bullet$}{\partopsep 4pt\topsep 0pt\itemsep -2pt}}%
{\end{list}\egroup}%

\begin{document}
My solutions
\be
\item Proof by Contrapositive\\
\fbox{
\begin{tabular}{l}
\textbf{Proposition:} If $P$, then $Q$.\\
\textbf{Proof:} (by contrapositive)  Suppose $\sim Q.$\\
\quad\\
$\vdots$\\
\quad \\
Thus, $\sim P$. \hspace{1.5in} $\square$\\
\end{tabular}
}
\item Prove that for every integer $n$, if $n^2+3n$ is odd, then $n$ is odd.\\

\begin{proof}[Proof by Contrapositive]
We prove the contrapositive: If $n$ is even, then $n^2 + 3n$ is even.

Assume $n$ is even. Then $n = 2k$ for some integer $k$.

We have:
\begin{align*}
n^2 + 3n &= (2k)^2 + 3(2k)\\
&= 4k^2 + 6k\\
&= 2(2k^2 + 3k)
\end{align*}

Since $2k^2 + 3k$ is an integer, we see that $n^2 + 3n = 2(2k^2 + 3k)$ is even.

Therefore, by contrapositive, if $n^2 + 3n$ is odd, then $n$ is odd.
\end{proof}

\vfill
\item Prove that for every pair of real numbers $x$ and $y$, if $x+y$ is irrational, then either $x$ is irrational or $y$ is irrational.

\begin{proof}[Proof by Contrapositive]
We prove the contrapositive: If both $x$ and $y$ are rational, then $x + y$ is rational.

Assume $x$ and $y$ are both rational. Then $x = \frac{a}{b}$ and $y = \frac{c}{d}$ for some integers $a, b, c, d$ with $b \neq 0$ and $d \neq 0$.

We have:
\begin{align*}
x + y &= \frac{a}{b} + \frac{c}{d}\\
&= \frac{ad + bc}{bd}
\end{align*}

Since $a, b, c, d$ are integers, we know that $ad + bc$ and $bd$ are integers, and $bd \neq 0$ (since $b \neq 0$ and $d \neq 0$).

Therefore $x + y = \frac{ad + bc}{bd}$ is rational.

Hence, by contrapositive, if $x + y$ is irrational, then either $x$ is irrational or $y$ is irrational.
\end{proof}
\vfill

\item Use proof by contrapositive to prove each statement below.
	\begin{enumerate}
	\item If the product of two integers $ab$ is even, then $a$ is even or $b$ is even.
	
	\begin{proof}[Proof by Contrapositive]
We prove the contrapositive: If $a$ is odd and $b$ is odd, then $ab$ is odd.

Assume $a$ and $b$ are both odd. Then $a = 2m + 1$ and $b = 2n + 1$ for some integers $m$ and $n$.

We have:
\begin{align*}
ab &= (2m + 1)(2n + 1)\\
&= 4mn + 2m + 2n + 1\\
&= 2(2mn + m + n) + 1
\end{align*}

Since $2mn + m + n$ is an integer, we see that $ab = 2(2mn + m + n) + 1$ is odd.

Therefore, by contrapositive, if $ab$ is even, then $a$ is even or $b$ is even.
\end{proof}
	\vfill
	\item If $n^2$ is a multiple of $3,$ then $n$ is a multiple of 3.
	
	\begin{proof}[Proof by Contrapositive]
We prove the contrapositive: If $n$ is not a multiple of 3, then $n^2$ is not a multiple of 3.

Assume $n$ is not a multiple of 3. By the division algorithm, we can write $n = 3q + r$ where $q$ is an integer and $r \in \{0, 1, 2\}$. Since $n$ is not a multiple of 3, we have $r \neq 0$, so $r \in \{1, 2\}$.

\textbf{Case 1:} $r = 1$, so $n = 3q + 1$.

Then:
\begin{align*}
n^2 &= (3q + 1)^2\\
&= 9q^2 + 6q + 1\\
&= 3(3q^2 + 2q) + 1
\end{align*}

Since $3q^2 + 2q$ is an integer, $n^2 = 3(3q^2 + 2q) + 1$ leaves a remainder of 1 when divided by 3, so $n^2$ is not a multiple of 3.

\textbf{Case 2:} $r = 2$, so $n = 3q + 2$.

Then:
\begin{align*}
n^2 &= (3q + 2)^2\\
&= 9q^2 + 12q + 4\\
&= 9q^2 + 12q + 3 + 1\\
&= 3(3q^2 + 4q + 1) + 1
\end{align*}

Since $3q^2 + 4q + 1$ is an integer, $n^2 = 3(3q^2 + 4q + 1) + 1$ leaves a remainder of 1 when divided by 3, so $n^2$ is not a multiple of 3.

In both cases, $n^2$ is not a multiple of 3.

Therefore, by contrapositive, if $n^2$ is a multiple of 3, then $n$ is a multiple of 3.
\end{proof}
	\vfill
	\newpage
	\item Suppose $x \in \mathbb{R}.$ If $x^7-3x^4+10x^3-x^2-\pi \geq 0,$ then $x \geq 0.$
	
	\begin{proof}[Proof by Contrapositive]
We prove the contrapositive: If $x < 0$, then $x^7 - 3x^4 + 10x^3 - x^2 - \pi < 0$.

Assume $x < 0$. We analyze the sign of each term in the expression $x^7 - 3x^4 + 10x^3 - x^2 - \pi$:
\begin{itemize}
\item $x^7 < 0$ (negative number to an odd power is negative)
\item $-3x^4 < 0$ (since $x^4 > 0$ for $x \neq 0$, so $-3x^4 < 0$)
\item $10x^3 < 0$ (negative number to an odd power is negative, and $10 \cdot (\text{negative}) < 0$)
\item $-x^2 < 0$ (since $x^2 > 0$ for $x \neq 0$, so $-x^2 < 0$)
\item $-\pi < 0$ (since $\pi > 0$)
\end{itemize}

Therefore:
$$x^7 - 3x^4 + 10x^3 - x^2 - \pi = (\text{negative}) + (\text{negative}) + (\text{negative}) + (\text{negative}) + (\text{negative}) < 0$$

Hence, by contrapositive, if $x^7 - 3x^4 + 10x^3 - x^2 - \pi \geq 0$, then $x \geq 0$.
\end{proof}
	\vfill 
	\end{enumerate}
\end{enumerate}
\end{document}  















