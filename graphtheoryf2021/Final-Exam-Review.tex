%%%%%%%%%%%%%%%%%%%%%%%%%%%%%%%%%%%%
% 
% This top part of the document is called the 'preamble'.  Modify it with caution!
%
% The real document starts below where it says 'The main document starts here'.

\documentclass[12pt]{article}

\usepackage{amssymb,amsmath,amsthm}
\usepackage[top=1in, bottom=1in, left=1.25in, right=1.25in]{geometry}
\usepackage{fancyhdr}
\usepackage{enumerate}
%for the picture. 
\usepackage{tikz, calc}
%% The next line says how the "vertex" style of nodes should look: drawn as small circles.
\tikzstyle{vertex}=[circle, draw, inner sep=3pt, minimum size=6pt]
%%
%% Next, we make a \vertex command as a shorthand in place of \node[vertex} to get that style.
\newcommand{\vertex}{\node[vertex]}
% Comment the following line to use TeX's default font of Computer Modern.
\usepackage{times,txfonts}

\newtheoremstyle{homework}% name of the style to be used
  {18pt}% measure of space to leave above the theorem. E.g.: 3pt
  {12pt}% measure of space to leave below the theorem. E.g.: 3pt
  {}% name of font to use in the body of the theorem
  {}% measure of space to indent
  {\bfseries}% name of head font
  {:}% punctuation between head and body
  {2ex}% space after theorem head; " " = normal interword space
  {}% Manually specify head
\theoremstyle{homework} 

% Set up an Exercise environment and a Solution label.
\newtheorem*{exercisecore}{Exercise \@currentlabel}
\newenvironment{exercise}[1]
{\def\@currentlabel{#1}\exercisecore}
{\endexercisecore}

\newcommand{\localhead}[1]{\par\smallskip\noindent\textbf{#1}\nobreak\\}%
\newcommand\solution{\localhead{Solution:}}

%%%%%%%%%%%%%%%%%%%%%%%%%%%%%%%%%%%%%%%%%%%%%%%%%%%%%%%%%%%%%%%%%%%%%%%%
%
% Stuff for getting the name/document date/title across the header
\makeatletter
\RequirePackage{fancyhdr}
\pagestyle{fancy}
\fancyfoot[C]{\ifnum \value{page} > 1\relax\thepage\fi}
\fancyhead[L]{\ifx\@doclabel\@empty\else\@doclabel\fi}
\fancyhead[C]{\ifx\@docdate\@empty\else\@docdate\fi}
\fancyhead[R]{\ifx\@docauthor\@empty\else\@docauthor\fi}
\headheight 15pt

\def\doclabel#1{\gdef\@doclabel{#1}}
\doclabel{Use {\tt\textbackslash doclabel\{MY LABEL\}}.}
\def\docdate#1{\gdef\@docdate{#1}}
\docdate{Use {\tt\textbackslash docdate\{MY DATE\}}.}
\def\docauthor#1{\gdef\@docauthor{#1}}
\docauthor{Use {\tt\textbackslash docauthor\{MY NAME\}}.}
\makeatother

% Shortcuts for blackboard bold number sets (reals, integers, etc.)
\newcommand{\Reals}{\ensuremath{\mathbb R}}
\newcommand{\Nats}{\ensuremath{\mathbb N}}
\newcommand{\Ints}{\ensuremath{\mathbb Z}}
\newcommand{\Rats}{\ensuremath{\mathbb Q}}
\newcommand{\Cplx}{\ensuremath{\mathbb C}}
%% Some equivalents that some people may prefer.
\let\RR\Reals
\let\NN\Nats
\let\II\Ints
\let\CC\Cplx

%%%%%%%%%%%%%%%%%%%%%%%%%%%%%%%%%%%%%%%%%%%%%%%%%%%%%%%%%%%%%%%%%%%%%%%%%%%%%%%%%%%%%%%
%%%%%%%%%%%%%%%%%%%%%%%%%%%%%%%%%%%%%%%%%%%%%%%%%%%%%%%%%%%%%%%%%%%%%%%%%%%%%%%%%%%%%%%
% 
% The main document start here.

% The following commands set up the material that appears in the header.
\doclabel{Graph Theory: Review for Midterm 3}
\docauthor{}
\docdate{\date}

\begin{document}
The final exam will be given Monday 6 December from 8:00am-10:00am.\\

You may bring a page of notes, handwritten, front and back.\\

\noindent \textbf{Topics by Section}\\

\fbox{1.1}  graph, vertex set, edge set, endpoints, loop, multiple edges, adjacent, neighbors, complement, clique, independent set, stable set, bipartite graph, subgraph, connected, disconnected, isomorphism, isomorphism class\\


\fbox{1.2}  walk, trail, path, internal vertices of a walk/trail/path, endpoints of a walk/trail/path, length, closed, components, isolated vertex, cut-edge, cut-vertex, induced subgraph, bipartition, $X,Y$-bigraph, Eulerian circuit, Eulerian graph, even graph, maximal path, maximum path, $C_k$, $P_k$, $K_n$, $K_{n,m}$, biclique, Petersen graph\\

\fbox{1.3} degree of a vertex, $d_G(v)$, $\Delta(G)$, $\delta(G)$, maximum degree, minimum degree, $k$-regular graph, $N_G(v)$, order, size, $n(G)$, $e(G)$, degree sequence, graphic sequence\\

\fbox{2.1} acyclic, forest, tree, leaf, pendant vertex, spanning subgraph, spanning tree, star, distance, $d_G(u,v)$, diameter, $diam(G)$, eccentricity, $\epsilon(v)$, radius, $rad(G)$, center\\

\noindent \textbf{major theorems / useful results}\\

\fbox{1.2}\\
Every $uv$-walk contains a $uv$-path.\\
Every graph with $n$ vertices and $k$ edges contains at least $n-k$ components.\\
An edge is a cut-edge iff it lies on a cycle.\\
A graph is bipartite iff it has no odd cycle.\\
A graph is Eulerian iff it has at most one nontrivial component and all vertices have even degree.\\
Even graphs decompose into cycles.\\
Every graph with a nonloop edge has at least two vertices that are not cut-vertices.\\

\fbox{1.3}\\
The Degree-Sum Formula\\
A $k$-regular bipartite graph has the same number of vertices in each partite set.\\
If the $n$-vertex graph $G$ has $\delta(G) \geq (n-1)/2,$ then $G$ must be connected.\\
The minimum number of edges in an $n$-vertex graph is $\lfloor n^2/4 \rfloor.$\\
For every degree sequence with an even sum, there is a (multi)graph that realizes that sequence.\\
Havel-Hakimi Theorem and implied algorithm\\

\fbox{2.1}\\
Every tree with at least two vertices has at least two leaves.\\
Deleting a leaf from a tree on at least two vertices gives a tree on $n-1$ vertices.\\
A list of equivalent definitions of a tree.\\
Adding a single edge to a tree produces a unique cycle.\\
Every connected graph contains a spanning tree.\\
If $G$ is a simple graph with $diam(G) \geq 3$, then $diam(\overline{G}) \leq 3$\\
The center of a tree is $K_1$ or $K_2.$\\
If $H \subseteq G$, then for every $u,v \in V(H)$ $d_H(u,v) \geq d_G(u,v).$\\


\fbox{2.2}  Pr\"{u}fer code, Cayley's Formula\\


\fbox{2.3}  minimum weight spanning tree, weighted graph, Kruskal's Algorithm for finding minimum weight spanning trees, Dijkstra's Algorithm for finding the distance between all vertices and a given vertex, Breadth-First Search for finding distances in an unweighted graph, Chinese Postman Problem\\

\fbox{3.1} matchings, perfect matching, saturated/unsaturated vertices, maximal/maximum matchings, $M$-alternating path, $M$-augmenting path, symmetric difference, Berge's Theorem (3.1.10), Hall's Condition, Hall's Theorem (3.1.11), vertex cover, K\"{o}nig-Egerv\'{a}ry Theorem (3.1.16)\\

\fbox{3.2} Augmenting Path Algorithm, Hungarian Algorithm\\

\fbox{3.2}  Stable matchings, Gale-Shapley Proposal Algorithm\\

\fbox{4.1} separating set/vertex cut, connectivity, $\kappa(G)$, $k$-connectivity, disconnecting set of edges, $k$-edge-connectivity, edge-connectivity, $\kappa'(G),$ edge cut\\

Whitney's Theorem (4.1.9) If $G$ is a simple graph, $\kappa(G) \leq \kappa'(G) \leq \delta(G).$\\

Them 4.1.11 If $G$ is 3-regular, then $\kappa(G) = \kappa'(G).$\\

\fbox{4.2} internally disjoint $uv$-paths, $xy$-cut\\

Whitney's Theorem (4.2.2) A graph $G$ with at least 3 vertices is 2-connected if and only if for each pair $u,v \in V(G)$ there exist internally disjoint $uv$-paths.\\

Theorem 4.2.4 (a list of 4 statements equivalent to being 2-connected)\\

Menger's Theorem: If $x,y$ are nonadjacent vertices of the graph $G,$ then the minimum number of edges in an $xy$-cut is equal to the maximum number of pairwise internally disjoint $xy$-paths.\\

\fbox{5.1} $k$-coloring, color class, proper coloring, $k$-colorable, chromatic number, $k$-chromatic, color-critical, clique number, greedy coloring algorithm\\

Proposition 5.1.7 For every $G$, $\chi(G) \geq \omega(G)$ and $\chi(G) \geq \frac{n(G)}{\omega(G)}.$\\

Proposition 5.1.13 $\chi(G) \leq \Delta(G)$\\

Brooks' Theorem (5.1.22)\\

\end{document}


