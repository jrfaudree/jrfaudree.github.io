%%% Preamble starts here.
\documentclass{amsart}
%for the heading
\usepackage{fancyhdr, enumerate,hyperref}
%for the picture. 
\usepackage{tikz, calc}
%adjust the page width
\usepackage[margin=1in]{geometry}


\hypersetup{
    colorlinks=true,
    linkcolor=blue,
    filecolor=magenta,      
    urlcolor=blue,
}

\urlstyle{same}
%% The next line says how the "vertex" style of nodes should look: drawn as small circles.
\tikzstyle{vertex}=[circle, draw, inner sep=0pt, minimum size=6pt,fill=white]
%%
%% Next, we make a \vertex command as a shorthand in place of \node[vertex} to get that style.
\newcommand{\vertex}{\node[vertex]}

\linespread{1.1}


%special commands for number sets
\def\RR{{\mathbb R}}
\def\NN{{\mathbb N}}
\def\ZZ{{\mathbb Z}}
\def\QQ{{\mathbb Q}}
\def\CC{{\mathbb C}}

% header
\lhead{\sc  Graph Theory}
\chead{\sc Open Problem/Project Ideas } 
\rhead{Fall 2021}
\cfoot{}
\pagestyle{fancy}

%%%% Main document starts here.

\begin{document}
\thispagestyle{fancy}

\begin{enumerate}
\item Our textbook
	\begin{enumerate}
	\item the Reconstruction Conjecture (pg 38)
	\item Ringel's Conjecture \& Graceful Tree Conjecture (pg 87)
	\item the Overfull Conjecture \& the 1-factorization Conjecture (pg 279)
	\item Tutte's 4-flow Conjecture (pg 311)
	\item Cycle Double Cover Conjecture (pg 313)
	\item Strong Perfect Graph Conjecture (pg 320)\\
	\end{enumerate}

\item A site listing open problems specifically for undergraduates. \url{http://dimacs.rutgers.edu/~hochberg/undopen/graphtheory/graphtheory.html}\\

\item A site of open problems hosted by Doug West \url{https://faculty.math.illinois.edu/~west/openp/}\\

\item A massive site of open problems with an indicator (third column ``Rec") of problems recommended for undergraduates. \url{http://www.openproblemgarden.org/category/graph_theory}\\

\item Furman's Electronic Journal of Undergraduate Mathematics. \url{https://scholarexchange.furman.edu/fuejum/}\\

\item Mathematics Magazine and College Mathematics Journal (articles available through Rasmuson Library)\\

You would find articles here by googling "graph theory" and "Mathematics Magazine" and starting to follow a google-trail.\\

\item Electronic Journal of Combinatorics \url{https://www.combinatorics.org/}\\

\item Everyone should know about the searchable database: MathSciNet \url{https://mathscinet.ams.org/mathscinet}.\\
Note that it is accessible on-campus or via vpn.
\end{enumerate} 

\end{document}
