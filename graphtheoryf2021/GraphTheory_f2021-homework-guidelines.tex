\documentclass[11pt]{article}
\usepackage[margin=1in, head=1in]{geometry}
\usepackage{amsmath, amssymb, amsthm}
\usepackage{fancyhdr}
\usepackage{graphicx,hyperref}
%% set helvetica font
\renewcommand{\familydefault}{\sfdefault}
\usepackage[scaled=1]{helvet}
\usepackage[helvet]{sfmath}
\everymath={\sf}

\addtolength{\textwidth}{.5in}
\addtolength{\leftmargin}{-1in}
\addtolength{\textheight}{.5in}
\addtolength{\topmargin}{-0.5in}

\setcounter{secnumdepth}{0}
\newcommand{\R}{\mathbb{R}}
\newcommand{\N}{\mathbb{N}}
\newcommand{\Z}{\mathbb{Z}}
\newcommand{\clm}{\par\textit{Claim:}\par}
\newcommand{\diam}{\mathrm{diam}}
\newcommand{\sect}{\textsection}

\parindent=0in
\parskip=0.5\baselineskip

\begin{document}

\begin{center}Graph Theory (Math 430/663)  \\ Fall 2021 \\ Homework Guidelines
\end{center}
\hrulefill

Homework will be graded based on completion and effort. \\

For the most part, I will read your homework superficially. However, if you indicate I should look more carefully at a problem, I will do so.\\

All homework should be submitted in a single PDF file that has been typeset using \LaTeX. \\

Embedded figures (even hand-drawn ones) are acceptable provided they are legible.\\

The purpose of homework is to learn about the subject by engaging with it. Here is a suggested algorithm for getting the most out of the homework.

\begin{itemize}
\item Start the homework of Section X as soon as it has been discussed in class.
\item Type all problems into \LaTeX \: and see if you understand the problem.
\item Attempt the problem independently.
\item If you get stuck, consult appropriate sources. (Don't understand what the problem is asking? Ask Jill or a classmate. Can't think of a good strategy? Look back in the text for similar problems or theorems that might apply. Ask Jill or a classmate.)
\item Consult a complete solution only after you have established clearly what the problem is asking, what is a reasonable strategy for attacking the problem and have attempted that strategy.
\item Once you think you understand how solve the problem, write your solution without using any external sources.
\end{itemize}

\textbf{Bright Lines}
\begin{itemize}
\item It is ok to read a complete solution online. It is ok to have a classmate explain their complete solution.
\item It is \textbf{not} ok to ask to see another student's complete solution.
\item It is \textbf{not} ok to give another student your complete solution.
\item It is \textbf{not} ok to cut-and-paste a solution as your own in any circumstances, even if you give the other source credit.
\item It is \textbf{not} ok to repeatedly refer to or line-by-line paraphrase another's solution. 
\item If you \emph{always} write up your solution independently (ie without notes copied from another source, without repeatedly referring to another source) you will never violated any ethical boundary.
\end{itemize}

\end{document}